How do I thank all the people that helped me throughout my PhD?\@ I've literally
left this section for last since it's the most challenging.

Let me start by thanking Frans and Nickolai, who I've been extremely fortunate
to have as my advisors from the start of my PhD. Frans is a fantastic advisor in
every sense, from technical advice to life advice. He has a knack for keeping
track of the bigger picture, honing in on the contributions, finding the
\emph{technical nugget} in every research project, and telling just the right
story. It amazes me every time how under Frans's guidance a pile of code and
proofs turns into a research paper. Frans was also extremely supportive of all
my extra-curricular commitments, whether it was being in the CSC, the Comm Lab,
or going to PAX East. Nickolai greatly strengthened the work in this thesis. He
has an awesome ability to understand the high-level and low-level details of
just about anything. Nickolai equally contributed to high-level direction about
what to verify, as well as to technical details like figuring out how NFS works.
Frans and Nickolai are exceptionally hands-on: they contributed directly to the
implementation and proofs described in this thesis.

Next I have to thank Joe Tassarotti, my collaborator and also co-advisor for
this thesis. Our collaboration has always been totally natural and fun. Joe is
responsible for introducing me to Iris, and for much of the development of
Perennial.

Butler Lampson has been an amazing source of clear thinking around the value of
verification, both while teaching 6.826 and also as a reader on this thesis. I
also worked with Adam Chlipala and Robert Morris on projects not part of this
thesis but important to my journey as a researcher.

Many people helped me get started with research before grad school. My
undergraduate advisor, Indy, taught me a lot about the process of conducting
research through his patient and hands-on guidance, and was responsible in no
small part for getting me to MIT.\@ I also worked with Prof.~John D'Angelo in
the math department, who greatly increased my confidence in my mathematical
ability. Before that I'm also grateful to professors P.~R.~Kumar and Praveen
Kumar for giving me a chance to experience research in high school before I had
any idea what I was doing.

I've had a great group of friends keeping me sane throughout my PhD, so thanks
to all of you --- apologies for not listing all of you by name. Special thanks
go to Jon Gjengset, Davis Blalock, Sara Achour, Anish Athalye, Ben Sherman,
Cl\'ement Pit-Claudel, Ajay Brahmakshatriya, Alexandra Henzinger, Max Wolf, and
Ted Helm. Also thanks to all the folks I played board
games with through these many years, and Leilani Battle and Nathan Beckmann for
getting me into board games in the first place.

My time in the Comm Lab has been hugely valuable to me in my growth as a
communicator and coach. It has especially been due to the constant guidance and
mentorship of Diana and Deanna, who have been better managers than I could have
asked for.

Thanks to all of my mentees --- Daniel Ziegler, Alex Konradi, Lef Ioannidis,
Sydney Gibson, Sharon Lin, and Mark Theng --- for putting up with me. I learned
a lot from all of you. Alex, Sydney, and Mark in particular did work that
directly contributed to improving this thesis.

On a technical note, the artifacts of this thesis depend crucially on Coq and
Iris, so thanks to both of those communities, especially Ralf Jung and Robbert
Krebbers.

I am truly indebted to my parents in a way that I cannot really express. Thanks
for supporting me from the very beginning. I knew I wanted to get a PhD even
before undergrad, and they always encouraged me to follow that dream. My sister
has also always been there to provide advice and perspective.
