DaisyNFS draws on several lines of research. This chapter both explains the
research DaisyNFS directly builds upon and other work solving similar problems.

\resume

\section{The Flashix file system}

The Flashix project deserves special attention since it also develops a verified
concurrent and crash-safe file system~\cite{bodenmuller:concurrent-flashix}.
Flashix has different goals since it is a file system for flash storage, a
lower-level technology than the drives DaisyNFS targets. The verification
infrastructure used is designed for the specific use case of a flash file
system, including the techniques used for concurrency and crash safety, whereas
Perennial is a general reasoning system. Flashix is implemented using abstract
data types in a high-level language; a code generator transforms this code into
executable C, but this process is both not verified and has difficulty producing
the most efficient code using in-place updates.

While Flashix does have mechanized proofs, the approaches for both concurrency
and crash safety are layered on top in such a way that the top-level
specification and proof are not entirely within the proof assistant. Perennial
on the other hand formalizes the complete specification down to a simple
semantics for crashes and concurrency. We do need some argument to connect the
transaction system proof to the theorems proven in Dafny, which is a limitation
of combining verification frameworks, but the upshot is a much lower proof
overhead.

The Flashix project found that components in the stack intertwine due to the
realism of the system; modularity helped but the interfaces still had many
dependencies. We had a similar experience within the transaction system, where
performance constrained the APIs of internal layers and forced us to export
complicated APIs. GoTxn was perhaps trickier than the components of Flashix
because we model code at a lower level of abstraction, so even issues like
concurrency and memory safety show up in each interface. We were inspired to
develop the DaisyNFS design to get truly sequential reasoning by the pain in
verifying GoTxn, and found modularity and clean abstractions much easier once
code ran within a transaction.

Where it lacks in generality, Flashix does make up for in verifying a taller
storage stack than DaisyNFS. Flash storage has a more limited API than a
standard drive --- in particular flash blocks must be completely erased before
being reused. A flash file system works with this limitation with \emph{garbage
collection}, where occasionally the system identifies a mostly-unused block,
moves its valid data elsewhere, and then erases it to reclaim space. This is
implemented by maintaining a logical-to-physical block mapping similar to
virtual memory. DaisyNFS does not require any of this code. We do generally run
on SSDs which internally use flash blocks and themselves implement garbage
collection and a logical block mapping so that they can support overwriting data
in-place.

\section{Crash-safety foundations}

systems with custom infrastructure: FSCQ, Yggdrasil, VeriBetrKV

FTCSL

persistent memory semantics

\section{Concurrency foundations}

Old stuff like Views, FCSL; VST and CCAL. Few systems but do have CertiKOS.

Iris

\section{Other stuff}

ext4 semantics from Azalea's group

\section{Related to GoTxn}

comparison of system to jbd2

Push/Pull model

DFSCQ logging system. ARIES and its FTCSL proof

To the best of our knowledge, \txn is the first verified concurrent,
crash-safe journaling system. The verification of \txn builds on a
large body of previous work, as described in the rest of this
section.

\subsection{Perennial 2.0 vs Perennial 1.0}

The verification approach we take is based on a new version of our earlier
Perennial~\cite{chajed:perennial} framework, so we draw a contrast between the
two here. The new implementation is
conceptually similar in that it supports reasoning about concurrency and
crash-safety, it is implemented on top of the
Iris~\citep{jung:iris-jfp,jung:iris-1} concurrency verification system,
and it uses Goose~\cite{chajed:goose-coqpl} to enable verification of Go
programs by translating them into a model in Perennial 2.0.
However, to make verification of \txn feasible, we had to re-write many core parts of the framework.
To clarify which framework is being referenced we will write Perennial 1.0 for the
original framework and Perennial 2.0 for the new one in this section, in order
to highlight the new features Perennial 2.0 supports. The rest of the paper
generally refers only to Perennial 2.0.

%Perennial 1.0 supported verifying Go programs with a system called
%Goose~\cite{chajed:goose-coqpl} that translates Go into a model in Perennial.
%Perennial 2.0 also comes with a new version of Goose that supports the new
%verification infrastructure and extends translation to cover additional features
%in Go used in \txn.



Some of Perennial 2.0's features are needed to support the \txn top-level
specification and enable verification on top of this interface. The reason this
problem is complicated is because the journal does not make operations
automatically atomic but requires the caller to correctly manage ownership, and
Perennial 1.0's refinement specifications do not give a good way to talk about
ownership. The top-level specification of \txn relies on \emph{crash
framing}~(\autoref{s:design:crashframe}) and \emph{crash-aware
locks}~(\autoref{s:design:crashlock}) to enable application proofs that reason
about ownership of durable data.

Perennial 2.0 also scales to a larger system than the mail server verified in
Perennial 1.0. One of the challenges with the larger system is that it has many
internal layers that need their own specifications, so that the proof can be
carried out modularly. Normally a separation logic or refinement-based
specification would be sufficient, but we need internal specifications that capture
the crash and concurrent behavior of each internal library. To that end
Perennial 2.0 incorporates a new specification style which adds \emph{crash
atomicity} to the logically atomic specification styles developed in earlier
work~\citep{jacobs:logatom,svendsen:hocap,pinto:tada}. Modularity in the proof
was necessary to scale verification to
all of \txn's performance optimizations and concurrency.
At the same time, \txn's specification allows
the proof of \simplenfs to mostly avoid reasoning about crashes.

%% The specification for the simple NFS server is based on closely reading RFC
%% 1813~\cite{RFC:1813}, which defines the NFSv3 API in English prose. We do not
%% attempt to formalize all allowed behaviors in the specification, but did attempt
%% to write a formal specification in Coq which would meet the prose specification.
%% \mfk{maybe say we have some confidence in our formal spec, because
%%   \simplenfs can be mounted and used by a linux client.}

%% \joe{talk about JBD2 here}

%% Lifting and disk-object ownership are
%% new techniques that would not more complicated in Perennial due to the
%% need for leases. Perennial does demonstrate a complex example that
%% requires recovery helping to prove correct; our framework does not
%% support this pattern, but we did not need it to prove \txn or the file
%% system since no helping is involved.

%% The most obvious difference to Perennial is the difference in verified artifact:
%% we prove \txn, a \gotxnLOC-line, high-performance transaction system with interesting
%% crash safety and concurrency challenges, along with a \simplenfsLOC-line NFS server using
%% it; Perennial verified a 150-line mail server. Besides simply having much more
%% code, \txn is verified in several modular layers, with four internal APIs, and
%% then we use the specification to verify an application. Perennial did not
%% explain or demonstrate a modular verification story; our contribution of
%% logically atomic crash specifications allows us to scale verification to this
%% larger system by specifying internal APIs, which are more complex than the
%% client-visible one.

%% Perennial did not include explicit crash conditions, instead carrying out all
%% crash reasoning using an invariant. This works for small examples but is
%% inconvenient for modular verification. We instead build on a framework that uses
%% Iris but augments Hoare logic specifications with a crash condition, in the
%% style of FSCQ's Crash Hoare Logic, but extended to support concurrency.

% \subsection{Related verification efforts}

\subsection{Related verification frameworks}

\paragraph{Crash-safe systems.}

% Perennial 2.0's atomic crash specifications enable \txn to formalize the crash behavior
% of a layer in a lightweight fashion---that is, promising that a layer does not
% expose any unexpected intermediate states after a crash.
Any crash-safe system
must reason about the possible states after a crash, and several prior works
have formalized this in different ways for \emph{sequential} crash-safe systems.
FSCQ~\cite{chen:fscq,chen:dfscq} uses Crash Hoare Logic (CHL) to specify crash
behavior through a crash condition, which describes the state of a system if a
crash happens during execution of a function. Alternatively, a number of systems
verify crash safety using refinement
reasoning~\cite{sigurbjarnarson:yggdrasil,ernst:crash-refinement-asms,chajed:argosy,hance:veribetrkv},
but none support the combination of concurrency and crash-safety.

% However, refinement-based proofs can require additional layers for proof purposes
% that do not correspond to modules in the original code.

% However, refinement based proofs can be heavy-weight because they require
% formalizing the entire set of operations that can be performed at every layer of
% a refinement proof (including logically orthogonal operations like allocating
% memory, accessing data structures, and calling other libraries).
% \ralf{I don't understand the last sentence. It seems to be specifically about CSPEC, not refinement proofs in general?}

Although they are not concurrent, some of these systems address other
aspects of performant storage systems that are not found in \txn.
DFSCQ~\cite{chen:dfscq} verifies a high-performance file system built
on top of a logging system with asynchronous disks and log-bypass
writes, which are challenging optimizations that \txn does not
support. VeriBetrKV~\cite{hance:veribetrkv} verifies a key-value store
based on B\textsuperscript{$\epsilon$} trees, a data structure that also underlies BetrFS~\cite{jannen:betrfs}. \txn
and \simplenfs use simple data structures;
the challenge lies in accounting for concurrent accesses.
%where the challenge is concurrent access.
% concurrent access to these structures.

% The proof technique uses refinement between layers
% in a way that is lighter-weight and more flexible than in CSPEC,a and
% additionally uses the crash guarantees of one layer to prove the crash
% guarantees of a higher layer. VeriBetrFS is implemented in Dafny, which has
% tight integration with Hoare logic; this makes it easier to use but also harder
% to extend, particularly with concurrency. In Coq the program logic is not
% built-in, which gives the flexibility to implement a new logic (as FSCQ did) or
% to extend an existing one (as we do on top of the Iris program logic).

\paragraph{Concurrent systems.}

In addition to specifying behavior at intermediate crash points, Perennial 2.0's
specifications describe the atomic commit points of concurrent operations. A
range of verification techniques have been used to address this kind of
challenge in concurrent systems. AtomFS~\cite{zou:atomfs} uses a framework
called CRL-H (concurrent relational logic with helpers) to verify a concurrent
in-memory file system implemented in C. Refinement-based systems such as
CSPEC~\cite{chajed:cspec}, Armada~\cite{lorch:armada}, and Concurrent
CertiKOS~\cite{gu:certikos-ccal} typically prove that a function implements an
atomic operation at a more abstract layer.
However, in \txn, many internal APIs provide operations that are only atomic if
the caller owns some data. This kind of conditional atomicity is easy to express
in Perennial 2.0 using separation logic, but hard to express
as a precondition in a transition system.
% Additionally, as with using refinement for crash-safety reasoning, refinement-based approaches often introduce extra layers for proof purposes. On the other hand, proofs of refinement between
% layers in an SMT-based system like Armada are automated, while proofs are
% interactive in Perennial 2.0.

% Armada~\cite{lorch:armada} also verifies C code but uses a different approach
% based on several layers of refinement between levels, which are progressively
% more abstract versions of the code. Neither of these frameworks
% supports crash-safety guarantees, and neither supports modular verification of a
% lower-level layer independently of upper layers that use it.


\paragraph{Concurrent, crash-safe reasoning.}

Program logics other than Perennial
have been developed for formal reasoning about concurrent, crash-safe systems.
Fault-Tolerant Concurrent Separation Logic (FTCSL)~\cite{ntzik:faults} extends
the Views~\cite{dinsdale:views} concurrency logic to incorporate crash-safety.
POG~\cite{raad:pog} is a program logic for reasoning about the interaction of
x86-TSO weak-memory consistency and non-volatile memory.
Neither logic has a mechanism for modular proofs of layers,
which we found essential to scale verification to a system of \txn's
complexity. Both are restricted to pen-and-paper proofs, whereas both Perennial
1.0 and 2.0 have machine-checked proofs.

A specification called the Push/Pull model of
transactions~\cite{koskinen:pushpull} is similar to the \emph{lifting} technique
in the journal system's specification~(\autoref{s:design:lifting}) --- the core
problem addressed is that a journal operation atomically modifies a small number
of objects, but other objects can change between the start of the operation and when
it commits. The Push/Pull model also discusses reasoning on top of the
specification, using Lipton's reduction~\cite{lipton:movers} rather than
separation-logic ownership to handle concurrency. However that work is about
on-paper specifications and proofs, while we also prove an implementation meets
our specification and proved \simplenfs on top.


\section{Related to DaisyNFS}

some similarities to design of Yggdrasil

indirect block verification-friendly design from DFSCQ

The Dafny side of DaisyNFS is a new implementation but its design and aspects of
the proof strategy were inspired by other verified file systems like
DFSCQ~\cite{chen:dfscq} (especially its indirect block implementation described
in Konradi's master's thesis~\cite{akonradi-meng}) and
Yggdrasil~\cite{sigurbjarnarson:yggdrasil}.

AtomFS does concurrency reasoning

\sys is the first verified file system that has both concurrency and crash
safety, but it adopts many ideas from previous verified storage systems,
including some with concurrency. Our main contribution is an approach for
combining verification efforts in different proof systems, which isolates
concurrency reasoning to the transaction system. Prior work has also explored
how to compose proofs across layers for modularity, to contain concurrency, or
to cross between proof systems; we believe \sys advances this work by combining
two software stacks of significant size and complexity.

\paragraph{Other verified file systems}

%% Yggdrasil is implemented with a high degree of automation, with the
%% only lines of proof appearing as intermediate specifications, but the
%% infrastructure uses SMT solving directly and thus doesn't support
%% ordinary program verification like Hoare logic, recursive data
%% structures, and loop invariants, which \sys uses.

%% \mfk{in eval?} For example, in DFSQ, the file-system implementation on
%% top of the journal is about 28,000 lines of combined code and
%% proof. \sys is of comparable complexity in 2,500 lines of imperative
%% Dafny code, which is lower-level than the Gallina language used in
%% DFSCQ\@.

VeriBetrKV~\cite{hance:veribetrkv} is a verified key-value store of
the sort that underpins the BetrFS~\cite{jannen:betrfs} file system. It uses Dafny for
crash-safety reasoning but does not layer any file-system proof on
top. This file-system design does not involve general transactions, so
the code on top of the key-value store must still carry out crash
reasoning.

AtomFS~\cite{zou:atomfs} is a verified concurrent file system that
does not persist data. It uses a custom concurrent relational logic
implemented in Coq.  Because the system does not persist data, AtomFS
does not have any transaction system and implements the top-level
file-system operations together with appropriate locking for
concurrency control.

%\paragraph{GoJournal} \sys's transaction system uses
%GoJournal~\cite{chajed:gojournal}. GoJournal also has a notion of transactions,
%but the caller is required to use locks appropriately to guarantee concurrent
%transactions do not access the same data. Instead of doing this manually, \sys extends GoJournal with
%two-phase locking so that the file system code on top is automatically well-synchronized. Although verifying the two-phase locking requires
%sophisticated reasoning (see~\cref{sec:txn-proof}), this separation
%allows \sys to use Dafny for sequential reasoning and benefit from
%proof automation.

%% The GoJournal paper presents a
%% separation logic and claims to make reasoning mostly sequential once
%% the caller has established lock invariants. One could ask why we did
%% not directly prove an NFS server with that sequential reasoning. The
%% approach in this paper is better largely because it allows us to use
%% the best tool for that sequential reasoning, rather than using
%% Perennial, which was designed to make proofs of concurrent, crash-safe
%% systems possible without trying to make sequential proofs easy and
%% productive. We think this makes progress towards more accessible
%% verification tools while still reasoning about realistic, concurrent
%% code.

%% Another question relating to GoJournal is why we still needed to verify a
%% transaction system rather than directly relying on its existing results. One
%% simple reason why this was necessary is that GoJournal requires the caller to
%% ensure transactions are non-conflicting, and we needed to prove this somehow. We
%% chose to achieve this in the code with a generic transaction system. One can
%% view the transaction system as a particular use of GoJournal's ``mostly
%% sequential'' specification in order to support not application-specific
%% invariants and transaction but arbitrary ones. Because this is such a general
%% theorem and not the use case envisioned in the GoJournal specification we did
%% have to do some work to prove the transaction system correct, but it largely
%% builds upon rather than replaces the GoJournal specification. If we did not
%% handle the concurrency in a verified transaction system, it would be difficult
%% to set up Dafny to prove this obligation, since once we're using Dafny all code
%% is implicitly sequential. We were concerned that we had no principled way to
%% argue that the overall system is really correct if Dafny was responsible for
%% anything related to concurrency.

\subsection{Concurrency verification}

%% \paragraph{Reducing concurrency to sequential reasoning}
%% \joe{I'm not sure the subsection header works for this one either. Maybe we should kill the subsections? I would not really consider movers a form of composing proofs.}

%% One technique for composing proofs is Lipton's reduction proof technique
%% (sometimes called ``movers'')~\cite{lipton:movers}. The idea is to show that a
%% concurrent system's execution is always equivalent to a rearranged execution where
%% operations execute atomically, reducing the number of interleavings that must be
%% considered in the remainder of the proof.
%% There are many examples of frameworks that use reduction to reason about concurrent and distributed systems, including
%% CIVL~\cite{hawblitzel:civl}, CSPEC~\cite{chajed:cspec}, Armada~\cite{lorch:armada} and
%% IronFleet~\cite{hawblitzel:ironfleet}.

A number of verification frameworks address concurrency, including
CIVL~\cite{hawblitzel:civl}, CSPEC~\cite{chajed:cspec},
Armada~\cite{lorch:armada}, Iris~\cite{jung:iris-jfp}, CCAL~\cite{gu:certikos-ccal},
and FCSL~\cite{sergey:fcsl}, among many others. These frameworks use a range of
methods, such as movers~\cite{lipton:movers} and concurrent separation
logic~\cite{brookes:csl}. Here we focus on those that use techniques closely related to \sys:

% IronFleet~\cite{hawblitzel:ironfleet}.

%% \sys has a similar idea in proving that code running on the transaction system
%% behaves as if the transactions execute atomically. However, instead of using movers,
%% \sys uses Perennial's concurrent separation logic to show that the transaction system's
%% two-phase locking scheme ensures concurrent transactions access disjoint objects, and therefore do not interfere.
%% % However, we do not use the
%% % reduction proof technique directly, instead obtaining this property with a
%% % logical relations proof \tej{what do you cite for logical relations?}.
%% % This proof is able to use Perennial's modularity features to reason about a large
%% %implementation with crash safety and concurrency.
%% Reduction might be hard to scale to such an implementation, and moreover we do not know of any work that
%% extends the movers approach to also address crashes and recovery.


\paragraph{Isolating concurrency}
IronFleet~\cite{hawblitzel:ironfleet} uses Dafny to reason about distributed
system implementations, and VeriBetrKV~\cite{hance:veribetrkv} uses a similar
approach for a sequential storage system. To justify sequential reasoning for
each handler, these approaches use a \emph{reduction} argument, which is
separately formalized in Dafny. The approach in \sys is more general because the
illusion of atomic execution comes from the transaction system implementation,
rather than the nature of distributed execution. Rather than using only a
reduction argument, we use a general-purpose program logic, which is important
for scaling to a large implementation and extending existing techniques with
crash safety reasoning. Even systems which support reduction arguments like
CSPEC~\cite{chajed:cspec}, CIVL~\cite{hawblitzel:civl}, and
Armada~\cite{lorch:armada}, do not rely on only reduction to reason about
programs; we do not know of any work that has extended these systems and all
their proof techniques to incorporate crash safety.

% One difference is that IronFleet's
% protocol reasoning does not involve reasoning about additional code, but only
% the behavior of the lower-level handlers when run together; \sys combines a
% proof about a library with a proof about a code that uses the library.

% \paragraph{Concurrency and crash-safety reasoning} Two frameworks support a
% combination of crash-safety and concurrency: fault-tolerant concurrent
% separation logic (FTCSL)~\cite{ntzik:faults} and
% Perennial~\cite{chajed:perennial,chajed:gojournal}. FTCSL was
% only used to verify a transaction system on paper.  \sys uses
% Perennial to extend GoJournal with two-phase locking, and to verify
% its correctness.

\paragraph{Concurrent library verification}

The specification we prove for the transaction system is a program refinement,
which says something about any code that calls into the transaction system. This
style of specification is powerful for verifying libraries, but it is
challenging since it requires reasoning about any calling program. Some prior
work has verified program refinement (sometimes called contextual refinement in
the literature), notably CertiKOS~\cite{gu:certikos-ccal}, which uses the
approach for several layers in a verified concurrent OS kernel. Program
refinement was also used at a smaller scale in CSPEC~\cite{chajed:cspec} for
concurrent code and in the original Perennial work~\cite{chajed:perennial} for a
concurrent, crash-safe program.


% Another benefit of our approach is that we are not limited to a single
% proof approach: we leverage the custom program logic from Perennial in the
% transaction system's proof and switch to the Dafny sequential program logic for
% the transactions, then compose these proofs despite carrying them out in very
% different systems.

%\tej{moved from soundness section, need to fit into this section}
%
% A solution that leaves the Dafny proofs as simple as possible is to show outside
% of Dafny that the code appears to execute sequentially, justifying the implicit
% assumption in Dafny. IronFleet~\cite{hawblitzel:ironfleet} uses this approach
% with a verified reduction from a distributed (and thus concurrent) execution
% down to a sequential one, which holds as long as the Dafny code follows a
% particular pattern of interactions with the outside world.
% VeriBetrKV~\cite{hance:veribetrkv} takes a similar approach to deal with
% concurrency between a key-value store and a disk.

% In these approaches, the sequential code that is verified in Dafny is a method
% that handles a single request from the network or storage system. In \sys in
% contrast, the atomicity of transactions is due to the correctness of the entire
% transaction system implementation. \tej{haven't figured out how to explain this}

\subsection{Verified two-phase locking}

% Instead of
% acquiring locks throughout a transaction, the system tracks a \emph{read set} of
% values read.

% To commit, a thread checks if there have been any writes, and if so
% checks if anything in the read set has changed. If the read set is still valid,
% then the commit proceeds (while holding a global lock), and otherwise the
% transaction is forced to abort and start over. This implementation has low
% overhead for low contention workloads, and higher scalability for read-only
% workloads than our two-phase locking design. \tej{probably don't need to say so
% much, just say it's more sophisticated}

\subsection{NFS servers}

We chose to verify an NFS server because it is widely used in practice
and the expected behavior of NFS operations is well documented in
RFCs.  FUSE is an alternative for implementing file systems in user
space, but its operations have a less clear specification.

To be conducive to verification, \sys is implemented differently than
many NFS servers; in particular using two-phase locking is not common
practice.  Other user-level NFS servers are typically implemented on
top of an existing file system, relying on the underlying file system
for logging and locking. The Linux NFS server is implemented inside
the kernel using VFS and the ext3/ext4 file systems (if exporting an
ext3/ext4 file system).  Ext3 and ext4 use a journaling system, but
the file system and VFS layers perform locking.  WAFL~\cite{wafl:hitz}
is NFS appliance that provides snapshots and logs NFS requests to
NVRAM.  It has evolved its locking plan to obtain good
parallelism~\cite{curtis:wafl}.  Both the Linux NFS server and WAFL
are more complicated and have more features than \sys.

% Think Global, Act Local: A Buffer Cache Design for Global Ordering
% and Parallel Processing in the WAFL File System (ICPP)

