\section{Goals and motivation}
\label{sec:goose:goals}

There are three main goals for Goose: supporting \textbf{efficient} code,
\textbf{soundness} for the translation process, and \textbf{convenient
reasoning}. The subset of Go that Goose supports is
intended to enable writing and verifying high-performance code. Goose
is sound if the model captures the behaviors of the code, which is required for
the proofs to say something meaningful about the executable Go code. If the model
misses some buggy behavior in the code, then a correctness proof wouldn't mean
anything about the code. Finally, Goose aims to make reasoning convenient by developing reasoning
principles for the model of Go primitives like structs, slices, and maps.

There are alternate setups for verification where the connection between the
proofs and the code is not given by a model and translation. Goose supports
efficient code both by connecting to ordinary Go, which benefits from the Go
compiler and runtime, and by supporting a variety of features. Sometimes
efficient code is more complex to prove, but this is a tradeoff the user can
make.

The
design of Goose tries to achieve soundness through simplicity and careful choice
of what to model. There is sometimes a tradeoff between simplicity and
supporting efficient code. If a language feature is needed for good performance (for
example, taking pointers to individual struct fields), then Goose models it,
even if the feature is complex. If
a feature would only result in more idiomatic code and modeling it seems
subtle, then it might not be implemented (for example, simple uses of
\texttt{defer} could be modeled but aren't because the feature is complicated in
general). The result is that Goose is generally pleasant and productive enough
to write in, but requires some practice for a Go programmer.

Convenient reasoning remains a goal for Goose, but not one that was always
achieved. All of the verified libraries are usable but pain points remain; some
of these are simply a matter of engineering effort or fixing bugs, but we have
also found code patterns that weren't well captured in the reasoning.

\subsection{Why Go?}

Go is a convenient language for building verified systems. It is productive
enough to build systems that get good performance. The language is simple,
facilitating a sound translation.

For our first goal, efficiency, Go has enough features to build good systems
in. It has efficient and useful built-in slice and map data structures. The
runtime handles concurrency efficiently and has good support for synchronization
using locks and condition variables, allowing a low-level implementation.

There is also an advantage to Go as a programming environment rather than
programming language. The tooling for testing, debugging, and profiling is
extremely good, making it easy to fix bugs (before verification or in unverified
code) and find performance problems while optimizing. We were able to use
low-level interfaces to the operating system to access the disk --- these are
easier to understand in isolation, compared to understanding the combination of
a file-system library and the operating system's behavior. Garbage collection
simplifies writing code, particularly with concurrency, and carries a relatively
low performance impact due to the efficient runtime.

For the second goal, soundness, it helps that Go is a simple language. The Goose
translator effectively gives a semantics to the source code; in a complex
language this can be a daunting task (such as attempts to formalize JavaScript and
Python~\cite{guha:lambda-js,politz:python-semantics}). It isn't too
difficult to give Go a semantics, especially the Goose subset. Go's tooling
helped, including libraries for parsing and type-checking Go source code. Not
only do these libraries save time in implementing Goose, they greatly improve
reliability since they are written by experts (the Go compiler team, extracting
code from the compiler itself).

\subsection{Why not C or Rust?}

C is not too different from Go as a basis for Goose. The main differences are
the need to implement and verify manual memory management, and it would be more
challenging to parse and type-check C code.

Using Rust as a source language seems attractive but would likely not be much
better than Go. One subtlety is that while the source code is type checked, the
model is an untyped program. It would be difficult to take advantage of the fact
that the code is type-checked, and thus the verification engineer would anyway
re-prove memory safety in the same way as Goose requires with Go. The main
benefit would be reduced bugs before reaching verification.

Another difficulty with Rust would be the size and complexity of the language
--- the subset supported might be restrictive enough that the experience no
longer feels like Rust. For example, the \cc{Vec<T>} type has 152 methods
without even including trait implementations. Using any of these methods would
require assuming a semantics for it, which is trusted to be sound (that is,
getting the semantics wrong could compromise the whole verification). Thus in
practice the expansive standard library would mostly not be available; for
soundness only a core subset would be modeled and the rest manually implemented
and verified.
