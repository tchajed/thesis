\section{Related work}%
\label{sec:goose:rel-work}

There are several areas of related work. First and foremost Goose is an approach
to verifying executable programs, so this section discusses alternate
approaches. Second, Goose implicitly gives a semantics to Go, and there are
related projects giving semantics to other programming languages.

\subsection{Verification approaches}

At a high level, all verification systems need to solve the problem of
connecting the world of the proof assistant to the ``real world'' that runs the
code. The proofs are always over a model of the code, and the theorems are
always contingent on the assumption that the running code has been modeled correctly. The approach
imposes some requirements on the code that can be written and how proofs are
written. Thus each approach makes a tradeoff between how efficient verified code
can be, soundness, and
convenient reasoning, as outlined above in \cref{sec:goose:goals}.

There are basically three ways to verify programs (with several interesting
caveats). We can translate code to a model (as in Goose), pretty-print a
model to code, or compile from a language intended for both verification and
implementation. In addition, verified compilation and translation validation can
implement parts of the translation process from model to code in a verified way;
this can usefully make reasoning more convenient without sacrificing soundness.

It is easiest to understand this characterization in the context of concrete
examples. Goose translates from Go code to a model in Coq. Both
CFML~\cite{chargueraud:cfml} and hs-to-coq~\cite{spector-zabusky:hstocoq} work
in a similar way, for OCaml and Haskell respectively.
Fiat-Crypto~\cite{erbsen:fiat-crypto} prints a model of C code to a string that
is compiled with a C compiler. Programs in Dafny~\cite{leino:dafny} are
typically run using Dafny's built-in compiler (which has backends for C\#, Java,
JavaScript, and Go at the time of this writing). Similarly, languages like Coq
and \fstar have support for \emph{extraction} that translate functional programs
in those languages to something executable like OCaml.

There are approaches that are somewhere in between using built-in compilation
and writing code in a model. For example, FSCQ~\cite{chen:fscq} uses Coq
extraction, but it works from a model of I/O interaction that is implemented by
combining the extracted code with a Haskell library. This combines Coq's
extraction (compilation) with translating from a model to code. For \fstar,
there is a specialized toolchain called KaReMeL~\cite{protzenko:lowstar} that
targets a subset of \fstar called \lowstar and extracts imperative C code.

Verified compilation can reduce trust when using either approach.
VST~\cite{cao:vst-floyd} in its most basic form looks like a code-to-model
translation, using a tool called \cc{clightgen} to translate C code to an
abstract syntax tree (AST) in Coq, which the user can then specify and verify in
VST. However, it is then possible to run this AST through the CompCert verified
compiler and produce assembly code (or more specifically, a model of assembly
represented in Coq). The proof than covers this assembly model, which is
pretty-printed and compiled with an assembler to run it. While the user writes C
code, this system only requires trusting the model of assembly. On the other
hand, the performance of the code is determined by what features \cc{clightgen}
supports and the quality of the CompCert compiler.

Verified compilation is used in a way similar to VST in
Vale~\cite{bond:vale,fromherz:vale-fstar}. Code is written in a specialized
assembly-like language called Vale, with proof annotations. This is then
compiled to Dafny or \fstar (Vale has two backends) where the proofs are carried
out before finally being printed back down to assembly. Vale supports highly
efficient code because it is written in assembly, but this code pays a price in
being harder to verify.

Cogent~\cite{amani:cogent,oconnor:cogent-lang} also uses a form of verified
compilation known as translation validation to make proofs easier without compromising on sounds. Code
written in the Cogent language (an imperative language with linear types) is
translated to a model in Isabelle/HOL along with a functional specification for
the code and a proof of correspondence. Thus the user can write proofs on top of
the functional specification, but the proofs go down to an imperative model.

\subsection{Modeling programming languages}

Goose gives a semantics to a subset of Go. Each function in Go is translated to
a term in GooseLang, which has a semantics in Coq in the form of a transition
system. Giving a semantics to a real-world language is interesting in its own
right, not just for verifying code in that language. Many similar semantics
efforts are focused on the goal of completeness, in order to understand how all
of the language features interact.

CH20~\cite{krebbers:c-coq} is a fairly complete formalization of the C standard.
CH20 defines C in operational and axiomatic styles, along with an executable
semantics to test the semantics on (small) concrete programs. All of these
semantics are formally related to each other. Having multiple, related semantics
helps give confidence in each of them, since they independently express the same
thing in different ways.

VST~\cite{cao:vst-floyd} uses CompCert C to model C code. One similarity to
Goose is support for structs in terms of their individual fields. Both systems
need to model the semantics of features like taking a pointer to an individual
struct field (as CH20 also handles). What sets VST and Goose apart from CH20 is
to also have reasoning principles for verifying code that uses structs in
interesting ways, such as structs where a subset of fields are protected by a
lock.

RustBelt~\cite{jung:rustbelt} is intended as a model of Rust code, at the Rust
MIR (mid-level IR) level of abstraction. The modeling language in RustBelt,
\lambdarust, has many similarities to GooseLang, and both are used together with
Iris~\cite{jung:iris-1} and the Iris Proof Mode~\cite{krebbers:ipm}. RustBelt
has different goals, being intended for reasoning about Rust as a whole rather
than program verification. This is a somewhat subtle difference: the goal in
RustBelt is to prove properties about \emph{all} \lambdarust programs, whereas
we only prove correctness of \emph{particular} GooseLang programs. As a result
it is important for \lambdarust, together with its type system, to rule out
programs that Rust does not allow, whereas GooseLang is more expressive than Go
with unsafe features like pointer arithmetic. Due to different goals, RustBelt
also does not have an automatic translation from Rust to \lambdarust, though
some libraries have been translated by hand and verified using Iris.

Finally, WebAssembly is a rare example of a production language with a formal
semantics~\cite{haas:wasm}, and moreover it was designed with the formal
semantics in mind.
