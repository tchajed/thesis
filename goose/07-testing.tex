\section{Testing Goose}%
\label{sec:goose:testing}

Goose is a trusted component in the entire verification process. For the
overall system's proof to be sound, we rely on the model to produce all
of the behaviors of the Go code; that is, the behaviors of the Go code
(in practice, using the Go compiler) should be a subset of the behaviors
of its translated GooseLang (according to the Coq semantics). As long as
this is case, the proof is sound in that if the modeled system always
satisfies some property the code will, too.

One subtlety to Goose soundness is that the system is automatically sound if
Goose fails to translate some code, or the code does not compile in Coq, or the
model has undefined behavior; in each of these cases it is impossible to verify
incorrect code. These might still reflect a bug in Goose, or at least an
unsupported feature, but they do not compromose soundness. Therefore the
most important bugs are those where the translation's behavior differs
from that of Go; these can compromise soundness of the system and lead
to a proof that is not borne out in practice.

To increase out confidence in Goose, we implemented a large suite of
unit tests. On their own these tests check that Goose continues to translate
existing code (and check that the translation has not unexpectedly
change). To test the soundness of the translation, the relevant comparison is
between Go and GooseLang. Unfortunately GooseLang is not natively an executable language.
Its semantics is expressed as a Coq relation that describes the valid executions
of an expression, but not how to run a particular expression.

To test GooseLang code, we implemented an interpreter in Coq, which can
run GooseLang code and produce either an error due to undefined behavior
or a result. The interpreter is itself
verified to match the semantics. The specification for the interpreter is slightly
subtle in that the interpreter produces only one possible execution rather than
all the executions allowed by the semantics, but
the non-determinism is only due to the choice of what locations to use
for pointers, which should not affect any visible behavior.

GooseLang is a lambda calculus, so
its semantics is expressed as a transition system between expressions.
It is easy to
interpret \emph{pure} reductions like \cc{x + y} where \cc{x}
and \cc{y} are values, since the semantics of these pure expressions on their own
is already given as a Gallina function rather than a relation. The semantics of each core primitive is
given by a transition relation, which the interpreter implements as a function
and its correctness theorem shows this function produces an allowed transition.
For example, the semantics of allocation stores a value in an unused address,
whereas the interpreter concretely identifies an unused location (the maximum
used address, plus one).

The challenge in the interpreter's correctness theorem comes from \emph{context} reductions,
which specify how to find a sub-expression within \cc{e} to reduce
if the head is not immediately a value. The semantics follows a standard
presentation of context reduction using \emph{evaluation contexts}. The
idea is to define a type of evaluation contexts $E \in \mathcal{E}$ that
represent an expression with a hole; $E[e]$ represents filling that hole with
the expression $e$. The possible evaluation
contexts give all the context reductions in one compact rule, \ruleref{context-reduce}: if $e$
can step to $e'$, then $E[e]$ can step to $E[e']$. This rule applies whenever
such an $E$ exists, while the
interpreter recurses through an expression (in the right order) and
evaluates a sub-expression, then fills it into the context. The interpreter's
proof shows that
this traversal is correct, proving that the interpreter and semantics agree on an
evaluation order. Specifically, the interpreter proof shows the interpreter
produces a valid evaluation order, and a separate proof shows that evaluation
contexts are unique.\footnote{See the lemma
\href{https://github.com/mit-pdos/perennial/blob/6f5ed5e7c2d3e8d657a0022c51e1d1e32a81e671/src/goose_lang/lang.v\#L1443-L1447}%
{\cc{head_redex_unique} in src/goose\_lang/lang.v}.} There is other non-determinism in the semantics that the
interpreter does not fully explore, though, such as for allocation.

The test suite is structured as a number of test functions, each producing a
boolean that should be true. To check that the test itself is written correctly,
a Go test checks that it produces \cc{true}. Then to check the semantics of
the translation, the GooseLang test infrastructure translates the test and runs
it through the interpreter, checking that this produces $\goosetrue$ in
GooseLang. While the interpreter is not extremely efficient, it is fast enough
to run the tests in the test suite.

The interpreter and test framework was designed and implemented by
Sydney Gibson, and is described in greater detail in her master's
thesis~\cite{gibsons-meng}. That thesis includes more details on evaluating the
interpreter itself, for example documenting bugs caught by the test suite and
other bugs that are now part of our regression tests.
