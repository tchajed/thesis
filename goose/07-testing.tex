\section{Testing Goose}
\label{sec:goose:testing}

Goose is a trusted component in the entire verification process. For the
overall system's proof to be sound, we rely on the model to produce all
of the behaviors of the Go code; that is, the behaviors of the Go code
(in practice, using the Go compiler) should be a subset of the behaviors
of its translated GooseLang (according to the Coq semantics). As long as
this is case, the proof is sound in that if the modeled system always
satisfies some property the code will, too.

One subtlety in the trust we place in Goose is that it only applies when
Goose translates code successfully, that code compiles in Coq, and the
model has no undefined behavior. If any of these fail, then the proof of
the system would either not be possible or not go through. Therefore the
most important bugs are those where the translation's behavior differs
from that of Go; these can compromise soundness of the system and lead
to a proof that is not borne out in practice.

To increase out confidence in Goose, we implemented a large suite of
unit tests. While these tests check that Goose continues to translate
existing code (and check that the translation has not unexpectedly
change), for soundness the relevant test is to compare Go to the Goose
output. Unfortunately GooseLang is not natively an executable language.
Its semantics is expressed as a Coq relation that specifies how an
expression is evaluated (or gets stuck, indicating undefined behavior).
To test GooseLang code, we implemented an interpreter in Coq, which can
run GooseLang code and produce either an error due to undefined behavior
or a result. While the interpreter is not very efficient, it has good
enough performance to run the Goose unit tests.

The interpreter is an important part of the testing strategy, but
ultimately the comparison is intended to be between Go and the GooseLang
semantics. Thus we verified that the interpreter produces executions in
accordance with the semantics. The correctness theorem is slightly
subtle in that the interpreter produces only one possible execution, but
the non-determinism is only due to the choice of what locations to use
for pointers, which should not affect any visible behavior.

The technical challenge with implementing and verifying the interpreter
is that the semantics uses a convenient but non-executable way of
expressing the order of evaluation. GooseLang is a lambda calculus, so
its semantics is expressed as a transition system between expressions.
It is easy to give the semantics of a primitive at the \emph{head} of an
expression; for example, we can say what $\goosekw{Store}(l, v)$ does in a given
heap if $l$ and $v$ are already values (it stores $v$ in the heap
and evaluates to \texttt{\#()}, the unit value). It is also easy to
interpret \emph{pure} reductions like \cc{x + y} where \cc{x}
and \cc{y} are values since the semantics of these pure expressions
is already given as a Gallina function.

The challenge in the interpreter comes from \emph{context} reductions,
which specify how to find a sub-expression within \cc{e} to reduce
if the head is not immediately a value. The semantics follows a standard
presentation of context reduction using \emph{evaluation contexts}. The
idea is to define a type of evaluation contexts $E \in \mathcal{E}$ that
represent an expression with a hole; $E[e]$ represents filling that hole with
the expression $e$. The possible evaluation
contexts give all the context reductions in one compact rule, \ruleref{context-reduce}: if $e$
can step to $e'$, then $E[e]$ can step to $E[e']$. This rule applies whenever
such an $E$ exists, while the
interpreter recurses through an expression (in the right order) and
evaluates a sub-expression, then fills it into the context. We prove
this correct, showing that the interpreter and semantics agree on an
evaluation order. Specifically, the interpreter proof shows the interpreter
produces a valid evaluation order, and a separate proof shows that evaluation
contexts are unique.\footnote{See the lemma
\href{https://github.com/mit-pdos/perennial/blob/6f5ed5e7c2d3e8d657a0022c51e1d1e32a81e671/src/goose_lang/lang.v\#L1443-L1447}%
{\cc{head_redex_unique} in src/goose\_lang/lang.v}.} There is other non-determinism in the semantics that the
interpreter does not fully explore, though.

The test suite is structured as a number of test functions, each producing a
boolean that should be true. To check that the test itself is written correctly,
we test that it produces \cc{true} in Go first. Then to check the semantics of
the translation, in GooseLang we check that the interpreter succeeds and returns
true for each test function. While we could compare more sophisticated results
like integers or structs between the two, this strategy is especially easy to
implement, since there is no need to correlate Go and GooseLang outputs and
compare structured data.

The interpreter and test framework was designed and implemented by
Sydney Gibson, and is described in greater detail in her master's
thesis. The thesis includes more details on evaluating the interpreter
itself, for example documenting bugs caught by the test suite and other
bugs that are now part of our regression tests.
