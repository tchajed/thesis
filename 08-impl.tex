This chapter describes some details of the DaisyNFS implementation.
Implementation details on the Perennial framework are given in
\cref{sec:perennial:impl} and details on Goose are in \cref{sec:goose:impl}.

The implementation of DaisyNFS is split into several public code repositories:
\begin{itemize}[noitemsep,topsep=0pt,parsep=0pt,partopsep=0pt]
  \item \url{https://github.com/mit-pdos/daisy-nfsd} builds the overall
        file-system binary. The repository has the Dafny code and proofs. Most
        of the evaluation and benchmarking code is also here.
  \item \url{https://github.com/mit-pdos/go-journal} has the Go code for GoTxn
        (the name is a remnant from GoJournal, which was only a journaling
        system). This repository is just a Go library, since the proofs are
        elsewhere.
  \item \url{https://github.com/mit-pdos/go-nfsd} is an unverified NFS server
        which we used to evaluate GoJournal~\cite{chajed:gojournal}. Some of the
        evaluation code relies on scripts here, even if not comparing against
        this server.
  \item \url{https://github.com/mit-pdos/perennial} has the Perennial framework,
        GooseLang, and GoTxn. The repository also holds several other proofs
        built using Perennial not described in this thesis.
  \item \url{https://github.com/tchajed/goose} has the Goose translator and some
        small support libraries, in particular the library that exposes the disk
        to verified code.
  \item \url{https://github.com/tchajed/marshal} is the code for a verified
        serialization library used in GoTxn. The proofs are in the Perennial
        repository.
  \item \url{https://github.com/zeldovich/go-rpcgen} is a library for generating
        code from XDR and SUN RPC specifications.
\end{itemize}

DaisyNFS has some unverified code wrapping the verified implementation of the
NFS operations. This glue code uses \cc{go-rpcgen}, a library that parses XDR
struct definitions~\cite{RFC:4506}; XDR is a generic serialization protocol
similar to protobuf, Cap'n Proto, or FlatBuffers. The XDR code is unverified,
but we did use fuzzing to test both memory safety and that deserialization is
the inverse of serialization. We use XDR definitions for SUN RPC~\cite{RFC:1057}
and the NFSv3-specific definitions from RFC 1813~\cite{RFC:1813}.

The file-system code interacts with GoTxn via an axiomatized interface in Dafny
that is substituted with a real implementation that calls the GoTxn library. A
similar approach is used to represent Go byte slices.

Similar to VeriBetrKV~\cite{hance:veribetrkv}, we followed a
discipline of identifying and addressing timeouts in the proof.
% , which
% hamper development by giving the developer slower feedback (because
% proofs fail only after 30s) and less specific feedback (because Dafny
% is unable to pinpoint which assertion or postcondition failed and
% reports a timeout for an entire method).
As a result, the overall
build is fast: compiling the proofs takes only 12 minutes on a slow
machine in continuous integration and 4 minutes on a laptop
using eight CPU cores.

% Tej: takes 10m on my laptop with just one core (no VC parallelism)
% time make -j1 DAFNY_CORES=1 verify

GoTxn is implemented using fairly standard Go code. Goose imposes some
restrictions, but we nonetheless have fairly idiomatic code, split into modules
and multiple files as expected. One interesting library in GoTxn is its
\cc{lockmap} library, which logically implements per-address locking but does
not materialize all of these locks for better memory usage (this is similar to
Guava's ``striped'' locking library~\cite{guava-striped}). The specification for
the library mirrors the specification for the usual locks, with a lock invariant
associated with each possible address, hiding how the data structure is
implemented.

\section{Lines of code}

\begin{figure}
\centering
\small
\begin{tabular}{lr}
\toprule
\bf Component & \bf Lines of Coq \\
\midrule
  Helper libraries (maps, lifting, tactics) & \loc{5760} \\
Ghost state and resources & \loc{5125} \\
Program logic for crashes & \loc{9375} \\
\midrule
\textbf{Total} & \loc{20260} \\
\bottomrule
 \\
  \bottomrule
\end{tabular}
\caption{Lines of specs and proofs for Perennial.}
\label{fig:perennial:lines}
\end{figure}

\begin{figure}
\centering
\begin{tabular}{lrrr}
\toprule
  & \bf Lines of code & \bf Lines of proof & \bf Ratio \\
  & (Go) & (Coq) & \\
\midrule
  % auto-generated by go-txn/loc.py
\textsc{circular} & \loc{109} & \loc{1907} & $17\times$ \\
\textsc{wal-sts} & \loc{563} & \loc{13013} & $28\times$ \\
\textsc{wal} & --- & \loc{2854} & --- \\
\textsc{obj} & \loc{250} & \loc{4467} & $18\times$ \\
\textsc{jrnl-sts} & \loc{172} & \loc{1729} & $19\times$ \\
\textsc{jrnl} & --- & \loc{1616} & --- \\
\textsc{lockmap} & \loc{118} & \loc{1778} & $15\times$ \\
Misc. & \loc{211} & \loc{1707} & $8\times$ \\
\textsc{txn} & \loc{255} & \loc{2153} & $8\times$ \\
Transaction refinement & --- & \loc{6293} & --- \\
Simulation transfer & --- & \loc{1854} & --- \\
\midrule
GoTxn total & \loc{1674} & \loc{39371} & --- \\
\bottomrule
 \\
\bottomrule
\end{tabular}
\caption{Lines of code and proof for the components of GoTxn.
Ratio is the proof:code ratio, a rough measure of verification overhead.}
\label{fig:txn:loc}
\end{figure}

% ./daisy-nfs/loc.py --latex daisy-nfs/
\def\daisyMethodSpecs{273}
\def\daisyNfsSpec{357}
\def\daisyCode{4051}
\def\daisyTotal{10838}
\def\daisyTrustedSpec{558}
\def\daisyTrustedCode{1142}
\def\daisySpec{630}
\def\daisyProof{6787}


\begin{figure}
\centering
\begin{tabular}{lrrl}
  \toprule
  & \bf proof & \bf code & \bf spec \\
  \midrule
  GoTxn (Go + Coq) & \loc{39371}  & \loc{1674} & \loc{932} (Thm~\ref{thm:gotxn-transfer}) \\
  File system (Dafny) & \loc{\daisyProof}  & \loc{\daisyCode} & \loc{\daisySpec} (Thm~\ref{thm:dafny}) \\
  Trusted interfaces (Dafny) & --- & --- & \daisyTrustedSpec{} \\
  \cc{daisy-nfsd} & \emph{unverified} & \loc{\daisyTrustedCode} & --- \\
  \bottomrule
\end{tabular}
\caption{Lines of proof, code, and trusted specification for DaisyNFS.}
\label{fig:daisy:loc}
\end{figure}

The lines of proof, code, and specification for the layers of the system are
summarized in \cref{fig:daisy:loc}. The GoTxn correctness proof, \cref{thm:gotxn-transfer}, is
relatively large because code executed in atomically blocks can include many Go
operations modeled by Perennial, and the proof has cases to handle
each operation. However the result of the proof is a relatively concise
specification as a plain Coq statement that doesn't refer to the Perennial
logic.

% spec for refinement proof
% wc -l src/goose_lang/{typed_translate.v,ffi/jrnl_ffi_spec.v,refinement.v} src/program_proof/txn/{typed_translate.v,op_wrappers.v}
% 932 lines of Coq

The file-system operations are implemented in Dafny, which helped us verify a
relatively complete system without too much tedium. The
proof-to-code ratio (where code is the number of lines extracted by Dafny's
\cc{/printMode:NoGhost} flag) is about $2\times$ for the file system code.
The proof summarizes the implementation well, with about $6\times$ fewer lines
of specification as code (about half that specification is quite verbose and concerns
error codes and attributes). For efficiency, the Dafny code has trusted interfaces to
primitives like byte slices and integer-to-byte encoding. Together these are
written in \loc{\daisyTrustedSpec{}} lines of trusted Dafny code.
% The rest of the system has good leverage from the proofs: there are $8\times$
% as many lines of code as specification.
Finally, to complete the NFS server required around \loc{1000} lines of Go
code, about half of which bridge between the Dafny method signatures and the
actual NFS structs.

% fish script:
\begin{comment}
begin
    rm -rf src-compiled
    for file in src/*/**.dfy
        set -l path (string sub --start 4 $file)
        set -l dir (dirname $path)
        mkdir -p src-compiled/$dir
        dafny /printMode:NoGhost /dafnyVerify:0 /rprint:src-compiled/$path $file &
    end
    wait
    cloc --read-lang-def ~/dafny-lang.txt src-compiled
    cloc --read-lang-def ~/dafny-lang.txt src
end
\end{comment}

% cloc (compiled)
% 3227 source lines of code

% cloc (total)
% 9585 source lines total (minus code, 6358 proof lines)

\begin{comment}
begin
    gsed -n '/method[^(]* [A-Z]*(/,/^\s*{/p' src/fs/dir_fs.dfy
    cat src/fs/nfs.s.dfy
end | wc -l
\end{comment}
%
% 382

\tej{describe something similar for GoTxn proof and code; some spec number would
  be helpful, and separate transaction refinement and simulation transfer since
  they're large}

\section{Achieving good performance for DaisyNFS}
\label{sec:impl:dafny-perf}

An important aspect of the Dafny proof was to write code in a way that produces
high-performance Go code.
% problem because Dafny's built-in immutable collections (sequences and maps) are
% extremely inefficient in Go due to an impedance mismatch between Dafny and Go
% semantics.
Compared to Dafny's C\# backend, the generated Go code for Dafny's built-in
immutable collections has much
additional pointer indirection and defensive copying. Using these data
structures for byte sequences would simplify proofs, but has unacceptably poor
performance in Go.

To avoid this performance problem we use an axiomatized interface to
Go byte slices (\cc{[]byte} in Go) whenever raw data is required, including file
data and paths, and then modify these slices in-place. It was possible to
axiomatize this API without any changes to Dafny; we use a standard Dafny
feature of \cc{:extern} classes to specify a Dafny class \cc{Bytes} in terms of
ghost state of type \cc{seq<byte>} but then implement it as in Go as a thin
wrapper around the native \cc{[]byte} type. This API is trusted, so we
test it. To catch off-by-one errors in the specification, we wrote
tests like \verb![]byte{1,2,3}[2]! and ran them in Go and
(equivalent) Dafny.
%(this test should return 3 because both languages are
% zero-indexed).

% We implement the file system fairly efficiently, taking advantage of Dafny's
% support for imperative code.
The on-disk data structures---inodes, indirect blocks, and directories---are
represented in memory in their serialized form and modified by updating this
representation directly, a ``zero-copy'' implementation that avoids the cost of
an extra memory copy between representations. These were first written as
ordinary parsers that produced immutable data structures, which was then
migrated to this more efficient implementation. The proofs for this code are
ad-hoc rather than a systematic; DaisyNFS only has a handful of formats which
did not justify something more general. There is interesting work on
verified parsing with zero-copy support~\cite{swamy:everparse3d}, albeit not in Dafny.

Dafny's default integer type \cc{int} is unbounded and compiled to big-integer
operations. We used Dafny's
\cc{nativeType} support to instead define a type of 64-bit integers (that
is, natural numbers less than $2^{64}$) and compile this to Go's \cc{uint64}.
This requires overflow reasoning, but
automation makes this palatable in the proof and the performance gain is
significant.
