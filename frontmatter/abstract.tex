\tej{do another pass over the abstract}
The file system is an important service of the operating system, yet bugs are
still occasionally found in
implementations that can lead to incorrect results
or even data loss. What makes it difficult to write a correct file system is
that even simple operations can execute in many ways due to concurrency and
system crashes (after which the file system should still preserve data). Crashes
and concurrency make it challenging to correctly handle and test all
possibilities. Formal verification offers a promising solution by using a proof
to show the file-system code always meets its specification.

This thesis develops DaisyNFS, a verified, concurrent, file system that achieves
good performance. Several contributions make this possible. The first set of
techniques provide new foundations for verification of systems with concurrent
implementations and crash guarantees: a program logic, a specification pattern
to make proofs modular, and an approach for applying these techniques to
executable code. The file system is built on top of a verified transaction
system called GoTxn. The second set of techniques enables this design, which
allows an efficient design due to the concurrency in GoTxn while using the
guarantees afforded by transactions to verify the rest of the file system with
much simpler sequential reasoning. The overall file system is performant
compared to Linux NFS exporting ext4 over a range of benchmarks --- at least
90\% the throughput in a setting with a single client and in-memory disk, and
comparable scalability from multiple concurrent clients.
