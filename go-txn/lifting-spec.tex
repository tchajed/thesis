\subsection{Lifting specification for journaling (\scc{jrnl})}
\label{sec:txn:lifting}

Lifting is defined at an intermediate layer, \scc{jrnl} (for journaling), rather
than for \scc{txn} which is the top-level API of GoTxn. Journaling uses what we
call ``operations'' to distinguish them from ``transactions'' --- an operation
supports reads and writes, and like a transaction can be committed atomically to
disk, but the difference is that the caller is responsible for guaranteeing
operations manipulate disjoint addresses.

In the lifting specification, there is a global, logical disk representing the
state of the journaling system, and within each ongoing operation there is a
durable and operation-local view of the disk. The local views are both
initialized from the global disk, and the operation-local view is like the disk
view but also incorporates buffered writes. The durable views are disjoint, so
that an address is either in the global disk or ``lifted'' into a particular
operation. Disjointness is what guarantees that the views are really local and
don't change due to concurrent threads.

The lifting-based specification uses a logical concept of ownership to keep all
of these views disjoint and consistent. Each view is expressed using ghost state
in Iris, then the proof uses separation-logic resources to express ownership over that
ghost state; for a general introduction to the idea of
ownership in separation logic, see \cref{sec:perennial:concurrency}. The
journaling proof issues three types of resources, all per-address:
$a \mapstoDisk o$, $a \mapstoOp o$, and $a \mapstoLftd o$. The first is the most
straightforward: $a \mapstoDisk o$ asserts that the global durable disk has
value $o$ at address $a$ (we use the metavariable $o$ since these addresses have
objects that can be smaller than a block, such as a 128-byte inode value or even
a single bit). In addition, $a \mapstoDisk o$ expresses ownership over the
address $a$. The second resource, $a \mapstoOp o$, asserts ownership of the
operation-local view associated with an in-memory operation $\mathit{op}$; it
can mean that either the on-disk value is $o$ or that $\mathit{op}$ contains a
buffered write with the value $o$ to address $a$.

\subsubsection{Initialization}

The first step in using the journaling specification is to initialize the whole
system. Initialization is somewhat complicated by some restrictions GoTxn places
on how sub-block objects are used. The proof requires the caller to fix ahead of
time (when the whole journaling system is initialized) what size the objects
within each block is. Currently our proofs assume that objects are either a single bit, 128
bytes, or a full block, but supporting an arbitrary power of two should be a
straightforward extension. To initialize the system, the caller supplies a
``schema'' which gives the size of every disk address (that is, every 4KB
aligned address, not the journal's logical sub-block addresses); this enforces
that all the objects within a block are of the same size, which also simplifies
the proof.

The journal's initialization is specified by giving the caller a lemma that
transform ownership over the entire physical disk into the precondition for the
journal's recovery procedure; from this point forward the usual crash and
recovery reasoning in Perennial applies. For each offset $i$ and block size $k$, the initial ownership for that disk
block is just all the objects within that block:
\[
  \operatorname{zeroObjects}(i, k) \defeq \bigast_{0 \leq k \times o < 4096} (i, k \times o) \mapstoDisk 0
\]


The initialization lemma creates resources for all of these initial objects,
starting from an all-zero disk. The lemma takes a schema
$\mathit{kinds}$ that maps disk addresses to block sizes and a disk size $s$;
this schema is arbitrary but fixed at initialization time. We use
$i \mapsto_{\mathrm{disk}} 0$ to denote ownership over a physical disk offset $i$
(not to be confused by the $a \mapstoDisk o$ issued by the journaling proof).
\begin{align*}
  &513 \leq s \land {} \\
  &\operatorname{dom}(\mathit{kinds}) = \{ i \mid 513 \leq i < s \} \land {} \\
  &\bigast_{0 \leq i < s} i \mapsto_{\mathrm{disk}} 0 \wand {} \\
  &\vs \cc{is_txn_crash_condition}(\mathit{kinds}) \sep {} \\
  &\quad \bigast_{(i,k) \in \mathit{kinds}} \operatorname{zeroObjects}(i, k)
\end{align*}

Notice that this initialization is just a logical operation; the schema is not
even required at runtime for the code. It creates \cc{is_txn_crash_condition},
which is the precondition and crash condition for the transaction system's
recovery procedure. The journal objects in this lemma's conclusion are initially
fully owned by recovery, which then shares them with threads. The journaling
system's proof does not impose how access to these resources is mediated; it
simply requires exclusive access to an address in the form of a
$a \mapstoDisk o$ resource. In practice, GoTxn mediates access to
$a \mapstoDisk o$ by protecting it with a per-address lock. In our paper on the
journaling system itself we wrote a proof of a simple file system where the
locks were instead per inode, so that one lock protected more than one journal
resource; see the GoJournal paper for details on that use
case~\cite{chajed:gojournal}.

\subsubsection{Preparing operations}

The specification for \cc{Begin()} is straightforward, since the operation has
not yet interacted with the disk:
\hypertarget{tgt:begin-spec}{}
%
\begin{align*}
  \hoareV{\TRUE}%
  {\cc{Begin}()}%
  {\Ret{op} \cc{is_op}(\mathit{op})}
   %\tagH{BeginSpec}
\end{align*}
The specification has a trivial precondition of $\TRUE$ and its postcondition
returns an assertion \cc{is_op}, which says that $\mathit{op}$ is a valid
\cc{*Op} object that represents a journaling operation.  (This is different from
the \cc{*Txn} object that represents for a transaction, which has additional
state to track locking.) The operation starts out with an empty operation-local
view and ownership over no addresses; the ghost state for this operation's view
is initialized by the proof of this theorem and represented as part of
$\cc{is_op}(\mathit{op})$.

In order to interact with an address within an operation, the specification
requires the caller to start with $a \mapstoDisk o$ and then \emph{lift} it to
obtain an operation-local resource, where lifting is the following purely
logical operation:
\[
  a \mapstoDisk o \vs a \mapstoOp o \sep a \mapstoLftd o
\]

Notice that the result of lifting is both an operation-local assertion and
$a \mapstoLftd o$; this latter assertion is much like $a \mapstoDisk o$ in that
it asserts the on-disk value of address $a$, but it cannot be lifted again. This
is required for soundness; only one of $a \mapstoDisk o$ and $a \mapstoOp o$ is
allowed for ownership to actually be exclusive. Not shown is that lifting does
also require $\cc{is_op}(\mathit{op})$ which reflects that the ghost state has
been set up.

The specification for \cc{OverWrite} describes the effect of writing to the
local memory of a buffered journal operation:
%
\begin{align*}
  \hoareV{\cc{is_op}(\mathit{op}) \sep a \mapstoOp o \sep \cc{buf_obj}(buf, o')}%
        {\mathit{op}.\cc{OverWrite}(a, buf)}%
        {\cc{is_op}(\mathit{op}) \sep a \mapstoOp o'}
  %\tagH{OverWriteSpec}
\end{align*}

The precondition includes $\cc{buf_obj}(buf, o')$ to say that the in-memory
buffer $buf$ encodes the object to be written $o'$.
The \cc{is_op} predicate is both
required and returned by the specification, which reflects the fact that
\cc{OverWrite} operates on the in-memory and ghost state covered by
this predicate.

The specification for \cc{ReadBuf} is similar. \cc{ReadBuf}
returns a buffer that points into the \cc{op} struct, so it has a more
sophisticated spec:
%
\begin{align*}
  \hoareV{\cc{is_op}(\mathit{op}) \sep a \mapsto_{\mathit{op}} o}%
        {op.\cc{ReadBuf}(a)}%
        {\Ret{buf}
  \begin{aligned}
  &\cc{buf_obj}(\mathit{buf}, o) \sep \phantom{} \\
  &\quad( \cc{buf_obj}(\mathit{buf}, o) \wand \\
  &\quad\quad \cc{is_op}(\mathit{op}) \sep a \mapsto_{\mathit{op}} o )
    \end{aligned}}
    %\tagH{ReadBufSpec}
\end{align*}

This states that, when \cc{ReadBuf} finishes, it returns a buffer $\mathit{buf}$
and two resources: $\cc{buf_obj}(\mathit{buf}, o)$ says the buffer has the old
object $o$, while the second is a separating implication or \emph{wand} $\wand$.
The wand says that if
the caller returns ownership of $\cc{buf_obj}(\mathit{buf}, o)$, it can get back
the $\cc{is_op}(\mathit{op})$ predicate. This allows the caller to use the
buffer to read the data before continuing to read and write within the
operation. The complete specification in GoJournal also supports modifying the
object using the buffer returned by \cc{ReadBuf}~\cite{chajed:gojournal}, but
GoTxn does not expose that feature.

\subsubsection{Commit}

The final method in the journaling API is \cc{Commit()}, which atomically
persists all the writes buffered in the operation. The specification is based on
two maps $m$ and $m'$ whose domains are the set of all addresses lifted into the
operation. The first map $m$ gives the on-disk, pre-operation values while $m'$
has the new values buffered in the operation.
\begin{align*}
  \hoareCV{ \begin{aligned}
              &\dom(m) = \dom(m') \sep {} \\
  &\left( \bigast_{(a,o) \in m} a \mapstoLftd o \right) \sep %
  \left( \bigast_{(a,o') \in m'} a \mapstoOp o' \right)
            \end{aligned}} %
  {\mathit{op}.\cc{Commit}()}%
  {\Ret{\mathit{ok}} \mathrm{if~} \mathit{ok} %
  \mathrm{~then~} \bigast_{(a,o') \in m'} a \mapstoDisk o' \mathrm{~else~} %
    \bigast_{(a,o) \in m} a \mapstoDisk o}%
  {\left( \bigast_{(a,o) \in m} a \mapstoDisk o \right) \lor
    \left( \bigast_{(a,o') \in m'} a \mapstoDisk o' \right) }
\end{align*}

The precondition in this specification requires ownership over a subset of the
logical disk given by $\dom(m) = \dom(m')$, both its durable values and the
operation-local view. If the system does not crash and returns
$\mathit{ok} = \goosetrue$, \cc{Commit} returns ownership over the new values
on disk. It returns full disk points-to assertions $a \mapstoDisk o'$ because
these addresses are ``lowered'' so that subsequent operation can lift them
again. If the commit aborts (which only happens if the transaction does not fit
in memory), then the specification still returns full disk ownership albeit with
the old values. The crash condition of the \cc{Commit} specification captures
that even if the system crashes, it still returns ownership over the disk
resources, with either the old or new values. Whether this disjunction goes to
the left- or right-hand side depends on exactly when the system crashes; see
\cref{sec:perennial:recovery-spec} for a discussion of how the proof handles
this.

It's a little easier to see the overall structure of \cc{Commit}'s specification
if we specialize it to a single address (although the specification is powerful
because it captures atomicity across multiple addresses):
\begin{align*}
  \hoareCV{a \mapstoLftd o \sep a \mapstoOp o'} %
  {\mathit{op}.\cc{Commit}()}%
  {\Ret{\mathit{ok}} \mathrm{if~} \mathit{ok} %
  \mathrm{~then~} a \mapstoDisk o' \mathrm{~else~} a \mapstoDisk o}%
  {a \mapstoDisk o \lor a \mapstoDisk o'}
\end{align*}
