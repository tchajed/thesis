\section{Specifying GoTxn using refinement}%
\label{sec:txn:spec}

At a high level, GoTxn makes transactions atomic. This chapter formalizes this
intuitive definition in the form of \emph{transaction refinement}. Later in
\cref{ch:daisy-nfs} we build upon this specification to show how it enables
sequential reasoning. Transaction refinement is defined in terms of
\emph{refinement}, which relates code to a specification.

\subsection{Crash-safe, concurrent refinement}%
\label{sec:txn:refinement-def}

Abstractly, refinement from a code program to a specification says that the
behaviors of the code are a subset of the behaviors of the specification.
Refinement relates two programs in terms of their visible behavior, which is
used in the specification to connect a specification where transactions are
defined to be atomic to an implementation where they physically take many steps.
In this thesis we use a \emph{concurrent, crash-safe} notion of refinement that
allows the code to have concurrency (that is, it can spawn new threads) and that
captures that crashes in the code behave according to a specification of crash
behavior. For
the purposes of this work, the visible behavior is always network I/O,
corresponding to receiving requests for the system or responding to them (for
example, processing NFS requests).

\begin{definition}[Crash-safe, concurrent refinement]
  A concurrent implementation $p_{c}$ \emph{crash refines} (or simply
\emph{refines}) a specification program $p_{s}$,
written $p_{c} \refines p_{s}$, if whenever there are initial states
$\sigma_{s}$ and $\sigma_{c}$ satisfying $\mathrm{init}(\sigma_{s}, \sigma_{c})$
and $p_{c}$ can execute from $\sigma_{c}$ and produce a trace of network I/O
$\textit{tr}$, then $p_{s}$ can execute from $\sigma_{s}$ and produce the same trace
$\textit{tr}$.  Execution might involve crashing and restarting a program (potentially
multiple times), wiping out any in-memory state after each crash.
  \label{def:refinement}
\end{definition}

\tej{clarify this is concurrent, crash-safe refinement essentially everywhere}
The intuition behind the notation $p_{c} \refines p_{s}$ is that the set of
behaviors of $p_{c}$ (the set of traces of network I/O $\textit{tr}$) is a subset of the
behaviors of $p_{s}$. The statement $p_{c} \refines p_{s}$ leaves implicit
a definition of initial states $\mathrm{init}(\sigma_{s}, \sigma_{c})$, which
will generally say both states are all zeros and of the same size. This notion
of refinement extends a standard definition of concurrent refinement with
crash-safety by allowing crashing in the execution of the implementation, which
must correspond to a specification-level crash transition.

\subsection{Modeling programs for refinement}%
\label{sec:txn:go-layers}

To define the transaction refinement specification, we need to be more precise about what a program is
and how it executes. We
write $p : \gooselayer{X}$ to say $p$ is a Go program written using operations
from layer X.
Layer operations are always atomic transitions in a state machine. Layers will
be one of Sys, Txn, or Disk, where Sys is a
stand-in for an arbitrary system implemented on top of GoTxn. We write
``Sys'' generically to emphasize that the GoTxn specification does not fix how the
caller uses transactions; in practice it will be instantiated with the NFS state
machine for DaisyNFS, as described in \cref{sec:daisy:refinement-spec}.
The Txn layer allows the operations of the GoTxn API, with reads and writes over a logical
disk and the ability to wrap these into an atomic transaction. The
Disk transition system is formalized in Coq as part of the GoTxn proof,
and assumes reads and writes of 4KB blocks are atomic.

The concept of a layer is part of how GooseLang formalizes Go. The definition of
GooseLang is parameterized by some ``external'' operations and state, as
described in \cref{sec:goose:lang}. Regardless of layer, the type
$\gooselayer{X}$ includes a $\goosekw{Fork}$ expression to spawn a thread, heap operations
on pointers, slices, and maps, and computation on primitives like integers and
structs. A program $p : \gooselayer{X}$ represents a Go program which uses any
of these Go primitives and only imports additional operations from the layer
$X$. This factoring is used to represent a class of Go programs that uses
external state in a controlled way (limited to a layer), for the purpose of
defining the specification. Each GooseLang program is limited to one external
layer to simplify the definitions, but of course Go permits unlimited imports.

The transaction refinement specification for GoTxn below in
\cref{sec:txn:transaction-refinement} relates a \emph{spec program}
$p : \gooselayer{Txn}$ that uses transactions, to its executable version
which invokes the GoTxn implementation. These specification programs can invoke
transactions, which we write $\atomically{\cc{f}}$ to represent a transaction
running \cc{f}, which in turn might have calls to Read and Write from the GoTxn
API. The Read and Write operations in this model are primitive, external calls.
Each specification program $p$ has an associated implementation program
$\mathrm{link}(p, \txncode) : \gooselayer{Disk}$. The notation is intended
to suggest the linking process where each call to GoTxn is replaced with an
implementation that uses the Disk operations. Linking has the additional effect,
specific to the transaction system, of replacing a specification transaction
$\atomically{\cc{f}}$ with a sequence \cc{tx := Begin(); f(tx); tx.Commit()}
(with some additional code handling aborts).

\subsection{Transaction refinement}
\label{sec:txn:transaction-refinement}

The specification we give to GoTxn uses \emph{transaction refinement},
which formalizes serializability (sometimes known as atomicity and isolation)
for transactions running on top of GoTxn.
To set up this specification, consider a program $p : \gooselayer{Txn}$ that
uses transactions.
To run $p$, it is combined with the transaction-system implementation, producing
a program $\mathrm{link}(p, \txncode) : \gooselayer{Disk}$ that can be run on
top of a disk.
Transactions in the linked program continue to have the expected atomic
behavior, so long as transaction code in $p$ follows certain restrictions, such
as not accessing shared state outside the journal system.  We write
$\mathrm{safe}(p)$ to mean $p$ is ``safe'' in the sense that it follows these restrictions.

% At a high level of abstraction, the main difficulty is to give a specification
% for the transaction system, which we do in several steps:
%
% \begin{enumerate}
%   \item First, we define an arbitrary Go program running on top of
%         the transaction system. For reasons we will explain shortly we will use
%         $p : \gooselayer{Txn}$ for such a program. To run such a program it
%         first needs to be linked with the transaction system implementation,
%         producing a program denoted $\mathrm{link}(p, \txncode)$.
%   \item The second idea is to say what the semantics of a program
%         $p : \gooselayer{Txn}$ is. Transactions are atomic in this semantics in
%         that the whole transaction transitions at once, without interleaving
%         other threads. The program can issue reads and writes within a
%         transaction, and they follow a simple state machine.
%   \item The final idea is to define ``safe'' programs $\mathrm{safe}(p)$, those
%         that follow the restrictions of the transaction system. The
%         specification only applies to safe programs.
% \end{enumerate}

The correctness of the transaction system is expressed by the following theorem:
%
\begin{theorem}[Transaction refinement]
  The transaction system's implementation $\txncode$ is a \emph{transaction refinement}, meaning for
  all $p : \gooselayer{Txn}$, if $\mathrm{safe}(p)$, then
  $\mathrm{link}(p, \txncode) \refines p$. The definition of
  $init(\sigma_{s}, \sigma_{c})$ in this refinement relates an all-zero physical
  disk to an all-zero transactional disk of the same size.
  \label{thm:gotxn-transaction-refinement}
\end{theorem}
%
What the theorem says is that if a program is safe, the program linked with the
transaction system always behaves as if its transactions were atomically
accessing a transactional disk logically maintained by the transaction system.
Unlike the definition of concurrent, crash-safe refinement, which is between two complete programs, transaction
refinement is a \emph{contextual refinement} property about a library that is
stated in terms of all callers of that library.
The theorem is stated in Coq and has a fully mechanized proof in Perennial.

The definition of $\mathrm{safe}(p)$ formalized in Coq requires that any code
within a transaction not access any shared memory outside of the transaction
layer; other than that, transactions can use the GoTxn operations to interact
with the logical disk and the allocator, and do any computation in between these
operations. For example it is safe for transaction to issue data-dependent
operations, where the addresses in a transaction depend on earlier reads. The
restriction to not use other shared state is a natural one for the system's
correctness; for example, reads and writes to global variables would clearly be
non-atomic since the transaction system does not have any concurrency control or
protection over such variables.

Safety also requires that transactions follow the preconditions of the \cc{Read}
and \cc{Write} operations, which require a discipline of accessing each object
with a fixed size --- this is a limitation of the current GoTxn implementation
and proof, which does not handle concurrency between overlapping addresses.
Finally, safe programs can only \cc{Abort} or \cc{Commit} a
given transaction once. The notion of safe program will be important when
linking this proof with the Dafny proofs, since the transaction system's proof
only applies to a safe caller.

The allocator is part of the transaction API so that safe transactions have
access to this important in-memory data structure. However, the allocator
operations \cc{Alloc} and \cc{Free} do not hold a lock throughout the
transaction, and thus they would appear to have an affect on concurrent
transactions before \cc{Commit}. The reason why transactions are serializable
despite this behavior is that we under-specify the allocator's behavior to cover
possible concurrent interference. In particular the specification for \cc{Alloc}
merely promises to return an in-bounds address, and GoTxn's specification does
not even track the set of allocated/free addresses for each in-memory allocator.

In practice the way the caller uses this specification for a given allocator is
to store the ground-truth allocated/free state in the on-disk state of the
transaction system, then use an in-memory allocator as an efficient way to find
a free bit. Without some in-memory
allocator, searching for a free bit over the on-disk state would be difficult to do in an efficient
way. The return value of \cc{Alloc} must be checked against the durable bitmap,
which is easily done since GoTxn supports accessing an individual bit with
\cc{ReadBit}. There is a chance that \cc{Alloc} returns a used address, if it was
freed in memory by a concurrent transaction that hasn't committed to disk, but the allocator
is designed not to return recently freed addresses to avoid this issue.
Similarly in addition to calling \cc{Free} the caller also issues
$\cc{WriteBit}(a, \goosefalse)$ to mark the correct bit free on disk. During
recovery, for the in-memory allocator to be useful it must know what numbers
have already been allocated. The caller initializes the in-memory allocator to
the same state as what is stored on disk with\cc{MarkUsed(n)}.
