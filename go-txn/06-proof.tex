\section{Verifying GoTxn}
\label{s:proof}

GoTxn consists of multiple layers, as described in \cref{s:gotxn:impl}. This
section provides some highlights of the complexity involved in GoTxn's
implementation, along with the proof techniques required to formally
reason about that complexity.

\subsection{Write-ahead logging (\scc{wal})}

The write-ahead log layer is responsible for updating multiple disk
blocks (a multiwrite) atomically.
Each multiwrite is a list
of updates, where an update consists of a disk block number and the new data to write in that block.
%The write-ahead log makes transactions atomic by first writing the
% updates to an on-disk log, and then installing them to their respective
% disk blocks.
A background logger thread moves multiwrites from an in-memory buffer to an
on-disk log. To make this atomic, the logger first writes
the contents of a multiwrite in a log entry, and then updates a designated header block to indicate
the entry is complete. If a crash
happens before the header is updated, none of the multwrite's updates
are applied; if a crash happens after the header update, the multiwrite
will be applied during recovery.
Meanwhile, an installer thread applies entries in the log to the disk, clearing
space for new multiwrites.
If a crash happens before the updates in an entry are fully installed,
recovery installs the updates again from the on-disk log.

The write-ahead log implements two optimizations related to combining
multiwrites. Two or more multiwrites can be \emph{group committed} by logging
them together, which still guarantees their atomicity. If multiwrites being
committed together update the same block, the first update can be
\emph{absorbed} and replaced with the second. These optimizations trigger both
for multiwrites that are committed without waiting for durability and also for
concurrent, synchronous multiwrites.

% When a transaction is written in the log,
% updates to the on-disk log is made atomic by first writing the updates
% to the log, and then updating a designated header block.
%
% If a crash
% happens before the header is updated, none of the transaction's updates
% are applied; if a crash happens after the header update, the transaction
% will be applied during recovery.

% Write-ahead logging is a standard approach for atomic disk writes, but it
% is complicated by performance and crash safety requirements.  For example,
% writing transactions to disk requires waiting for a disk write, which
% can be slow.  To improve performance, many storage applications do not
% flush transactions to disk immediately (e.g., unstable \scc{write} RPCs
% in NFSv3), but do require that the on-disk state be consistent (i.e.,
% no partially applied or out-of-order transactions) after a crash.
% Furthermore, because disk writes are slow, many operations need to
% happen in parallel, without holding locks while waiting for disk reads
% and writes: committing new transactions in memory; logging transactions
% from memory to disk; waiting for transactions to be made durable; and
% installing transactions.  Concurrency ensures that in-memory operations
% need not wait for any in-flight disk reads or writes, and that many
% disk reads and writes can happen at the same time.  Finally, to reduce
% the number of disk writes, multiple in-memory transactions are logged
% to disk together (``group commit''), and if they update the same disk
% block multiple times, only the most recent update of that disk block is
% written to the log (``absorption'').

% Formally verifying a high-performance write-ahead log is challenging
% because of the lock-free concurrency, and because the log must maintain
% crash-safety at all times.


\paragraph{Internal abstract state: logical log.}
To prove the write-ahead log layer correct, \txn represents the
state of the write-ahead log as a logical list of multiwrites, as
shown in \cref{fig:log}.  Multiwrites before \cc{memStart} have
already been installed, and their log entries do not physically exist in memory or on disk.
Multiwrites from \cc{memStart} to \cc{diskEnd} are already logged on
disk.  Multiwrites from \cc{diskEnd} to \cc{nextDiskEnd} are currently being logged
from memory to disk.  Finally, multiwrites between \cc{nextDiskEnd}
and \cc{memEnd} are purely in-memory, and are eligible for absorption.

\begin{figure}[ht]
    \begin{tikzpicture}[>=latex]

  \tikzstyle{log}=[thick,rectangle, draw, minimum width=3cm,minimum
    height=1.5cm, align=center,fill=blue!10];

  \node[log] (installed) {Installed \\ writes};
  \node[log,right=0cm of installed] (logged) {Logged \\ writes};
  \node[log,right=0cm of logged] (logging) {Writes being \\ logged};
  \node[log,right=0cm of logging] (unstable) {Unstable \\ writes};

  \node[] at (-1.3cm, -1cm) {$\uparrow$ 0};
  \node[] (memstart) at (2.2cm, -1cm) {$\uparrow$ \cc{memStart}};
  \node[] (diskend) at (5.2cm, -1cm) {$\uparrow$ \cc{diskEnd}};
  \node[] (nextdiskend) at (8.55cm, -1cm) {$\uparrow$ \cc{nextDiskEnd}};
  \node[] (memend) at (11.1cm, -1cm) {$\uparrow$ \cc{memEnd}};

  \node[font=\footnotesize, below=0cm of memstart, align=center, xshift=-0.6cm] {Advanced by \\ installer};
  \node[font=\footnotesize, below=0cm of diskend, align=center, xshift=-0.5cm] {Advanced by \\ logger};
  \node[font=\footnotesize, below=0cm of nextdiskend, align=center, xshift=-0.85cm] {Advanced by \\ \cc{Flush}};
  \node[font=\footnotesize, below=0cm of memend, align=center, xshift=-0.5cm] {Advanced by \\ \cc{Commit}};

\end{tikzpicture}

    \vspace{-\baselineskip}
    \caption{The logical write-ahead log.  Vertical arrows indicate
        designated positions in the logical log.  Labels below the arrows
        indicate what thread or function is responsible for advancing
        that logical position to the right.}
    \label{fig:log}
\end{figure}

This representation allows \txn to precisely specify how concurrent
operations modify this abstract state, and how the state changes on crash.
For example, although the installer thread performs many disk writes to
install multiwrites, its only effect on the abstract state is that it
advances \cc{memStart}.  Similarly, the logger thread's only change to
the abstract state is to advance \cc{diskEnd}.  Calling \cc{Flush()}
advances \cc{nextDiskEnd}, freezing the data to be logged, then waits
for the logger to advance \cc{diskEnd} up to that point.  Committing a
new multiwrite simply appends it at \cc{memEnd}.  Finally, on crash,
an arbitrary suffix of the log from \cc{diskEnd} onwards is discarded.


\paragraph{External abstract state: durable lower bound.}
Although the details of the logical log are important for proving the
\scc{wal} layer, the caller (i.e., the \scc{obj} layer) does not need
to know about installation, group commit, etc. To abstract away these
details, the \scc{wal} provides a simplified state as its interface,
as shown in \cref{fig:wal-spec}.  The simplified state consists of
the same list of multiwrites, together with \cc{durable_lb}, which is
a lower bound on what set of multiwrites will be preserved on crash.
Using a lower bound instead of precisely exporting \cc{diskEnd} means
that this abstract view does not need to change if the logger thread
adds more multiwrites to disk in the background, and thus hides
this concurrency.

\begin{figure}[ht]
\begin{Verbatim}[commandchars=\\\{\},codes={\catcode`\$=3\catcode`\^=7\catcode`\_=8},fontsize=\small]
\PY{k+kn}{Record} \PY{n}{update} \PY{o}{:=} \PY{o}{\PYZob{}} \PY{n}{addr}\PY{o}{:} \PY{n}{u64}\PY{o}{;} \PY{n}{data}\PY{o}{:} \PY{n}{Block}\PY{o}{;} \PY{o}{\PYZcb{}}\PY{o}{.}
\PY{k+kn}{Record} \PY{n}{State} \PY{o}{:=}
  \PY{o}{\PYZob{}} \PY{n}{multiwrites}\PY{o}{:} \PY{n}{list} \PY{o}{(}\PY{n}{list} \PY{n}{update}\PY{o}{)}\PY{o}{;}
    \PY{c}{(*}\PY{c}{ at least durable\PYZus{}lb elements are durable }\PY{c}{*)}
    \PY{n}{durable\PYZus{}lb}\PY{o}{:} \PY{n}{nat}\PY{o}{;} \PY{o}{\PYZcb{}}\PY{o}{.}

\PY{k+kn}{Definition} \PY{n}{mem\PYZus{}append} \PY{o}{(}\PY{n}{ws}\PY{o}{:} \PY{n}{list} \PY{n}{update}\PY{o}{)} \PY{o}{:}
    \PY{n}{transition} \PY{n}{State} \PY{n}{unit} \PY{o}{:=}
  \PY{n}{modify} \PY{o}{(}\PY{k}{set} \PY{n}{multwrites} \PY{o}{(}\PY{k}{fun} \PY{n}{l} \PY{o}{=\PYZgt{}} \PY{n}{l} \PY{o}{+}\PY{o}{+} \PY{o}{[}\PY{n}{ws}\PY{o}{]}\PY{o}{)}\PY{o}{)}\PY{o}{;}
  \PY{n}{ret} \PY{n}{tt}\PY{o}{.}

\PY{c}{(*}\PY{c}{ non\PYZhy{}deterministically pick how many}
\PY{c}{   multiwrites survive the crash. }\PY{c}{*)}
\PY{k+kn}{Definition} \PY{n}{crash} \PY{o}{:} \PY{n}{transition} \PY{n}{State} \PY{n}{unit} \PY{o}{:=}
  \PY{n}{durable} \PY{o}{\PYZlt{}\PYZhy{}} \PY{n}{suchThat} \PY{o}{(}\PY{k}{fun} \PY{n}{s} \PY{n}{i} \PY{o}{=\PYZgt{}} \PY{n}{durable\PYZus{}lb} \PY{n}{s} ≤ \PY{n}{i}\PY{o}{)}\PY{o}{;}
  \PY{n}{modify} \PY{o}{(}\PY{k}{set} \PY{n}{multiwrites} \PY{o}{(}\PY{k}{fun} \PY{n}{l} \PY{o}{=\PYZgt{}} \PY{n}{l}\PY{o}{[}\PY{o}{:}\PY{n}{durable}\PY{o}{]}\PY{o}{)}\PY{o}{)}\PY{o}{;}
  \PY{n}{modify} \PY{o}{(}\PY{k}{set} \PY{n}{durable\PYZus{}lb} \PY{o}{(}\PY{k}{fun} \PY{o}{\PYZus{}} \PY{o}{=\PYZgt{}} \PY{n}{durable}\PY{o}{)}\PY{o}{)}\PY{o}{;}
  \PY{n}{ret} \PY{n}{tt}\PY{o}{.}
\end{Verbatim}

\vspace{-\baselineskip}
\caption{Parts of the specification for the \scc{wal} interface.}
\label{fig:wal-spec}
\end{figure}


\paragraph{Lock-free logging and installation.}
For performance, \txn has dedicated threads that perform logging and
installation.  However, these threads do not hold any locks while reading
or writing to disk.  To allow these threads to run concurrently, \txn
uses two separate header blocks, as shown in \cref{fig:physlog}.
One header block (owned by the installer thread) stores the start of
the on-disk log, and another header block (owned by the logger thread)
stores the end of the on-disk log.  This lets the installer and logger
concurrently advance their pointers (\cc{memStart} and \cc{diskEnd}
respectively) without locks.

\begin{figure}
    \centering
    \begin{tikzpicture}[>=latex]

  \tikzstyle{circlog}=[thick,rectangle, draw,minimum height=1.5cm, align=center,fill=magenta!10];
  \tikzstyle{log}=[thick,rectangle, draw,minimum height=1.5cm, align=center,fill=green!10];

  \node[circlog,minimum width=1cm] (logger) {Logger \\ end \\ pointer};
  \node[circlog,minimum width=1cm, right=0cm of logger] (installer) {Installer \\ start \\ pointer};

  \node[circlog,minimum width=2cm, right=0cm of installer] (log) {Logged \\ multiwrites};

  \node[log,minimum width=3cm, right=0cm of log] (log) {Installed \\ blocks};

  \node[] at (-0.45cm,-1cm) {$\uparrow$ 0};
  \node[] at (4.2cm,-1cm) {$\uparrow$ 513};

  \draw [decorate,decoration={brace,mirror,amplitude=10pt},xshift=-4pt,yshift=0pt]
    (-0.45cm,-1.2cm) -- (4cm,-1.2cm) node [black,midway,yshift=-0.6cm]
    {\scc{circular}};

\end{tikzpicture}

    \caption{The physical write-ahead log.}
    \label{fig:physlog}
\end{figure}

Although the logger and installer threads can perform lock-free disk
writes, they must still coordinate with one another.  For example,
the installer cannot run ahead of the logger thread, and the logger
thread must coordinate with threads that are appending new multiwrites
in memory. \txn's proof uses the notion of \emph{monotonic counters} to reason
about the safety of the logger and installer's lock-free operations.

The logger thread needs to check that \cc{memStart} is far enough along that the
log will have space for the new multiwrite. The proof gets a \emph{lower bound}
on the \cc{memStart} variable while holding a lock, which remains true even
after releasing the lock. Even though \cc{memStart} might grow after the initial
check, the log will only have more space and thus the multiwrite will still fit.

The installer has a similar lock-free region that also reasons using a lower
bound. The installer retrieves the updates from the current \cc{memStart} to
\cc{diskEnd} in order to start installing them to disk. When the installer
eventually trims the log, it needs to be sure not to advance beyond the current
logger position, which the proof demonstrates using a lower bound on
\cc{diskEnd} from when the logger initially started.

\subsection{Logically atomic crash specifications}
\label{s:proof:logatom}

Throughout the \txn stack we specify internal layers using a transition-system
specification, such as the examples illustrated in \cref{fig:wal-spec} for
the \scc{wal} layer. Perennial formalizes what it means for the code in a layer to
implement a transition using Hoare triples in a style we call
\emph{logically atomic crash specifications}. While the precise encoding
involves some technical details of Iris, we explain here the intuition behind these
specifications as well as why they are useful.

As a motivating example, consider the moment when the logger thread commits a
new batch of multiwrites to the physical log in order to advance the durable
point \cc{diskEnd} in the logical log of the \scc{wal} layer. It does this by calling into the
\cc{Append} method of the \scc{circular} layer, which appends to the small
buffer of logged multiwrites. The code for \cc{Append} commits at some internal
step when it writes the header block and makes the data valid, and it is at this
instant that the logical log's \cc{diskEnd} should be incremented.
How can we verify \cc{Append} in the \scc{circular} layer separately from the \scc{wal} layer,
while still executing the right update in the logger proof?

Logically atomic specifications achieve this separation by having the precondition to \cc{Append}
take a logical \emph{callback}~\cite{jacobs:logatom}, which the proof promises to ``execute'' at the commit point.
This callback is a Hoare triple of the form $\hoare{P}{\SKIP}{Q}$, where $P$ and $Q$ are later selected
by the logger proof to update the \cc{diskEnd} ghost state of the logical log, as shown in \cref{fig:circ-callback}.
% The logger proof maintains a
% ghost variable $\gamma \mapsto \cc{diskEnd}$, much like other points-to predicates,
% but variable $\gamma$ exists only in the proof. It passes an update to this
% variable
% $\hoare{\gamma \mapsto \cc{diskEnd}}%
% {\SKIP}%
% {\gamma \mapsto (\cc{diskEnd} + \cc{len}(\cc{txns}))}$
% as the callback, as illustrated in \cref{fig:circ-callback}.
%(we use the
%shorthand $\gamma\ \cc{+=}\ \cc{len(txns)}$ for
%space reasons).
This
specification for \cc{Append} provides modularity in that the \cc{Append} proof
does not need to know about the logical log and its \cc{diskEnd}, and the logger
proof does not need to worry about why \cc{Append} is atomic.  A key
advance of Perennial's logically atomic crash specs lies in additionally capturing the
crash behavior in this callback style, so as to enable a complete
proof of crash safety across layers.

\begin{figure}
  % \begin{tikzpicture}[remember picture, >=latex]
%   \draw node (append) [label=above:{Append proof}]{
%     \begin{minipage}{0.3\textwidth}
%    \begin{Verbatim}[commandchars=\\\{\},codes={\catcode`\$=3\catcode`\^=7\catcode`\_=8},fontsize=\small,numbersep=6pt,xleftmargin=0in]
% \PY{k+kd}{func} \PY{n+nx}{Append}\PY{p}{(}\PY{n+nx}{txns}\PY{p}{)} \PY{p}{\PYZob{}}
%   \PY{c+c1}{// write data}
%   \PY{n+nx}{hdr} \PY{o}{:=} \PY{o}{...} \PY{c+c1}{// prep header}
%   \PY{n+nx}{disk}\PY{p}{.}\PY{n+nx}{Write}\PY{p}{(}\PY{n+nx}{LOGHDR}\PY{p}{,} \PY{n+nx}{hdr}\PY{p}{)\tikzmark{write}}
%   \PY{o}{...}
% \PY{p}{\PYZcb{}}
%    \end{Verbatim}
%    \end{minipage}};
% 
%   \draw node (logger) [right=4cm of append.north, label=above:{Logger proof}, anchor=north] {
%     \begin{minipage}{0.1\textwidth}
%    \begin{Verbatim}[commandchars=\\\{\},codes={\catcode`\$=3\catcode`\^=7\catcode`\_=8},fontsize=\small,xleftmargin=0in]
% 
% 
% \PY{o}{...}
% \PY{n+nx}{\tikzmark{append}Append(txns)}
% \PY{o}{...}
%         \end{Verbatim}
%     \end{minipage}};
% 
%   \tikzstyle{edge}=[->,thick];
%   \draw [edge, color=black, bend right] (pic cs:append) to (pic cs:write);
%   \draw [edge, color=black, bend right] (pic cs:write) to (pic cs:append);
% \end{tikzpicture}

\begin{tikzpicture}[remember picture, >=latex]
  \draw node (append) [label=above:{Append proof in \scc{circular}}]{
    \begin{minipage}{0.3\textwidth}
   \begin{Verbatim}[commandchars=\\\{\},codes={\catcode`\$=3\catcode`\^=7\catcode`\_=8},fontsize=\small,numbersep=6pt,xleftmargin=0in]
\PY{k+kd}{func} \PY{n+nx}{Append}\PY{p}{(}\PY{n+nx}{txns, }\PY{pf}{$\{P\}\SKIP\{Q\}$}\PY{p}{)} \PY{p}{\PYZob{}}
 \PY{c+c1}{... // write data}
 \PY{n+nx}{hdr} \PY{o}{:=} \PY{o}{...}\PY{c+c1}{}
 \PY{n+nx}{disk}\PY{p}{.}\PY{n+nx}{Write}\PY{p}{(}\PY{n+nx}{LOGHDR}\PY{p}{,} \PY{n+nx}{hdr}\PY{p}{)}\tikzmark{write2}
 \PY{o}{...}
\PY{p}{\PYZcb{}}
   \end{Verbatim}
   \end{minipage}};

  \coordinate (pt) at (pic cs:write2) {};
  \draw node (logger) [right=4.25cm of append.north, label=above:{Logger proof in \scc{wal}}, anchor=north] {};
  \draw node (cbcall) [right=-.1cm of pt, yshift=.5ex,text width=1.5cm,align=center] {use \\ $\textcolor[rgb]{0.35,0.35,0.35}{\{P\}\SKIP\{Q\}}$};
  \draw node (cbcode) [right=1.75cm of pt, yshift=.5ex] {%
\textcolor[rgb]{0.35,0.35,0.35}{$\cc{diskEnd} \cc{+=}\ \cc{len(txns)}$}
% \draw node (cbcode) [right=1.95cm of pt, yshift=-0ex, align=left] {%
%   \textcolor[rgb]{0.35,0.35,0.35}{$\begin{aligned}P :=&  \gamma \mapsto \cc{diskEnd} \\
%   Q :=& \gamma \mapsto \cc{diskEnd} \\ &\ \ \ \phantom{} + \cc{len}(\cc{txns})
%      \end{aligned}$}
%   \textcolor[rgb]{0.35,0.35,0.35}{$P :=  \gamma \mapsto \cc{diskEnd}$} \\
%   \textcolor[rgb]{0.35,0.35,0.35}{$\begin{aligned}& Q := \\ & \qquad {\gamma \mapsto
%         \cc{diskEnd} + \cc{len}(\cc{txns})}
%      \end{aligned}$}
%$\hoareV{\gamma \mapsto \cc{diskEnd}}%
%{\SKIP}%
%{\gamma \mapsto \left(\begin{aligned}
%    &\cc{diskEnd} + \phantom{} \\
%    &\cc{len}(\cc{txns})
%\end{aligned}\right)}$
};

  \tikzstyle{edge}=[->,thick];
  \draw [edge, color=black, bend left, transform canvas={yshift=1.2em}] (pt) to (cbcode.west);
  \draw [edge, color=black, bend left, transform canvas={yshift=-1.1em}] (cbcode.west) to (pt);
\end{tikzpicture}

%%% Local Variables:
%%% mode: latex
%%% TeX-master: "paper.tex"
%%% End:

  \caption{Illustration of how the proof of \cc{Append} executes a logical
callback $\{P\}\SKIP\{Q\}$. The logger passes a callback that adds \cc{len(txns)} to \cc{diskEnd}.}
% $\gamma\ \cc{+=}\ \cc{len(txns)}$.
% Note that the proof of
% \cc{Append} does not need to know the details of what \textcolor[rgb]{0.35,
% 0.35, 0.35}{cb} does.}
  \label{fig:circ-callback}
\end{figure}

% \begin{figure}
%   \includegraphics[width=\columnwidth]{drawn-diagrams/circ-callback.png}
%   \caption{Illustration of how the proof of \cc{Append} executes a callback of
%     ghost updates from the logger thread. Note that the proof of \cc{Append}
%     does not need to know the details of what \cc{cb} does. \joe{make this figure}}
%   \label{fig:circ-callback}
% \end{figure}

%  The proof separates verifying the
% circular buffer that manipulates the on-disk header blocks and transactions (the
% first 513 blocks in \cref{fig:physlog}) from reasoning about the logger
% thread.
%
%  Transition systems are useful as a specification because
% they can precisely capture the behavior of even complex internal APIs like that
% of the \scc{wal}, including crash behavior resulting from deferring logging and
% the split \cc{ReadMem} and \cc{ReadInstalled} operations.
%
% Perennial formalizes what it means for the code of a layer to implement a transition
% system by encoding the \emph{refinement} between the code and the transition
% system within separation logic using Iris ghost state, a style we call
% \emph{logically atomic crash specifications}.  Logically atomic crash triples
% are important in Perennial for modularity. Transition systems are an expressive
% specification, and we can use them to precisely express the behavior of even
% complicated internal APIs with lots of concurrency.  The way refinement is
% encoded in a logically atomic crash specification is to expose the abstract
% state of a layer as a ghost variable which the calling code maintains.  Giving
% the caller control using a ghost variable is a powerful technique in Iris, and
% lets us express invariants that are maintained by upper layers \emph{without
% lower layers being aware of these invariants}.
%
% For a concrete example, we describe how logically atomic crash specifications
% help factor out a library for managing the on-disk circular buffer (the first
% 513 blocks on disk in \cref{fig:physlog}) in the \scc{wal}. This internal
% layer handles just two operations: \cc{Append} writes updates to the buffer
% atomically to log a new batch of transactions and \cc{Advance} trims from the
% front of the buffer to make room for new updates after installing old ones.
% These operations can be called concurrently by the logger and installer threads.
% They behave like a queue of updates, which is updated atomically. Importantly
% the circular buffer's state is durable, which is reflected by specifying that on
% crash its abstract state is unaffected and proving that the queue can be
% reconstructed based on the header blocks.
%
% The write-ahead log uses the circular buffer's atomic specification by relating
% the buffer's queue to the logical write-ahead log seen in \cref{fig:log}.
% Slightly simplified, the invariant says that the data in the queue is exactly
% the updates between \cc{memStart} and \cc{diskEnd}. More precisely during the
% lock-free code in the installer the queue might have advanced beyond
% \cc{memStart}: this is safe because in that case the invariant also says the
% missing updates have already been installed. During the lock-free code in the
% installer the queue might have the updates all the way to \cc{nextDiskEnd}: this
% is safe because it persists more data than promised by \cc{diskEnd}, and as
% explained the external abstraction of the \scc{wal} only promises a lower-bound
% on durability, not precisely what is on disk.
%
% How does the proof of the \scc{wal} reason about calls to the circular buffer?
% We illustrate this refinement reasoning for the specific case of the call to
% \cc{Append} that logs a new batch of transactions.
%
% \includegraphics{drawn-diagrams/circ-hocap.png}
%
% In this diagram, the logger thread has called \cc{Append} with the teal updates.
% The circular buffer at some point atomically commits the effect of the
% \cc{Append} (at the point where it updates the logger header block), at which
% point it transitions from the blue updates to the blue + teal updates. What the
% specification allows the logger proof to do is simultaneously update ghost state
% in the write-ahead log representing the logical write-ahead log. At this point
% it is safe to advance the logical \cc{diskEnd} variable, since the teal updates
% from the log are now durable. Eventually the logger thread will re-acquire the
% lock and update the in-memory \cc{diskEnd} value and calls to \cc{Flush} will
% know that these updates are durable.
%
% The key idea is that the write-ahead log has some invariant about what updates
% are actually stored in the circular buffer that relates them to the updates in
% the log and ensures that durable updates are in the circular buffer or are
% installed. However, these details are unimporting to proving that the circular
% buffer correctly and atomically writes updates to disk. Logically atomic crash
% specifications separate the two proofs, simplifying the reasoning in both the
% circular buffer and write-ahead log.

% One way to think about this specification is that when calling a method in the
% layer, the caller supplies a \emph{callback} function that takes two states that
% are a valid transition of the transition system as input. The specification for
% the method promises atomicity by guaranteeing to call the callback at a single
% instant --- the commit point of the code --- and move the transition system
% forward. Other than these callback functions, the code in the layer is not
% allowed to affect the abstract state of the system since other effects might
% break an invariant over the abstract state being maintained by the caller.
%
% \begin{figure}
%   \[
%     \hoare%
%     {\begin{aligned}
%        &\cc{layer_state}(\gamma) \sep\\
%       &\quad(\forall \sigma, \sigma', r.\, \cc{allowed}(\cc{op}, \sigma, \sigma', r) \to\\
%       &\quad\quad\gamma \mapsto \sigma \vs \gamma \mapsto \sigma' \sep Q(r))
%      \end{aligned}}%
%     {\cc{op}()}%
%     {\Ret{r} Q(r)}
%   \]
%   \caption{The generic form of a logically atomic crash specification for an
%     operation \cc{op}. The specification allows the caller to pick a
%     postcondition $Q$; the operation promises to return a value that satisfies
%     $Q$. The way the operation does so is to update a ghost variable $\gamma$, which
%     $\cc{layer_state}(\gamma)$ asserts holds the abstract state of the layer
%     (and that the layer has previously been initialized). It can only do so by
%     calling the update in the precondition, which is an update from $\sigma$ to
%     $\sigma'$ and $r$ that follows the transition system, represented by
%     $\cc{allowed}(\cc{op}, \sigma, \sigma', r)$.}
%   \label{fig:crash-hocap}
% \end{figure}

\subsection{Concurrency within a block (\scc{obj})}
\label{s:proof:obj}

\txn's \scc{obj} layer allows the caller to issue reads and writes that
are smaller than a full block.  This finer granularity helps increase
concurrency: for example, the NFS file server packs multiple inodes into
a single disk block, and \scc{obj} allows threads to concurrently read
and write multiple inodes even if they share a disk block.

At commit time, \scc{obj}'s \cc{Commit} may need to perform an
``installation read'' and read a full block, update the range that was
modified by the caller as part of a journal operation, and write back the
full block using the \scc{wal} layer.  To ensure correctness of this
read-modify-write operation, \cc{Commit} uses a lock to serialize
all commit operations.  However, \cc{Read} operations are lock-free:
they can execute concurrently with one another and concurrently with
\cc{Commit}.

Lock-free reads pose a verification challenge because the disk
block can be modified during the read.  Consider the example shown
in \cref{fig:txn-concur}, where a single disk block stores many
inodes. Inode 1 initially contains the value A, while inode 4 contains B. Thread 1 is committing a write of B' to inode 4 in that block, while
thread 2 concurrently reads inode 1 from the same block.  To read
inode 1, thread 2 will read the entire block, and then copy out the part
of the block corresponding to inode 1.  The block seen by
thread 2 will differ depending on whether thread 1's write happens
before or after the read, but inode 1 will contain A in either case.

%\begin{figure}[ht]
%\centering
%\includegraphics{drawn-diagrams/sub-block-concurrency.png}
%\caption{An example of a concurrent \cc{Read} and \cc{Commit}
%  in the \scc{obj} layer.}
%\label{fig:txn-concur}
%\end{figure}

\begin{figure}[ht]
\centering
\scalebox{.9}{%
\begin{tikzpicture}[>=latex, ampersand replacement=\&]

  \tikzstyle{blk1}=[fill=blue!10]; 
  \tikzstyle{blk6write}=[fill=blue!40]; 

  \matrix (disk) [matrix of nodes, nodes={rectangle,draw,minimum width=5ex, minimum height=4ex, anchor=center},
    nodes in empty cells,
    execute at empty cell=\node{\vphantom{?}};]
  { \& |[blk1]|\text{A} \& \& \& B \& \& \& \dots \\ };
  \node (disklabel) [left=.1cm of disk] {Disk Block:};

  \foreach \i [count=\xi from 0] in  {1,...,8}{
      \node also [label=below:\xi] (disk-1-\i) {}; 
  }

  \matrix (thread1) [matrix of nodes, above=1.45cm of disk.west, anchor=west, nodes={rectangle,draw,minimum width=5ex, minimum height=4ex, anchor=center},
    execute at empty cell=\node{\vphantom{?}};]
  { \& |[blk1]|\text{A} \& \& \& |[blk6write]|\text{B'} \& \& \& \dots \\ };
  \node (thread1label) [left=.1cm of thread1] {Thread 1:};

  \matrix (thread2) [matrix of nodes, below=1.65cm of disk.west, anchor=west, nodes={rectangle,draw,minimum width=5ex, minimum height=4ex, anchor=center},
    execute at empty cell=\node{\vphantom{?}};]
  { \& |[blk1]|\text{A} \& \& \& B/B' \& \& \& \dots \\ };
  \node (thread2label) [left=.1cm of thread2] {Thread 2:};

  \tikzstyle{edge}=[->,thick, shorten >=.1cm, shorten <=.1cm];
  \draw [edge, color=black, transform canvas={xshift=-.5em}] (disk-1-2) -- (thread1-1-2);
  \draw [edge, color=red, transform canvas={xshift=.5em}] (thread1-1-2) -- (disk-1-2);

  \draw [edge, color=black, transform canvas={xshift=0em}] (disk-1-2.south)+(0,-.4cm) -- (thread2-1-2);

  \draw [edge, color=red] (thread1-1-5)+(0,.9cm) -- (thread1-1-5);


%  \tikzstyle{disk}=[thick,rectangle, draw, minimum width=1.8cm,minimum
%    height=1.5cm, align=center];
%
%  \node[disk] (installed) {Installed \\ writes};
%  \node[disk,right=0cm of installed] (logged) {Logged \\ writes};
%  \node[disk,right=0cm of logged] (logging) {Writes \\ being \\ logged};
%  \node[disk,right=0cm of logging, minimum width=2.2cm] (unstable) {Unstable \\ writes};
%
%  \node[] at (-0.8cm,-1cm) {$\uparrow$ 0};
%  \node[] (memstart) at (1.6cm, -1cm) {$\uparrow$ \cc{memStart}};
%  \node[] (diskend) at (3.4cm, -1cm) {$\uparrow$ \cc{diskEnd}};
%  \node[] (nextdiskend) at (5.5cm, -1cm) {$\uparrow$ \cc{nextDiskEnd}};
%  \node[] (memend) at (7.3cm, -1cm) {$\uparrow$ \cc{memEnd}};
%
%  \node[below=0cm of memstart, align=left] {$|\Rightarrow$ \\ Installer};
%  \node[below=0cm of diskend, align=left] {$|\Rightarrow$ \\ Logger};
%  \node[below=0cm of nextdiskend, align=left] {$|\Rightarrow$ \\ \cc{Flush}};
%  \node[below=0cm of memend, align=left] {$|\Rightarrow$ \\ \cc{Commit}};

\end{tikzpicture}
}

\vspace{-\baselineskip}
\caption{An example of a concurrent \cc{Read} of inode 1 and \cc{Commit}
  modifying inode 4
  in the \scc{obj} layer.}
\label{fig:txn-concur}
\end{figure}

Formally reasoning about the \cc{Read} operation requires the \scc{obj}
layer to connect the $a \mapsto_{\mathit{op}} o$ predicate about a disk object
(such as an inode) to the disk block containing that object at the
\scc{wal} layer.  However, due to the race condition described above,
the \cc{Read} implementation might observe many possible values of the
containing disk block.  As a result, it is important for the \scc{obj}
invariant to relate the $a \mapsto_{\mathit{op}} o$ predicate not just to
the latest value of the containing block, but to all recent contents
of that block.  Specifically, the invariant for $a \mapsto_{\mathit{op}} o$
requires that all recent writes to $a$'s block (since \cc{Read(a)}
started) must agree on the part of the block storing $o$.  As a result,
regardless of what block happened to be read,
the caller is guaranteed to see the correct object $o$.


% \subsection{Lock-free reads in \scc{wal}}
%
% \cref{fig:walread} shows the implementation of \cc{Read} in the
% \scc{wal} layer.  This implementation does not hold any lock for the
% duration of the read; instead, it uses a lock to check the in-memory state
% (consisting of updates between \cc{memStart} and \cc{memEnd}), and, if
% no in-memory updates match the address being read, \cc{Read} falls back
% to reading from the installed area on disk.  As a result, other threads
% can run between the call to \cc{ReadMem()} and \cc{ReadInstalled()}.
% In particular, a thread running \scc{obj}'s \cc{Commit()} could write
% to the same block that another thread is reading, if the two threads
% are accessing different files, as shown in \cref{fig:txn-concur}.
%
% \begin{figure}[ht]
% \begin{Verbatim}[commandchars=\\\{\},numbers=left,firstnumber=1,stepnumber=1,codes={\catcode`\$=3\catcode`\^=7\catcode`\_=8},fontsize=\small,numbersep=6pt,xleftmargin=0.2in]
\PY{k+kd}{func} \PY{p}{(}\PY{n+nx}{l} \PY{o}{*}\PY{n+nx}{Wal}\PY{p}{)} \PY{n+nx}{Read}\PY{p}{(}\PY{n+nx}{a} \PY{k+kt}{uint64}\PY{p}{)} \PY{n+nx}{Block} \PY{p}{\PYZob{}}
  \PY{n+nx}{b}\PY{p}{,} \PY{n+nx}{ok} \PY{o}{:=} \PY{n+nx}{l}\PY{p}{.}\PY{n+nx}{ReadMem}\PY{p}{(}\PY{n+nx}{a}\PY{p}{)}
  \PY{k}{if} \PY{n+nx}{ok} \PY{p}{\PYZob{}}
    \PY{k}{return} \PY{n+nx}{b}
  \PY{p}{\PYZcb{}}
  \PY{k}{return} \PY{n+nx}{l}\PY{p}{.}\PY{n+nx}{ReadInstalled}\PY{p}{(}\PY{n+nx}{a}\PY{p}{)}
\PY{p}{\PYZcb{}}

\PY{k+kd}{func} \PY{p}{(}\PY{n+nx}{l} \PY{o}{*}\PY{n+nx}{Wal}\PY{p}{)} \PY{n+nx}{ReadMem}\PY{p}{(}\PY{n+nx}{a}\PY{p}{)} \PY{p}{(}\PY{n+nx}{Block}\PY{p}{,} \PY{k+kt}{bool}\PY{p}{)} \PY{p}{\PYZob{}}
  \PY{n+nx}{l}\PY{p}{.}\PY{n+nx}{Lock}\PY{p}{(}\PY{p}{)}
  \PY{c+c1}{// get b from in\PYZhy{}memory log, if present}
  \PY{n+nx}{l}\PY{p}{.}\PY{n+nx}{Unlock}\PY{p}{(}\PY{p}{)}
  \PY{k}{return} \PY{n+nx}{b}\PY{p}{,} \PY{n+nx}{ok}
\PY{p}{\PYZcb{}}

\PY{k+kd}{func} \PY{p}{(}\PY{n+nx}{l} \PY{o}{*}\PY{n+nx}{Wal}\PY{p}{)} \PY{n+nx}{ReadInstalled}\PY{p}{(}\PY{n+nx}{a} \PY{k+kt}{uint64}\PY{p}{)} \PY{n+nx}{Block} \PY{p}{\PYZob{}}
  \PY{k}{return} \PY{n+nx}{disk}\PY{p}{.}\PY{n+nx}{Read}\PY{p}{(}\PY{n+nx}{a}\PY{p}{)}
\PY{p}{\PYZcb{}}
\end{Verbatim}

% \caption{The implementation of \cc{Read} in the \scc{wal} layer.}
% \label{fig:walread}
% \end{figure}
%
% This poses a challenge in specifying the behavior of \cc{Read}, because
% there are two possible commit points: either during \cc{ReadMem()} or
% during \cc{ReadInstalled()} \tej{this is wrong --- the other commit point is
% part of another thread's commit. Suppose $a$ is written and then $b$, and $a$
% gets installed before the ReadInstalled runs. Then the value read is consistent
% with any point after $a$ was written but before $b$. We simplify the situation
% drastically by allowing ReadInstalled to return a non-deterministic value, so
% its linearization point is within its code.}.  Fortunately, the
% \scc{obj} layer maintains
% a strong invariant about all of the possible blocks that \cc{Read} could
% return (either from \cc{ReadMem()} or \cc{ReadInstalled()}.  As a result,
% \scc{wal}'s \cc{Read()} does not provide a single atomic specification;
% instead, it exposes both \cc{ReadMem()} and \cc{ReadInstalled()}, and
% \scc{obj}'s \cc{Read()} uses its strong invariant to reason about the
% result returned from either of the two read functions.
%
% Formally, we believe that proving the \scc{wal} \cc{Read()} operation to
% be linearizable requires using \emph{prophecy variables}.  Perennial
% does not support them, which forced us to adopt the non-atomic
% specification described above.  Recent results on prophecy variables
% in Iris~\cite{jung:prophecy} could be used to avoid such a non-atomic
% specification for the \scc{wal} \cc{Read()}.

% splice this in from DaisyNFS paper
% subsection of GoTxn proof section
\subsection{Refinement to atomic language}
\label{sec:txn:refinement}

\newcommand{\txnmapsto}{\mapsto_{\cc{txn}}}
\newcommand{\thdmapsto}{\Rightarrow}

Recall that \cref{thm:gotxn-program-refinement} is the overall correctness
theorem for GoTxn. It says that the GoTxn implementation is a \emph{program
refinement} between the Txn layer (with programs that use transactions) and the
Disk layer (with an implementation that interacts with a disk).
That is, given a program
$p : \gooselayer{Txn}$, if the transaction operations of $p$ are linked with
GoTxn (implemented in $\gooselayer{Disk}$), the implementation program's
observable behaviors (from running on a disk) are a subset of the
specification's behaviors (with high-level transaction-system operations, and in
particular atomicity for transactions). The proof of this theorem leverages the
journaling specification from \cref{sec:txn:lifting}, extending it to also
reason about the concurrency control implemented with two-phase locking.

In general, the Perennial framework supports proving refinements between a
specification layer $\gooselayer{S}$ and an implementation layer $\gooselayer{I}$ by constructing
a simulation between the specification and its implementation. The simulation is
expressed in terms of refinement conditions that are written using the usual
Perennial Hoare triples specifications (with pre- and postconditions),
so that they can be proven using the full spectrum of techniques in Perennial.

Following an
approach developed by \citet{turon:caresl}, in a refinement proof in Perennial the execution of the specification
program is represented by \emph{ghost state}. The logic then has assertions for
describing this ghost state.  For example, for the Txn layer, the assertion $\cc{a}
\txnmapsto b$ says that address $a$ contains the value $b$ in the ghost
transaction system's state. In addition, there are \emph{thread points-to}
assertions, written $j \thdmapsto \cc{e}$, which says that thread $j$ in the
specification program is executing program $\cc{e}$. Perennial has rules
for updating the ghost state by ``executing'' these ghost threads, such as the
following view shift:
\[
  (\cc{a} \txnmapsto b) \sep (j \thdmapsto \cc{Write(a, b')}) \vs
  \cc{a} \txnmapsto b'
\]
which updates the value stored at $\cc{a}$ as a result of a write.

To establish the refinement, the proof engineer first defines a
\emph{representation invariant} $I$, an assertion in the logic that describes a
relation between the specification state and the implementation state.
Perennial's proof rules ensure that this designated invariant must hold before
and after each step of a program throughout the proof. Next, the proof engineer
proves a Hoare triple for each operation $o$ of $\gooselayer{S}$ and its
corresponding implementation $p_o$ in $\gooselayer{I}$:
\[
\hoare{(j \thdmapsto o) * \knowInv{}{I}}{p_o}{\Ret{v} (j \thdmapsto v)}
 \]
Such a \emph{refinement triple} says that if a specification thread is executing
$o$ and an implementation thread is running $p_o$, then the representation
invariant $I$ is maintained and the value $v$ returned by running $p_o$ is a valid
return value of operation $o$. In the proof of this triple, the ghost execution
rule above is used at the linearization point of $p_o$, to mark
when the operation logically takes effect by executing $o$ in the ghost code.

These refinement triples imply a refinement between programs in $\gooselayer{S}$
and $\gooselayer{I}$. This is formally stated and proven in Coq, using the Perennial
framework's soundness theorem.

In the particular case of the transaction system, the key refinement triple to
prove is for a block of code $f$ enclosed in transaction \cc{Begin} and
\cc{Commit} operations; for example the triple might look the following (for a
particular transaction $f$ that copies from address 0 to 1):
\[
  \hoareV{ \left( j \thdmapsto %
      \atomically{v \gets \mathit{Read}(0);\, \mathit{Write}(1, v)} %
    \right) %
    \sep \knowInv{}{I}}%
  {\begin{aligned}
&\cc{tx := Begin();} \\
&\cc{v := tx.Read(0); tx.Write(1, v);} \\
&\cc{tx.Commit()}
\end{aligned}}%
    {j \thdmapsto ()}
\]
The difficulty in proving this triple is that the linearization point is at the
very end when the code calls \cc{Commit}, at which point the actual earlier
execution of $f$ becomes visible to other threads. The proof must show that
ghost-executing the specification's \cc{atomically} block at this point is valid by
tracking the behavior of $f$.

\tej{approximately here explain how the lock invariants are set up to protect
  the GoJournal disk resources and their crash invariants}

To show this, our proof maintains a stronger invariant during a transaction's
execution. As the transaction executes, we track the initial, on-disk value of any
objects accessed in a map $J$. The domain of this map
$\Sigma = \operatorname{dom}(J)$ is the \emph{footprint} of the transaction,
which two-phase locking keeps locked during the transaction. The intuition
behind the invariant is that if the transaction only depends on $J$, the
transaction's execution can be delayed to take place atomically at the call to
\cc{Commit} and its behavior will be the same since the subset of the journal
$J$ is the same.

The proof needs to reason about the two-phase locking concurrency control in
order to use the journaling layer's lifting specification as part of this proof.
Perennial has a crash-aware specification for locks, described in
\cref{sec:perennial:wpc}, that allows us to reason about the per-address locks
while also reasoning about crashes in the middle of a transaction. To use this
specification, the proof defines a per-address lock invariant and crash
invariant --- the lock invariant is a property that holds when the lock is
acquired (and must be shown when it is released), while the crash invariant is a
guarantee that also holds on crash. In the case of the two-phase locking code,
both the lock and crash invariants for the lock associated with address $a$
contain ownership of $\exists o, a \mapstoDisk o$, which gives the transaction
the ability to lift address $a$ and read and write it with the journaling
system's interface. The invariant has additional constraints to assert that the
value $o$ is the same as the one in the transactional disk.

More formally, the proof constructs a second simulation relation during the
execution of a transaction $\cc{f}$.  Let $J$ be a map giving the values of each
object in the transaction's footprint $\Sigma$ at the first time they are
accessed by $\cc{f}$, and let $J'$ be a mapping giving the transaction's current
buffered in-memory view of the same addresses.  Then, the invariant requires
that after $n$ steps of execution:
%
\begin{enumerate}

\item The transaction holds the lock for every address $a \in \Sigma$. The
transaction has ownership over $\bigast_{(a,o) \in J} a \mapstoLftd o$ due to
holding these locks and $\bigast_{(a,o') \in J'} a \mapstoOp o'$ from the
buffered writes.

\item Executing $n$ steps of $\cc{f}$ in \emph{any} starting state that has the same
  values as $J$ for the addresses in $\Sigma$ can lead to a state with values given
  by $J'$.

\end{enumerate}
%
At the start of a commit, the locking described by the first part of the
invariant ensures that the durable value of each address still match the values
in $J$. The second part of the invariant means that even though other
parts of the state outside of $\Sigma$ may have changed, those changes do not
affect execution of $\cc{f}$. Thus, the ghost execution of $\cc{f}$ at this point will
have the same behavior as the implementation. The transaction's invariant
maintains ownership of the resources to use the journal's \cc{Commit}
specification. The postcondition and crash condition of that specification
return new disk points-to assertions consistent with either $J$ or $J'$ as
appropriate. If the commit is successful, then the second part of the
transaction system's invariant allows issuing a ghost-execution update to change
the transactional disk from $J$ to $J'$ (for the subset $\Sigma$), and it
guarantees that this correctly simulates \cc{f}.

Showing that the second part of the invariant holds requires that code within a
transaction must not access global state outside of the
transaction system, as mentioned in \cref{sec:proof:linking}. Accesses to such global state
would violate the invariant because their behavior would then depend upon
things outside of the footprint $\Sigma$. Because those global values could change
by the time the transaction commits, the above argument would no longer work if they were allowed.

The allocator creates another subtlety related to the second part of this
invariant. Allocations do not hold the allocator lock throughout the remainder
of a transaction. This seems to violate the two-phase locking pattern, since
allocations could be implicitly observed by other concurrent transactions from
the fact that an allocated address is no longer free. Correspondingly, in the
proof, the footprint $J$ of a transaction does not describe the allocator state.
Thus, at the ghost-execution linearization point, the addresses returned by the
allocator may no longer be free. However, because the specification for the
allocator does not guarantee that returned addresses are actually free, the
second part of the invariant above still holds.

% Suppose the transaction being simulated is $\operatorname{Atomically}{f}$, and the implementation
% has executed $n$ steps of $f$, accessing addresses in the set $\sigma$.

% Let $J_1 : \cc{addr} \rightarrow Obj$ be a map giving the durable state of the journal at the start of the transaction, and let $J_2$ be the current buffered in-memory view of the state from the perspective of the transaction.
% Then, the invariant requires that:
% \begin{enumerate}
% \item the transaction holds the lock for every address $a \in \sigma$
% \item for each $a$, $J_1(a)$ matches the ghost state value for the address $a$
% \item if $J_1'$ is a mapping such that $\forall a \in \sigma$, $J_1(a) = J_1'(a)$, then executing $f$ for $n$ steps with an initial journal $J_1'$
% \end{enumearte}
% \joe{this would probably be vastly improved by using math notation that matches whatever will have been used in the previous two sections}

