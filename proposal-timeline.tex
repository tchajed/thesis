\section{Outline of thesis}

\begin{itemize}
  \item Introduction. This proposal is a draft of the introduction.
  \item Related work. We will give a more detailed overview of related work.
        There are two main strands: testing file systems to find bugs, and
        reasoning about storage systems (with or without concurrency).
  \item Goose. This system connects an implementation in Go to a model for proof
        purposes in Perennial.
  \item GoJournal. A transaction system with a proof that transactions (if the
        obey a particular contract) are atomic, both to other threads and on
        crash. This thesis focuses on the design and proof strategy of
        GoJournal, without going into the formal details of the Perennial
        program logic.
  \item DaisyNFS. This chapter is both about the file-system proof and how the
        specification for GoJournal is used from Dafny.
  \item Evaluation. We will evaluate the performance of DaisyNFS. We wil also
        evaluate the proof effort and discuss the assumptions in the proof.
  \item Conclusion.
\end{itemize}

The main body of the thesis is the three chapters on Goose, GoJournal, and
DaisyNFS.\@ We will make these independent of each other. Goose is easy to
abstract away in the GoJournal chapter, since we can pretend like the reasoning
is directly over the Go code. A specification we call program refinement bridges
GoJournal and DaisyNFS, so to decouple the two in the DaisyNFS chapter we will
re-introduce the specification for GoJournal, from the perspective of using it
rather than proving it.

\section{Timeline}
The file system and transaction system implementations and proofs are complete,
which represents the bulk of the work of this thesis. We also have a fairly
complete understanding of DaisyNFS's performance compared to Linux. I aim to
complete a draft of the thesis by October 1st and leave several weeks for
feedback and revision. There are three main tasks: revising the story of the
DaisyNFS paper based on feedback from SOSP; removing redundancy between the
GoJournal and DaisyNFS papers; and describing a new and more complete
performance evaluation.
