\section{Crash and recovery reasoning}%
\label{sec:perennial:recovery-spec}

The strategy behind proving crash safety is to prove a theorem about running a
whole system, restarting after every crash. For example, for DaisyNFS this
theorem applies to the server's main loop that receives a message from the
network, processes it in a background thread, and replies over the network.
Immediately following boot, before processing any requests, the server runs a
recovery procedure to restore in-memory state. This whole procedure --- recovery
followed by running the system --- gets an idempotent specification that is
proven with \ruleref{wpr-idempotence} from \cref{sec:perennial:recovery}. The
crash condition for this theorem is a description of the state of the whole
system at any intermediate step; describing this directly would be daunting, but
it is doable since each layer describes its part of the crash condition.

The circular buffer is the lowest layer of the implementation, so the way it
fits into the larger plan is that the whole system's recovery starts by
recovering the circular buffer's state, then uses that state to recover the next
layer, and so on until the whole system is ready to run. Schematically this
looks like the following:
%
\begin{minted}[linenos]{go}
func RunSystem() {
  c, start, upds := RecoverCircular()
  // recover rest of system from c, start, upds
  fs := recoverFilesystem(...)
  for {
    req := GetRequest()
    go func() {
      ret := fs.Handle(req)
      SendReply(req, ret)
    }()
  }
}
\end{minted}

The circular buffer supplies three things to fit into the whole-system
recovery proof: (1) an abstract crash predicate for the state the circular
buffer requires for recovery, (2) a crash specification for the circular
buffer's recovery procedure itself, and (3) a post-recovery init theorem that
helps the caller maintain the circular buffer's crash predicate and also
initializes the circular buffer's invariant.

The specification for the library's recovery procedure \cc{RecoverCircular()} is:
%
\begin{align*}
  &\{ \circstate(\gamma, \sigma) \sep \cc{circ_resources}(\gamma) \} \\
  &\qquad\cc{RecoverCircular()} \\
  &\{ \Ret{(\ell, \cc{diskStart}, \cc{upds})} \sigma = (\cc{diskStart}, \cc{upds}) \sep {} \\
  &\quad \circstate(\gamma, \sigma) \sep {}  \\
  &\quad \startIs(\cc{diskStart}) \sep \diskendIs(\cc{diskStart} + \cc{len}(\cc{upds})) \sep {} \\
  &\quad \cc{circ_appender}(\gamma, \ell) \} \\
  &\{ \circstate(\gamma, \sigma) \sep \cc{circ_resources}(\gamma) \}
\end{align*}
%
The first thing to notice about the recovery specification is that it preserves
$\circstate(\gamma, \sigma)$, which gives the current state of the circular
buffer, both on crash and if recovery completes. It is also the case that
$\cc{circ_appender}(\gamma, \ell) \proves \cc{circ_resources}(\gamma)$, so that
$\cc{circ_resources}(\gamma)$ is also preserved by recovery. These two together
are the crash predicate for the circular buffer. We say the crash predicate is
abstract because the user of this theorem does not need to know how it is
defined.

\newcommand{\cfupdw}{\mathrm{cfupd}}
\newcommand{\cfupd}[1]{\cfupdw\left(#1\right)}
\newcommand{\cinvw}{\mathrm{cinv}}
\newcommand{\cinv}[1]{\mathrm{cinv}\left(#1\right)}

The circular buffer's proof also supplies the following \emph{post-recovery
init} theorem, which sets up the circular buffer for use after recovery:
%
\begin{align*}
  &\forall \sigma, P(\sigma) \vs %
  P_{\mathrm{rec}}(\sigma) \sep %
  P_{\mathrm{crash}}(\sigma) \proves \\
  &\circstate(\gamma, \sigma) \sep P(\sigma) \vs \\
  &\exists \gamma'.\, \cc{is_circular}(\gamma) \sep {} \\
  &\quad \cfupd{ \exists \sigma.\, \circstate(\gamma', \sigma) \sep %
    \cc{circ_resources}(\gamma', \sigma) \sep %
    P_{\mathrm{rec}}(\sigma) } \sep {} \\
  &\quad \cinv{ \exists \sigma.\, \circstate(\gamma, \sigma) \sep %
    \cc{circ_exchanger}(\gamma, \gamma') \sep %
    P_{\mathrm{crash}}(\sigma) }
\end{align*}

There are several components to this theorem. First, the circular buffer has the
caller provide a way to split the HOCAP predicate $P(\sigma)$ into two parts,
$P_{\mathrm{rec}}(\sigma)$ and $P_{\mathrm{crash}}(\sigma)$ --- intuitively the
former is transferred to recovery as part of its precondition, whereas the
latter is ``immutable'' and held inside an invariant. Second, the theorem
consumes $\circstate(\gamma, \sigma) \sep P(\sigma)$ and allocates three things:
(1) $\cc{is_circular}(\gamma)$ (notice that the \cc{is_circular} is defined to
be an invariant, and the premise of this theorem is the contents of that invariant), (2) a ``$\cfupdw$'', and a (3) ``$\cinvw$''. The
latter two are low-level features of Perennial that we'll now introduce.

The assertion $\cfupd{R}$ (``crash fancy update'') is similar to a view-shift
$\TRUE \vs R$ in that it is a single-use update that produces the resources $R$.
However, the ``crash'' part indicates that this update can only be fired after a
crash, so it can be used to prove a crash condition but is otherwise unusable.
Practically speaking, having $\cfupd{R}$ in a precondition allows the user to
\emph{cancel} $R$ from the proof's crash condition, as reflected in the
following rule:

\begin{mathpar}
  \inferH{cfupd-use}
  {P \proves \wpc{e}{Q}{Q_{c}}}%
  {P \sep \cfupd{R} \proves \wpc{e}{Q}{Q_{c} \sep R}}
\end{mathpar}

In the context of the circular buffer, what the $\cfupdw$ sets up for the caller
is that if the system crashes, the recovery proof can dismiss the circular
buffer's crash predicate, along with $P_{\mathrm{rec}}(\sigma)$, from the
overall crash condition. Thus using \ruleref{cfupd-use} before running the
system both sets up the circular buffer for use (by establishing
$\cc{is_circular}(\gamma)$) and guarantees the circular buffer's part of the
crash condition from now on. A simpler but abstract version of this
reasoning appears in the \ruleref{wpc-inv-alloc} rule in \cref{sec:perennial:wpc};
this theorem is a concrete use case (split into creating the cfupd and later
using it, rather than in one step as in \ruleref{wpc-inv-alloc}). What Perennial
makes possible is to take advantage of the invariant for crash reasoning ---
what must be carefully handled is that while the invariant is shared by all
threads, only one thread can get ``credit'' for this invariant holding at crash
time, in the sense of permission to cancel it from its crash condition.

Notice in the circular buffer's post-recovery init theorem that the ghost name
for the circular buffer changes on crash to a new $\gamma'$; the reason for this
is that other threads (the logger and installer) have access to some of the
circular buffer resources associated with $\gamma$. Rather than requiring those
threads to ``return'' those resources on crash and impose a crash condition on
that code, this theorem simply creates a new instance of the circular buffer and stops using
the old one. As a result on crash the theorem can produce
$\cc{circ_resources}(\gamma', \sigma)$ and hand out these resources to the
caller.

The third component of the theorem is a $\cinvw$ (``crash invariant'').
$\cinv{R}$ behaves very similarly to an invariant assertion $\knowInv{}{R}$, except
that like a crash fancy update it can only be used after a crash. This crash
invariant ``freezes'' the old state of the circular buffer (notice that
$\circstate(\gamma, \sigma)$ is now unchanging, since future operations will
interact with the instance named $\gamma'$). This process also produces
$\cc{circ_exchanger}(\gamma, \gamma')$, which includes additional ghost state
relating the old and new instances in the form of the predicate. The circular buffer in
particular does not have an interesting exchanger predicate, but this feature of
the specification pattern shows up in the GoTxn proof.

In general, each layer supplies a post-recovery init theorem. The view shift for
each of these
theorems is invoked in the proof of a program like \cc{RunSystem()} after
running all the recovery code and just
before normal processing (after line 4 in the code above). Firing these view shifts allocates all the layers'
invariants while simultaneously getting ``credit'' for this allocation in the
form of the $\cfupdw$ assertions, similar to how \ruleref{wpc-inv-alloc} works.
Prior to calling this initialization, the recovery procedure has access to the
inner contents of all the invariants, reflecting that it has exclusive access,
but in turn this proof is required to maintain a complicated crash condition.
(One interesting side effect is that recovery code can safely call library
functions without using locks, since this code is still guaranteed to be
single-threaded.) Once the system starts running the invariant allows multiple
threads to share access to this state, and the invariant implicitly guarantees
the crash condition recovery depends on.

\section{Exchanging resources}%
\label{sec:perennial:exchanging}

The final aspect of the logically-atomic specification pattern is the role of
the exchanger, one of the conclusions of the post-recovery init theorem. The
purpose of this predicate is to give the caller a way to, on crash, relate
resources issued by the library before a crash to those after a crash. In the
circular buffer, the $\startIs(\cc{start})$ and $\diskendIs(\cc{end})$ resources
are thrown away on crash and re-created during recovery; there are only two of
them, so it is easy enough to reconstruct them globally. In contrast, some
resources are used locally by a thread and we would like a way to retain
ownership across a crash, into that thread's crash condition. This is exactly
what happens with GoTxn's lifting-based specification for the journaling layer, explained in greater
detail in \cref{sec:txn:lifting}. Exchanging gives threads a way to ``trade''
ownership during a crash, trading ownership associated with the old $\gamma$ instance for
associated with $\gamma'$; both forms of ownership can be
exclusive because the process destroys the old ownership and thus can only be
performed once.

\newcommand{\mapstoDisk}{\mapsto_d}
\newcommand{\mapstoOp}{\mapsto_{\mathit{op}}}
\newcommand{\mapstoLftd}{\mapsto_d^{\mathrm{lftd}}}
\newcommand{\jrnlToken}[1]{\textlog{token}(#1)}

To make this a bit more concrete, let us look at a part of the GoTxn lifting
specification. This specification issues three related resources: the two more important
ones are $a \mapstoDisk o$ and $a \mapstoOp o$. Both represent exclusive
access to address $a$. $a \mapstoDisk o$ is a durable statement about the
logical disk, while $a \mapstoOp o$ gives the value at address $a$ that
an in-progress operation $op$ would observe. The lifting-based specification
gets its name from a logical ``lift'' operation that trades $a \mapstoDisk o$
for $a \mapstoOp o \sep a \mapstoLftd o$. This brings us to the
third type of GoJournal resource, $a \mapstoLftd o$, which is
almost identical to $a \mapstoDisk o$ except that it has been lifted and thus
cannot be lifted again.\footnote{In the implementation,
$a \mapstoLftd o$ is the primitive resource. There is a
resource $\jrnlToken{a}$ that gives the right to lift $a$. Then
$a \mapstoDisk o$ is simply defined to be
$a \mapstoLftd o \sep \jrnlToken{a}$.} We can give
the journal's \cc{Commit} operation a
(highly-simplified) specification like the following:
%
\begin{align*}
  \hoareCV{a \mapstoLftd o \sep a \mapstoOp o'} %
  {\mathit{op}.\cc{Commit}()}%
  {\Ret{\mathit{ok}} \mathrm{if~} \mathit{ok} %
  \mathrm{~then~} a \mapstoDisk o' \mathrm{~else~} a \mapstoDisk o}%
  {a \mapstoDisk o \lor a \mapstoDisk o'}
\end{align*}

In the non-error, non-crash return, the postcondition turns $a \mapstoOp o'$
into a durable fact $a \mapstoDisk o'$. If the system doesn't crash but the
transaction fails (which happens if it is too large to fit in the circular
buffer), then the caller gets back an unlifted disk fact $a \mapstoDisk o$,
reflecting that the transaction did not affect the disk. In the case of a crash
the specification promises to give one of these assertions, and which one
depends on exactly when the crash occurs.

This specification is simplified to only give the case of committing an
operation that modifies a single address, when in reality \cc{Commit}'s main purpose
is to atomically commit multiple addresses; a single address is enough to
understand the proof issues involved. If the system doesn't crash, it is
relatively straightforward to reverse the lifting process and restore full
ownership of the disk values. $a \mapstoDisk o'$ reflects that the commit
actually succeeded in changing the disk value, while $a \mapstoDisk o$ results
from aborting and throwing away buffered writes in $op$.

On crash it is more difficult to show $a \mapstoDisk o \lor a \mapstoDisk o'$.
The challenge is how to propagate whether or not the writes were made durable
from the lower-level abstractions up to this proof; indeed the durability of
this high-level operation depends on whether or not the lowest level of the
system, the circular buffer, successfully wrote a single header block!

The write-ahead logging layer conveys durability by issuing two resources: one
gives the history of multiwrites, and another gives a lower bound on how many
multiwrites are durable. Its exchanging resource allows the caller to trade an
assertion about the history before a crash for a similar assertion after a
crash, albeit with some recent writes lost. Crucially the effect of a crash is
constrained by the durable lower bound, which can also be exchanged across a
crash. On top of this exchanging, the journaling proof implements a fine-grained
exchange of ownership over individual addresses.

The actual implementation of exchanging uses a neat separation logic trick. To
distinguish between $a \mapstoDisk o$ in two different generations, we'll
annotate the resource with a $\gamma$. The exchanger for the journaling proof is
defined in terms of facts like:
\[
  \cc{exchange_addr}(a) \defeq (a \mapsto^{\gamma}_{d} o) \lor (a \mapsto^{\gamma'}_{d} o)
\]
When proving the post-recovery init theorem, we initially prove
$\cc{exchange_addr}(a)$ using the right-hand side; this is easy since $\gamma'$
is fresh, so we've just allocated all of its associated ghost state and it
hasn't been used by any threads so far. We can then prove exchange lemmas like
$a \mapsto^{\gamma}_{d} o \sep \cc{exchange_addr}(a) \proves a \mapsto^{\gamma'}_{d} o \sep \cc{exchange_addr}(a)$.
The reason this proof works is that $a \mapsto_{d} o$ is \emph{exclusive}, thus
from the premise the proof can rule out that $\cc{exchange_addr}(a)$ is proven with the
left disjunct and learn that the right-hand side holds. The proof takes out this
right disjunct ($a \mapsto^{\gamma'}_{d} o$) and gives up
$a \mapsto^{\gamma'}_{d} o$ re-prove $\cc{exchange_addr}(a)$, this time using
the left disjunct. This all ensures that exchanging is only performed once,
which is necessary since all the resources involved are exclusive.

% look at \cc{exchange_mapsto_commit} for the interesting case (ephemeral values
% that might now be durable) and \cc{exchange_durable_mapsto} for the
% straightforward case (durable values that are now proven to still be durable,
% exchanger shows old lower bounds are still valid)

\section{Summary}

The overall logically-atomic crash specification pattern has four components for
every layer of the system:

\begin{itemize}
  \item An opaque predicate for the abstract state of the system (e.g.,
        $\circstate(\gamma, \sigma)$).
  \item Proofs for each operation that use a client-specified predicate,
        modified by requiring a view shift that updates the predicate in
        accordance with the operation's effect on abstract state.
  \item An opaque crash predicate for the system.
  \item A recovery theorem, with a crash condition, that preserves the crash
  predicate for the system and re-builds any in-memory state needed.
  \item A post-recovery init theorem, to be called after the whole system has
        recovered, which sets up the layer's invariant for normal execution,
        creates ghost state for the next generation, and produces an exchanger
        resource to relate the old state to the new one.
\end{itemize}

When layers are composed, the upper layer proof generally hide the lower level
in its own proofs --- for example, the write-ahead log's crash predicate
includes the circular buffer's crash predicate as a sub-term, and subsumes the
circular buffer's recovery and post-recovery init theorems.

This pattern can be combined with user-defined ghost resources in order to give
more powerful specifications (rather than requiring every operation to be
unconditionally atomic). User-defined ghost resources can interact with crashes
and recovery; for example, they can be returned from the recovery theorem and
used across a crash using exchanging lemmas specific to the library.
