Perennial is a framework for reasoning about crash safety and concurrency that
we developed in order to verify GoTxn. The main component of Perennial is a
program logic based on concurrent separation logic, with extensions for
reasoning about crash and recovery behaviors.

\paragraph{Who is this chapter for?}
This chapter describes Perennial for an audience with some programming languages
or verification background, but not necessarily experience with concurrency
specifically. The high-level ideas are meant to be broadly accessible even if
some background on Iris is needed to appreciate the details. To that end the
presentation is somewhat simplified, with details and side conditions omitted
from the logical rules. A more systems-oriented reader can safely skip this
chapter and the next (\cref{ch:crash-logatom}) and still understand the later
chapters.

\section{Primer on Iris and separation logic}%
\label{sec:perennial:iris}

A program logic is a formal system for specifying and reasoning about programs.
One of the simplest program logics is Hoare logic, still the basis for much
sequential reasoning today. The judgments of Hoare logic consist of
specifications of the form $\hoare{P}{e}{Q}$, which is interpreted as meaning
``if $e$ is run in a state where $P$ holds and it terminates, then the final
state will satisfy $Q$''. The logic has various rules for proving and combining
these specifications.

\subsection{Separation logic}

Separation logic is an extension of Hoare logic that has proven profitable for
reasoning about heap-manipulating programs with pointers and concurrency
(surprisingly, the same techniques help solve both problems). A good
introduction to the basic ideas of separation logic is found in O'Hearn's
``Separation Logic'' article~\cite{ohearn:seplogic}. This section gives a more
terse overview, especially to introduce the relevant notation.

Separation logic introduces some notation for the logical assertions that
describe the heap. The core assertion to talk about pointers is $p \mapsto v$,
pronounced ``$p$ points to $v$'',
which says that the pointer $p$ when dereferenced has value $v$. The new logical
connective of separation logic is
$P \sep Q$, pronounced ``$P$ and separately $Q$'', which says that the heap can
be divided into two disjoint pieces, one satisfying $P$ and the other satisfying
$Q$. Entailment between propositions is written $P \proves Q$, read as ``$P$
entails $Q$'' or ``$P$ proves $Q$'', which says that in any heap where $P$
holds, $Q$ must also hold.

When working in separation logic, specifications like $\hoare{P}{e}{Q}$ are
generally stated in a ``small footprint'' style where $P$ mentions only the
state $e$ relies on for its execution. This intuition is backed by the
celebrated frame rule, which says that if $\hoare{P}{e}{Q}$ holds, any disjoint
state is unaffected, namely $\hoare{P \sep F}{e}{Q \sep F}$.

Instead of working with Hoare triples, it is convenient to instead define
specifications in a different style of \emph{weakest preconditions} (WPs). We will use
$\wpre{e}{Q}$ to denote the weakest precondition of $e$ with postcondition $Q$;
if $e$ is run in a state satisfying $\wpre{e}{Q}$ and terminates, the final
state will satisfy $Q$. Note that the $\wprew$ is a \emph{predicate over
states}, not a judgment of the logic like a Hoare triple. To build intuition, the statement
$P \proves \wpre{e}{Q}$ is equivalent to $\hoare{P}{e}{Q}$. An excellent
comparison between weakest preconditions and Hoare triples can be found in
``Separation logic for sequential programs''~\cite{chargueraud:seq-seplogic}.

The term ``weakest precondition'' is because $\wpre{e}{Q}$ is supposed to be the
\emph{weakest} predicate that implies $Q$ holds after $e$'s execution, in the
sense that any other precondition would imply $\wpre{e}{Q}$, but our work does not
emphasize this aspect of weakest preconditions. Furthermore the literature will
sometimes distinguish between weakest liberal preconditions that only guarantee
$Q$ if $e$ terminates and reserve the term weakest preconditions for a predicate
that also guarantees termination. This thesis uses the term weakest precondition for the
``liberal'' version (also called \emph{partial correctness} as opposed to
\emph{total correctness}), because proving termination in the presence of
concurrency is quite challenging.

\Cref{fig:wp-rules} shows some basic rules of separation logic, phrased in terms of weakest preconditions.
As an example, the frame rule becomes
\ruleref{wp-frame} in terms of the \wpw assertion. Reading this rule forwards, if in a proof
the assumptions include $F$
and separately $\wpre{e}{Q}$, then the proof can move $F$ to the postcondition because
separation logic guarantees the proof of the WP does not affect or invalidate
the part of the heap covered by the frame $F$.

\begin{figure}
\begin{mathpar}
\inferH{wp-frame}%
{}%
{F * \wpre{e}{Q} \proves \wpre{e}{F * Q}}

\inferH{wp-mono}%
{P \proves P' \and \forall v.\, ([v/x] Q' \proves [v/x] Q) \and \hoare{P'}{e}{Q'}}%
{\hoare{P}{e}{Q}}

\inferH{wp-seq}%
{\hoare{P}{e_1}{Q} \and \hoare{Q}{e_2}{R}}%
{\hoare{P}{e_1;\, e_2}{R}}

\inferH{wp-load}%
{}{\hoare{p \mapsto v}{\load{p}}{\Ret{v} p \mapsto v}}

\inferH{wp-store}%
{}{\hoare{p \mapsto v}{\store{p}{v'}}{p \mapsto v'}}

\end{mathpar}
\caption{Selection of proof rules for sequential separation logic.}
\label{fig:wp-rules}
\end{figure}

\subsection{Ghost state and concurrency in Iris}
\label{sec:perennial:concurrency}

Concurrent separation logic~\cite{brookes:csl} generalizes separation logic to
also reason about concurrency. Iris is a type of concurrent separation logic,
with several advances beyond the original formulation. A full
explanation of the Iris logic is out-of-scope for the thesis; ``Iris from the
ground up''~\cite{jung:iris-jfp} is a comprehensive introduction while the
original ``Iris 1.0'' paper~\cite{jung:iris-1} is a shorter introduction for a
reader already familiar with separation logic. Two features of Iris are most
relevant since they are used in the GoTxn proof: ghost state and invariants.

A key technique in Iris is to verify a program by augmenting its
physical state (local variables and the heap) with some additional \emph{ghost
state} which is maintained only for the sake of the proof and has no effect on
the program's execution (hence the term ``ghost''). It is easier to understand
ghost state via its API in Dafny as a programming-language feature, so let
us first see how they help there and then return to Iris.

In Dafny, a variable can be marked \cc{ghost}. Ghost variables can be written
and read in the proof, but Dafny enforces that the ghost variables' values never
influences execution; they can only be used to inform uses of lemmas,
assertions, and other proof annotations. Then at run time ghost variables and all
uses of them are \emph{erased} before running the program. Why would adding a
ghost variable to a program help with its proof? The simplest examples are code
where a ghost variable holds the old value of some variable, say prior to a
loop; this lets the proof refer to the old value while clarifying that the
regular execution does not need it.\footnote{For a concrete example, see the
bubble sort example in
\url{https://www.doc.ic.ac.uk/~scd/Dafny_Material/Lectures.pdf}.}

\pagebreak[1]
Ghost variables can also be used to give abstract specifications to a piece of
code. For example, consider a ``statistics database'' that maintains the running
mean of a sequence of numbers (written in Go for readability):
\nopagebreak
\vspace{\baselineskip}
\hrule\nopagebreak
\vspace{-12pt}\nopagebreak
\noindent\begin{tabular}{p{0.5\textwidth} p{0.5\textwidth}}
\begin{minted}[frame=none]{go}
type StatDB struct {
  count int
  sum   float64
}

func (db *StatDB) Add(n float64) {
  db.count++
  db.sum += n
}
\end{minted}
&
\begin{minted}[frame=none]{go}
func (db *StatDB) Mean() float64 {
  if db.count == 0 {
    panic("empty db")
  }
  return db.sum/db.count
}
\end{minted}
\end{tabular}
\vspace{-8pt}
\hrule
\vspace{\baselineskip}

The code only tracks the count of elements and
their sum, but the behavior of the library is easiest to state in terms of the list of all
numbers added. Thus in Dafny such a library can use a ghost variable to track
the full database, relate this ghost variable to the physical variables
of the code, and then prove that the code returns the correct running mean in
terms of the ghost state.

One intuition for the technique of ghost variables is that it augments the
execution of the program with additional information, which is used only for the proof
and thus not tracked at run time. For every actual execution, there is a
corresponding execution where the ghost variables are maintained and updated.
The proof is carried out on this augmented execution,
but the proofs apply to the normal execution because by design they have the
same behavior. Verifying the program with ghost
variables is easier because the ghost variables can track important
information about the history of the program, such as in the example above of
the pre-loop values of the local variables.

In Iris, the proof is a separate entity from the code. The program logic still
has a way to use ghost variables, with proof rules that construct and update a
ghost variable, applied at the appropriate points in the proof rather than added
to the code. The high-level idea for why this works --- that there is an
augmented execution with the ghost variables ---
remains the same. In fact in Iris it is more obvious
that the ghost variables do not affect program execution, since their creation
and updates only appear in the proof.

So far, we've explained ghost state in terms of ghost variables, with the familiar
API where they can be read and written. Iris ghost state is a bit more
sophisticated in order to support concurrency reasoning. Iris has
separation logic assertions for \emph{ownership} of ghost state, which can be
split and divided among threads. In conjunction with this analogy to ownership,
in concurrent separation logic a piece of ghost state is also referred to as a
\emph{resource} that a thread can own by having an assertion over the ghost
state in its precondition.  A key principle in concurrent separation logic is that in
a proof about a thread of interest, any resources or ownership in that proof's
precondition can never be invalidated by the actions of other threads. Ghost
state can have restrictions on how it may be updated to reason about a shared
protocol that all threads respect.

\newcommand{\dashedbox}[1]{\boxedassert[densely dashed]{#1}[]}
\newcommand{\ghostvar}[2][]{\dashedbox{\gamma \mapsto_{#1} #2}}

A simple example to illustrate both principles is Iris's fractional ghost
variables. The assertion $\ghostvar[q]{v}$ says that the ghost variable $\gamma$ has value
$v$ (of any fixed type) and asserts ownership over a (positive) fraction $q$ of
it --- any fraction $q < 1$ represents read-only access to the ghost variable,
while full ownership $q = 1$ allows writing as well.
Full ownership $\ghostvar[1]{v}$ is common enough that it is often
abbreviated to
$\ghostvar{v}$, with no fraction. The dashed box around this assertion emphasizes that this
assertion is about ghost state and not about the heap, as in the points-to
assertion $p \mapsto v$.  There are several rules for manipulating and
using this fractional ghost state:

\begin{mathpar}
  \inferH{frac-alloc}{}%
  {\proves \upd \exists \gamma.\, \ghostvar[1]{v}}

  \inferH{frac-update}{}%
  {\ghostvar[1]{v} \proves \upd \ghostvar[1]{v'}}

  \inferH{frac-split}{}%
  {\ghostvar[q_1 + q_2]{v} \provesIff \ghostvar[q_1]{v} \sep \ghostvar[q_2]{v}}

  \inferH{frac-agree}{}%
  {\ghostvar[q_1]{v_1} \sep \ghostvar[q_2]{v_2} \proves v_1 = v_2}

  \inferH{upd-fire}%
  {P \proves \wpre{e}{Q}}%
  {\upd P \proves \wpre{e}{Q}}
\end{mathpar}

The new notation $\upd$ is an Iris \emph{update modality}. The assertion
$\upd P$ expresses ownership of resources which could be used to become $P$ with
some update to ghost state. As an example of proving an update modality,
\ruleref{frac-alloc} shows that starting with no assertions it is possible to
allocate a new ghost variable $\gamma$ with value $v$ and complete ownership
over it; this is analogous to how the Hoare triple for allocation has no
precondition. The formal rule that allows the user to get access to $P$ is
\ruleref{upd-fire}. It corresponds to advancing the proof of $\wpre{e}{Q}$ by
changing whatever ghost state is needed to turn $\upd P$ into $P$. As long as
the user of the logic is proving a weakest precondition as the goal, they can
apply this rule to ``eliminate'' an update modality.

Fractional ghost state can be updated after creation with \ruleref{frac-update}.
The update requires full ownership. Fractional ghost variables can instead be
split into smaller pieces with \ruleref{frac-split}; these pieces can no longer
be updated (since this would invalidate the rest of the assertions), but two
assertions for the same ghost variable must be for equal values
(\ruleref{frac-agree}) since the underlying variable has only one value.
Fractional ownership expresses a simple protocol among threads where when a
thread owns a fraction less than one, it can read but not write, and full
ownership is sufficient to write to a ghost variable.

This thesis describes a few constructions for ghost state to carry out parts of
the GoTxn proof, such as the example of fractional ghost state described above. In reality all
ghost state in Iris is defined using a single, general mechanism. Ghost state
can come from any instance of an algebraic structure $M$ called
a ``resource algebra'', where ownership really means
ownership of an element $a \in M$. This thesis does not explain the details of
how ghost state is constructed using resource algebras --- see
``Iris from the ground up''~\cite{jung:iris-jfp}. For the ghost
state in this thesis, we will only give the API, in terms of resources and
rules that allow updating those resources. The Iris logic ensures that the
updates are ``sound'', enforcing a global property that updates to a resource in
one part of the proof never invalidate resources owned by concurrent threads at
the same time.

\begin{figure}[ht]
  \begin{mathpar}
    \inferH{wp-inv-alloc}%
    {P \sep \knowInv{}{R} \proves \wpre{e}{Q}}%
    {P \sep R \proves \wpre{e}{Q}}

    \inferH{inv-atomic}%
    {\atomic(e) \and R \sep P \proves \wpre{e}{R \sep Q}}%
    {\knowInv{}{R} \sep P \proves \wpre{e}{Q}}

    \inferH{wp-fork}%
    {P \proves \wpre{e}{\TRUE} \and Q \proves \wpre{e'}{R}}%
    {P \sep Q \proves \wpre{(\operatorname{fork} \{e\}; \, e')}{R}}
  \end{mathpar}
  \caption{Key concurrency rules in Iris for invariants and forking}
  \label{fig:invariants}
\end{figure}

A fundamental reasoning principle for concurrency is the notion of an
\emph{invariant}. Threads eventually do share state, and invariants
are the main way to reason about how threads coordinate on that shared state.
The assertion $\knowInv{}{I}$ expresses the knowledge that
$I$ is an invariant. Once this invariant is established, the
proof rules in Iris guarantee that $I$ holds at all steps of the program. A
thread that has $\knowInv{}{I}$ in its precondition can make use of the
invariant by ``opening'' it to obtain ownership over $I$, but only for a single program step; it must be
returned afterward to guarantee the invariant holds for other threads. Finally,
invariants are freely \emph{duplicable} --- that is,
$\knowInv{}{I} \proves \knowInv{}{I} \sep \knowInv{}{I}$ --- reflecting that
knowledge of an invariant, once it is established, is an assertion that all
threads agree on which cannot be invalidated.

The formal rules for using invariants in Iris (which are all valid rules in
Perennial) are given in \cref{fig:invariants}. To create an invariant
$\knowInv{}{R}$, a thread gives up $R$, as given in \ruleref{wp-inv-alloc}. The
rule for using an invariant is \ruleref{inv-atomic}, which can only be used over
an ``atomic'' instruction $e$. Perennial is defined for a general language, as in
Iris, and the semantics of the language defines what is modeled to be atomic.
Formally $\atomic(e)$ says that the expression $e$ reduces to a value in a
single reduction step, so that other threads cannot run in between. See
\cref{sec:goose:semantics} for the details on the specifics of what GooseLang
makes atomic.

Ownership transfer is reflected in the rule for forking new threads,
\ruleref{wp-fork}. If a thread has $P \sep Q$ in its precondition, it can pass
some of those resources $P$ to a newly-forked thread and retain the remainder
$Q$ for the subsequent code. Note that due to the separating conjunction these
resources must be separate, so that the rules of the separation logic guarantee
$e$ cannot invalidate $Q$ using its resources $P$. However, invariants give a
way for the two threads to safely share resources: both can have access to
$\knowInv{}{I}$ because the knowledge of the invariant can be duplicated, which
is sound because each thread only uses $I$ for an atomic step and then
guarantees $I$ holds for concurrent threads.

% \tej{Look at Perennial 1.0 paper and Later Credits paper for some inspiration on
% introducing Iris}

\section{Crash weakest preconditions}
\label{sec:perennial:wpc}

\newcommand{\propc}{P_c}
\newcommand{\propcB}{Q_c}
\newcommand{\propcC}{R_c}

\newcommand{\wpcseqfig}{%
\begin{mathpar}
\inferH{wpc-value}
{}{\propc \land [\val/\var]\prop \proves \wpc{\val}{\Ret\var \prop}{\propc}}

\inferH{wpc-mono}
{\forall \val.\,\left([\val/\var]\prop \proves [\val/\var]\propB\right) \and
\propc \proves \propcB}
{\wpc\expr{\Ret\var \prop}{\propc} \proves \wpc\expr{\Ret\var \propB}{\propcB}}

%\inferH{wpc-bind}
%{\text{$\lctx$ is an evaluation context}}
%{\wpc\expr{\Ret\var  \wpc{\lctx[\var]}{\Ret\varB \propB}{\propcB}}{\propcB} \proves \wpc{\lctx[\expr]}{\Ret\varB \propB}{\propcB}}

\inferH{wpc-let}
{}
{\wpc{e_1}{ %
    \Ret{v} \wpc{\subst{e_2}{x}{v}}{\prop}{\propc} % nested postcondition
}{\propc} \proves
\wpc{\gooselet{x}{e_1}{e_2}}{\prop}{\propc}}

\end{mathpar}
}

Iris gives tools for proving specifications that capture the concurrency
behavior of a program, but storage systems need stronger specifications that
also cover crash safety. The formal definition of crash safety is ultimately
stated as a property of a storage system combined with a recovery procedure.
Crash safety is \emph{defined} in terms of the results possible after a program
crashes mid-operation, the system reboots, and subsequently a recovery procedure
re-initializes the program.

In Perennial, most of the reasoning required for this recovery-based definition
goes into specifying what happens if the system halts at some intermediate point
and all threads stop running. The core specification idea for reasoning about
this situation is a \emph{crash weakest precondition} $\wpc{e}{Q}{Q_{c}}$. Similar
to the weakest precondition, if $e$ is run from a state satisfying this
predicate and terminates, the resulting state will satisfy $Q$. However, in
addition if the system halts at any time $Q_{c}$ is guaranteed to hold. We also
sometimes write a \emph{crash Hoare quadruple} $\hoareC{P}{e}{Q}{Q_{c}}$ that is
defined to be $P \proves \wpc{e}{Q}{Q_{c}}$. There is one subtlety in the
terminology: though we use the term ``crash,'' these postconditions hold
\emph{just prior} to the system actually shutting down, so that the contents of
memory is unchanged.\footnote{They might more properly be called \emph{halt
conditions}, but the original FSCQ paper used ``crash conditions'' and the term has
stuck since then.} In \cref{sec:perennial:recovery} we'll connect these
crash specifications to the memory wipe and reboot that happens right afterward.

Some basic structural rules for these crash weakest preconditions are given in
\cref{fig:wpc-structural}. These largely mirror structural rules from Iris and do not
say anything crash-specific. The rule \ruleref{wpc-mono} allows weakening a
WPC by replacing the postcondition $Q$ with a weaker assertion $P$ and replacing
the crash condition $Q_{c}$ with a weaker assertion $P_{c}$. The rule
\ruleref{wpc-let} is a bit long, but it expresses formally sequentially
reasoning about $e_{1}$ followed by $e_{2}$. What it expresses is that to verify
$\gooselet{x}{e_1}{e_2}$, a proof can verify $e_{1}$ first, then in its post-condition
reason about $e_{2}$ with the return value $v$ from $e_{1}$ substituted for the
bound variable $x$. The crash condition $P_{c}$  is carried throughout, since
both $e_{1}$ and $e_{2}$ must maintain it.  Perennial has a similar but more
general rule for arbitrary \emph{evaluation contexts} (of which $\goosekw{let}$
is just an example), to reason about sequencing between a sub-expression and its
context.

\begin{figure}[ht]
  \wpcseqfig
\caption{Basic structural rules for crash weakest preconditions.}%
\label{fig:wpc-structural}
\end{figure}

\begin{figure}[ht]
  \begin{mathpar}
    \inferH{wpc-frame}
    {}{\propB * \wpc\expr{\prop}{\propc} \proves \wpc\expr{\propB*\prop}{\propB*\propc}}

    \inferH{wp-wpc}
    {}{\wpre\expr{\Ret\var \prop} \dashv\proves \wpc\expr{\Ret\var \prop}{\TRUE}}
  \end{mathpar}
  \caption{Interesting crash-related structural rules for crash weakest
    preconditions.}%
\label{fig:wpc-seq}
\end{figure}

Some more interesting structural rules for WPCs with sequential code are listed in \cref{fig:wpc-seq}.
One such rule that is often used in Perennial is the crash frame rule,
\ruleref{wpc-frame}. Like the traditional frame rule, this is a reasoning
principle for ignoring some resources while proving part of a program. When
reasoning about crashes, framing is a useful way to dismiss the crash condition
when it refers to durable resources that aren't needed for reasoning about some
part of the code.

The way the rule works is that in analogy to \ruleref{wp-frame}, the premise
is a proof of $Q$ (the frame) and separately $\wpc{e}{P}{P_{c}}$. However, in
addition to framing from the postcondition, Perennial also frames from the crash
condition. A common case is where $P_{c} = \TRUE$, which is useful when $e$ is a
purely in-memory piece of code. In that case a proof can combine framing
and \ruleref{wp-wpc} to reason about part of a crash Hoare quadruple using
crash-free reasoning, by temporarily ignoring the durable resources $P_d$ in the
precondition, using a derived rule:

\[
  \infer{\hoare{P}{e_1}{Q} \and \hoareC{Q \sep P_d}{e_2}{R}{Q_c}}%
  {\hoareC{P \sep P_d}{e_1;\, e_2}{R}{Q_c}}
\]

Another example of combining WPC and WP (crash and crash-free) reasoning is in
the \ruleref{wpc-atomic} rule. The precondition $\atomic(e)$ says this rule only
applies to atomic expressions, which take a single step, and the conclusion is a
WPC for this expression. The premise involves a connective $P \land Q$. This is
a \emph{non-separating conjunction} or ``logical and''. $P \land Q$ holds in
some state when $P$ and $Q$ both hold, but unlike $P \sep Q$ they do not have to
be over disjoint parts of the state, so for example
$p \mapsto v \proves p \mapsto v \land p \mapsto v$ is trivially true.

The logical ``and'' makes this spec much more powerful. If at some point at the
proof we have resources $R$ and want to prove $\wpc{e}{P}{P_{c}}$, the rule says
it is sufficient to prove two things: $R \proves P_{c}$ and
$R \proves \wpre{e}{P_{c} \land P}$.
Observe that this rule reduces crash reasoning to non-crash reasoning. Also
notice that unlike most separation logic reasoning, we don't need to split up
$R$ to prove these two entailments; the full contents of $R$ are available in both.  There are two
reasons why this reasoning is sound: first, $e$ takes only a single step, and
the system either crashes or not, but not both, so the proof only needs to show either
the crash or post condition. If the system crashes before $e$ then
$R \proves P_{c}$ shows the crash condition holds now, and otherwise the
resources $R$ are still available to prove $\wpre{e}{P_{c} \land P}$. That proof
shows that the post-crash resources satisfy both $P_{c}$ (if the system crashes
right after $e$) and $P$ (to show the postcondition).

\begin{figure}
  \begin{mathpar}
    \inferH{wpc-inv-alloc}%
    {P \sep \knowInv{}{R} \proves \wpc{e}{Q}{Q_c}}%
    {P \sep R \proves \wpc{e}{Q}{Q_c \sep R}}

    \inferH{wpc-atomic}
    {\atomic(\expr)}
    {\propc \land \wpre\expr{\Ret\var  \propc \land \prop}
    \proves \wpc\expr{\Ret\var \prop}{\propc}}
  \end{mathpar}
  \caption{Rules for reasoning about concurrency and crashes.}
  \label{fig:wpc-concurrent}
\end{figure}

To reason about concurrency, Perennial needs some more principles for sharing
ownership between threads. The core mechanism in concurrent separation logic for
reasoning about concurrency is the \emph{invariant}. The rules for working
with invariants in a non-crash setting are given in \cref{fig:invariants}: \ruleref{wp-inv-alloc} gives up $R$
in exchange for $\knowInv{}{R}$, and \ruleref{inv-atomic} ``opens'' an
invariant for a single atomic step.
$\knowInv{}{R}$ is duplicable --- that is,
$\knowInv{}{R} \proves \knowInv{}{R} \sep \knowInv{}{R}$.

Invariants have a special role in Perennial. \Cref{fig:wpc-concurrent} lists
Perennial's rules for concurrency reasoning. Perennial extends invariant allocation
with a rule \ruleref{wpc-inv-alloc}. The non-crash parts of this rule are
identical to \ruleref{wp-inv-alloc}, but applying the rule has the additional
benefit of \emph{removing $R$ from the crash condition}. The intuitive reason
this is sound is that since threads maintain $R$ at all intermediate steps, it
must also hold in case the system crashes, and the thread that created the
invariant can get ``credit'' for this on crash. After allocating an invariant it
has no special role for crashes, and proofs can use \ruleref{inv-atomic} as
usual for opening an invariant across an atomic step.

Invariants are especially useful for lock-free reasoning, but concurrent code
commonly coordinates between threads using locks. A lock guarantees that no two
threads own the lock at the same time. This is expressed in separation logic by
associating a \emph{lock invariant} $P$ with each lock when it is created, which
we think of as something the lock ``protects''. When a thread acquires the lock, it obtains
ownership of $P$, and when it release the lock, it gives up the same ownership.
This is sound precisely because either a single thread holds the lock, or it is
free (in which case intuitively we can think of the lock as holding ownership).

What happens if the system crashes while a lock is held?
The standard lock specification gives a thread exclusive access to $P$ during a
critical section (between $\lock{\ell}$ and $\unlock{\ell}$), and says nothing
about the locked state if the system crashes in the middle of a critical
section --- this is fine if the lock protects in-memory state that is anyway
lost on crash, but insufficient for reasoning about locks that protect durable
state. One intuition for what goes wrong is that a crash ``steals'' ownership of
the locked state and forcibly transfers it to recovery, which is something the
proof of a concurrent, crash-safe system needs to reason about.

\begin{figure}
  \begin{mathpar}
    \inferH{wp-lock-alloc}%
    {}%
    {P \proves \wpre{\newlock}{\Ret \ell \islock{\ell}{P}}}

    \inferH{wp-lock-use}%
    {P \sep R \proves \wpre{e}{P \sep Q}}%
    {\islock{\ell}{P} \sep R \proves \wpre{(\lock{\ell};\, e;\, \unlock{\ell})}{Q}}

    \inferH{wpc-lock-alloc}%
    {P \proves P_c}%
    {P \proves \wpc{\newlock}{\Ret \ell \iscrashlock{\ell}{P}{P_c}}{P_c}}

    \inferH{wpc-lock-use}%
    {P \sep R \proves \wpc{e}{P \sep Q}{P_c}}%
    {\iscrashlock{\ell}{P}{P_c} \sep R \proves \wpre{(\lock{\ell};\, e;\, \unlock{\ell})}{Q}}

  \end{mathpar}
  \caption{Rules for reasoning about concurrency and crashes with locks.}
\end{figure}

Perennial addresses this issue by introducing a new crash-aware lock
specification that also gives guarantees if the system crashes during a critical
section. Compared to the regular lock spec (\ruleref{wp-lock-alloc} and
\ruleref{wp-lock-use}), the crash-aware specification (\ruleref{wpc-lock-alloc}
and \ruleref{wpc-lock-use}) associates both a regular lock invariant $P$ with
the lock and a crash condition $P_{c}$. The rule for allocating a lock dismisses
the crash condition, similar to allocating an invariant, and intuitively the
reason why this rule is sound is that when the lock is used with
\ruleref{wpc-lock-use} the caller is obliged to prove $P_{c}$ holds throughout
the entire critical section (whereas the stronger lock invariant $P$ only needs
to be restored at the end).

\section{Crash model}
\label{sec:perennial:crash-model}

Perennial has to model a system crash and reboot. A crash is represented as
another transition for a program to take, one which wipes in-memory state. The
reboot is modeled by running a dedicated procedure that restores the storage
system on boot, which we call a \emph{recovery procedure}. So far, we have used
Perennial for reasoning about systems that interact with a disk, a device with
an interface for reading and writing blocks (assumed to be 4KB in our
development). These disks might be a block device like \cc{/dev/sda1} in Linux,
or can be a file hosted in another file system. The latter is useful for testing
and performance experiments that run on an in-memory file hosted in tmpfs.

A crash transition resets the heap to empty while preserving the disk state.
Encoding the effect on the heap in separation logic is a bit tricky, since it would appear that
assertions about in-memory pointers like $\ell \mapsto v$ need to be
invalidated. The logic handles this by instead parameterizing all heap
assertions and weakest preconditions with a \emph{generation number}. Across a
crash, the generation number is incremented, so that old assertions are true but
no longer work for loads and stores since they do not apply to the current heap.
A class of assertions $\durable(P)$ only concern durable state like the disk, which in the
logic means that $P$ is independent of the current generation number. These
assertions are special in that they can be proven as the crash condition of the
system and then used in the precondition of recovery, as described subsequently
in \cref{sec:perennial:recovery}.

The Perennial logic, like the underlying Iris framework, is parameterized over a
language and its semantics, which defines the crash transition. The
language we generally work with is GooseLang, described in detail in
\cref{ch:goose}. Our proofs so far with GooseLang use a simple crash model,
where the heap is reset to an empty map and the disk is unchanged. This
corresponds to a synchronous semantics, since it means that disk writes are
immediately considered durable. Perennial also supports defining an asynchronous
disk by modeling buffered writes and then non-deterministically choosing which
buffered write is persistent at crash time, similar to the model in
FSCQ~\cite{chen:fscq}. Because the crash transition is defined by the user, it
is not necessary in Perennial to strictly partition the state into ephemeral and
durable state.

\section{Recovery reasoning}
\label{sec:perennial:recovery}

To allow the user to reason about the recovery process, Perennial
has a \emph{recovery} weakest precondition $\wpr{e}{e_{r}}{P}{P_{r}}$,
where $e$ represents the storage system, $e_{r}$ is a recovery procedure that
will run on restart, $P$ is the postcondition for normal execution, and $P_{r}$
is the \emph{recovery postcondition}. When $e$ is run in a state satisfying this
$\wprw$, if it terminates normally then $P$ holds, and if the system crashes and
$e_{r}$ terminates then $P_{r}$ holds. The latter is true even if the system
crashes while running $e_{r}$ and restarts all over again.

Perennial has only one rule to prove a $\wprw$, which reduces it to proving a
$\wpcw$ about $e$ and an \emph{idempotent} specification for
$e_{r}$~\cite{chen:fscq}:

\[
  \inferH{wpr-idempotence}%
{\operatorname{durable}(P_{c}) \and P_{c} \proves \wpc{e_{r}}{P_{r}}{P_{c}}}%
{\wpc{e}{P}{P_{c}} \proves \wpr{e}{e_{r}}{P}{P_{r}}}
\]

At a high level, the user proves $\wpr{e}{e_{r}}{P}{P_{r}}$ in two steps:
\begin{enumerate}
  \item First, prove $\wpc{e}{P}{P_c}$ to cover normal termination and establish a
  global crash invariant $P_{c}$ for the program.
  \item Second, prove $P_{c} \proves \wpc{e_{r}}{P_{r}}{P_{c}}$ to establish the
  recovery postcondition from the crash condition and show that recovery
  maintains the crash invariant so that crashes during recovery are also
  handled.
  \item Finally, prove $\durable(P_{c})$, which asserts that the
  crash invariant is stated using only \emph{durable} resources that survive a
  crash.
\end{enumerate}

The final step, the durability side condition, is where the proof takes into
account the effect of a crash. For example $\wpc{e}{P}{P_{c}}$ proves that
$P_{c}$ holds just prior to a crash while
$P_{c} \proves \wpc{e_{r}}{P_{r}}{P_{c}}$ starts reasoning just after the crash.
Durability is defined so that if $P_{c}$ is durable, it holds across a crash.

In practice a user of the logic proves a $\wprw$ assertion about an expression $e$
that is a server loop that accepts operations, executes them, and replies. The
same loop $e$ is also $e_{r}$ since it restores its state from disk. Finally,
the user will separately prove that $e$ is safe to run from an initial state
with an all-zero disk.

The recovery code sets up the global invariants and appropriate crash locks for
the whole system and thus cancels all of these assertions from its crash
condition; the separating conjunction of all of these invariants will be
$P_{c}$. Then the proof rules in Perennial guarantee that recovery can assume
$P_{c}$ holds if the system crashes and reboots at any time, as given by the
idempotence rule.

\section{Soundness}

Above we presented an intuitive meaning for $P \proves \wpr{e}{e_{r}}{Q}{Q_{c}}$, but it is
not obvious that the definitions of $\wprw$ and $\wpcw$ imply that this
intuitive meaning holds up. Interpreting the mechanisms of the logic is
complicated since it requires relating logical features like ghost state,
invariants, and ownership to the execution of the program. To address these
concerns, Perennial comes with a \emph{soundness theorem} that formally connects
the definitions to a guarantee about $e$ and $e_{r}$ independent of the details
of the logic. The soundness theorem establishes what an end-to-end theorem about
the system \emph{means} without trusting the interpretation and implementation
of these features.

The basic version of the soundness theorem says:

\newcommand{\bigast}{\mathop{\scalebox{3}{\raisebox{-0.3ex}{$\ast$}}}}

\begin{theorem}[Perennial soundness]
\label{thm:soundness}
  Let $d \in \textdom{Disk}$ be a disk. Let $\phi$ and $\phi_r$ be predicates over
  values.
  Suppose that
  \[\bigast\limits_{a \mapsto v \in d} a \dmapsto v \proves
  \wpr{e}{e_r}{\Ret{x} \phi(x)}{\Ret{y} \phi_{r}(y)} \] is derivable in
  Perennial. Running $e$ (recovering with $e_{r}$) in a state $\sigma$ with an
  empty heap and disk $d$ will not get stuck, and if the execution
  terminates with an expression $e'$ in state $\sigma'$ (possibly
  with additional forked threads), then
  \begin{enumerate}
    \item If this execution is a normal termination and $e'$ is a value, then
    $\phi(e')$ holds.
    \item If this execution is a crash and recovery execution and $e'$ is a
    value produced by $e_{r}$, then $\phi_{r}(e')$ holds.
    \item Any forked threads are either a value or are reducible in $\sigma'$.
  \end{enumerate}
\end{theorem}
To satisfy the premise of the soundness theorem, the user gets to assume points-to
facts for all of the addresses in the initial disk. In return the theorem has
several postconditions. First, the theorem promises
that $e$ runs ``safely'' in the sense of not getting stuck. This immediately
rules out a number of errors, including any calls to Go's built-in \cc{panic()}
function and lower-level issues like reading slices out-of-bounds and data
races.  Second, the properties (1) and (2) state that the normal postcondition and
recovery postcondition have the intended meaning for the return value of the
main thread $e$ or the recovery procedure $e_{r}$, respectively. Finally,
property (3) is a safety statement that says forked threads also do not
encounter errors.

This soundness theorem is machine-checked in the Perennial implementation. Its
proof hides complexity in the definitions of $\wpcw$ and $\wprw$, and in the
definitions of all the Perennial mechanisms using lower-level Iris primitives.

The reader might observe that this theorem doesn't say much about intermediate
results of the program. For a system like a server, which has no interesting
return value, the main outcome of this theorem is the safety statement.
It is possible to go beyond this soundness theorem to a \emph{refinement}
soundness theorem that specifies a program by relating its observable I/O
behavior to a simpler, more abstract program. We used this more general support
to prove a theorem for GoTxn that relates an arbitrary program using the
transaction system to a simpler specification program. Both the simple soundness
theorem above and the refinement soundness theorem are proven using lower-level
mechanisms from Iris and Perennial.

\section{Implementation}
\label{sec:perennial:impl}

Perennial is implemented on top of Iris. The Iris logic is divided into two
parts: a ``base logic'' which defines a notion of resources, ownership, and
separation logic, and on top of this base logic a \emph{program logic} that
defines the weakest precondition $\wpre{e}{Q}$ as an assertion in the base logic
about the execution of $e$. All of the proof rules in the program logic are
defined as theorems in the base logic, using the definition of $\wpw$.

Perennial re-uses the Iris base logic with no changes, and builds a custom
program logic on top with $\wpc{e}{Q}{Q_{c}}$ and $\wpre{e}{Q}$. The non-crash
part of these definitions are similar to Iris, but the crash aspects required
significant implementation work. This thesis aims only to explain how the logic
is used, at a high level of abstraction; the code is currently the only
source of truth for how the logic is actually implemented in the base logic (the
\emph{model} of Perennial's \wpcw).

One aspect of the implementation that should be noted is that the rules as
presented in this thesis have been simplified for exposition and are not sound
as written. The real theorem statements require Iris features like the later modality,
invariant namespaces, and masks to be sound. The later modality addresses issues with circularity that
arises from Iris's flexible ghost state. Namespaces and masks are used to
prevent re-entrancy where an invariant is opened twice by the same thread, which
would be unsound. The
on-paper presentation still aims to capture the essence, but the code is again
the source of truth for the Perennial logic (which has an associated soundness
proof in Coq).

Iris comes with a proof mode called MoSeL~\cite{krebbers:ipm,krebbers:mosel} for
interactive proofs. The proof mode supports proving separation logic theorems
and proving programs correct interactively. The style of proof highly resembles
the Coq proof mode, but extended for reasoning in separation logic. The proof
mode is highly extensible, and we used these features to integrate support for
MoSeL with Perennial.

There are two interesting aspects in the Perennial MoSeL support that we
developed.
The first is \emph{named propositions}, a general feature for
interactive separation logic proofs that arose out of our experience writing and
changing large invariants. The second is tactics specific to the crash reasoning
in Perennial: \emph{proof caching} reduces the burden of crash-safety reasoning
where the crash condition must often be re-proven at intermediate steps of the
proof, and \emph{crash framing} simplifies the mechanics of framing away
resources that are needed for the crash condition but not being used otherwise.

\subsection{Named predicates in separation logic}

In proving GoTxn, every layer defines an invariant that is preserved throughout
the entire execution. Coming up with these invariants is a challenging part of
the proof, and requires frequent cycles of finding an invariant doesn't work,
editing it, and then revising the proofs. The proof will typically need to
\emph{destruct} the invariant, which breaks it into separate hypotheses for each
conjunct; for example if we have an invariant \cc{P * Q * R}, we might
need to write \cc{iDestruct "H" as "(HA & HB & HC)"} to turn it into three
hypotheses. The problem is that updating proofs when the invariant changes is
tedious and error-prone. If we add \cc{* S} to the end, then ``HC'' will now be
\cc{R * S}; the situation is worse if a conjunct is added to the middle of the
invariant.

\emph{Named propositions} help solve this problem. With this feature, the
invariant is written combining the logical statement with names for each conjunct:
\begin{verbatim}
Definition inv := "HA" :: P *
                  "HB" :: Q *
                  "HC" :: R.
\end{verbatim}
The notation \cc{"HA" :: P} means the exact same thing as \cc{P}, logically, but
named propositions extend the proof mode with the ability to destruct a hypothesis and
automatically name its conjuncts. This has two advantages: first, we don't repeat
the names for the conjuncts throughout the proof, and second, extensions to the
invariant are minimally invasive to the proof (even if conjuncts are reordered)
since each name always refers to the same thing it did before.

Named conjuncts made a huge difference to proof burden --- many individual
definitions have 5--6 conjuncts, and across the entire GoTxn proof we name some
900 total conjuncts. Implementing them was actually quite simple, taking only
about 400 lines of code in Coq on top of the Iris Proof Mode. The implementation
is separate from Perennial, depending only on Iris.\footnote{The code is
available at \url{https://github.com/tchajed/iris-named-props}.}

% cat src/program_proof/{buf,wal,txn,obj,jrnl}/**.v | grep -c '∷
% 907

\subsection{Perennial-specific interactive proofs}

\paragraph{Caching:} Crash-safety reasoning involves repeatedly showing
that the crash condition holds at each intermediate point.
Writing local specifications (by ``framing away'' anything not needed)
helps reduce this burden, but does not fully eliminate it.
Perennial includes a tactic for caching and reusing proofs of the crash condition.
A proof
engineer proves the crash condition as it stands at some point, using
as few assumptions as possible for locality, and the caching infrastructure saves
the proof.\footnote{For readers familiar with Iris, the proof is a persistent
implication that can be reused as many times as desired.}
% This implication is
% proven persistently (that is, without
% spatial resources), so it is re-usable as many times as desired.
Whenever the
crash condition appears later in the proof with the same assumptions available,
the caching infrastructure proves the goal automatically.
% This approach is
% convenient, easy to implement, and more general than what could be implemented
% within the program logic itself.

\paragraph{Framing:} Recall the \ruleref{wpc-frame} rule for
proving a $\wpcw$ using a $\wpw$ when no durable resources are required for a
sub-part of the proof. This is commonly used, so Perennial has a \cc{wpc_frame}
tactic to apply the theorem. The tactic takes a list of hypotheses and applies
\ruleref{wpc-frame}, leaving the user to prove that the listed assumptions
together imply the crash condition, with the remaining hypotheses available to
prove a $\wpw$ for the original postcondition. In addition, the hypotheses used
for the crash condition are restored after proving the $\wpw$. This support
integrates with the caching support, automatically using a cached proof to prove
the crash condition if possible.

% \paragraph{Modalities:} We extend existing tactics for working with modalities to support $\cfupd$.
% For example, the user can apply the \ruleref{cfupd-frame} rule
% to eliminate $\cfupd P$ while proving a $\wpcw$, with the same tactic they would use to
% eliminate $\pvs Q$  while proving a $\wpw$.
% Of
% course these are just helpers for applying the relevant theorems, but they
% contribute greatly to making it feel like the user is proving in a specialized
% tool for Perennial even though the framework is implemented foundationally within
% Coq.


\section{Limitations}
\label{sec:perennial:limitations}

Perennial has some limitations. The logic can prove \emph{safety} properties
(``nothing bad happens'') but not \emph{liveness} (``something good eventually
happens''), a limitation inherited from Iris and in fact due to deeper
theoretical reasons related to step indexing --- Transfinite Iris is an approach
to solve this issue that discusses the reasons well~\cite{spies:transfinite}.
Due to different technical reasons related to step indexing, it is not currently
possible to put weakest precondition assertions inside crash conditions; such a
feature is useful for reasoning about procedures stored on disk, something that
shows up in practice with logical logging~\cite{mohan:aries}.

The current version of Perennial does not support \emph{recovery helping} from
the original version~\cite{chajed:perennial} (although it has many new
features), a reasoning principle where in a simulation proof recovery logically
completes an operation from before the crash. As described here, operations must
be simulated during the crash. It also does not implement Iris's support for
prophecy variables~\cite{jung:prophecy}, though we believe this is doable --- an
interesting extension to work out would be prophecies that predict a crash, or
prophesize actions during recovery.

\section{Take-aways beyond formal verification}

\paragraph{Crash-aware locks.} One of the benefits of concurrent separation
logic that is useful even without verification is that it crystallizes the idea
of a ``lock invariant'' as a way to think about the \emph{purpose} of locks.
Rust's \cc{sync::Mutex} API implements a basic version of the lock invariant
idea by allowing the programmer to at least express what data is protected by
the lock, but the idea of a separation logic invariant is more general (for
example, locks can protect just a part of a data structure). Perennial's
crash-aware lock specification formalizes an idea that is not well-known, of how
locking lock invariants interact with durable state: such a locks should have
both the usual lock invariant and a weaker crash invariant.

\paragraph{Invariants for crash reasoning.} Perennial has a rule where
allocating an invariant gives ``credit'' for it on crash. This idea formalizes
an intuition about how ownership flows between threads and on crash. A system
called RECIPE has an insight along these lines, observing that a lock-free data
structure should work on persistent memory~\cite{lee:recipe}. Intuitively this
is true because the code already maintains an invariant at all intermediate
steps due to other threads, and thus the same invariant should hold of the
persistent memory on crash, but some more conditions are required for the
intuition to hold up. RECIPE has no proof of correctness, and a mechanized proof
would be challenging, but the ideas in this thesis could be used to think about
its correctness and the required conditions with greater precision.
