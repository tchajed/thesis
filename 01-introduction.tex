One crucial service in an operating system is its \emph{file system}, the piece
of software that takes a disk (which simply stores a long sequence of bytes) and
makes it more useful by giving applications the abstraction of files and directories. The file
system is important because applications rely on it to store data, and to
persist that data even if the computer shuts down (perhaps unexpectedly, say due
to a power failure) and reboots. Due to their importance to nearly all applications, many
file systems have internally complicated implementations, with optimizations and
concurrency for high performance. However, due to this complexity, file systems
sometimes have bugs, especially subtle bugs that only manifest if the system
crashes at an inopportune time or if concurrent operations interleave in just
the wrong way. These bugs can result in the file system misbehaving, for example
losing some or all files when the file system's internals are corrupted.

A file system's developers generally use testing to identify and fix bugs in the
implementation. Unfortunately, testing is fundamentally difficult due to the
combination of a concurrent implementation and the requirement that the file
system be correct even when the computer crashes. Both of these facets of a file
system mean even a handful of operations can have a large number of possible
executions, making it difficult to scale testing and have high confidence in the
implementation.

This thesis takes an alternate approach to testing to obtain a correct file
system: we formally verify a new file-system implementation against a high-level
specification of the file system's behavior. The high-level approach of this
thesis is to \emph{implement} a file system, define a mathematical
\emph{specification} of what the file system is supposed to do, and then
\emph{prove} that the implementation always meets the specification regardless
of how the program executes and even under crashes. To increase confidence in
the proof, we carry it out with a computer; such a machine-checked proof
formalizes the specification and proof as code that are automatically checked
for correctness. The main contributions of this thesis lie in the proof, whereas
we use standard techniques in the implementation and give a straightforward
specification.

More concretely the verified
artifact from this thesis is \textbf{DaisyNFS}, which implements the Network
File System (NFS) protocol on top of a disk with block-based access. NFS is a
widely-used protocol for exporting a file system across a network. We chose to
implement NFS since the behavior of an NFS server is relatively well-specified
by the RFC 1813 standard, compared to the requirements on a Linux file system,
for example. The specification for our file system stipulates that each
operation is implemented atomically (with respect to both other threads and on
crash), and behaves according to a formal model of the NFS specification as laid
out in prose in RFC 1813.

\todo{How to add some structure to intro while still giving a short, coherent
overall view? Or should we not bother and leave that short overview to the
abstract or end of the intro?}

DaisyNFS is implemented in two main layers, a division which is important to
situate our contributions. While there are many intermediate abstractions, the
most important is an intermediate abstraction that supports starting and
committing transactions that consist of a sequence of reads and writes which
appear to execute atomically, including if the system crashes. The lower level
of the system is a \emph{transaction system} that implements this abstraction on
top of a raw disk. Next, the \emph{file-system layer} implements each NFS
operation as a single transaction, using the transaction system to automatically
make each file-system operation atomic for
concurrency and crashes. Transactions greatly simplify making a file
system correct, since they handle concurrency and crash safety so the
file system can focus on correctly implementing its data structures and
algorithms.

The structural division is important because a core contribution of the thesis
is an approach to give proofs for the two layers with different techniques.
The transaction
system exposes a simple API but its internals involve lots of
concurrency and crash-safety reasoning. Performance is important since
this layer limits the performance and concurrency of the file system.
This layer's verification uses specialized infrastructure we developed
and describe in the thesis: \textbf{Perennial} is a new program logic
for reasoning about the combination of concurrency and crash safety, and
\textbf{Goose} is a tool that translates the Go implementation of the
system to a model that we can apply Perennial to.

For the file-system implementation and proof, we use the Dafny
verification language. The file-system operations interact with the
transaction system to store and retrieve data. To run the system, we
compile the Dafny code to Go, which imports and calls into the
transaction system as a library. Dafny only supports sequential
reasoning, which is sufficient at this layer because the transaction
system guarantees that the Dafny code appears to run sequentially. In contrast
with Perennial, Dafny is an automated reasoning framework, resulting in simpler
and more productive proofs in this layer.

Using Perennial for the transaction sytem and Dafny for the file-system
operations lets us use the sharpest tool for each part of the proof. However, it
introduces a challenge in the proof: it is not obvious how to connect a verified
transaction system with the Dafny proof. This thesis develops an appropriate
specification for the transaction system and then uses an on-paper proof to
argue that the entire system is correct. The on-paper proof stitches together
the Dafny and Coq proofs, so that most of the proof is machine-checked while the
overall proof spans two very different proof systems. The overall system's
correctness then depends on some manual audits over the Dafny code to check that
the reasoning steps in the on-paper proof are sound.

The contribution of this thesis lies in how we prove DaisyNFS correct; the
system implementation itself uses standard techniques to achieve good
performance. We developed new verification techniques to reason about crash
safety and concurrency, to reason about code written in an efficient language,
and to take advantage of the transaction system for simpler proofs in the
file-system layer. In particular this thesis makes the following contributions:
\begin{itemize}
  \item Perennial 2.0 is a verification framework for reasoning about crash
  safety and concurrency. We extend concurrent separation logic with new
  techniques to bring its power to make proofs modular to programs with crash
  safety in their specification.
  \item Goose is a system for reasoning about Go using Perennial. We needed a
  new and flexible system to support our custom verification infrastructure, and
  the use of Go improved productivity for implementation and especially
  performance optimization.
  \item A verified transaction system gives code the
  illusion of atomic transactions, with respect to both crashes and other
  threads. The proof uses Perennial to establish a proof about any client of the
  transaction system.
  \item DaisyNFS is a file system that implements each file system operation as
  a transaction, using the verified transaction system to safely run
  concurrently. A key contribution is an on-paper soundness proof that shows how
  the Dafny and Perennial proofs can be composed.
\end{itemize}

DaisyNFS implements the NFSv3 protocol, a standard protocol for exposing
a file system over the network. We use DaisyNFS by mounting it with the
Linux NFS client and then interacting with the mounted file system using
the standard system-call API. A performance evaluation of DaisyNFS shows
that it is competitive with the Linux kernel NFS server exporting an
ext4 file system.

\section{State of the art}

crash testing, fuzzing, symbolic execution

forward reference to related work for state-of-the-art verification techniques

\section{Approach}

What does it mean to give a machine checked, formal proof of a system? At a high
level, program proofs always have three components: an implementation, a
specification, and a proof. When doing machine-checked proofs, all three are
physically represented as code in a verification system. The verification system
checks the proof against the implementation and specification, ensuring that the
proof is complete.

This thesis integrates interactive, foundational proofs using custom
infrastructure (in Coq) as well as automated verification using a
verification-aware programming language (Dafny). These are both machine-checked,
formal proofs, but the interaction models of the two systems are different
enough that we explain them separately.

In Coq, the core feature is proofs based on dependent type theory, which is
expressive enough to represent essentially any math. A first step when using Coq
is to connect the code to a model in Coq of its semantics. The model encodes any
assumptions we make about how the program behaves; the rest of the proof will be
about these behaviors. The semantics is typically structured as a transition
system, where an execution is a sequence of states and as the program executes,
along with some notion of observable behavior, like external I/O or return
values.

Armed with a model of the physical, executable code, we can now carry out
reasoning about that model. I will refer to this model as simply a program, but
when thinking about the meaning of verification the distinction is useful. While
there are many ways to reason about programs in a system as general as Coq, a
common structure is to use a \emph{program logic}. The program logic is a way of
expressing and proving statements about the program. \todo{finish this
explanation up}

\section{How to read this thesis}

\Cref{sec:daisy-nfs} describes the programming interface, specification,
implementation, and proof for the overall file system. When describing the
implementation and proof, a significant component is the transaction system,
which we only give an interface and specification for. The details of that proof
are in \cref{sec:txn}. The transaction system's proof required two
contributions that are part of this thesis. \Cref{sec:goose} is optional
reading on how the proof connects to the code. \Cref{sec:perennial} gives
background needed to understand how the proof itself works.
