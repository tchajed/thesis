Storage systems are important because applications rely on them to store
data, and to persist that data even if the computer shuts down (perhaps
unexpectedly, say due to a power failure) and reboots. However, file
systems have internally complicated implementations, with optimizations
and concurrency for high performance, which lead to bugs. Sometimes
these bugs result in the file system misbehaving, either by incorrectly
storing data or failing to retrieve it.

The approach taken in this thesis is to implement a file system with
good performance and concurrency, then formally verify that it always
meets its specification. More concretely the verified artifact from this
thesis is \textbf{DaisyNFS}, which implements the Network File System
(NFS) protocol on top of a disk. The specification for this file system
stipulates that each operation is implemented atomically (with respect
to both other threads and on crash), and behaves according to a formal
model of the NFS specification as laid out in prose in RFC 1813.

A file system is a large program, so we divided both the implementation
and proof of DaisyNFS into two layers. First, a \textbf{transaction
system} implements support for transactions that consist of a sequence
of reads and writes which appear to execute atomically, including if the
system crashes. Next, the \textbf{file-system layer} uses the
transaction system by implementing each NFS operation as a single
transaction, automatically making a file-system operation atomic for
concurrency and crashes. Transactions greatly simplify making a file
system correct, since they handle concurrency and crash safety so the
file system can focus on correctly implementing its data structures and
algorithms.

The proofs for the two layers are handled differently. The transaction
system exposes a simple API but its internals involve lots of
concurrency and crash-safety reasoning. Performance is important since
this layer limits the performance and concurrency of the file system.
This layer's verification uses specialized infrastructure we developed
and describe in the thesis: \textbf{Perennial} is a new program logic
for reasoning about the combination of concurrency and crash safety, and
\textbf{Goose} is a tool that translates the Go implementation of the
system to a model that we can apply Perennial to.

For the file-system implementation and proof, we use the Dafny
verification language. The file-system operations interact with the
transaction system to store and retrieve data. To run the system, we
compile the Dafny code to Go, which imports and calls into the
transaction system as a library. Dafny only supports sequential
reasoning, which is sufficient at this layer because the transaction
system guarantees that the Dafny code appears to run sequentially.

It is not obvious how to connect a verified transaction system with the
Dafny proof. This thesis develops an appropriate specification for the
transaction system and then uses an on-paper proof to argue that the
entire system is correct. The overall system's correctness then depends
on some manual audits over the Dafny code to check that the reasoning
steps are sound.

DaisyNFS implements the NFSv3 protocol, a standard protocol for exposing
a file system over the network. We use DaisyNFS by mounting it with the
Linux NFS client and then interacting with the mounted file system using
the standard system-call API. A performance evaluation of DaisyNFS shows
that it is competitive with the Linux kernel NFS server exporting an
ext4 file system.
