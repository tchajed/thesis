One crucial service in an operating system is its \emph{file system}, the piece
of software that takes a disk (which simply stores a long sequence of bytes) and
makes it more useful by giving applications the abstraction of files and directories. The file
system is important because applications rely on it to store data, and to
persist that data even if the computer shuts down (perhaps unexpectedly, say due
to a power failure) and reboots. Due to their importance to nearly all applications, many
file systems have internally complicated implementations, with optimizations and
concurrency for high performance. However, due to this complexity, file systems
sometimes have bugs, especially subtle bugs that only manifest if the system
crashes at an inopportune time or if concurrent operations interleave in just
the wrong way. These bugs can result in the file system misbehaving, for example
losing some or all files when the file system's internals are corrupted.

A file system's developers generally use testing to identify and fix bugs in the
implementation. Unfortunately, testing is fundamentally difficult due to the
combination of a concurrent implementation and the requirement that the file
system be correct even when the computer crashes. Both of these facets of a file
system mean even a handful of operations can have a large number of possible
executions, making it difficult to test all of them and gain high confidence in the
implementation.

This thesis takes an alternate approach to testing to obtain a correct file
system: we formally verify a new file-system implementation against a high-level
specification of the file system's behavior. The high-level approach of this
thesis is to \emph{implement} a file system, define a mathematical
\emph{specification} of what the file system is supposed to do, and then
\emph{prove} that the implementation always meets the specification regardless
of how the program executes and even under crashes. To increase confidence in
the proof, we carry it out with a computer; such a machine-checked proof
formalizes the proof as code that is automatically checked
for correctness against the specification and implementation.

The thesis's contributions address two main challenges in verifying a file system. The
first challenge is reasoning about the combination of crash safety and
concurrency at all, for which we build a framework that applies to efficient
code (written in Go) and can handle the sorts of reasoning in the file system.
The second challenge is the complexity of the implementation, which we address
with a novel, verification-friendly file-system design that allows us to reason
about each operation as if it were atomic, resulting in much simpler sequential proofs.

More concretely the verified
artifact from this thesis is \textbf{DaisyNFS}, which implements the Network
File System (NFS) protocol on top of a disk with block-based access. NFS is a
widely-used protocol for exporting a file system across a network. We chose to
implement NFS since the behavior of an NFS server is relatively well-specified
by the RFC 1813 standard, compared to the requirements on a Linux file system,
for example. The specification for our file system stipulates that each
operation is implemented atomically (with respect to both other threads and on
crash), and behaves according to a formal model of the NFS specification as laid
out in prose in RFC 1813.

The novelty in DaisyNFS's design is a split into two main layers, first a
lower-level \emph{transaction system} that gives the illusion of atomic
operations with reads and writes and then a \emph{file-system layer} that
implements each file-system operation as a single transaction to automatically
make them all atomic. While many file systems (including the most common Linux
file system, ext4) use transactions, DaisyNFS uses transactions in a principled
way so that each operation is automatically atomic, whereas many transaction
systems still have state outside the transaction system. Combining the
transaction system with other state is subtle, since it means the developer
cannot rely entirely on the transaction system for atomicity but must instead
coordinate threads, for example with additional locking.

The design enables us to give proofs for the two layers with different
techniques, notably using \emph{sequential proofs} for the file-system layer and
isolating the difficult concurrency and crash-safety reasoning to the
transaction system's implementation.
Performance in the transaction system is important since
this layer limits the performance and concurrency of the file system.
This layer's verification uses specialized infrastructure we developed
and describe in the thesis: \textbf{Perennial} is a new program logic
for reasoning about the combination of concurrency and crash safety, and
\textbf{Goose} is a tool that translates the Go implementation of the
system to a model that we can apply Perennial to.

For the file-system implementation and proof, we use the Dafny
verification language. The file-system operations interact with the
transaction system to store and retrieve data. To run the system, we
compile the Dafny code to Go, which imports and calls into the
transaction system as a library. Dafny only supports sequential
reasoning, which is sufficient at this layer because the transaction
system guarantees that the Dafny code appears to run sequentially. In contrast
with Perennial, Dafny is an automated reasoning framework, resulting in simpler
and more productive proofs in this layer.

Using Perennial for the transaction sytem and Dafny for the file-system
operations lets us use the sharpest tool for each part of the proof. However, it
introduces a challenge in the proof: it is not obvious how to connect a verified
transaction system with the Dafny proof. This thesis develops an appropriate
specification for the transaction system and then links the Dafny proof with the
transaction system's specification, to
argue that the linked code is correct. This linking theorem has a proof which stitches together
the Dafny and Coq proofs, so that all of the code is verified with a
machine-checked proof but we verify the system using two very different proof
systems.

This thesis aims for the goal of making verifying a concurrent file system
feasible and scalable in terms of features and performance. In particular this thesis
makes the following contributions:
\begin{itemize}
  \item Perennial 2.0: a verification framework for reasoning about crash
  safety and concurrency. A new primitive for transferring ownership of durable
  state helped us reason about complex interactions in the transaction system's
  implementation.
  \item Goose: a system for reasoning about Go using Perennial. We needed a
  new and flexible system to support our custom verification infrastructure, and
  Go was a good platform for writing code and improving its performance.
  \item A proof of a transaction system showing that transactions have the
    illusion of sequential, atomic execution, with respect to both crashes and other
    threads. The proof uses Perennial to establish a theorem about any client of the
    transaction system.
  \item DaisyNFS: a file system that takes advantage of the transaction system's
    correctness theorem to reason about each transaction in Dafny, a
    sequential verification language. A key contribution is a linking theorem
    that shows how the Dafny and Perennial proofs can be composed.
\end{itemize}

DaisyNFS implements the NFSv3 protocol, a standard protocol for exposing
a file system over the network. We use DaisyNFS by mounting it with the
Linux NFS client and then interacting with the mounted file system using
the standard system-call API.\@ A performance evaluation of DaisyNFS shows
that it is competitive with the Linux kernel NFS server exporting an
ext4 file system.

\section{State of the art}

Production file systems are generally validated by testing. While testing is
indispensible for development, the nature of a file system makes it difficult to
catch all bugs with only testing. The fundamental difficulty is a high degree of
non-determinism from two sources: crashes in the middle of execution, and
concurrency in the implementation that is needed for good performance.

The importance of file-system correctness has been recognized by the academic
community, thus there are many approaches for increasing confidence with
improved testing. One line of work has explored systematically testing crashes
at intermediate points~\cite{mohan:crashmonkey,pillai:appcrash}. Another line of
work has focused on fuzz testing as a way to induce crash-safety
bugs~\cite{xu:janus,kim:hydra}. These approaches have been successful for
finding bugs, including crash-safety bugs, but they only test sequential
executions, missing any bugs related to concurrency or crashes while multiple
threads are running. Furthermore, unlike formal verification, testing cannot
cover all executions of a program, even without crashes and concurrency,
potentially missing bugs.

The research community has also recognized the value of formal verification for
reasoning about a file-system implementation. The closest related work is
another verified, concurrent file system, Flashix~\cite{bodenmuller:concurrent-flashix}. The use case for Flashix is
more specific than DaisyNFS, as it is designed to run directly on flash chips
rather than more standard SSDs.
Its concurrency is primarily between regular operations and garbage collection,
and read-only concurrency. In contrast, DaisyNFS is a general-purpose file
system which efficiently implements even write-write concurrency, on top of a
standard SSD that is more sophisticated than raw flash chips. In addition to different
goals for the overall artifacts, DaisyNFS has a different file system design
where crash safety is based on a general-purpose transaction system, whereas
Flashix accounts for flash failures and thus crashes in every write to storage.

There are other verified file systems, especially the sequential file systems
FSCQ~\cite{chen:fscq} and Yggdrasil~\cite{sigurbjarnarson:yggdrasil} and an
in-memory but concurrent file system AtomFS~\cite{zou:atomfs}. These systems use
verification techniques that do not support both crashes and concurrency, so we
cannot extend them in a straightforward way to achieve the realism goals in this
thesis.

\section{Approach}

What does it mean to give a machine checked, formal proof of a system? At a high
level, program proofs always have three components: an implementation, a
specification, and a proof. When doing machine-checked proofs, all three are
physically represented as code in a verification system. The verification system
checks the proof against the implementation and specification, ensuring that the
proof is complete.

This thesis integrates interactive, foundational proofs using custom
infrastructure (in Coq) as well as automated verification using a
verification-aware programming language (Dafny). These are both machine-checked,
formal proofs, but the interaction models of the two systems are different
enough that we explain them separately.

In Coq, the core feature is proofs based on dependent type theory, which is
expressive enough to represent essentially any math. A first step when using Coq
is to connect the code to a model in Coq. The model and its semantics encodes any
assumptions we make about how the program behaves; the rest of the proof will be
about these behaviors. The semantics is typically structured as a transition
system, where an execution is a sequence of states and as the program executes,
along with some notion of observable behavior, like external I/O or return
values. From now on I'll generally refer to the model as simply being the original
program being verified,
but we'll make a distinction when discussing what it means to have verified
the system.

Now that we have a program in Coq, we can reason about it. The goal of
verification is to prove that the program meets its specification, and
specifications tend to give the allowed behaviors of the program. These
specifications forbid universally incorrect behavior, like reading an
out-of-bounds address in an array, but more precisely specify what the program's
allowed return values are. In principle it might be possible to prove a theorem
about all the behaviors of a program directly, but such a proof would be
monstrous to write, hard to split up, and challenging to maintain as the code
evolved.

A common structure to tame the complexity of reasoning about a program is to use
a \emph{program logic}. The program logic is a way of expressing and proving
statements about the program, including properties about individual functions
and language constructs like \cc{if}-statements and \cc{while}-loops. The proof
in a program logic will often mirror the structure of the code, since each
function has its own specification and groups of related functions have related
specs.

Program logics for concurrency are still an active area of research; only
recently have they reached the maturity to give completely mechanized proofs of
moderate-sized programs. There are few logics that also can reason about crash
safety. Our approach in this thesis is to build a new program logic with all the
concurrency-reasoning features of a modern program logic, plus new features for
reasoning about crashes. What makes this feasible is Iris, a modular framework
for concurrency. Iris includes a concurrent program logic which we are able to
extend with crash-safety reasoning while preserving the concurrency reasoning
features, without reimplementing them from scratch.

Using our new program logic, we verified a transaction system.
What we prove about the transaction system is that any program that uses
transactions really has transactional behavior: its execution is equivalent to a
version of the program where the transactions run atomically. The complete
specification includes some important details, but this intuition still holds.

Next, we use the transaction system in Dafny as the basis for a file system.
The transaction system is implemented as an ordinary Go package, with methods to
start a transaction, execute reads and writes within it, and finally commit or
abort it. Using Dafny's \cc{extern} feature we make these methods available with
assumed specifications that match the specifications proven in Coq.

Dafny verification works quite differently from Coq. Dafny is a programming
language with verification features; contrast this with Coq, which supports
general math that can \emph{model} programs. A Dafny method can be annotated
with a specification. The Dafny checker converts a method and its specification
to a logical formula, which is true if and only if the specification holds for
the method. It then queries a \emph{solver} to determine if the formula is true.
Contrast all of this with Coq, where the user manually develops the program
logic and connects the rules of this logic. The process cannot be perfectly
automated because it is impossible to answer whether a general logical formula
is true or not, but the user can insert annotations to help out the solver, and
generally fewer annotations are needed than lines of proof for Coq. The main
downside to this approach is that it fixes a sequential programming language, so
unlike Coq it isn't possible to reason about concurrency and crashes.

One of the contributions of this thesis is that the file-system design isolates
concurrency and crash safety into the transaction system. This structure leaves
the rest of the implementation only to \emph{sequentially} implement the
file-system logic and data structures. Because this is sequential, crash-free
execution, we use Dafny to implement and verify each operation, then run this
code wrapped in a transaction. Later we will have more to say about the file
system's proof, but for now suffice it to say that we show it correctly
implements the NFS protocol.

Since we use two different formal systems, we do not set up and prove a single,
unified theorem about the whole file system, which is indeed not even
implemented in one language. Instead, the file system is compiled by linking the
Dafny code, after compilation to Go, with the transaction system. Similar to
this code linking step, we prove a linking theorem, showing that the proof from
Coq and proof from Dafny compose properly. The proofs from both systems are
checked automatically so we call them mechanized, but this final linking step is
outside of both systems so we must give it on paper. However, we take care to
make the transaction system's specification strong enough that the two theorems
compose easily. Later we discuss what parts of the system are \emph{trusted} to
guarantee that the whole system is correct; notably, we learn from Coq and Dafny
that the transaction system and file-system code are independently correct, and
the trust only lies in the interface between them.

\section{Challenges}

What is hard about verifying a concurrent file system? There are two broad
challenges to solve: how to reason about crash safety and concurrency at all,
and how to manage the complexity of the system.

The first challenge is to develop techniques for reasoning about crashes and
concurrency. To give a flavor of the kinds of interactions these two can bring
about, we give a simple example from the transaction system. Writes to the
transaction system are buffered in memory before being flushed to disk, with the
goal of coalescing concurrent transactions in a single flush. Perhaps
surprisingly, even if all writes are immediately followed by a flush, the system
does not appear to have atomically durable writes. If one thread writes, another
thread can concurrently read the buffered value before it is flushed, but a
crash immediately afterward will lose the write and revert to the old value.

Once we have techniques that can in principle express crash and recovery
behavior, we need some structure to make reasoning about the real
transaction-system and file-system implementations feasible. Within the
transaction system this takes the form of modular proofs, matching the structure
of the code. The overall file system relies heavily on the transaction system to
lower the proof burden, isolating the tricky crash safety and concurrency
reasoning to a generic system and then verifying the file-system operations with
sequential proofs. Transactions are not fundamental to correctly implementing a
file system: notably Featherstitch~\cite{frost:featherstitch} offers a
principled way to think about correctness without transactions, while
B$e$trFS~\cite{jannen:betrfs} and the production btrfs file systems use
copy-on-write as the core atomicity primitive. However, our rigorous use of
transactions is amenable to a simpler proof since we can show that transactions
are atomic once, independent of the actual code written on top. Furthermore this
design makes the file system easy to grow over time since further development
just involves writing more sequential code and proofs.

While in the introduction we have given nearly equal weight to the transaction
system and the file-system implementation, much more of the thesis is about
verifying the transaction system. The transaction system is more complex than
the file-system design, largely by design: the transaction system uses novel
verification techniques whereas the file-system
design explicitly aims to make the higher-level proofs simple and sequential.
Furthermore the transaction system involved new infrastructure that is part of
this thesis, both the program logic and Goose, our tool for connecting the
implementation to the proof.
