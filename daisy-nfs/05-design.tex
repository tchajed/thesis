\section{Verifying the Dafny implementation}%
\label{sec:daisy:design}

\Cref{sec:daisy:proof-dafny}
explains how DaisyNFS connects sequential verification in Dafny to concurrency
and crash safety in GoTxn. This instead section focuses on the sequential
verification and file-system design themselves.

% The proof is given by annotating the code with proof steps, which include
% updates to ghost state, assertions to assist the automated verification, and
% calls to lemmas.

DaisyNFS is implemented and verified in several layers of abstraction, depicted
in \cref{fig:dafny-layers}. Each layer is implemented as a class that wraps the
lower layer as a field, until finally the transaction system is an assumed interface in Dafny.
The \cc{daisy-nfsd} binary implements the NFS wire protocol in
unverified Go code and calls the top-level Dafny class and its verified
methods to handle each operation.

\begin{figure}
\small \centering
\begin{tabular}{ll}
  \toprule
  \textbf{Layer} & \textbf{Functionality} \\
  \midrule
  daisy-nfsd & NFS wire protocol. \\
  dir & Directories and top-level NFS API. \\
  typed & Inode allocation. \\
  byte & Implement byte-level operations using blocks. \\
  block & Gather blocks for each file into a single sequence. \\
  indirect & Indirect blocks organized in a tree. \\
  inode & In-memory, high-level inodes; block allocation. \\
  txn & Assumed interface to external transaction system. \\
  \bottomrule
\end{tabular}
\caption{Layers in the Dafny implementation and proof of the file-system
operations.}
\label{fig:dafny-layers}
\end{figure}

The layers of the file system
can be organized into groups that implement three difficult pieces of
functionality: organizing data blocks into metadata and data (the
indirect and block layers), translating byte-level operations into
block operations (the byte and typed layers), and implementing
directories as special files that the file system itself reads and
writes (the dir layer). The modularity was essential to complete the proof in
manageable chunks (to avoid overwhelming the developer and prover), and it would
have been natural even without verification.

\subsection{Implementing the file system using transactions}

The design of DaisyNFS is broadly similar to the file system in xv6~\cite{xv6},
as well as Yggdrasil~\cite{sigurbjarnarson:yggdrasil}, a verified sequential
file system. We also adopt the recursive strategy for implementing and
verifying indirect blocks from DFSCQ~\cite{akonradi-meng}; recursion simplifies
the implementation of triply-indirect blocks, which are needed to reach a
reasonable maximum file size of 512GB.\@ Unlike most file systems, DaisyNFS is designed
to fit every operation into a transaction in order to support our goal of
sequential reasoning. This is a non-standard design and we encountered some
unique challenges in doing so. In this section we highlight difficulties in
fitting two features into transactions: rename and freeing space from deleted
files.

\subsubsection{Rename}
\label{sec:dafny:rename}

The NFS RENAME operation is similar to the \cc{rename} system call: it moves a
source file or directory to a destination location. What makes it tricky is that
it involves more than one inode and hence introduces the possibility for
deadlock.
% , which we would like to avoid even if the theorems do not forbid it.
We
use the standard strategy of enforcing a global ordering where inodes are always
locked in numerical order (smaller inode numbers first); this avoids a deadlock
where a cycle of threads is waiting on each other.

In a rename operation, the source and destination are each specified by a
combination of the parent directory inode and name within that directory. Rename
has an additional functionality of overwriting the destination if the source and
destination are files, or if both are directories and the destination is empty.
It is this latter check that makes deadlock avoidance difficult: it is necessary
to lock the source and destination directories first to lookup the source and
destination names, but those might be files that are earlier in the inode lock
order. We address this in the code by returning an error from the Dafny
transaction before the lock order would be violated. The error comes with the
set of inodes that should have been acquired.  The rename is then re-run with
this set of inodes as a lock hint which are all acquired in the correct
order at the beginning of the operation. The inodes to lock are only a hint and must be compared
against the current source and destination in case those inodes have changed in
the mean time. The rename operation runs in a loop until the lock hint succeeds;
the loop is potentially unbounded, but each iteration can only fail due to
concurrent renames that involve the same inodes.

At this point it is worth discussing the performance considerations that lead to
handling lock ordering in the file
system, rather than generically in GoTxn. The transaction system could
avoid deadlocks by either enforcing a global order over addresses or by
timing-out operations. Enforcing a global order is inefficient for the file
system; data blocks will never cause deadlock because the file system only
accesses a block after locking the (unique) inode that owns it. Timing-out
operations would lead to slow and spurious transaction failures that could more
rapidly be avoided in the higher-level code, hence we do not attempt to detect
deadlock dynamically.


\subsubsection{Freeing space}
\label{sec:dafny:freeing}

Freeing space becomes surprisingly tricky with large files. The problem is that
a large-enough file may reference too many blocks to be
freed in a single transaction. Transactions are bounded by the size of the
on-disk log (which can hold 511 blocks), whereas freeing a file requires
writing 0s to the block allocator for all of its formerly used blocks to mark them as free.
DaisyNFS handles this issue by splitting file removal and space reclamation
into separate transactions. The latter is implemented with an operation
\cc{ZeroFreeSpace(ino)} which frees and zeros the unused space in an inode that
we prove has no effect on the logical file-system state. Because this operation is a
logical no-op, it is safe to call it at any time.

The implementation is careful to call \cc{ZeroFreeSpace} after any operation
that leaves unused blocks, in particular \cc{REMOVE}, which deletes a file, and
\cc{SETATTR}, which can shrink a file by reducing its size. Since
\cc{ZeroFreeSpace} doesn't affect the user-visible data, it can return early to
avoid overflowing a transaction. The unverified code that manages freeing space
runs the operation in a loop until it covers all of the unused space in an
inode.

There is one case where freeing blocks is important for correctness and not just
to reclaim space. Growing a file is supposed to logically fill the new space
with zeros. If the file had old data in that space, it might not be zero but
instead contain previously written and deleted data, which both violates the specification and
is a potential security risk. The way we handle this with background freeing is
with validation: when the \cc{SETATTR} operation grows a file, it checks if the
free space is already zero first, and if not fails with a special error code. The
unverified code uses this error as a signal to immediately call
\cc{ZeroFreeSpace} and try the operation again. The same support also handles
holes created by writing past the end of a file, which are similarly supposed to
be zero.

The freeing implementation is an interesting example of using validation in
verification. The specification for much of the freeing code is loose, allowing
any data to be written to the free space. Only the code that checks if the
zeroing is done needs to be verified against a strong specification. The rest of
the code does still needs to be correct for this check to succeed, but we
aren't required to prove it.

\subsection{Verifying the indirect block implementation}%
\label{sec:dafny:indirect}

DaisyNFS supports large files using indirect blocks. A file's inode has a fixed
number of addresses for block addresses, some of which are used as indirect
blocks that hold another layer of addresses rather than direct blocks that have
file data. A single level of indirection is insufficient for a practical file
system, so DaisyNFS implements support for arbitrarily indirect blocks, and in
practice the file system uses up to triply-indirect blocks to support files up
to 512GB.\@

The indirect and block layers together implement an abstraction of a file as a
sequence of blocks, hiding the fact that some of the blocks are used as
metadata. These sequences are always of the maximum size, and only the next
layer reasons about file sizes. To efficiently represent files of the maximum
size, the code uses a convention that a zero block address is treated implicitly
as encoding a zero block, including for indirect blocks, an idea borrowed from
DFSCQ~\cite{akonradi-meng}. Indirect blocks are implemented recursively, where a
$k$-indirect block is always treated as containing 512 pointers to
$(k-1)$-indirect blocks, and a 0-indirect block contains file data. Zeros are
also treated recursively so that a single zero in an inode for the root of a
triply-indirect block efficiently stores a whole tree corresponding to many
gigabytes of zeros.

An inode has space for twelve 64-bit block addresses, after accounting for space
used by its attributes and type information. In principle all of these addresses
could be uniformly used as triply-indirect blocks. However, this would create a
lot of indirection for small files and lower performance for the common case.
Thus instead of organizing them in this way, the code uses a whole range, with
mostly direct blocks and a handful of indirect blocks. To keep the code general,
the indirection level of each of the twelve blocks is given in a global
indirect-block configuration, and most of the code is generic over the configuration. We currently
configure DaisyNFS with 8 direct blocks, two 1-indirect blocks, a 2-indirect
block, and a 3-indirect block. Just the direct and indirect blocks can address 4
MB fairly efficiently, but the triply-indirect block allows files to be large.

Indirect blocks pose a challenge for verification due to the classic problem of
\emph{aliasing}. The proof must show that modifying a data block or indirect
block has no effect on other files. In the DFSCQ proof, the invariant
captures the non-aliasing between files using separation logic, which makes
disjointness easy to express. In Dafny we have no such logical
technique, so we instead use a standard SMT-friendly trick for the invariant: in
addition to the physical mapping that tracks how to dereference a block address,
the indirect layer proof tracks a ghost \emph{reverse} mapping that tracks where
each in-use block number is stored. The invariant states that the forward and reverse
mappings are inverses of each other, which implies that modifying an address
only affects its owner and nothing else.

To encode the reverse mapping, the code uses a ``position'' datatype \cc{Pos} to
represent the location of a block within an inode. With indirect blocks, the
metadata blocks themselves also need to be considered locations, since the
invariant must also rule out metadata aliasing with data or other metadata. A
\cc{Pos} encodes an inode, an indirection level, and an offset for the blocks at
that indirection level to uniquely identify where a block is used. If we imagine
that an inode's block pointers are organized in a tree, the roots are stored
directly in the inode while the leaves are direct blocks. An indirection level
which is higher than the leaf level describes a metadata block.

The indirect block proof is split into the indirect and blocks layers. In the
indirect layer, the abstract state maps a \cc{Pos} to a block, and separately
tracks the size and attributes of each inode. The invariant also expresses that
the block allocator's used blocks have an associated \cc{Pos} and that the free
ones do not. The interface for the indirect layer exposes reads and writes for
positions, regardless of whether they are metadata or data blocks. The block
layer above instead exposes a map from inode number to a flat sequence of blocks by
mapping each leaf position to its linear index within the inode. Separating
these two made it easier to work on the indirect layer while exposing a much
more natural abstraction of a file as a sequence of blocks for the rest of the
file-system implementation.


% \subsection{Random notes on development process}
%
% \begin{itemize}
%   \item Used inefficient functional Dafny code at first, then slowly migrated to
%         in-memory data structures and improved performance.
%   \item Hard to debug and fix timeouts. Profiling verification performance is
%         hard.
%   \item Profiling Go code is great as usual. The generated code looks strange,
%         but I think after code generation it's pretty ordinary (the weird things
%         are mostly bad variable names, lot of unused assignments, and anonymous
%         functions that are immediately called).
%   \item Used some unit tests, but very few and only at the top level. Mainly
%         debugged Go compilation issues and cases where errors were
%         unintentionally being returned.
%   \item Trusted code isn't easy, had bugs in it before testing it thoroughly.
%         Also violated preconditions in top-level specs, triggering memory-safety
%         bugs.
% \end{itemize}
