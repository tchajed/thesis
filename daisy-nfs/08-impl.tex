\section{Implementation}

%% daisy-nfsd spec:
% gsed -n '/public/,/^\s*{/p' src/fs/dir_fs.dfy | wc -l
%   261
% cloc --read-lang-def ~/dafny-lang.txt src/fs/nfs.s.dfy
%   306

% getting lines of code and total lines (proof is total - code)
\begin{comment}
begin
  rm -rf src-compiled
  for file in src/*/**.dfy
          set -l path (string sub --start 4 $file)
          set -l dir (dirname $path)
          mkdir -p src-compiled/$dir
          dafny /printMode:NoGhost /dafnyVerify:0 /rprint:src-compiled/$path $file &
      end
  wait
  cloc --read-lang-def ~/dafny-lang.txt src-compiled
  cloc --read-lang-def ~/dafny-lang.txt src
end
\end{comment}

\newcommand{\daisynfsCode}{3479}
\newcommand{\daisynfsTotal}{10137}

\begin{figure}
\small
\begin{tabular}{lrrr}
  \toprule
  & \bf proof & \bf code & \bf spec \\
  \midrule
  GoJournal & \loc{26000} & \loc{1345} & \cite{chajed:gojournal} \\
  Transaction system & \loc{10000}  & \loc{250} & \loc{564} (Thm~\ref{thm:gotxn-transfer}) \\
  File system & \loc{\fpeval{\daisynfsTotal-\daisynfsCode}}  & \loc{\daisynfsCode} & \loc{\fpeval{261+306}} (Thm~\ref{thm:dafny}) \\
  \cc{daisy-nfsd} & \emph{unverified} & \loc{1015} & --- \\
  \bottomrule
\end{tabular}
\caption{Lines of proof, code, and trusted specification. GoJournal is included
  only for comparison; its specification is subsumed by the transaction system's.
  Not shown are 635 lines of assumptions in Dafny about the
  transaction system and Go primitives.}
\label{fig:daisy:loc}
\end{figure}

We implemented DaisyNFS in a combination of Go and Dafny, with proofs in the
Perennial framework (which is a library in the Coq proof assistant, heavily
based on Iris~\cite{jung:iris-1}) and inline in Dafny. The Go side
uses GoJournal, which we extend with a transaction system and
concurrent allocator. The implementation is
publicly available.\footnote{
  %The transaction system is part of
  %\url{github.com/mit-pdos/go-journal} while the Dafny side is at
  %\url{github.com/mit-pdos/daisy-nfsd} (which imports the former repo).
  URLs omitted for double-blind review.
}


The lines of proof, code, and specification for the layers of the system are
summarized in \cref{fig:daisy:loc}. GoJournal is prior work but included for
comparison purposes. The GoTxn correctness proof, \cref{thm:gotxn-transfer}, is
relatively large because code executed in atomically blocks can include many Go
operations modeled by Perennial, and the proof has cases to handle
each operation. However the result of the proof is a relatively concise
specification as a plain Coq statement that doesn't refer to the Perennial
logic.

% wc -l program_proof/txn/{txn_proof,twophase_refinement_proof,twophase_sub_logical_reln_defs,typed_translate_facts,wrapper_proof,wrapper_init_proof}.v goose_lang/ffi/{jrnl_ffi_facts,atomic_refinement}.v
% 9999 lines of Coq

% allocator is 459 lines of proof for 123 lines of code

% spec for refinement proof
% wc -l src/goose_lang/typed_translate.v src/goose_lang/ffi/jrnl_ffi_spec.v
% 564 lines of Coq

The file-system operations are implemented in Dafny, which helped us verify a
relatively complete system without too much tedium. The
proof-to-code ratio (where code is the number of lines extracted by Dafny's
\cc{/printMode:NoGhost} flag) is about $2\times$ for the file system code.
The proof summarizes the implementation well, with about $6\times$ fewer lines
of specification as code (about half that specification is quite verbose and concerns
error codes and attributes). For efficiency, the Dafny code has trusted interfaces to
primitives like byte slices and integer-to-byte encoding. Together these are
written in \loc{635} lines of trusted Dafny code.
% The rest of the system has good leverage from the proofs: there are $8\times$
% as many lines of code as specification.
Finally, to complete the NFS server required around \loc{1000} lines of Go
code, about half of which bridge between the Dafny method signatures and the
actual NFS structs.

% wc -l src/jrnl/*.s.dfy src/machine/*.s.dfy
% 635 trusted lines of code

% fish script:
\begin{comment}
begin
    rm -rf src-compiled
    for file in src/*/**.dfy
        set -l path (string sub --start 4 $file)
        set -l dir (dirname $path)
        mkdir -p src-compiled/$dir
        dafny /printMode:NoGhost /dafnyVerify:0 /rprint:src-compiled/$path $file &
    end
    wait
    cloc --read-lang-def ~/dafny-lang.txt src-compiled
    cloc --read-lang-def ~/dafny-lang.txt src
end
\end{comment}

% cloc (compiled)
% 3227 source lines of code

% cloc (total)
% 9585 source lines total (minus code, 6358 proof lines)

\begin{comment}
begin
    gsed -n '/method[^(]* [A-Z]*(/,/^\s*{/p' src/fs/dir_fs.dfy
    cat src/fs/nfs.s.dfy
end | wc -l
\end{comment}
%
% 382

% additional unverified lines of Go in dafny-nfsd
% wc -l nfsd/{fh,mkfs,mount,ops}.go cmd/daisy-nfsd/main.go
% 1015

Similar to VeriBetrKV~\cite{hance:veribetrkv}, we followed a
discipline of identifying and addressing timeouts in the proof.
% , which
% hamper development by giving the developer slower feedback (because
% proofs fail only after 30s) and less specific feedback (because Dafny
% is unable to pinpoint which assertion or postcondition failed and
% reports a timeout for an entire method).
As a result, the overall
build is fast: compiling the proofs takes only 12 minutes on a slow
machine in continuous integration and 4 minutes on a laptop
using eight CPU cores.

%% Similar to
%% VeriBetrKV~\cite{hance:veribetrkv}, when a proof took a long time,
%% before moving on to another layer, we would hide definitions and add
%% proof annotations to make verification fast enough for interactive
%% development. This discipline was particularly important between
%% layers: making the invariant of a layer opaque makes verification of
%% the layer above predictable and fast, since the automation only sees
%% one abstraction rather than all the lower levels.

% Tej: takes 10m on my laptop with just one core (no VC parallelism)

% VeriBetrKV is 1.8h CPU time and 11min on 32 cloud machines (of unspecified
% number of cores)
