\section{Verifying DaisyNFS using Dafny}
\label{sec:daisy:proof}

A key contribution of DaisyNFS is a proof structure that isolates concurrency and crash
reasoning to the transaction system. The proof works in two steps. First, a
\emph{simulation transfer} theorem extends GoTxn's transaction refinement to show
how transactions enable sequential reasoning for an arbitrary system implemented on
top (\cref{sec:daisy:simulation-transfer}). Second, simulation transfer is
applied to DaisyNFS and its sequential reasoning carried out in Dafny
(\cref{sec:daisy:proof-dafny}).

\subsection{Simulation transfer}%
\label{sec:daisy:simulation-transfer}

The GoTxn specification from \cref{sec:txn:spec} uses transaction refinement to
capture that transactions are atomic in the sense that every interleaved
execution has a corresponding execution where each transaction runs all at once.
This section describes a \emph{simulation transfer} theorem, proven in Coq, that
uses transaction refinement to show how GoTxn enables \emph{sequential
reasoning} for a system implemented with transactions on top.

The idea behind the simulation transfer specification is to express that a system
verified using sequential reasoning for each transaction is also correct when
run concurrently through GoTxn --- intuitively, this follows from the atomicity
provided by transaction refinement.
% , at which point the sequential reasoning
% applies, but we have a more precise and formal proof in Coq.
To make this precise, we define formally what we mean by
``sequential reasoning''. Suppose we have an
implementation of layer $S$ using operations from $T$. The implementation $i$
consists of a function $i(\opv) : \gooselayer{T}$ for each operation $op \in S$. The statement
$\seqrefinement \targ{T, S}(i)$ says that $i$ is a correct sequential
implementation of $S$ using $T$. The effect of each operation is given by a
transition relation $\sigma \overset{\opv}{\leadsto} \sigma'$ that is part of the layer
$S$. To specify correctness under crashes, the
definition of sequential refinement refers to $\crash(\sigma, \sigma')$, which is a
layer-specific crash transition that models, for example, clearing the
contents of memory.

\begin{definition}
  The implementation $i : S \to \gooselayer{T}$ is a \emph{sequential
    refinement}, written \newline
  % latex makes a really overfull line, hence a manual line break
  $\seqrefinement \targ{T, S}(i)$, if there exists an abstraction relation
  $R : \Sigma_{S} \to \Sigma_{T} \to \textdom{bool}$ such that: \newline
(1) for every operation
  $op \in S$, the following sequential Hoare triple holds:
  \[
    \hoare{R(\sigma)}{i(\opv)}{\exists \sigma'.\, R(\sigma') \land \sigma \overset{\opv}{\leadsto} \sigma'},
  \]
(2) $\mathrm{init}(\sigma_{S}, \sigma_{T})$ implies
$R(\sigma_{S}, \sigma_{T})$, and \\
(3) if $R(\sigma_{S}, \sigma_{T})$ holds and $\crash(\sigma_{T}$, $\sigma_{T}')$,
then there exists a $\sigma_{S}'$ such that $R(\sigma_{S}', \sigma_{T}')$ and
$\crash(\sigma_{S}, \sigma_{S}')$.%
  \label{def:seqrefinement}
\end{definition}
%
Conditions (1) and (2) in this definition are standard for sequential
verification of refinement, while condition (3) is a condition for sequential crash-safety~\citep{sigurbjarnarson:yggdrasil,chajed:argosy}. Though condition (3) requires the
abstraction relation to be preserved by crashes, the proof engineer does \emph{not} have to reason about crashes in the middle of operations.
% is a standard condition for sequential crash-safety~\cite{chajed:argosy}
% for sequential crash safety~\cite{chajed:argosy}.
The
diagram in \cref{fig:refinement:seq} depicts the main
refinement condition (1) diagrammatically.
% For example, it is fairly easy to use them to show
% that for any sequential program $p : \gooselayer{S}$, the traces of the code
% $\mathrm{link}(p, i)$ are a subset of the traces of the spec $p$ (we will not
% use exactly this theorem since we are interested in concurrent code using
% transactions).
% \tej{I mentioned this but maybe we don't want/need to say it?}. Such reasoning would be familiar in previous sequential verified
% systems like FSCQ~\cite{chen:fscq}, IronFleet~\cite{hawblitzel:ironfleet}, and
% VeriBetrKV~\cite{hance:veribetrkv}.

The simulation transfer theorem takes a proof of \emph{sequential} refinement
conditions for a system implemented using transactions and derives a
\emph{concurrent and crash-safe} refinement. Like transaction refinement, it is
stated as a theorem about an arbitrary program $p : \gooselayervar{Sys}$ using
the specification-level API.\@ To model that program using the transactions given
by implementation $i$, we write $\mathrm{link}(p, \atomiccomp i)$. The function
$(\atomiccomp i)(\opv) = \atomically{i(\opv)}$ wraps the implementation
$i(\opv)$ in an \cc{atomically} block. Physically this \cc{atomically} block
corresponds to running code in a pattern like the following, written using Go's
support for generics:
%
\begin{minted}{go}
type txnBody[T any] func(tx *Txn) (T, bool)
func runTxn[T any](f txnBody[T]) (v T, ok bool) {
  tx := Begin()
  v, ok := f(tx)
  if ok { tx.Commit() } else { tx.Abort() }
  return v, ok
}
\end{minted}

In order for simulation transfer to work, every transaction must satisfy some
conditions to ensure atomicity. We write $\safe(i(\opv))$ to say that $i(\opv)$ is a
valid transaction. The main restriction is that $i(\opv)$ cannot access global state
such as the heap, since the transaction system does not make such accesses
atomic; further details are discussed in \cref{sec:txn:transaction-refinement}.

\begin{theorem}[Simulation transfer]
  Let $\textdom{Sys}$ be a layer implemented using transactions with
$i : \textdom{Sys} \to \gooselayer{Txn}$, such that
$\seqrefinement\targ{\gooselayer{Txn}, \textdom{Sys}}(i)$ and
$\forall \opv.\, \safe(i(\opv))$ hold. Then
\[
  \forall p : \gooselayervar{Sys}, \linked{\mathrm{link}(p, \atomiccomp i)} \refines p.
\]
\label{thm:gotxn-transfer}
\end{theorem}
\nopagebreak
The theorem is about a program $p$ using some API $\Sys$, such as the server
loop for DaisyNFS seen in \cref{sec:daisy:refinement-spec}, and an
implementation $i$ of this API.\@
The conclusion is a refinement that uses $p$ as the \emph{specification}; recall
that the semantics of $p$ defines all of the $\Sys$ operations to be atomic at this
layer. The left-hand side is the executable code for this program, derived by
composing two functions: first $\mathrm{link}(p, \atomiccomp i)$ takes each abstract operation
$\opv$ in $p$ and replaces it with $\atomically{i(\opv)}$, and second
$\linked{\mathrm{link}(p, \atomiccomp i)}$ takes the result of this process and
further replaces $\atomically{i(\opv)}$ with executable code that uses the GoTxn
implementation for each call to the Txn API and the \cc{runTxn} pattern above to
make the snippet atomic. The overall refinement says that this executable code
has a subset of behaviors of $p$, so that each operation is not only atomic but
follows the abstract specification of the $\Sys$ layer.

Simulation transfer is stated and proven in Coq. The proof builds on top of
transaction refinement. For every execution of the code, transaction refinement promises
a corresponding atomic execution of $\mathrm{link}(p, \atomiccomp i)$, and the
sequential refinement assumption is enough to show that the transactions
correctly simulate $p$ at the $\mathit{Sys}$ abstraction level.

\subsection{Using simulation transfer with Dafny}%
\label{sec:daisy:proof-dafny}

\newlength{\stepw}
\newlength{\dstepw}
\newlength{\nfstop}

Simulation transfer reduces verifying DaisyNFS's correctness to reasoning about
the transaction that implements each NFS operation. Because this reasoning is
sequential, we carry it out in Dafny~\cite{leino:dafny}, a verification-oriented programming language.
Let $\infs : \mathrm{NFS} \to \gooselayer{Txn}$ denote a model of the
Dafny implementation of each NFS operation, as a program using the GoTxn
interface. Note that this is merely a hypothetical model of the Dafny code,
since Goose does not support enough of Go to explicitly construct these terms in
GooseLang. The Dafny annotations on these methods prove sequential refinement
conditions that we will denote $\seqrefinementdfy(i)$, as
illustrated in \cref{fig:refinement}. This sequential refinement is indexed by
``dfy'' to indicate that it refers to what is proven in Dafny. It is intended
that this obligation captures $\seqrefinement(i)$ from the simulation-transfer
theorem, but using a formal distinction allows to say that the connection
between these definitions is trusted. Dafny checks the following lemma that
$\infs$ is correct, which will eventually be used to show the overall DaisyNFS
correctness theorem:
%
\nopagebreak
\begin{lemma}
  $\seqrefinementdfy(\infs)$ holds, as checked by Dafny.
  \label{thm:dafny}
\end{lemma}

We now state more formally what $\seqrefinementdfy$ means.
The Dafny code is implemented as a class with ghost variables for its abstract
state and an invariant that expresses the proof's refinement relation $R$
between the abstract state and its concrete state (which is limited to constants
and the transaction system's logical disk). The top-level interface is outlined
in \cref{fig:dafny-outline}. Condition (1) in the definition of
sequential refinement is encoded exactly in Dafny, using \cc{requires} and \cc{ensures}
clauses. Condition (2) for initialization and condition (3) for crashes relate
to the Dafny code in a slightly more subtle way.

\begin{figure}
  \begin{verbatim}
class Filesys {
  // the external abstract state
  ghost data: FilesysData;
  var txs: TxnSystem;

  // abstraction relation that relates txs.disk to data
  predicate Valid() { ... }

  // see text for these signatures
  constructor Init() { ... }
  constructor Recover() { ... }

  method GETATTR(ino: uint64) returns (r: Result<Attrs>)
    requires Valid() ensures Valid()
    ensures GETATTR_spec(old(fs), fs, ino, r) { ... }

  // ... methods for other NFS operations ...
}
  \end{verbatim}
  \caption{Outline of how the Dafny proof is expressed.}
  \label{fig:dafny-outline}
\end{figure}

\newcommand{\Op}{\mathbb{M}}

To describe how Dafny encodes this form of refinement, we start by describing its
specification transition system in more formal detail. The specification
transition system is
a triple $S = (\Sigma, \Op, \delta)$ where $\sigma \in \Sigma$ is a set of states,
$m \in \Op$ is a set of operations,\footnote{we use the metavariable $m$ to
suggest that these are methods} and $\delta$ is a relation, where
$\delta(\sigma, m, \sigma', v)$ is true when in state $\sigma$ operation $m$ can
transition to $\sigma'$ and return $v$. To simplify the notation, the transition
system has a special operation $\mathsf{crash} \in \Op$ to represent a crash transition.

\newcommand{\altsys}[1]{\tilde{#1}}

To define $S$, we build on a subset of its transitions $\altsys{S}$ that only
defines the methods of the Dafny class.
A state $\sigma \in \altsys{\Sigma}$ of this sub-transition system
consists of the values for all the ghost variables that define the API; in the
case of the file system this is just the \cc{data} field of type
\cc{FilesysData}, which was defined in \cref{fig:dafny-state}. Each NFS method and its
arguments is an operation $m \in \altsys{\Op}$, and predicates like \cc{GETSZ_spec}
define the transition relation $\altsys{\delta}$. Building on $\altsys{S}$, we
can now define $S$ by incorporating
initialization and recovery, which are implemented
in Dafny as \emph{constructors} \cc{Init} and \cc{Recover}. Constructors are
needed to set up the system's global constants, and to initially establish the
abstraction relation. Below we define $S$ in terms of $\altsys{S}$.
To write down the definition of $\delta$, we use $\sigma \xrightarrow[v]{m} \sigma'$ to mean that
$\delta(\sigma, m, \sigma', v)$ is true, and let $\delta$ implicitly be false
elsewhere if not specified.

\begin{figure}[ht!]
  \begin{mathpar}
  \begin{array}{llcl}
    &\Sigma &\defeq & \mathsf{UnInit} \ALT \mathsf{Running}(\sigma:
                      \altsys{\Sigma}) \ALT \mathsf{Crashed}(\sigma:
                      \altsys{\Sigma}) \\
    &\Op &\defeq & \mathsf{Init} \ALT \mathsf{Recover} \ALT \mathsf{Method}(m:
                   \altsys{\Op}) \\
    &\delta& \defeq& \mathsf{Running}(\sigma) \xrightarrow[v]{\mathsf{Method}(m)}
        \mathsf{Running}(\sigma') \text{ when } \sigma \xrightarrow[v]{o}

        \sigma' \text{ (in  $\altsys{\delta}$)} \\
    &&& \mathsf{UnInit} \xrightarrow{\mathsf{Init}} \mathsf{Running}(\sigma_0) \\
    &&& \mathsf{Crashed}(\sigma) \xrightarrow{\mathsf{Recover}} \mathsf{Running}(\sigma) \\
    &&& \mathsf{Running}(\sigma) \xrightarrow{\mathsf{crash}}
        \mathsf{Crashed}(\sigma') \text{ when } \sigma
        \xrightarrow{\mathsf{crash}} \sigma' \text{ (in  $\altsys{\delta}$)} \\
    &&& \mathsf{UnInit} \xrightarrow{\mathsf{crash}} \mathsf{UnInit} \\
    &&& \mathsf{Crashed}(\sigma) \xrightarrow{\mathsf{crash}} \mathsf{Crashed}(\sigma) \\
  \end{array}
  \end{mathpar}
  \caption[Formal definition of the Dafny specification transition
  system.]{Defining the specification transition system $S$ for the Dafny code,
    in terms of $\altsys{S}$ which only defines the transitions for the
    methods.}
  \label{fig:daisy:formal-spec}
\end{figure}

The transition system $S$ embeds $\altsys{S}$ in the transitions over
$\mathsf{Running}(\sigma)$, and these are the usual transitions during normal
operation. However, the Dafny code has to transition into the
$\mathsf{Running}(\sigma)$ state in two places: first the $\mathsf{Init}$
operation starts the system in state $\sigma_{0}$, and following a crash, the
system generally transitions from $\mathsf{Running}(\sigma)$ to
$\mathsf{Crashed}(\sigma)$. The system has to be restored to its usual running
state in order to use any methods.

In the Dafny code, $\mathsf{Running}(\sigma)$ corresponds to when there \emph{is
an instance} of the \cc{Filesys} class. The formal $\mathsf{Init}$ and
$\mathsf{Recover}$ operations are special insofar as they are implemented as
\emph{constructors} \cc{Init} and \cc{Recover}, which set up the system's global
constants and create the instance in the first place.

The Dafny code implements the initialization using a \cc{NewTxnSystem} method
that creates a fresh instance of GoTxn and sets up its initial object schema, as
described in \cref{sec:txn:lifting}. The \cc{Init} constructor establishes the
class invariant \cc{Valid()} starting from nothing, implicitly assuming that the
disk is all-zero to be ready for initialization. It promises an initial
file-system abstract state $\sigma_{0}$ with just a single root inode (the
attributes include a time stamp and are thus not fully specified):

\begin{verbatim}
constructor Init()
  ensures Valid()
  ensures exists attrs0 ::
          && init_attrs(attrs0)
          && data == map[1 := Dir([], attrs0)]
\end{verbatim}

The \cc{Recover} constructor has a more interesting verification story.
According to $S$ above,
it is permitted to assume that when recovery runs, the system starts out in the state
$\mathsf{Crashed}(\sigma)$. The only difference between
$\mathsf{Crashed}(\sigma)$ and $\mathsf{Running}(\sigma)$ as far as the
simulation is concerned is that in $\mathsf{Crashed}(\sigma)$, the \cc{Filesys}
instance is no longer accessible in memory --- the system always maintains an
abstraction relation \cc{fs.Valid()} that relates the physical state
(\cc{txs.disk} in the Dafny code) to the abstract state (the \cc{data} ghost
field). The proof of recovery shows it transitions from
$\mathsf{Crashed}(\sigma)$ to $\mathsf{Running}(\sigma)$, expressed in Dafny
with the following specification:

\begin{verbatim}
constructor Recover(txs: TxnSystem,
    ghost fs: Filesys)
  requires fs.Valid()
  requires txs.disk == fs.txs.disk
  ensures Valid()
  ensures data == fs.data
\end{verbatim}

The specification takes a \emph{ghost} \cc{fs} that satisfies the invariant, as
well as an assumption that the transactional disk from GoTxn is the same as the
one for the old file system. The former is true since the system is in the state
$\mathsf{Crashed}(\sigma)$, and the latter is due to the transaction system's
specification defining a crash as a no-op on the logical state. In order to
transition to the state $\mathsf{Running}(\sigma)$, recovery returns a fully
initialized file system (implicit in the fact that this is a constructor), which
satisfies the abstraction relation \cc{Valid()} and has the same underlying
state $\sigma$ as before (written \cc{data == fs.data} in Dafny). The proof of
recovery is not that involved, since all that must be verified is that recovery
restores the global constants of the file system to their old values, which is
implemented by storing them on disk during initialization in a fixed-location
file-system super block and reading them back in the recovery procedure.

The methods in Dafny are verified with the usual pre- and post-conditions. By
virtue of being methods, they implicitly assume that the system in the state
$\mathsf{Running}(\sigma)$; the \cc{Filesys} object must be initialized to call
any method on it. Initialization and recovery implicitly require that the user
follows the protocol above; specifically initialization must be called only
once, and subsequently recovery should be used to restore the previous state.
The code tries to check these assumptions: for example, the super block contains
a ``magic number'' (a fixed, initially randomly chosen 64-bit integer) and
recovery fails if the super block has the wrong magic number, indicating an
attempt to recover from a disk that has not been initialized. Ext4 implements a
similar feature to prevent mounting an invalid image.

\begin{figure}
  \centering
  \begin{subfigure}{0.25\textwidth}
    \includegraphics{fig/sequential-refinement.png}
  \end{subfigure}~~~~~\vrule~~~~%
\begin{subfigure}{0.3\textwidth}
  {\small
\begin{verbatim}

method CREATE(d_ino: uint64,
              name: Bytes)
 returns (r: Result<Ino>)
 requires R(txn_disk, fs)
 ensures R(txn_disk, fs)
 ensures r.Ok? ==>
 nfs3create_spec(d_ino, name,
   old(fs), fs, r.v)
\end{verbatim}
}
  %\caption{Dafny encoding}%
  %\label{fig:refinement:dafny}
\end{subfigure}
\vspace{0.5\baselineskip}
  \caption[Illustration of sequential refinement and its Dafny encoding]%
  {Illustration of $\seqrefinement(\infs)$ (left) and its encoding
in Dafny $\seqrefinementdfy(\infs)$ (right), for one particular operation.
In the diagram, the solid parts are assumed, and the
dashed parts must be shown to exist. The complete Dafny spec is more precise about
errors.}
  \label{fig:refinement}
\end{figure}

% hack to get theorem to have same number as original (only works because that's
% the first theorem, and there are no subsequent theorems)
\setcounter{theorem}{0}

We can now take simulation transfer and the Dafny proof and combine them to get
the overall DaisyNFS correctness theorem:
\begin{theorem}[DaisyNFS correctness, restated]
  DaisyNFS atomically implements the NFS protocol, formally stated as the
  refinement $\linkedcode \refines \snfs$. Note that
  $\sdfy = \mathrm{link}(\snfs, \infs)$. The system must be initialized with the
  \cc{Init} constructor on an empty disk, and then after every reboot
  re-initialized with the \cc{Recover} constructor.
  \label{thm:daisy-correctness-restatement}
\end{theorem}

This theorem re-states the specification developed in
\cref{sec:daisy:refinement-spec}. Its proof is a simple on-paper argument that
connects \cref{thm:gotxn-transfer} (which \emph{assumes} a sequential
refinement) to \cref{thm:dafny} (which \emph{proves} this sequential refinement).
\Cref{fig:refinement-execs} illustrates
the intuition for this theorem in terms of an example execution of $\snfs$: the transaction system proof guarantees an
atomic execution while the sequential refinement guarantees the transactions
themselves are correct.

\begin{figure}
  \centering
  \includegraphics{fig/refinement-execs.png}
  \caption[Overall DaisyNFS proof strategy]{Illustration of the DaisyNFS proof
    strategy in terms of one
    possible execution of DaisyNFS, receiving parallel \cc{MKDIR} and \cc{LOOKUP}
    operations, at its three abstraction levels. Operations in each row are
    color-coded green or orange according to which operation they correspond to
    (the top-level \cc{MKDIR} or \cc{LOOKUP} respectively). The refinement proof first
    shows that for every code execution (bottom row), there exists an atomic
    execution at the Txn layer (middle row), as proven in
    \cref{thm:gotxn-transaction-refinement}. This justifies sequential reasoning to
    show the transactions on top follow the NFS specification (top row), as
    proven in \cref{thm:dafny}. Putting the two together,
    \cref{thm:daisy-correctness-restatement} shows the entire DaisyNFS server atomically
    implements the NFS specification.}
  \label{fig:refinement-execs}
\end{figure}

There are two assumptions needed for the theorems to compose. First,
$\seqrefinement_{\mathrm{dfy}}(i_{NFS})$ should imply $\seqrefinement(i_{NFS})$,
to bridge the premise in simulation transfer and the theorem being proven in
Dafny. That is, the
encoding of the refinement conditions in Dafny must be correct, but also the
semantics of the transaction system operations modeled in Dafny must match the
Coq proof. Second, every Dafny transaction must be safe, that is
$\safe(i_{NFS}(op))$ must hold. Safety has a static restriction that transactions
should not modify global state, which the Dafny code satisfies because the only
mutable state in the file-system Dafny class is the transaction system, so
file-system operations cannot make mutations other than through GoTxn. The
dynamic restrictions for safety are expressed with preconditions on the GoTxn
interface so that Dafny automatically enforces them.

% We have some
% confidence this holds due to a simple check over the Dafny code: the only
% mutable state in the Dafny class that implements the file system is the ghost
% variables and the transaction system, so it cannot make mutations other than
% through GoTxn (ghost variables cannot influence execution due to the design of
% Dafny).
