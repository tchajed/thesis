\section{Verifying DaisyNFS}
\label{sec:proof}

How do we prove that the implementation of the NFS server running on a disk
implements the NFS transition system formalized in Dafny? The overall structure
of the proof resembles the division in the implementation: we prove that the
transaction system really makes the calling code's transactions atomic, and
separately prove that the file system's transactions are implemented correctly.
By isolating the concurrency and crash-safety reasoning to the transaction system, we can use
Dafny, with highly automated proofs, to reason about the file-system
implementation (since it behaves sequentially) while using Perennial to reason
about that concurrency and crash safety in the transaction system. Since the
proofs happen in separate formal systems, we also prove a linking theorem about
the entire file system whose proofs connects the transaction system's
correctness to the correctness of the file-system code.

Using a careful choice of interface between the transaction system and Dafny, we
minimize the work that goes into linking, which has to be proven on paper.
In particular the transaction system's proof
shows that any calling program appears to have atomic transaction, provided the
transactions follow some rules. For example, transactions can only modify state
protected by the two-phase locking system; if the caller accesses a global
variable, the transaction system can't see this access and acquire a lock so the
transaction would not be atomic. At a high level the on-paper reasoning
addresses why the Dafny transactions follow the rules required by the proof
mechanized in Coq.
% Dafny always assume methods run
% sequentially, so it is necessary that the implementation meet any restrictions
% of the transaction system for the Dafny proof to be meaningful.

% to understand how it guarantees transactions behave
% atomically and what the requirements on the caller are.

% While there is no fully
% machine-checked theorem covering the end-to-end system, all of the proofs within
% the transaction system and within the Dafny code are machine-checked and this
% proof merely connects those theorems. The manual step carried out here is
% an audit over the Dafny code to establish that it fits the transaction system's
% rules.

% It would be an interesting direction for future work to automate this step.

\begin{figure}[ht]
\begin{tabular}{lp{4.5cm}r}
\toprule
Layer & Operations & \\
\midrule
  NFS
      & \cc{CREATE(d_ino, name)}, \cc{READDIR(d_ino)}, \dots & \autoref{fig:nfs} \\
  Txn
      & \cc{Read(tx, a, sz)}, \cc{Commit(tx)}, \cc{Alloc(a)},
        \dots & \autoref{fig:txn-api} \\
  Disk
      & \cc{Read(a)}, \cc{Write(a, b)} & \\
\bottomrule
\end{tabular}
\caption{API layers of \sys.}
\label{fig:layers}
\end{figure}

The proof relates the \sys server loop at three different layers of abstraction,
corresponding to the three API layers in \autoref{fig:layers}. Refinement from a code program to
a specification program says that the behaviors of the code are a subset of the
behaviors of the specification; we will shortly give a more precise definition.
At the top layer, the specification end, we'd like to think of \sys as atomically responding to
NFS operations according to the state machine developed in Dafny.
The NFS operations are implemented using methods from transaction API, the middle layer.
Finally, the running code links the Dafny code with the transaction system's
implementation and runs on top of a raw disk-block API.

To define refinement we need to be more precise about what a program is and how it
executes. We
write $p : \gooselayer{X}$ to say $p$ is a Go program written using operations
from layer X, where X is one of NFS, Txn, or Disk.
Layer operations are always atomic transitions in a state machine. In the
NFS layer, the NFS operations behave according to the NFS state machine
described previously in \autoref{sec:soundness} and defined formally in Dafny.
The Txn layer is specified in both Coq where it is part of the transaction
system's correctness theorem and in Dafny where it appears as an assumption. The
Disk transition system is formalized in Coq as part of the GoJournal proof,
and assumes reads and writes of 4KB blocks are atomic. Each layer
includes concurrent threads that interleave layer
operations, basic heap operations on pointers, slices, and maps, and computation
on primitives like integers and structs.

There are three programs involved in defining and proving the overall
correctness of \sys, corresponding to the server loop at each abstraction layer.
At the top, the specification is a loop $\snfs : \gooselayer{NFS}$ which
atomically processes each NFS operation according to the NFS state machine. The
next level is $\sdfy : \gooselayer{Txn}$, where each handler is the atomic body of the
corresponding Dafny method, operating on top of the transaction system. Finally,
the executable code is written
$\mathrm{link}(\sdfy, \txncode) : \gooselayer{Disk}$, indicating ``linking'' the
Txn-layer server $\sdfy$ with the transaction system by taking each call to a
Txn API and plugging in its implementation on top of a disk.

Refinement relates two programs in terms of their visible behavior, which we
will use to connect the server loop at the disk layer to the transaction layer
and finally to the NFS layer. For the purposes of this paper, all of the
programs involved are servers that issue network I/O, either receiving an NFS
request or responding to one. Regardless of the level of abstraction, each model
of the server defines a trace of network I/O consisting of requests and
responses, and this is the behavior refinement talks about:

\begin{definition}[Refinement]
  An implementation program $p_{c}$ refines a specification program $p_{s}$,
written $p_{c} \refines p_{s}$, if whenever there are initial states
$\sigma_{s}$ and $\sigma_{c}$ satisfying $\mathrm{init}(\sigma_{s}, \sigma_{c})$
and $p_{c}$ can execute from $\sigma_{c}$ and produce a trace of network I/O
$tr$, then $p_{s}$ can execute from $\sigma_{s}$ and produce the same trace
$tr$.  Execution might involve crashing and restarting a program (potentially
multiple times), wiping out any in-memory state after each crash.
  \label{def:refinement}
\end{definition}

The intuition behind the notation $p_{c} \refines p_{s}$ is that the set of
behaviors of $p_{c}$ (the set of traces of network I/O $tr$) is a subset of the
behaviors of $p_{s}$. Whenever we state $p_{c} \refines p_{s}$ we leave implicit
a definition of initial states $\mathrm{init}(\sigma_{s}, \sigma_{c})$, which
will generally say both states are all zeros and of the same size.

\begin{figure}[ht]
  \center
  \begin{tikzpicture}[>=latex, node distance=1.5cm]

 \tikzset{
    genericnode/.style={rectangle,draw,minimum width=2cm, minimum height=.85cm, align=center,},
    layer/.style={
      genericnode,
      alias=genericnode,% <- alias added
      label={[anchor=south west,shift={(genericnode.north west)},inner sep=2pt]{\tiny #1}}% position the label using the alias
    }}

%\tikzstyle{layer}=[rectangle, draw, minimum width=2cm, minimum height=.85cm, align=center];
\tikzstyle{genlayer}=[dashed, layer={}];
\tikzstyle{edge}=[->,thick];

\draw node (dispatch) [layer={$\gooselayer{NFS}$}] {$\snfs$};
\draw node (goout) [layer={$\gooselayer{Txn}$},below of=dispatch] {$\sdfy$};
\draw node (txn) [layer={$\gooselayer{Disk}$},below of=goout] {$\mathrm{link}(\sdfy, \txncode)$};

\draw node (dafny) [left=1.25cm of goout, layer=Dafny] {File-system \\ Operations};

%\draw [thick] (goout.south) -- (txn.north);
\path (txn) edge[draw=none]
                node (incl1) [sloped, auto=false,
                 allow upside down] {$\refines$} (goout);
\path (goout) edge[draw=none]
                node (incl2) [sloped, auto=false,
                 allow upside down] {$\refines$} (dispatch);


\path (goout) edge[draw=none]
                node (thm2) [xshift=1.3cm, auto=true] {Thm \ref{thm:dafny} (Dafny)} (dispatch);
\path (txn) edge[draw=none]
                node (thm1) [xshift=1.5cm, auto=true] {Thm \ref{thm:txn} (Perennial)} (goout);

\draw [edge] (dafny.east) -- node[above] {\texttt{dafny}} (goout.west);


\end{tikzpicture}

  \caption{The overall structure of the proof combines theorems proven in Perennial
    and Dafny, each tool used for the reasoning it is best suited to.}
  \label{fig:proof-overview}
\end{figure}

\autoref{fig:proof-overview} gives the overall structure of the proof, which
relates these three programs via refinement. The high-level strategy is to break the proof down
into three steps: 1) state and prove \autoref{thm:txn} in Perennial, the correctness
of the transaction system, 2) state and prove \autoref{thm:dafny} in Dafny, the
correctness of the Dafny methods (as transactions), and 3) prove a specification
for the whole compiled system, \autoref{thm:correctness}, by applying the other
two theorems.

% \begin{figure}
%   \center
%   \includegraphics[width=\columnwidth]{drawn-diagrams/proof-overview.png}
%   \caption{The overall structure of the proof combines theorems proven in Coq
%     and Dafny, each tool used for the reasoning it is best suited to.}
%   \label{fig:proof-overview}
% \end{figure}

\subsection{Correctness theorem for the transaction system}
\label{sec:proof:txn}

\cc{Txn} layer semantics serves as the interface between the Perennial and Dafny
proofs.  It specifies that transactions are atomic in the sense that code
enclosed within a transaction \cc{Begin} and \cc{Commit} happens all at once (or
does nothing, if \cc{Abort} is executed), without interleaving of steps by other
threads.

We define the correctness of the transaction system's implementation as a
\emph{program refinement}.
To set up this specification, consider a program $p : \gooselayer{Txn}$ that
uses transactions.
To run $p$, it is combined with the transaction-system implementation, producing
a program $\mathrm{link}(p, \txncode) : \gooselayer{Disk}$ that can be run on
top of a disk.
Transactions in the linked program continue to have the expected atomic
behavior, so long as transaction code in $p$ follows certain restrictions, such
as not accessing shared state outside the journal system.  We write
$\mathrm{safe}(p)$ to mean $p$ is ``safe'' in the sense that it follows these restrictions.

% To connect this idealized semantics to the behavior of implementation, we define
% an

% At a high level of abstraction, the main difficulty is to give a specification
% for the transaction system, which we do in several steps:
%
% \begin{enumerate}
%   \item First, we define an arbitrary Go program running on top of
%         the transaction system. For reasons we will explain shortly we will use
%         $p : \gooselayer{Txn}$ for such a program. To run such a program it
%         first needs to be linked with the transaction system implementation,
%         producing a program denoted $\mathrm{link}(p, \txncode)$.
%   \item The second idea is to say what the semantics of a program
%         $p : \gooselayer{Txn}$ is. Transactions are atomic in this semantics in
%         that the whole transaction transitions at once, without interleaving
%         other threads. The program can issue reads and writes within a
%         transaction, and they follow a simple state machine.
%   \item The final idea is to define ``safe'' programs $\mathrm{safe}(p)$, those
%         that follow the restrictions of the transaction system. The
%         specification only applies to safe programs.
% \end{enumerate}

%With these specification ideas in place, what we prove in Coq is the following
%theorem:

The correctness of the transaction system is summarized by the following theorem:

\begin{theorem}
  The transaction system's implementation $\txncode$ is a program refinement, meaning for
  all $p : \gooselayer{Txn}$, if $\mathrm{safe}(p)$, then
  $\mathrm{link}(p, \txncode) \refines p$. The definition of
  $init(\sigma_{s}, \sigma_{c})$ in this refinement relates an all-zero physical
  disk to an all-zero transactional disk of the same size.
  \label{thm:txn}
\end{theorem}

\autoref{thm:txn} is stated in Coq and has a fully mechanized proof in Perennial.
What it says is that if a program is safe, the program linked with the
transaction system always behaves as if its transactions were atomically
accessing a transactional disk logically maintained by the transaction system.
The definition of safety formalized in Coq requires that code within a
transaction not access any shared memory outside of the transaction layer; other
than that, transactions are permitted to issue reads, writes, and do other
computation. Safety also requires that transactions follow the preconditions of
the \cc{Read} and \cc{Write} operations, which require a discipline of accessing
each object with a fixed size. Finally, safe programs can only \cc{Abort} or
\cc{Commit} a given transaction once. The notion of safe program will be
important when linking this proof with the Dafny proofs, since the transaction
system's proof only applies to a safe caller.

In the Coq development, \autoref{thm:txn} uses Goose~\cite{chajed:goose-coqpl}
to translate the transaction system's Go implementation into a model in Coq that
Perennial supports.

\subsection{Correctness theorem for sequential file-system transactions}
\label{sec:proof:dafny}

The top-level file-system is a program denoted $\sdfy : \gooselayer{Txn}$
written against the transaction-system API, where this program models the
top-level dispatch loop that repeatedly accepts an NFS request and responds to
it in a separate thread. What we prove using Dafny is that this implementation
refines a more abstract dispatch loop $\snfs$ where the transitions atomically
follow the NFS transition system:

\begin{theorem}
  $\sdfy \refines \snfs$. To establish the init relation in this definition,
  the caller runs a dedicated method in Dafny that assumes an all-zero
  transaction system and establishes the invariant all other operations rely on
  (including recovery).
  \label{thm:dafny}
\end{theorem}

We prove this theorem in Dafny using a standard \emph{forward simulation}
technique.  Instead of directly reasoning about the $\sdfy$ concurrent program
(which is not possible in Dafny) we instead consider each of its handler methods
separately. Because transactions run atomically, it is sound to use Dafny's
sequential specifications in terms of pre- and post-conditions to reason about
an individual method. If every method satisfies a refinement obligation using a
common abstraction relation, we know that the entire program satisfies
refinement as defined above. This proof method of forward simulation is
so common that many systems verified in Dafny do not mention it, or treat
simulation and refinement synonymously.

Crash safety does add one interesting case to forward simulation. In general for
a sequential storage sytem, recovery should satisfy a specification that assumes
a \emph{crash invariant} that the whole system maintains at each intermediate
point and establishes the abstraction relation, with a new abstract state
consistent with the specification's crash behavior~\cite{chajed:argosy}. Due to
our transaction system we can make one simplification: on crash, the system will
satisfy the abstraction relation and not just a weaker crash invariant, since
operations are atomic; furthermore, the file system automatically maintains the
abstraction relation for crashes during recovery since recovery itself uses a
single transaction. Recovery must still do some work since the abstraction
relation held prior to the crash using in-memory state that has been lost. We
give the precise specification proven in Dafny in \autoref{appendix:proof}, and
also argue how it fits into the forward simulation proof.

\subsection{Linking the transaction system and Dafny correctness theorems}
\label{sec:proof:linking}

Using the mechanized theorems, we can show that the overall system is correct,
namely that the running code implements the abstract NFS dispatch loop:

\begin{theorem}[\sys correctness]
  $\mathrm{link}(\sdfy, \txncode) \refines \snfs$. Initialization requires
  starting from an empty disk, then running the initialization implemented in
  Dafny. After that, the system boots by first recovering the transaction
  system's state, then running file-system recovery.
  \label{thm:correctness}
\end{theorem}

This theorem's proof is a straightforward consequence of the mechanized theorems
described as long as $\sdfy$ is safe; that is, the Dafny code follows the rules of
the transaction system. As long as this
holds we simply apply the theorems in order to show
$\mathrm{link}(\sdfy, \txncode) \refines \sdfy \refines \snfs$, since
the definition of refinement is transitive.

The proof requires Dafny operations to be encapsulated in transactions
and follow the transaction system's safety restrictions.
The code in Dafny does not generally manage starting and
committing a transaction; this is handled by a single wrapper function written
in Go. The wrapper creates a transaction, calls a Dafny method on it,
and then aborts if the method returns an error code and commits otherwise.
It is easy to confirm that this function follows the calling sequence required
by the transaction system.

Many transaction safety restrictions are preconditions on the
transaction-system APIs, which are enforced as Dafny preconditions.  The most
restrictive part of safety is that transactions are not allowed to read or write
shared memory, other than through the Txn layer. The Dafny code does allocate
heap objects and read them, but these are only used locally, and thus are
unaffected by concurrency; we work through this argument more formally in
\autoref{appendix:proof}. To confirm that there is no shared mutable state, we
checked that the Dafny class implementing the file system has no mutable
variables, other than the transaction system and its allocators.  Dafny does not
support mutable global variables outside of any class, so checking the class is sufficient.

Notice that this linking proof makes minimal assumptions about the code in
Dafny, other than the fact that it is verified. In particular, the same argument
would equally apply to a different NFS implementation, or even a system with an
entirely different specification, as long as the equivalent of
\autoref{thm:dafny} was still proven in Dafny, and the Dafny code did not use
shared mutable state.

% Formally, even if $\sdfy$ is not safe, it is sufficient if there
% exists any other program $\widetilde{\server}_{\mathrm{dfy}}$ with equivalent behavior which is
% safe; we can apply \autoref{thm:txn} to this alternate program and get the same
% refinement result. Such a program would be a systematic transformation of the
% Dafny code that replaced any allocation with a local variable forwarded to the
% remainder of the code, and inlined any global constants. This transformation
% would could fail if Dafny wrote outside of its local allocations, but there are
% no mutable variables to write to. Finally, we manually check that any buffers
% returned from Dafny are used in a read-only manner.

The Txn layer's allocator methods are important for this approach to work.
The allocator cannot be implemented in Dafny because it is shared mutable state.
Instead, we expose the allocator API as part of a transaction, albeit with a loose
specification that says \cc{Alloc} may return
any number and \cc{Free} may be called on any unused block.
As we explain in \autoref{sec:txn-proof}, under this specification, \cc{Alloc} and \cc{Free} both behave
atomically in a transaction. The
true allocation state is stored on disk, in the bitmap blocks for
example, and the Dafny code must validate that the address returned by \cc{Alloc} is actually
free.

% It is possible that one transaction calls
% \cc{Free} and another allocates the same number before the disk is updated,
% because freeing does not happen at commit time, but the allocator's policy is
% designed to delay allocating recently freed blocks so this is extremely
% unlikely.
