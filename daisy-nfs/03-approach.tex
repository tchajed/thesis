\section{Specifying DaisyNFS}%
\label{sec:daisy:spec}

The specification for DaisyNFS is a state machine describing an ideal NFS server in
the form of an abstract state and a transition for each operation. The
implementation of DaisyNFS is a binary \cc{daisy-nfsd} that implements the NFS
protocol, running on top of a
disk. Then the DaisyNFS correctness
theorem is a \emph{refinement} property, which intuitively says that
for any interaction with the
implementation, the ideal, atomic NFS state machine could produce the responses;
this section shortly gives a more formal definition.
As a result a client interacting with the server can pretend
that it is the NFS state machine and ignore the complexities of its
implementation.

\subsection{Formalizing NFS}%
\label{sec:daisy:nfs}

RFC 1813 specifies the NFS protocol, which we make mathematically precise with a
state-machine representation defined in Dafny.
The formalization requires first
defining an abstract state, and then a transition for each
NFS operation that specifies how it changes that state and what return
values are allowed. While most of the specification is deterministic,
some operations have to be specified with non-determinism; for
example, we allow returning an out-of-space error in many operations,
and the specification allows any timestamp to be picked for the
current time. The RFC is precise about arguments and allowed return
values, and the text is good about explaining the intended behavior,
but it does not separately describe an abstract state to make that behavior
mathematically precise.  We define
the NFS server state as shown in \cref{fig:dafny-state}.

\begin{figure}[ht!]
\begin{minted}[frame=lines]{csharp}
// the abstract state of the file system
type FilesysData = map<Ino, File>

datatype File =
  | ByteFile(data: seq<byte>, attrs: Attrs)
  | Dir(dir: map<FileName, Ino>, attrs: Attrs)

type Ino = uint64
type FileName = seq<byte>
datatype Attrs = Attrs(mode: uint32, ...)
\end{minted}
  \tightenspace
\caption{Dafny definition of the NFS server state (simplified).}
\label{fig:dafny-state}
\end{figure}

This definition says that an NFS server conceptually maintains a mapping from
inode numbers to files, where a file can either be a regular file with
bytes, or a directory. Both types of files have a number of attributes, storing
metadata like the file's mode (permission bits) and modification time. A
directory is a partial map from file names
(which are just bytes) to inode numbers. Note that DaisyNFS doesn't
represent the file system as a tree but as a collection of
links, which is sufficient to model all NFS operations, because
NFS clients resolve path names.

% \mfk{the rest of this
%   paragraph is lacking a clear narrative.}
% In any case a
% tree wouldn't be a sufficient state for NFS since modifying one file
% affects any other hard links to the same file (note though that DaisyNFS
% does not currently support hard links).

The NFS state machine models each operation as a non-deterministic transition
that answers when it is allowed for an operation to change the state from
\cc{fs} to \cc{fs'} and return \cc{r}. The return value is always wrapped in a
\cc{Result} type, which can be either \cc{Ok(v)} for a normal return or an error
code for one of the errors defined in the standard. The file system systematically guarantees
that the state is unchanged when an operation returns an error (though this is
stronger than what the RFC mandates); the transaction system makes this easy to
achieve by aborting the whole transaction. For example,
\cref{fig:getsz} shows the
specification for a (hypothetical) \cc{GETSZ} operation that returns the size of
the inode \cc{ino}.

\begin{figure}[ht]
  % make this text because csharp doesn't add much, and incorrectly highlights
  % the 's as errors
\begin{minted}[frame=lines]{text}
predicate GETSZ_spec(ino: Ino, fs: FilesysData,
  fs': FilesysData, r: Result<uint64>)
{
  fs' == fs &&
  (r.ErrBadHandle? ==> ino !in fs) &&
  (r.ErrIsDir? ==> (ino in fs) && fs[ino].Dir?) &&
  (r.Ok? ==> (ino in fs) && fs[ino].ByteFile? &&
             r.v == |fs[ino].data|)
}
\end{minted}
  \tightenspace
  \caption[Transition-system specification for a hypothetical \cc{GETSZ} operation.]%
  {Specification of a hypothetical \cc{GETSZ} operation, a simplification
  of the real \cc{GETATTR} operation. When \texttt{GETSZ\_spec(ino, fs, fs', r)}
  returns true, it is valid when the server receives \cc{GETSZ(ino)} to transition from
the state \cc{fs} to \cc{fs'} and return \cc{r}. The return value can be an
error (e.g., \cc{r.ErrBadHandle?}), or if \cc{r.Ok?} holds then \cc{r.v} is the
\cc{uint64} return value from the operation.}
\label{fig:getsz}
\end{figure}

There are four clauses in the specification. The first just says that this
operation is read-only. The second is one possible error: if the server returns
\cc{ErrBadHandle}, then \cc{ino} is not allocated. The third is a different
error, which says this operation returns \cc{ErrIsDir} if passed a directory inode.
Finally the fourth clause says that if the operation is successful, it returns the
length of the data in \cc{fs[ino]}. Dafny checks several consistency properties
of this specification itself; for example, a use of \cc{fs[ino]} only compiles
if the specification earlier implies \cc{ino in fs}.

We developed a state-machine model of the regular file and directory operations
in NFS in this style, including specifying what certain errors
signify. \Cref{fig:nfs} lists the entire NFS API and what parts are verified in
DaisyNFS.
The file system has unverified implementations \cc{FSINFO} and \cc{PATHCONF}, which give the client static
configuration information about the file system (for example, the maximum supported
write size or the maximum path length). These return constants and thus have no
associated proof. DaisyNFS also implements \cc{FSSTAT} to report total and free space,
but it does not have a meaningful specification.


\renewcommand{\check}{\textcolor{ForestGreen}{\checkmark}}
\newcommand{\nope}{\textcolor{Maroon}{\ding{55}}}

\begin{figure}
\small \centering
\begin{tabular}{@{~}ll@{}c@{~}}
  \toprule
  \bf Category & \bf Operations & \bf Verified \\
  \midrule
  \textit{File and directory ops}
  & \cc{GETATTR}, \cc{SETATTR}, \cc{READ}, \cc{WRITE} & \check \\
  & \cc{CREATE}, \cc{REMOVE}, \cc{MKDIR}, \cc{RENAME} & \check \\
  & \cc{LOOKUP}, \cc{READDIR} & \check \\

  \textit{Unsupported features}
  & \cc{READLINK}, \cc{SYMLINK}, \cc{LINK}, \cc{MKNOD} & \nope \\
  & \cc{READDIRPLUS}, \cc{ACCESS} & \nope \\

  \textit{Configuration}
  & \cc{FSINFO}, \cc{PATHCONF}, \cc{FSSTAT} & \nope \\

  \textit{Trivial operations}
  & \cc{NULL}, \cc{COMMIT} & \check \\

  \bottomrule
\end{tabular}
\caption{NFS API and which operations DaisyNFS supports and verifies.}
\label{fig:nfs}
\end{figure}

DaisyNFS could support some of the remaining operations with some more effort.
A symbolic link is essentially a file that holds a path, which is created with
\cc{SYMLINK} and can be read with
\cc{READLINK}. \cc{MKNOD} similarly creates a new type of special file that
consists of a major and minor device number.
Specifying these operations would require mostly mechanical changes to the
specification to accommodate the new file types.
\cc{LINK} is more complicated because in addition to tracking
the link count of every file in the state, the specification for \cc{REMOVE}
needs to say that the link count is decremented and that the file is deleted if
its link count drops to zero.

The current implementation of \cc{READDIR} always returns the entire directory,
which is impractical for large directories; NFS supports a paginated API, where
\cc{READDIR} returns some directory entries and then an offset to resume
iteration. The complication with this API is that the directory could change
between reads, so the interface has provisions to handle
this kind of \emph{iterator invalidation} and the specification would need
similar adaptations. We believe it would be possible to adapt the work in
SibylFS~\cite{ridge:sibylfs}, a specification for the POSIX file-system API, which gives a
specification for the analogous \cc{readdir} system call. DaisyNFS does not
support \cc{READDIRPLUS}, which differs from \cc{READDIR} only in that it also
returns attributes along with names and inode numbers.

\subsection{Specifying correctness for DaisyNFS}%
\label{sec:daisy:refinement-spec}

The previous section (\cref{sec:daisy:nfs}) defines the transition system for NFS,
which this section relates to the actual DaisyNFS implementation in order to
define correctness. The definition uses the ideas from
\cref{sec:txn:refinement-def,sec:txn:go-layers}, namely the definition of
refinement and the definition of $\gooselayer{X}$, where X is instantiated
separately with both the Txn and NFS layers. \gooselayer{NFS} has a primitive for each operation supported by
DaisyNFS, and the semantics of these primitives is defined to be the NFS
transition system written in Dafny. Implicit in such a semantics is that each
operation is atomic both with respect to other threads and on crash.

With the NFS layer we can model a \emph{server loop} that recovers the
file-system state, then repeatedly accepts a new request, processes it with the
appropriate NFS transition, and sends the reply back. Let us denote this server
loop with $\snfs : \gooselayer{NFS}$. Note that when this (abstract) server
processes an NFS operation, it does so by definition atomically, and its state
is again by definition preserved on crash. $\snfs$ can be viewed as an
abstraction of the following pseudo-code that represents the essence of the
real \cc{daisy-nfsd} binary:

% minipage prevents splitting this code across pages
\begin{minipage}{\textwidth}
\begin{minted}[linenos]{go}
// this is the core of daisy-nfsd
func main() {
  tx := txn.Recover()
  fs := filesys.Recover(tx)
  for {
    req := GetRequest()
    go func() {
      switch req.Op {
      case CREATE:
        ret := fs.CREATE(req.Args)
        SendReply(req, ret)
      case LOOKUP:
        ret := fs.LOOKUP(req.Args)
        SendReply(req, ret)
      // ... other cases ...
      }
    }()
  }
}
\end{minted}
\end{minipage}

This code starts by recovering the state of the system, starting with the
transaction system on line 3 and then continuing on to the file-system state on
line 4. Then it repeatedly accepts new requests from the network, abstracted
with \cc{GetRequest()} (including parsing the NFS wire protocol). These requests
are each processed in a background thread due to the goroutine spawned on line
7. The processing for each request dispatches to the appropriate file-system
operation (e.g., lines 10 and 13). The implementations of these operations are
compiled from Dafny to Go and then linked with the transaction system.

The term $\snfs$ represents the abstract version of this loop where, for example,
the entire call to \cc{fs.CREATE} is an atomic primitive. In the Dafny code,
this corresponds to an entire method that calls the GoTxn API. We can think of
each method as having a corresponding GooseLang term that represents its
implementation at the Txn layer of abstraction, for example
$\mathit{CREATE} : \gooselayer{Txn}$ might model the \cc{fs.CREATE} method.
These constructions are all for the sake of the proof argument; it is part of
the assumptions of the proof that every method can be modeled using GooseLang.
Next, let $\sdfy$ denote the same overall server dispatch loop with
methods at the GoTxn abstraction level. Finally, the lowest-level model of the
server is derived by combining the Dafny implementation with the actual GoTxn
implementation, which we write as $\linkedcode$.

$\linkedcode$ and $\snfs$ are both models of the \cc{daisy-nfsd} server binary's
behavior. The former is a model of its implementation running on top of a disk,
while the latter is its specification at the level of NFS methods. These two
together are enough to state the overall DaisyNFS correctness theorem:
%
\begin{theorem}[DaisyNFS correctness] $\linkedcode \refines \snfs$; that is, the
  DaisyNFS code follows the NFS specification.
  Initialization requires initializing the
  transaction system on an empty disk and then running the \cc{Init}
  constructor for the file-system. Subsequently the system boots by recovering the
  transaction system and then the file system with its \cc{Recover} constructor.
  \label{thm:daisy}
\end{theorem}

This theorem connects the server's implementation at the disk layer,
$\linkedcode$, which combines a model of the Dafny code with the GoTxn
implementations, to the most abstract version of the server $\snfs$ which by
definition atomically processes operations, preserves data on crash, and follows
the NFS state machine written in Dafny. \Cref{sec:daisy:proof} explains how this
specification is proven. The basic idea is that the Dafny reasoning relates
$\sdfy$ to $\snfs$ with a sequential proof, and GoTxn guarantees that sequential
reasoning for transactions is sufficient to guarantee the whole system satisfies
concurrent, crash-safe refinement due to the atomicity of every transactions.
