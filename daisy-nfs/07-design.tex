\section{Verifying the Dafny implementation}%
\label{sec:daisy:design}

We follow the standard approach for verifying software in Dafny: each
file-system operation is implemented as a method on a class and its
specification is given using pre- and post-conditions. In \cref{sec:daisy:proof-dafny},
we explained how the Dafny proof shows the code is a correct implementation of NFS in terms of sequential refinement. This
section provides details about the file-system design and proof.

% The proof is given by annotating the code with proof steps, which include
% updates to ghost state, assertions to assist the automated verification, and
% calls to lemmas.

DaisyNFS is implemented and verified in several layers of abstraction, depicted in
\cref{fig:dafny-layers}. Each layer is implemented as a class that wraps the
lower layer as a field. The transaction system is an assumed interface in Dafny,
while the complete server implements the NFS wire protocol and calls into the
top-level Dafny class for each operation.

% The approach we follow is inspired by xv6. We borrow some tricks from FSCQ and
% Yggdrasil to accomodate verification (especially using more layers than the xv6
% C implementation). We also needed to adapt the proof a bit so that rather than
% using separation logic we encode disjointness in an SMT-friendly way. The
% approach we take is standard and the same as used in Yggdrasil (see how layer 3
% works).

\begin{figure}
\small \centering
\begin{tabular}{ll}
  \toprule
  \textbf{Layer} & \textbf{Functionality} \\
  \midrule
  dir & Directories and top-level NFS API. \\
  typed & Inode allocation. \\
  byte & Implement byte-level operations using blocks. \\
  block & Gather blocks for each file into a single sequence. \\
  indirect & Triple-indirect blocks organized in a tree. \\
  inode & In-memory, high-level inodes; block allocation. \\
  txn & Assumed interface to external transaction system. \\
  \bottomrule
\end{tabular}
\caption{Layers in the Dafny implementation and proof of the file-system
operations.}
\label{fig:dafny-layers}
\end{figure}

Between the layers of the file system
there are three difficult pieces of
functionality: organizing data blocks into metadata and data (the
indirect and block layers), translating byte-level operations into
block operations (the byte and typed layers), and implementing
directories as special files that the file system itself reads and
writes (the dir layer). The modularity was essential to complete the proof in
manageable chunks (to avoid overwhelming the developer and prover), and it would
have been natural even without verification.

\subsection{Implementing the file system using transactions}

The design of DaisyNFS is broadly similar to the file system in xv6~\cite{xv6},
as well as Yggdrasil~\cite{sigurbjarnarson:yggdrasil}, a verified sequential
file system. We also adopt the recursive strategy for implementing and
verifying indirect blocks from DFSCQ~\cite{akonradi-meng}; recursion simplifies
the implementation of triply-indirect blocks, which are needed to reach a
reasonable maximum file size of 512GB.\@ Unlike most file systems, DaisyNFS is designed
to fit every operation into a transaction in order to support our goal of
sequential reasoning. This is a non-standard design and we encountered some
unique challenges in doing so. In this section we highlight difficulties in
fitting two features into transactions: rename and freeing space from deleted
files.

\subsubsection{Rename}
\label{sec:dafny:rename}

The NFS RENAME operation is similar to the \cc{rename} system call: it moves a
source file or directory to a destination location. What makes it tricky is that
it involves more than one inode and hence introduces the possibility for
deadlock.
% , which we would like to avoid even if the theorems do not forbid it.
We
use the standard strategy of enforcing a global ordering where inodes are always
locked in numerical order (smaller inode numbers first); this avoids a deadlock
where a cycle of threads is waiting on each other.

At this point it is worth discussing the performance considerations that lead to
handling lock ordering in the file
system, rather than generically in GoTxn. The transaction system could
avoid deadlocks by either enforcing a global order over addresses or by
timing-out operations. Enforcing a global order is inefficient for the file
system; data blocks will never cause deadlock because the file system only
accesses a block after locking the (unique) inode that owns it. Timing-out
operations would lead to slow and spurious transaction failures that could more
rapidly be avoided in the higher-level code, hence we do not attempt to detect
deadlock dynamically.

In a rename operation, the source and destination are each specified by a
combination of the parent directory inode and name within that directory. Rename
has an additional functionality of overwriting the destination if the source and
destination are files, or if both are directories and the destination is empty.
It is this latter check that makes deadlock avoidance difficult: it is necessary
to lock the source and destination directories first to lookup the source and
destination names, but those might be files that are earlier in the inode lock
order. We address this in the code by returning an error from the Dafny
transaction before the lock order would be violated. The error comes with the
set of inodes that should have been acquired.  The rename is then re-run with
this set of inodes as a lock hint; these are first acquired in the correct
order, then compared against the current source and destination in case they
have been renamed concurrently.

%% \subsubsection{Large files}
%% \label{sec:dafny:indirect}
%%
%% The indirect and block layers together implement an abstraction of
%% files as a sequence of blocks, using some blocks in the transaction
%% system as indirect blocks that contain pointers to other blocks.  Most
%% of the code implementing indirect blocks is implemented recursively
%% using an approach borrowed from DFSCQ~\cite{akonradi-meng}.
%% In practice we configure the system with several direct and (single)
%% indirect blocks in an inode to make small files more efficient, plus a
%% triple-indirect block that allows files to grow to 512GB.

%% Using only a single level of
%% indirection would be convenient for verification, but it would limit the size of
%% files to 28MB. Instead, we use multiple levels of indirection. Indirect blocks
%% have a natural recursive structure, where a $k$-indirect block holds pointers to
%% $(k-1)$-indirect blocks, down to $k=0$ for the data blocks.

%% Files at the block layer are always of the maximum size, to avoid
%% reasoning about sizes until the higher layers. To efficiently
%% represent such large files, the zero block number is treated specially
%% as encoding a zero block, including for indirect blocks; an idea we
%% borrowed from DFSCQ~\cite{akonradi-meng}. Because this works
%% recursively, the single zero for the root of the triple-indirect block
%% address efficiently stores the many GBs of zeros in a typical small
%% file.

%% The representation of file data using zero addresses and
%% recursive indirect block implementation were directly inspired by the
%% DFSCQ indirect-block~\cite{akonradi-meng}.

%% Indirect blocks pose a challenge for verification due to the classic problem of
%% \emph{aliasing}. The proof must show that modifying a data block or indirect
%% block has no effect on other files. In the DFSCQ proof, the invariant
%% captures the non-aliasing between files using separation logic, which makes
%% disjointness easy to express. In Dafny we have no such logical
%% technique, so we instead use a standard SMT-friendly trick for the invariant: in
%% addition to the physical mapping that tracks how to dereference a block address,
%% the indirect layer proof tracks a ghost \emph{reverse} mapping that tracks where
%% each in-use block number is stored. The invariant states that the forward and reverse
%% mappings are inverses of each other, which implies that modifying an address
%% only affects its owner and nothing else.

%% To encode the reverse mapping, we need a data type to represent the location of
%% a block within an inode. With indirect blocks, the metadata blocks themselves
%% also need to be considered locations, since the invariant must also rule out
%% metadata aliasing with data or other metadata. We encode locations with a
%% position datatype that encodes an inode, an indirection level, and an offset for
%% the blocks at that indirection level. If we imagine that an inode's block
%% pointers are organized in a tree, the roots are stored directly in the inode
%% while the leaves are direct blocks. An indirection level which is higher than
%% the leaf level describes a metadata block.

%% The indirect block proof is split into the indirect and blocks layers. In the
%% indirect layer, the abstract state maps positions (including the inode number)
%% to data, and off to the side tracks the size and attributes of each inode. The
%% block layer changes this representation to a flat sequence of blocks by mapping
%% each leaf position to its linear index within the inode. Separating these two
%% made it easier to work on the indirect layer while giving the upper layers the
%% much more natural abstraction of a file as a sequence of blocks.

\subsubsection{Freeing space}
\label{sec:dafny:freeing}

Freeing space becomes surprisingly tricky with large files. The problem is that
a large-enough file may reference too many blocks to be
freed in a single transaction.
DaisyNFS handles freeing by removing a file from its directory and marking it free in
one transaction, and in separate transactions reclaiming the space it took by deallocating
its blocks.

Removal is implemented as a combination of two transactions,
one which performs the logical operation but leaks space, and an operation
\cc{ZeroFreeSpace(ino)} which frees and zeros the unused space in an inode that
we prove has no effect on the file-system state. Because this operation is a
logical no-op, it is safe to call it at any time. In practice the implementation
is careful to call it after any operation that leaves unused blocks, in
particular \cc{SETATTR}, which can shrink a file by reducing its size, and
\cc{REMOVE}, which deletes a file. Furthermore since \cc{ZeroFreeSpace} doesn't
affect the user-visible data, it may return early to avoid overflowing a
transaction, which GoJournal limits to 511 blocks.

There is one case where freeing blocks is important for correctness and not just to reclaim space. Growing a file is supposed to logically fill the new space with
zeros. If the file had old data in that space, it would not be zero but some
previously written and deleted data, which both violates the specification and
is a potential security risk. The way we handle this with background freeing is
with validation: when the \cc{SETATTR} operation grows a file checks, it checks if the
free space is already zero first, and if not fails with a special error code. The
unverified code interprets this as a signal to immediately call
\cc{ZeroFreeSpace} and try the operation again. The same support also handles
holes created by writing past the end of a file, which are similarly supposed to
be zero.

The freeing implementation is an interesting example of using validation in
verification. The specification for much of the freeing code is loose, allowing
any data to be written to the free space. We only needed a strong specification
for the code that checks if the zeroing is done; the rest of the code needs to
be correct for this check to ever succeed, but we aren't required to prove it.

\subsection{Achieving good performance}

An important aspect of the Dafny proof was to write code in a way that produces
high-performance Go code.
% problem because Dafny's built-in immutable collections (sequences and maps) are
% extremely inefficient in Go due to an impedance mismatch between Dafny and Go
% semantics.
Compared to Dafny's C\# backend, the generated Go code for Dafny's built-in
immutable collections has much
additional pointer indirection and defensive copying. Using these data
structures for byte sequences would simplify proofs, but has unacceptably poor
performance in Go.

To avoid this performance problem we use an axiomatized interface to
Go byte slices (\cc{[]byte} in Go) whenever raw data is required, including file
data and paths, and then modify these slices in-place. It was possible to
axiomatize this API without any changes to Dafny; we use a standard Dafny
feature of \cc{:extern} classes to specify a Dafny class \cc{Bytes} in terms of
ghost state of type \cc{seq<byte>} but then implement it as in Go as a thin
wrapper around the native \cc{[]byte} type. This API is trusted, so we
test it. To catch off-by-one errors in the specification, we wrote
tests like \verb![]byte{1,2,3}[2]! and ran them in Go and
(equivalent) Dafny.
%(this test should return 3 because both languages are
% zero-indexed).

% We implement the file system fairly efficiently, taking advantage of Dafny's
% support for imperative code.
The on-disk data structures---inodes, indirect
blocks, and directories---are represented in memory in their serialized form and
modified by updating this representation directly, avoiding copies to move
between representations. These were first written with slower purely
functional code, which was then migrated to imperative code that
used the functional code as a specification.


Dafny's default integer type \cc{int} is unbounded and compiled to big-integer
operations. We used Dafny's
\cc{nativeType} support to instead define a type of 64-bit integers (that
is, natural numbers less than $2^{64}$) and compile this to Go's \cc{uint64}.
This requires overflow reasoning, but
automation makes this palatable in the proof and the performance gain is
significant.



% \subsection{Random notes on development process}
%
% \begin{itemize}
%   \item Used inefficient functional Dafny code at first, then slowly migrated to
%         in-memory data structures and improved performance.
%   \item Hard to debug and fix timeouts. Profiling verification performance is
%         hard.
%   \item Profiling Go code is great as usual. The generated code looks strange,
%         but I think after code generation it's pretty ordinary (the weird things
%         are mostly bad variable names, lot of unused assignments, and anonymous
%         functions that are immediately called).
%   \item Used some unit tests, but very few and only at the top level. Mainly
%         debugged Go compilation issues and cases where errors were
%         unintentionally being returned.
%   \item Trusted code isn't easy, had bugs in it before testing it thoroughly.
%         Also violated preconditions in top-level specs, triggering memory-safety
%         bugs.
% \end{itemize}
