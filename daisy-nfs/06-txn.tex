% this continues the section that starts the GoJournal proof

\subsection{Two-phase locking and allocator proof}

Transaction atomicity relies on the guarantee that two different transactions
cannot concurrently access the same object. The two-phase locking system
guarantees this by logically maintaining a lock per object and then ensuring
that the lock for an object is held whenever it is accessed. Allocating a lock
upfront for every object would be expensive, so we use a more efficient lockmap
data structure already implemented and verified in GoJournal.

We isolate the reasoning about these locks into a set of Hoare-logic proofs
about the two-phase locking methods. This intermediate specification exposes the
durable state of the transaction system using ghost variables to the refinement
proof. \tej{that's all we have to say about the 2PL's Hoare specs?} GoJournal exposes logical resources representing the on-disk and
in-memory value of each object. The two-phase locking system wraps both of these
resources in a lock that maintains a correspondence with the transaction
system's ghost variables. To handle crashes, the proof uses Perennial's
crash-aware lock specification~\citep{chajed:gojournal} to show that the durable
state matches even if the system crashes while a transaction holds locks on some
objects.

% The lock reasoning uses the standard technique of wrapping access to resources in Iris invariants. When a lock is acquired, the proof gains access to the resources in its corresponding invariant; when the lock is released, the proof needs to return these resources to the invariant. Since the lock semantics guarantee that at most one thread can hold a particular lock at any point of time, this guarantees that at most one transaction can access the resources corresponding to a particular object at any point of time.

The transaction system's allocator is also verified in Perennial through a
series of Hoare triples.  Because \sys uses validation to check that the
returned addresses are free, the triples only need to guarantee that these
operations are memory safe and return in-bounds addresses.

\subsection{Refinement to atomic language}

\tej{copy the new version of this from DaisyNFS paper}

\newcommand{\txnmapsto}{\mapsto_{\cc{txn}}}
\newcommand{\thdmapsto}{\Rightarrow}

Recall that \cref{thm:txn} establishes refinement between the Txn layer and
Disk layer.  That is, given a program p written in $\gooselayer{Txn}$, if the
transaction operations of p are linked with an implementation written in
$\gooselayer{Disk}$, the implementation program's observable behaviors (from
running on a disk) are a subset of the specification's behaviors (with
high-level transaction-system operations).

In general, the Perennial framework supports proving refinements between a
specification layer $\gooselayer{S}$ and an implementation layer $\gooselayer{I}$ by constructing
a simulation between the specification and its implementation.  Following an
approach developed by \citet{turon:caresl} and implemented in Perennial 1.0~\cite{chajed:perennial}, the execution of the specification
program is represented by \emph{ghost state}. The logic then has assertions for
describing this ghost state.  For example, for the Txn layer, the assertion $\cc{a}
\txnmapsto b$ says that address $a$ contains the value $b$ in the ghost
transaction system's state. In addition, there are \emph{thread points-to}
assertions, written $j \thdmapsto \cc{e}$, which says that thread $j$ in the
specification program is executing program $\cc{e}$. Perennial has rules
for updating the ghost state by ``executing'' these ghost threads, such as:
\[ (\cc{a} \txnmapsto b) \sep (j \thdmapsto \cc{Write(a, b')}) \vdash
\cc{a} \txnmapsto b' \]
which updates the value stored at $\cc{a}$ as a result of a write.

To establish the refinement, the proof engineer first defines a
\emph{representation invariant} $I$, an assertion in the logic that describes a
relation between the specification state and the implementation state.
Perennial's proof rules ensure that this designated invariant must hold before
and after each step of a program throughout the proof. Next, the proof engineer
proves a Hoare triple for each operation $o$ of $\gooselayer{S}$ and its
corresponding implementation $p_o$ in $\gooselayer{I}$:
\[
\hoare{(j \thdmapsto o) * \knowInv{}{I}}{p_o}{\lambda v.\, (j \thdmapsto v)}
 \]
Such a \emph{refinement triple} says that if a specification thread is executing
$o$ and an implementation thread is running $p_o$, then the representation
invariant $I$ is maintained and the value $v$ returned by running $p_o$ is a valid
return value of operation $o$. Typically, in such proofs, a ghost execution rule
is used at the linearization point of $p_o$, to mark
when the operation logically takes effect by executing $o$ in the ghost code.

These refinement triples imply a refinement between programs in $\gooselayer{S}$
and $\gooselayer{I}$. We prove refinement in Coq by applying the soundness
theorem for the Perennial framework.

In the particular case of the transaction system, the key refinement triple to
prove is for a block of code $f$ enclosed in transaction \cc{Begin} and
\cc{Commit} operations.
%$\operatorname{Atomically}$:
%
%\[
%\hoareV{j \thdmapsto \operatorname{Atomically}(f) * \knowInv{}{I}}
%      {C(\operatorname{Atomically}(f))}{v.\, j \thdmapsto v}
%\]
%
%where $C$ is the translation function described in \cref{sec:soundness-thm1}.
The difficulty in proving this triple is that the linearization point is at the
very end when the code calls \cc{Commit}, at which point the actual earlier
execution of $f$ becomes visible to other threads. The proof must show that
ghost-executing the specification's Atomically block at this point is valid by
tracking the behavior of $f$.

To show this, our proof maintains a stronger invariant during a transaction's execution. As the transaction executes, we track the
initial value of any objects accessed in a map $J$. The domain of this map
$\Sigma = \operatorname{dom}(J)$ is the \emph{footprint} of the transaction,
which two-phase locking keeps locked during the transaction. The intuition
behind the invariant is that if the transaction only depends on $J$, the
transaction's execution can be delayed to take place atomically at the call to
\cc{Commit} and its behavior will be the same since the subset of the journal
$J$ is the same.

More formally, the proof constructs a second simulation relation during the
execution of a transaction $\cc{f}$.  Let $J$ be a map giving the values of each
object in the transaction's footprint $\Sigma$ at the first time they are
accessed by $\cc{f}$, and let $J'$ be a mapping giving the transaction's current
buffered in-memory view of the same addresses.  Then, the invariant requires
that after $n$ steps of execution:
%
\begin{enumerate}

\item The transaction holds the lock for every address $a \in \Sigma$.

\item Executing $n$ steps of $\cc{f}$ in \emph{any} starting state that has the same
  values as $J$ for the addresses in $\Sigma$ can lead to a state with values given
  by $J'$.

\end{enumerate}
%
At the start of a commit, the locking described by the first part of the
invariant ensures that the durable value of each address still match the values
in $J$. The second part of the invariant means that even though other
parts of the state outside of $\Sigma$ may have changed, those changes do not
affect execution of $\cc{f}$. Thus, the ghost execution of $\cc{f}$ at this point will
have the same behavior as the implementation. The existing spec from GoJournal
for \cc{Commit} ensures that the durable values of objects in the footprint are
atomically updated to match $J'$.

Showing that the second part of the invariant holds requires that code within a
transaction must not access global state outside of the
transaction system, as mentioned in \cref{sec:proof:linking}. Accesses to such global state
would violate the invariant because their behavior would then depend upon
things outside of the footprint $\Sigma$. Because those global values could change
by the time the transaction commits, the above argument would no longer work if they were allowed.

The allocator creates another subtlety related to the second part of this
invariant. Allocations do not hold the allocator lock throughout the remainder
of a transaction. This seems to violate the two-phase locking pattern, since
allocations could be implicitly observed by other concurrent transactions from
the fact that an allocated address is no longer free. Correspondingly, in the
proof, the footprint $J$ of a transaction does not describe the allocator state.
Thus, at the ghost-execution linearization point, the addresses returned by the
allocator may no longer be free. However, because the specification for the
allocator does not guarantee that returned addresses are actually free, the
second part of the invariant above still holds.

% Suppose the transaction being simulated is $\operatorname{Atomically}{f}$, and the implementation
% has executed $n$ steps of $f$, accessing addresses in the set $\sigma$.

% Let $J_1 : \cc{addr} \rightarrow Obj$ be a map giving the durable state of the journal at the start of the transaction, and let $J_2$ be the current buffered in-memory view of the state from the perspective of the transaction.
% Then, the invariant requires that:
% \begin{enumerate}
% \item the transaction holds the lock for every address $a \in \sigma$
% \item for each $a$, $J_1(a)$ matches the ghost state value for the address $a$
% \item if $J_1'$ is a mapping such that $\forall a \in \sigma$, $J_1(a) = J_1'(a)$, then executing $f$ for $n$ steps with an initial journal $J_1'$
% \end{enumearte}
% \joe{this would probably be vastly improved by using math notation that matches whatever will have been used in the previous two sections}
