This chapter is about Goose, which solves a practical
problem of reasoning about efficient code. To use Goose, a developer writes
code in Go, uses an automatic translator to convert the code to a model in the Coq proof
assistant, then carries out the proof on top of the model in the Perennial logic. Goose encompasses
the entire process: it includes the translation tool itself, the way it models Go
code, and finally the reasoning principles for proving properties of translated
Go. (We will also use ``Goose'' in some places to refer to the subset of Go
supported by the translation tool.)

This description of Goose is written to be accessible to someone relatively new
to verification or systems research. The hope is that others can learn from this
approach to modeling and verifying code, and apply it to other languages and
domains. Some familiarity reading programming language semantics will be helpful
to appreciate \cref{sec:goose:lang}, but fully understanding this section is not
needed to understand the overall idea and reasoning principles in the other
sections. \Cref{sec:goose:reasoning} gives specifications in separation logic,
starting with enough introduction to understand their intuitive meaning.

The chapter is intentionally fairly independent of the rest of the thesis, in
case the reader is not interested in the specifics of the file system or the
Perennial logic, and since there is no published paper describing the details of
how Goose works or its reasoning principles. Concurrency is an important part of
the Goose model, though pleasantly enough it doesn't complicate the sequential
parts of Goose. Crash-safety concerns do not show up here since crashes are
modeled simply by stopping execution and wiping out all of the state except for
the disk, which does not relate to the specifics of Go. Goose is not specific to
GoTxn and can handle a much broader range of Go than that single codebase.

\section{Goals and motivation}
\label{sec:goose:goals}

There are three main goals for Goose: \textbf{flexibility} in writing code,
\textbf{soundness} for the translation process, and \textbf{convenient
reasoning}. Flexibility is a goal so that the developer can write efficient,
high-performance code, as long as the proof engineer can also reason about it. Goose
is sound if the model captures the behaviors of the code, which is required for
the proofs to say something meaningful about the executable Go code. If the code
could go wrong in some way that the model doesn't capture, we might finish a
correctness proof but still have a bug; soundness makes sure this doesn't
happen. Finally, Goose ships with reasoning principles for how to prove
correctness of code using Go primitives like structs, slices, and maps, and aims
to make these easy to use.

There are alternate setups for verification where the connection between the
proofs and the code is not given by a model and translation, but the Goose
approach prioritizes flexibility in order to write fast code. The transaction
system sometimes does something efficient to shave off a bit of time, and its
proof pays the price by including a more complex correctness argument.

Soundness isn't a property we can immediately measure, since the translation
might miss subtleties of the source code that we have not noticed. However, the
design of Goose tries to achieve soundness through simplicity and careful choice
of what to model. The tradeoff between simplicity and expressivity was generally
made in favor if getting good performance and not always to write idiomatic
code. That is, if a language feature is needed for good performance (for
example, taking pointers to individual struct fields), then Goose models it. If
a feature would only result in more idiomatic code and modeling it seems too
subtle, then it might not be implemented (for example, simple uses of
\texttt{defer} could be modeled but aren't because the feature is complicated in
general). The result is that Goose is generally pleasant and productive enough
to write in, but requires some practice for a Go programmer.

Convenient reasoning remains a goal for Goose, but not one that was always
achieved. All of the verified libraries are usable but pain points remain; some
of these are simply a matter of engineering effort or fixing bugs, but we have
also found code patterns that weren't well captured in the reasoning.

\subsection{Why Go?}

Go turns out to be a very convenient language for building verification
infrastructure. It is productive enough to build systems that get good
performance. The language is simple, facilitating a sound translation.

For our first goal, flexibility, Go has enough features to build good systems
in. It has efficient and useful built-in slice and map collections, while we can
verify data structures writtn on top. The runtime handles concurrency
efficiently and has good support for synchronization using locks and condition
variables, allowing a low-level implementation.

There is also an advantage to Go as a programming environment rather than
programming language. The tooling for testing, debugging, and profiling is
extremely good, making it easy to fix bugs (before verification or in unverified
code) and find performance problems while optimizing. We were able to use
low-level interfaces to the operating system to access the disk, which even
helps with soundness since it's easier to understand the OS in isolation rather
than when combined with a library. Garbage collection simplifies some code and
carries a relatively low performance impact due to the efficient runtime.

For the second goal, soundness, it helps that Go is a simple language. The Goose
translator effectively gives a semantics to the source code; in a complex
language this can be a daunting task (such as attempts to formalize JavaScript and
Python~\cite{guha:lambda-js,politz:python-semantics}). It isn't too
difficult to give Go a semantics, especially the Goose subset. Go's tooling
helped, including libraries for parsing and type-checking Go source code. Not
only do these libraries save time in implementing Goose, they greatly improve
reliability since they are written by experts (the Go compiler team, extracting
code from the compiler itself).

\subsection{Why not C?}

C would actually make sense, but we felt Go had a better cost-benefit tradeoff
for GoTxn. Using C would have the benefit of supporting even lower-level code,
with manual memory management. C code is also easy to integrate into the kernel,
if one wanted to deploy a verified component of an operating system. The main
cost would be more difficult tooling to parse and typecheck C; the translation
might need to be built with clang or some other existing tool. It would make
sense to adopt the Goose approach to C, and a C version would probably retain
similar reasoning principles and restrictions on the source code.

\subsection{Why not Rust?}

Using Rust as a source language seems attractive but would likely not be much
better than Go. One subtlety is that while the source code is type checked, the
model is an untyped program. If the verification engineer wanted to somehow take
advantage of this type checking, it would be extremely difficult to express the
assumption that the source code is well-typed without making a mistake,
particularly for the guarantees of the Rust type system; it would be easier to
simply re-prove whatever memory safety the type system guarantees afresh, as we
end up doing in Goose. Type safety would still help write the original Rust
program correctly, and especially when it comes to ownership Rust could enforce
some discipline and catch errors before even reaching verification.

Another difficulty with Rust would be the size and complexity of the language
--- the subset supported might be restrictive enough that the experience no
longer feels like Rust and you might as well be writing C, except for the
lifetime system and tooling. For example, the \cc{Vec<T>} type has 152 methods
without even including trait implementations. Using any of these methods would
require assuming a semantics for it, which is trusted to be sound (that is,
getting the semantics wrong could compromise the whole verification). Thus in
practice the expansive standard library would mostly not be available; for
soundness only a core subset would be modeled and the rest manually implemented
and verified.

\section{Related work}%
\label{s:goose:rel-work}

There are several areas of related work. First and foremost Goose is an approach
to verifying executable programs, so this section discusses alternate
approaches. Second, Goose implicitly gives a semantics to Go, and there are
related projects giving semantics to other programming languages.

\subsection{Verification approaches}

At a high level, all verification systems need to solve the problem of
connecting the world of the proof assistant to the ``real world'' that runs the
code. The proofs are always over a model of the code, and the theorems are
always contingent on the assumption that the running code has been modeled correctly. The approach
imposes some requirements on the code that can be written and how proofs are
written. Thus each approach makes a tradeoff between how efficient verified code
can be, soundness, and
convenient reasoning, as outlined above in \cref{sec:goose:goals}.

There are basically three ways to verify programs (with several interesting
caveats). We can translate code to a model (as in Goose), pretty-print a
model to code, or compile from a language intended for both verification and
implementation. In addition, verified compilation and translation validation can
implement parts of the translation process from model to code in a verified way;
this can usefully make reasoning more convenient without sacrificing soundness.

It is easiest to understand this characterization in the context of concrete
examples. Goose translates from Go code to a model in Coq. Both
CFML~\cite{chargueraud:cfml} and hs-to-coq~\cite{spector-zabusky:hstocoq} work
in a similar way, for OCaml and Haskell respectively.
Fiat-Crypto~\cite{erbsen:fiat-crypto} prints a model of C code to a string that
is compiled with a C compiler. Programs in Dafny~\cite{leino:dafny} are
typically run using Dafny's built-in compiler (which has backends for C\#, Java,
JavaScript, and Go at the time of this writing). Similarly, languages like Coq
and \fstar have support for \emph{extraction} that translate functional programs
in those languages to something executable like OCaml.

There are approaches that are somewhere in between using built-in compilation
and writing code in a model. For example, FSCQ~\cite{chen:fscq} uses Coq
extraction, but it works from a model of I/O interaction that is implemented by
combining the extracted code with a Haskell library. This combines Coq's
extraction (compilation) with translating from a model to code. For \fstar,
there is a specialized toolchain called KaReMeL~\cite{protzenko:lowstar} that
targets a subset of \fstar called \lowstar and extracts imperative C code.

Verified compilation can reduce trust when using either approach.
VST~\cite{cao:vst-floyd} in its most basic form looks like a code-to-model
translation, using a tool called \cc{clightgen} to translate C code to an
abstract syntax tree (AST) in Coq, which the user can then specify and verify in
VST. However, it is then possible to run this AST through the CompCert verified
compiler and produce assembly code (or more specifically, a model of assembly
represented in Coq). The proof than covers this assembly model, which is
pretty-printed and compiled with an assembler to run it. While the user writes C
code, this system only requires trusting the model of assembly. On the other
hand, the performance of the code is determined by what features \cc{clightgen}
supports and the quality of the CompCert compiler.

Verified compilation is used in a way similar to VST in
Vale~\cite{bond:vale,fromherz:vale-fstar}. Code is written in a specialized
assembly-like language called Vale, with proof annotations. This is then
compiled to Dafny or \fstar (Vale has two backends) where the proofs are carried
out before finally being printed back down to assembly. Vale supports highly
efficient code because it is written in assembly, but this code pays a price in
being harder to verify.

Cogent~\cite{amani:cogent,oconnor:cogent-lang} also uses a form of verified
compilation known as translation validation to make proofs easier without compromising on sounds. Code
written in the Cogent language (an imperative language with linear types) is
translated to a model in Isabelle/HOL along with a functional specification for
the code and a proof of correspondence. Thus the user can write proofs on top of
the functional specification, but the proofs go down to an imperative model.

\subsection{Modeling programming languages}

Goose gives a semantics to a subset of Go. Each function in Go is translated to
a term in GooseLang, which has a semantics in Coq in the form of a transition
system. Giving a semantics to a real-world language is interesting in its own
right, not just for verifying code in that language. Many similar semantics
efforts are focused on the goal of completeness, in order to understand how all
of the language features interact.

CH20~\cite{krebbers:c-coq} is a fairly complete formalization of the C standard.
CH20 defines C in operational and axiomatic styles, along with an executable
semantics to test the semantics on (small) concrete programs. All of these
semantics are formally related to each other. Having multiple, related semantics
helps give confidence in each of them, since they independently express the same
thing in different ways.

VST~\cite{cao:vst-floyd} uses CompCert C to model C code. One similarity to
Goose is support for structs in terms of their individual fields. Both systems
need to model the semantics of features like taking a pointer to an individual
struct field (as CH20 also handles). What sets VST and Goose apart from CH20 is
to also have reasoning principles for verifying code that uses structs in
interesting ways, such as structs where a subset of fields are protected by a
lock.

RustBelt~\cite{jung:rustbelt} is intended as a model of Rust code, at the Rust
MIR (mid-level IR) level of abstraction. The modeling language in RustBelt,
\lambdarust, has many similarities to GooseLang, and both are used together with
Iris~\cite{jung:iris-1} and the Iris Proof Mode~\cite{krebbers:ipm}. RustBelt
has different goals, being intended for reasoning about Rust as a whole rather
than program verification. This is a somewhat subtle difference: the goal in
RustBelt is to prove properties about \emph{all} \lambdarust programs, whereas
we only prove correctness of \emph{particular} GooseLang programs. As a result
it is important for \lambdarust, together with its type system, to rule out
programs that Rust does not allow, whereas GooseLang is more expressive than Go
with unsafe features like pointer arithmetic. Due to different goals, RustBelt
also does not have an automatic translation from Rust to \lambdarust, though
some libraries have been translated by hand and verified using Iris.

Finally, WebAssembly is a rare example of a production language with a formal
semantics~\cite{haas:wasm}, and moreover it was designed with the formal
semantics in mind.

\section{High-level overview}

The goal of the translation is to model a Go program using GooseLang,
which is a programming language defined in Coq for this purpose. When we
say GooseLang is a programming language, we mean it in a theoretical
sense: GooseLang consists of a type of programs in Coq and a small-step
semantics of these programs. Since GooseLang programs support references
to model the Go heap, the semantics is written in terms of transitions
of (program, heap) pairs where the heap maps pointers to values. The
intention of the translation is that the semantics of the translated
function should cover all the behaviors of the Go code, in terms of
return values and effect on the heap. As long as this is true, a proof
that the translated code always satisfies some specification means that
the real running code will, too.

GooseLang is a low-level language, so many constructs in Go translate to
(small) implementations in GooseLang. This implementation choice proved
to be much more convenient than adding primitives to the language for
every Go construct. For example, a slice is represented as a tuple of
pointer, length, and capacity, and appending to a slice requires
checking for available capacity and copying if none is available.
Appending to a slice is a complicated operation, and it was easier to
write it correctly as a program rather than directly as a transition in
the semantics. The one cost to this design strategy is that an arbitrary
GooseLang program is much more general than translated Go programs. This
has no impact on verifying any \emph{specific} Go program.

The extra generality of GooseLang does have some downsides since at one point in
the development, in the transaction system's specification, we refer to an
arbitrary GooseLang program. This theorem is made a bit more complicated since
to there are some ill-defined GooseLang programs that no Go program could
generate which the theorem needs to exclude. The transaction system
specification uses a standard technique of restricting to well-typed GooseLang
programs, and encoding syntactic restrictions in that type system.

\tej{add a diagram, perhaps like the ones in the Goose talk, showing the
translation process and how Go/Goose/GooseLang all fit together}

An important aspect of GooseLang is supporting interactive proofs on top
of the translated code. The interactive proofs use separation logic, a
variant of Hoare logic, so specifications describe the behavior of each
individual function. In order to support verification of any translated
code, GooseLang comes with a specification for any primitive or function
that the translated code might refer to, including libraries like slices
used to model more sophisticated Go features. GooseLang has many
``pure'' operations that have no effect on the heap, due to many
primitive data types and operations (for example, there are both 8-,
32-, and 64-bit integers, and arithmetic and logical operations for
each). The specifications for these operations are handled with a single
lemma, which is applied automatically with a tactic \cc{wp_pures}.

Since our goal is to support interactive rather than automated proofs,
it is helpful to make the model simple to work with. We try to maintain
a strong correspondence between the model and source code: each Go
package translates to a single Coq file, and each top-level declaration
in the Go code maps to a Gallina definition (a GooseLang constant or
function). Goose has a special case for translating immutable variables
to let bindings in GooseLang (rather than allocating a pointer that will
only be read). As a result, factoring out a sub-expression to a variable
has little impact on proofs, since it just adds one more pure step.

While the model is simple in terms of control flow and structure, we can
safely translate any given Go operation to a sophisticated model as long
as the proof abstracts it away. The subsequent sections in this chapter
walk through several features of Go. In each case we first implement the
feature in GooseLang, which as a model of its behavior primarily aims to
be faithful to Go. Next, we develop reasoning principles for the
features, in the form of separation logic assertions (for example, to
represent a slice) and Hoare triples (for example, to specify the
behavior of Append). The key is that the model is trusted to capture
Go's behavior so some sophistication is useful, whereas the reasoning
principles aim to hide that complexity to make proofs practical.

\section{GooseLang syntax and semantics}%
\label{sec:goose:semantics}

\newcommand{\goosedef}[1]{\mathsf{#1}}
\newcommand{\goosekw}[1]{\goosedef{\textcolor{blue}{#1}}}
% spacing for application
\newcommand{\app}{\:}
% any pure binary op
\newcommand{\binop}{\circledcirc}
% any pure unary op
\newcommand{\unop}{\circleddash}

\newcommand{\external}{\mathsf{\textcolor{red}{\langle External \rangle}}}

\newcommand{\gooseif}[3]{\goosekw{if} \app #1 \app%
  \goosekw{then} \app #2 \app \goosekw{else} \app #3}

\newcommand{\recfx}{\goosekw{rec} \, f(x) = e}
\newcommand{\gooselambda}[1]{\goosekw{\lambda}#1.\,}

\newcommand{\reduces}{\rightsquigarrow}
\newcommand{\purereduction}{\overset{\mathrm{pure}}{\reduces}}

\newcommand{\seq}{;\,}
\newcommand{\defeq}{\triangleq}

GooseLang is an effectful, untyped lambda calculus, with mutable references and
concurrency. Defining the language involves defining a \emph{syntax} in the form
of expressions, and a \emph{semantics} for how those expressions execute in the
form of a transition system. The transitions also require some state, for
example to track the value stored at each allocated pointer. This language has
low-level primitives for modeling an imperative program, with nothing terribly
Go-specific; we will give a high-level overview in this section and details in
\autoref{sec:goose:reasoning}.

The basic unit of the language is the expression $e$, which supports recursive
functions $\recfx$ and basic data types like tuples $(e_1, e_2)$. Then, we give
these expressions a semantics in terms of the reduction relation
$(\sigma, e) \reduces (\sigma', e')$ which says that $e$ reduces to (or executes
to) $e'$ in the state $\sigma$, and the resulting state is $\sigma'$. This
reduction relation is eventually lifted to not just a single expression but a
whole \emph{threadpool}, a list of concurrently executing expressions. You
can think of each expression as representing the computation of a single thread,
and each reduction step as being running the thread for one transition. The
granularity of these transitions corresponds to what is considered atomic
between threads. For a reader unfamiliar with operational semantics for the
lambda calculus, X is a good reference \tej{figure out what to recommend for
  lambda calculus}.

\begin{figure}[hp!]
  \textbf{Syntax}
  \begin{mathpar}
  \begin{array}{llcl}
    &x,f &\in &\textdom{Var} \\
    &\ell &\in &\textdom{Loc} \\
    &n &\in & \textdom{U64} \cup \textdom{U32} \cup \textdom{U8} \\
    &s &\in &\textdom{String} \\
    &\binop &::= & + \ALT - \ALT * \ALT = \ALT < \ALT \dots \\
    &\unop &::= & - \ALT \goosekw{ToString} \ALT \goosekw{ToU64} \ALT
                         \dots \\
    \textdom{Var}& v &::= & () \ALT \goosekw{true} \ALT \goosekw{false} \ALT n
                            \ALT \ell \ALT s \\
    &&\ALT & \goosekw{inj}_1 \app v \ALT \goosekw{inj}_2 \app v \ALT
             \recfx \\
    \textdom{Exp}& e &::= & x \ALT v \ALT e \app e \ALT \recfx \\
    &&\ALT & \gooseif{e}{e}{e} \\
    &&\ALT & \goosekw{Fork}(e) \\
    &&\ALT & (e, e) \ALT \pi_1 \app e \ALT \pi_2 \app e \\
    &&\ALT & \goosekw{inj}_1 \app e \ALT \goosekw{inj}_2 \app e
    \ALT %
      \goosekw{case} \app e \app \goosekw{of} \app \goosekw{inj}_1 \app x
      \Rightarrow e \app \goosekw{or} \app \goosekw{inj}_2 \app x \Rightarrow e
    \\
    &&\ALT & e \binop e \ALT \unop e \ALT \goosekw{ArbitraryU64} \\
    &&\ALT & \goosekw{AllocN}(e, e)
             \ALT \goosekw{CmpXchg}(e, e, e) %
             \ALT \goosekw{Load}(e) \\
    &&\ALT & \goosekw{PrepareWrite}(e) %
             \ALT \goosekw{FinishStore}(e, e) %
             \ALT \goosekw{StartRead}(e) %
             \ALT \goosekw{FinishRead}(e) \\
    &&\ALT & \goosekw{Panic}(s) \\
    &&\ALT & \external
  \end{array}
  \end{mathpar}

  \textbf{Derived forms and notation}
  \begin{mathpar}
  \begin{array}{rcl}
    \gooselambda{x} e &\defeq & \goosekw{rec} \: \_(x) = e \\
    \gooselambda{x, y} e &\defeq & \gooselambda{x} \gooselambda{y} e \\
    \goosekw{let} \: x = e_1 \app \goosekw{in} \app e_2 %
                             &\defeq &%
                                   (\goosekw{\lambda} x.\, e_2) \app e_1 \\
    e_1\seq e_2 &\defeq &%
                   \goosekw{let} \: \_ = e_1 \app \goosekw{in} \app e_2 \\
    \goosekw{ref} \app e &\defeq &\goosekw{AllocN}(1, e) \\
    !x &\defeq & \goosekw{Load}(x) \\
    \goosedef{Store} &\defeq & \gooselambda{x, e} \goosekw{PrepareWrite}(x);
                               \goosekw{FinishStore}(x, e) \\
    x \gets e &\defeq & \goosedef{Store} \app x \app e \\
    % \goosekw{do}\,\goosekw{while} \app e &\defeq &%
    %                         (\goosekw{rec} \: loop(\_) = \gooseif{e \app ()}{loop
    %                         \app ()}{()}) \app () \\
  \end{array}
  \end{mathpar}
  \caption{GooseLang syntax}
  \label{fig:goose:syntax}
\end{figure}

The syntax for GooseLang programs is given in \autoref{fig:goose:syntax}. The
important top-level definition is $e \in \textdom{Exp}$, giving expressions in
GooseLang. Before getting to the expressions, it's worth noting that the values
in GooseLang are designed for Go's primitive data, in particular with
first-class support for bytes and 32-bit and 64-bit unsigned integers. We did
not need signed integers and so did not model them, but adding them would simply
require extending the type of literals and adding many more pure operations to
implement various integer conversions.

The basic lambda calculus primitives are given on the first line: variables $x$,
values $v$, function application $e \app e$, and (recursive) functions $\recfx$.
The first few derived forms encode more basic primitives like a non-recursive
lambda, let bindings, and sequencing, all on top of recursive functions. The
language has two types of composite data: products (used pervasively to model
structs), and sums (used only to model maps). There is also $\goosekw{if}$ as the
main control-flow primitive, and $\goosekw{Fork}$ to create concurrent threads.
Finally, there are many primitives for the heap, which are discussed in more
detail later.

We can build intuition for both the syntax and semantics of GooseLang by looking
at some very simple Go programs and their translations into GooseLang. First,
let's look at some \emph{pure} functions that don't use pointers:

\newenvironment{translatego}{
  % get proper top alignment
  % https://tex.stackexchange.com/questions/378548/vertical-alignment-of-side-by-side-minipages
  \begin{minipage}[t]{0.5\textwidth}
  \strut\vspace*{-\baselineskip}
}{
  \end{minipage}
}

\newenvironment{translategooselang}{
  \begin{minipage}[t]{0.5\textwidth}
  \strut\vspace*{-8pt}
}{
  \end{minipage}
}

\begin{translatego}
\begin{minted}{go}
func Midpoint(x uint64,
              y uint64) uint64 {
  return (x + y) / 2
}
\end{minted}
\end{translatego}
%
\begin{translategooselang}
  \begin{flalign*}
  &\goosedef{Midpoint} \defeq \gooselambda{x, y} (x + y) /2 &
  \end{flalign*}
\end{translategooselang}

\begin{translatego}
\begin{minted}{go}
func Max(x uint32,
         y uint32) uint32 {
  if x > y { return x }
  return y
}
\end{minted}
\end{translatego}
%
\begin{translategooselang}
  \begin{flalign*}
  &\goosedef{Max} \defeq \gooselambda{x, y} & \\
  &\quad \gooseif{x > y}{x}{y} &
  \end{flalign*}
\end{translategooselang}

\begin{translatego}
\begin{minted}{go}
func Arith(a uint64,
           b uint64) uint64 {
  sum := a + b
  if sum == 7 { return a }
  mid := Midpoint(a, b)
  return mid
}
\end{minted}
\end{translatego}%
%
\begin{translategooselang}
\begin{flalign*}
  &\goosedef{Arith} \defeq \gooselambda{a, b} & \\
  &\quad\goosekw{let} \app sum = a + b \app\goosekw{in}\app & \\
  &\quad\goosekw{if} \app sum = 7 \app\goosekw{then}\app a \app\goosekw{else} & \\
  &\quad\quad \goosekw{let} \app mid = \goosedef{Midpoint}\app a \app b \app\goosekw{in} &\\
  &\quad\quad mid &
\end{flalign*}
\end{translategooselang}

Notice that the translation maps each Go function to a GooseLang definition. For
readability the GooseLang definitions are written mathematically, but what the
Goose tool actually emits is a Coq file where each definition is a Gallina term
of type \cc{expr} (the type of GooseLang expressions). This is what allows
definitions to refer to each other, such as how $\goosedef{Arith}$ calls the
previously defined $\goosedef{Midpoint}$.

The next interesting feature of GooseLang is support for pointers. In this
aspect the language is a bit unusual; we get to the reasons why in
\autoref{sec:goose:pointers}. For now, it is sufficient to think of $x \gets e$
as the usual pointer store operation, and $!x$ as the usual load. Here are a couple
examples to illustrate:

\begin{translatego}
\begin{minted}{go}
func Swap(x *uint64,
          y *uint64) {
  tmp := *x
  *x = *y
  *y = tmp
}
\end{minted}
\end{translatego}
%
\begin{translategooselang}
\begin{flalign*}
  &\goosedef{Swap} \defeq \gooselambda{x, y} & \\
  % these {} are needed so LaTeX spaces the ! correctly, rather than putting it
  % next to the = or \gets
  &\quad\goosekw{let} \app tmp = {} !x \app\goosekw{in}\app \\
  &\quad x \gets {} !y \\
  &\quad y \gets tmp\\
\end{flalign*}
\end{translategooselang}

\begin{translatego}
\begin{minted}{go}
func NewPtr() *uint64 {
  return new(uint64)
}
\end{minted}
\end{translatego}
%
\begin{translategooselang}
\begin{flalign*}
  &\goosedef{NewPtr} \defeq \gooselambda{\_} \goosekw{ref} \app 0
\end{flalign*}
\end{translategooselang}

The $\goosedef{NewPtr}$ definition might need some explanation. First, even
though the Go code takes no arguments, the GooseLang expression takes an unused
argument; this is so that $\goosedef{NewPtr}$ is syntactically a function, a
requirement of the way GooseLang is encoded in Coq.\footnote{For expert readers,
GooseLang functions are actually \emph{values} and not expressions.} Second,
this definition allocates a pointer with $\goosekw{ref}$, using
$\goosekw{AllocN}$ which more generally allocates $n$ contiguous pointers. The
initial value of 0 is the ``zero value'' of the \cc{uint64} type, a promise made
by Go.

The Goose syntax includes an $\external$ alternative. The language, both syntax
and semantics, are parameterized over an external interface of operations, for
interacting with the outside world. For example, in GoTxn the code uses an
interface of external disk operations, presented separately in
\autoref{fig:goose:disk-ffi}. The parameterization is primarily useful for
refinement proofs that relate Goose with one set of external operations to
another (for example, we use this to specify GoTxn). Other than that use case,
the reader can imagine that the external rules are simply part of the definition
of expressions and the semantics.

Goose also has a formal, small-step operational semantics; this style of
specification and indeed most of the semantics is quite standard for languages of
this type. The full details are presented in \autoref{fig:goose:semantics}.

\begin{figure}[hp]
  \textbf{Pure reduction}%
  \hfill %
  \boxedassert[line width=0.4pt]{e \purereduction e'} %
  \hspace{20pt}

  \begin{mathpar}
  \begin{array}{rcll}
    v \binop v' &\purereduction & v'' & \mathrm{if} \: v'' = v \binop v' \\
    \unop v &\purereduction & v' & \mathrm{if} \: v' = \unop v \\
    \gooseif{\goosekw{true}}{e_1}{e_2} &\purereduction & e_1 \\
    \gooseif{\goosekw{false}}{e_1}{e_2} &\purereduction & e_2 \\
    \pi_i(v_1, v_2) &\purereduction &v_i \\
    \goosekw{case} \app \goosekw{inj}_i v \app\goosekw{of}\app %
    \goosekw{inj}_1 \app x_1 \Rightarrow e_1 \app\goosekw{or}\app%
    \goosekw{inj}_2 \app x_2 \Rightarrow e_2%
                &\purereduction & \subst{e_i}{x_i}{v} \\
    (\recfx) \app v &\purereduction & \subst{\subst{e}{f}{(\recfx)}}{x}{v} \\
    \goosekw{ArbitraryInt} &\purereduction &n & \forall n \in \textdom{U64} \\
  \end{array}
  \end{mathpar}

  \textbf{Per-thread head reduction}%
  \hfill %
  \boxedassert[line width=0.4pt]{((h, w), e) \reduces ((h', w'), e') } %
  \hspace{20pt}
  %
  \begin{mathpar}
  \begin{array}{llcl}
    \textdom{NonAtom}& z &::= & \goosekw{Reading} \app n \app v \ALT
                                \goosekw{Writing} \app v \\
    \textdom{Heap}& h &\in& \textdom{Loc} \overset{\mathrm{fin}}{\to}
                        \textdom{NonAtom} \\
    \textdom{World}& w & & \external  \\
    %\textdom{State}& \sigma &::= (h, w) \\
    \textdom{TPool}& \mathcal{E} &\in &\textdom{List}\app\textdom{Exp} \\
    %\textdom{Config}& \rho &::= (\sigma, \mathcal{E}) \\
  \end{array}

  \begin{array}{rcll}
    ((h, w), e) &\reduces & ((h, w), e') &\mathrm{if} \: e \purereduction e' \\
    ((h, w), \goosekw{CmpXchg}(\ell, v_1, v_2)) &\reduces %
                          &((h\mapupd{\ell}{v_2}, w), (v, \goosekw{true}))%
                                         &\mathrm{if} \: h(\ell) =
                                           \goosekw{Reading} \app 0 \app v \land
    v = v_1 \\
    ((h, w), \goosekw{CmpXchg}(\ell, v_1, v_2)) &\reduces %
                          &((h, w), (v, \goosekw{false}))%
                                         &\mathrm{if} \: h(\ell) =
                                           \goosekw{Reading} \app 0 \app v \land
                                           v \neq v_1 \\
    ((h, w), \goosekw{AllocN}(n, v)) &\reduces %
                          &((h[\ell + i \mapsto v \mid 0 \leq i < n], w), \ell) %
                                         &\mathrm{if} \: \forall 0 \leq i < n,
                                           \, \ell + i \notin \dom(h) \\
    ((h, w), \goosekw{PrepareWrite}(\ell)) &\reduces %
                          &((h\mapupd{l}{\goosekw{Writing} \app v}, w), ()) %
                                         &\mathrm{if} \: h[\ell] = \goosekw{Reading} \app 0
                                           \app v \\
    ((h, w), \goosekw{FinishStore}(\ell, v)) &\reduces %
                          &((h\mapupd{l}{\goosekw{Reading} \app 0 \app v}, w), ()) %
                                         &\mathrm{if} \: h[\ell] =
                                           \goosekw{Writing} \app v' \\
    ((h, w), \goosekw{StartRead}(\ell)) &\reduces %
                          &((h\mapupd{l}{\goosekw{Reading} \app (n+1) \app v}, w), ()) %
                                         &\mathrm{if} \: h[\ell] =
                                           \goosekw{Reading} \app n \app v \\
    ((h, w), \goosekw{FinishRead}(\ell)) &\reduces %
                          &((h\mapupd{l}{\goosekw{Reading} \app n \app v}, w), v) %
                                         &\mathrm{if} \: h[\ell] =
                                           \goosekw{Reading} \app (n+1) \app v \\
    ((h, w), \goosekw{Load}(\ell)) &\reduces %
                          &((h, w), v) &\mathrm{if} \: %
                                         h[\ell] = \goosekw{Reading} \app n \app v \\
  \end{array}
  \end{mathpar}

  \textbf{Context reduction}%
  \hfill%
  \boxedassert[line width=0.4pt]{((h, w), \mathcal{E}) \reduces ((h', w'), \mathcal{E}')}%
  \hspace{20pt}

  \begin{mathpar}
  \begin{array}{llcl}
    \textdom{ECtx}& E &::= & \square \ALT e \app E \ALT E \app v  %
                             \ALT E \binop e \ALT v \binop E \ALT \unop E \ALT \\
                  &&\ALT & \gooseif{E}{e}{e} \\
                  &&\ALT & (E, e) \ALT (v, E) \ALT \pi_i \app E \ALT
                           \goosekw{inj}_i \app E \\
                  &&\ALT & \goosekw{AllocN}(E, e) \ALT \goosekw{AllocN}(v, E)
                           \ALT \dots
  \end{array}

    \inferH{context-reduce}%
    {((h, w), e) \reduces ((h,', w'), e') \\ \text{$E$ is an evaluation context}}%
    {((h, w), \mathcal{E}\mapupd{i}{E[e]}) \reduces%
      ((h', w'), \mathcal{E}\mapupd{i}{E[e']})}

    \inferH{fork-reduce}%
    {j \notin \dom(\mathcal{E}) \cup \{j\} \\ \text{$E$ is an evaluation context}}%
    {((h, w), \mathcal{E}\mapupd{i}{E[\goosekw{Fork}(e)]}) \reduces%
      ((h, w), \mathcal{E}\mapupd{i}{E[()]}\mapupd{j}{e})
    }
  \end{mathpar}

  \caption{GooseLang semantics}%
  \label{fig:goose:semantics}
\end{figure}

\paragraph{Structure of the semantics}
The semantics is a lot to take in, especially if you've never seen such a
presentation. The semantics is defined in three relations: the first, the
\emph{pure reduction relation} $e \purereduction e'$, is the easiest to understand
because it describes ``pure'' expressions that do not depend on any external
state. The semantics of such expression is simply to ``reduce'' to simpler
expressions, eventually reaching values. The semantics of function application
is pure. We use $\subst{e}{x}{v}$ as notation for \emph{substituting} the expression
$v$ for the variable $x$ in $e$. To implement recursion, the function definition
is itself substituted for its name.

Next, we define the semantics of the heap operations. This relation is over
$((h, w), e)$. $h$ is the heap and $e$ is the expression being executed; $w$ is
the type of an external ``world'' that is used to define the semantics of
external operations. Even the type of this world state is a parameter at this
level, and is defined as part of defining a set of external operations and their
semantics. The specific case of the disk is presented in
\autoref{fig:goose:disk-ffi}.

Finally, we define the full semantics over $((h, w), \mathcal{E})$. Instead of a
single expression, we now have a \emph{threadpool}, a list of expressions, all
running concurrently. This semantics is presented in a standard style using
evaluation contexts. The previous relation involving a single expression
is called a \emph{head reduction relation} because it only gives the semantics
of expressions where the relevant operation is at the top level or ``head'' of
the expression, and when applied to values (notice for example that the rule for
$\goosekw{AllocN}(n, v)$ is only defined when applied to a number and a value,
not other expressions). Context reductions are how those arguments get executed
when we run something like $\goosekw{AllocN}(10, \goosekw{true || false})$.

Context reductions are defined by a type of evaluation contexts
$\textdom{ECtx}$. Rather than going through the details of how these are used
(the rule \ruleref{context-reduce}), here I'll just explain intuitively what
evaluation contexts are. Every context $E$ is an expression with a ``hole'' in
it, indicating where the next reduction should take place. The entire expression
might be a hole $\square$. As an example, the pair of contexts $E \binop e$ and
$v \binop E$ determine how all binary operators work (for example, $\binop$
might be $+$). These say that if we have an expression $e_{1} \binop e_{2}$, we can
always run the argument $e_{1}$ (due to the first context), but it must execute
to a value before we evaluate the second argument (the second context). Once
both are values, the pure reduction for binary operators will apply. The result
is a left-to-right evaluation order for binary operators.

GooseLang does have one extant bug related to evaluation contexts. The contexts
$e \app E$ and $E \app v$ define a right-to-left evaluation order for functions,
which is the opposite of Go. We haven't yet fixed this, either by adjusting the
GooseLang semantics or changing the translation to emit code that explicitly
evaluates all the arguments in the correct order before calling the function.

\begin{figure}[ht]
  \textbf{Disk operations}
  \begin{mathpar}
  \begin{array}{ccc}
    e &::= & \dots \ALT \goosekw{DiskRead}(e) \ALT \goosekw{DiskWrite}(e, e)
             \ALT \goosekw{DiskSize} \\
  \end{array}
  \end{mathpar}
  \textbf{Disk external semantics}
  \begin{mathpar}
  \begin{array}{lccc}
    \textdom{Block} & b &\in& \textdom{Vec} \app 4096 \app \textdom{U8} \\
    \textdom{World} & w &\in & \mathbb{N} \overset{\mathrm{fin}}{\to}
                               \textdom{Block} \\
  \end{array}
  \end{mathpar}

  \begin{mathpar}
    \infer{\forall 0 \leq i < 4096,\, \ell + i \notin \dom(h) \\ w[a] = b}%
    {((h, w), \goosekw{DiskRead}(a)) \reduces%
      ((h[\ell + i \mapsto \goosekw{Reading} \app 0 \app b[i] %
      \mid 0 \leq i < 4096], w), \ell)}

    \infer{\forall 0 \leq i < 4096,\, \exists k.\, h[\ell + i] =
      \goosekw{Reading} \app k \app b[i] }
    {((h, w), \goosekw{DiskWrite}(a, \ell)) \reduces %
      ((h, w[a \mapsto b]), ())}

    \infer{}%
    {((h, w), \goosekw{DiskSize}) \reduces %
    ((h, w), 1 + \textlog{max} \app \dom(w))}
  \end{mathpar}
  \caption{Syntax and semantics for an external disk.}
  \label{fig:goose:disk-ffi}
\end{figure}

\clearpage

\section{Modeling and reasoning about Go}%
\label{sec:goose:reasoning}

\subsection{Modeling pointers}%
\label{sec:goose:pointers}

Pointers turn out to be slightly subtle because of concurrency. In
short, GooseLang disallows concurrent reads and writes to the same
location by making such racy access undefined behavior (any
specification for a program implies that if the precondition holds, the
program never exhibits undefined behavior). The hardware provides some
guarantees, but they are relatively weak: for example, different cores
can observe writes at different times. Go's own memory model specifies
even weaker guarantees. Rather than attempt to formalize Go's rules
(which are complex and involve defining a partial order over all program
instructions), we side step the issue and make any races undefined,
which works for our intended use cases since we always synchronize
concurrent access with locks.

To disallow concurrent reads and writes we first detect them. The key is to make
$x \gets v$, the primitive store operation, \emph{non-atomic} by splitting it
into two operations. The GooseLang semantics tracks the behavior of these
operations by augmenting the heap with extra information; each address in the
semantics has a $\textdom{NonAtom}$ which can be
$\goosekw{Reading} \app n \app v$ if there are $n$ readers and the value is $v$,
or $\goosekw{Writing} \app v$ if a thread is currently writing. We actually only
use non-atomic stores for normal pointers, but we have support for non-atomic
reads as well for iterating over maps.

Ordinarily values in the heap are of the form $\goosekw{Reading} \app 0 \app v$,
to indicate no readers or writers. Writes are split into
$\goosekw{PrepareWrite}(\ell)$, which sets the value of $\ell$ to
$\goosekw{Writing} \app v_{0}$, and $\goosekw{FinishStore}(\ell, v)$ which sets
it to $\goosekw{Reading} \app 0 \app v$. A concurrent write will be undefined
since $\goosekw{PrepareWrite}(\ell)$ requires no concurrent writers, and
similarly for a concurrent read which is undefined if the address is being
written. Non-atomic reads are similar with $\goosekw{StartRead}$ and
$\goosekw{FinishRead}$; these increment and decrement the number of readers,
respectively, so that multiple readers can run concurrently but any concurrent
writer has undefined behavior.

Next, we need reasoning principles to abstract away this complexity from
program verification. Separation logic turns out to provide the right
language to reason about racy access. When a thread owns
$l \mapsto v$, we know no other thread can have access to location
$l$, so the specifications for reads and writes are unaffected by the
operations being non-atomic (although their proofs are a bit more
complicated to deal with the new semantics). The only change is that the
Read operation is no longer an atomic primitive but a function that
takes two execution steps. In Iris this means that two threads cannot
share memory with an invariant and must mediate access with a lock,
which transfers ownership of the $l \mapsto v$ for multiple execution
steps.

\subsection{Locks}

\newcommand{\Acquire}{\goosedef{Acquire}}
\newcommand{\CAS}{\goosedef{CAS}}

As is typical in Goose, locks are not built-in to GooseLang but modeled
using an implementation based on simpler primitives. Since locks are the
only synchronization primitive, implementing them requires shared
concurrent access, which ordinary pointers do not have in GooseLang.
Instead, the language also includes a primitive atomic compare-and-swap
operation that is only used to implement a model of locks. We could also
use the same operation to model Go's low-level atomic operations, like
\cc{atomic.CompareAndSwapUint64} and \cc{atomic.LoadUint64}, but
have not implemented this yet since we don't have code that uses these
low-level synchronization primitives.

The model of locks is simple enough to give the code in its entirety. The lock
is represented as a pointer to a boolean that is true if the lock is held. As a
helper we define $\CAS$ (compare-and-swap), a variant of compare-and-exchange
that only returns a boolean on success and not also the previous value.

\begin{align*}
  \CAS &\defeq \gooselambda{x, v1, v2} \pi_{2}\app \goosekw{CmpXchg}(x, v1, v2) \\
  \goosedef{NewLock} & \defeq \gooselambda{\_} \goosekw{ref} \app \goosekw{false} \\
  \Acquire &\defeq \gooselambda{l} \\
       &\goosekw{let}\app f = (\goosekw{rec}\: tryAcquire(\_) = \\
       &\qquad \gooseif{\CAS \app l \app \goosekw{true} \app \goosekw{false}}%
         {()}{tryAcquire \app ()}) \app\goosekw{in} \\
       &f \app () \\
  \goosedef{Release} &\defeq \gooselambda{l} l \gets \goosekw{false} \\
\end{align*}

Acquiring a lock is modeled as repeatedly using
$\CAS \app l \app \goosekw{false} \app \goosekw{true}$ to
atomically test that the lock is false and set it to true if so, while release
stores false to the lock. This implementation as a spin lock is merely an
operational model that captures what the lock does: acquire blocks until the
lock is free and sets it to locked, while release frees the lock. This code is
used to model Go's builtin \cc{*sync.Mutex}, which is implemented more
efficiently than spin locks with cooperation from the runtime and operating
system.

The reasoning principles for locks are more sophisticated than the spin-lock
implementation. As is typical in concurrent separation logic, we associate a
\emph{lock invariant} to the lock, which is a predicate that holds when the lock
is free. Because this is a separation logic, we can also interpret the lock
invariant as ownership over some data (for example, some region of memory); the
lock mediates access to this ownership, handing it out when the lock is acquired
and requiring it back when the lock is released. We prove this specification
sound against the GooseLang spin-lock implementation.

\subsection{Structs}

\newidentmacro{LoadTyped}
\newidentmacro{StoreTyped}
\newidentmacro{loadField}
\newidentmacro{storeField}

One of the most important features of Go to support is structs. Goose support
for structs uses a form of \emph{shallow embedding} using a combination of
Gallina (Coq's functional language) and GooseLang. We encode
higher-level primitives like field access on top of GooseLang primitives like
tuples and continguous memory.

First we need to handle struct values. We treat a struct value as just a tuple
of its fields. The definition of the struct gives an ordering of the fields.
This is enough to construct a struct from its fields and to access a field by
name. Here the shallow embedding comes in: we define struct declarations in
Gallina, and struct construction and field access are Gallina definitions that
produce GooseLang expressions, rather than all of this being directly
implemented in GooseLang.

Structs in memory are more interesting than struct values. Structs could
be stored in a single location; due to our non-atomic semantics for
memory, this would be sound even for structs larger than a machine word.
However, this model would be too restrictive: it is safe for threads to
concurrently access \emph{different fields}, just not the same field,
and we actually take advantage of this property (largely to write more
natural Go code; working around this restriction requires splitting
structs up if they are stored in memory).

To support this concurrency, we model a struct in memory with a
flattened representation, with each base element in a separate memory
cell. The flattening applies recursively to fields that are themselves
structs, until a base literal is reached (like an integer or boolean);
base elements are at most a machine word, but can be smaller. When
allocating a new pointer, the semantics flattens composite values and
stores the elements in a sequence of contiguous addresses.

\begin{figure}[ht!]
\begin{mathpar}
  \begin{array}{lrcl}
    \textdom{GooseType} & t &::=& \goosekw{uint64T} \ALT \goosekw{boolT} \ALT
                                  \dots \\
                        &&\ALT & t \times t \ALT t \to t \ALT \goosekw{ptrT} \\
  \end{array}
\end{mathpar}
\caption{GooseLang types. These are used in the semantics only to give a
  ``shape'' to data for accessing struct fields, and not to represent the Go
  type system.}%
\label{fig:goose:types}
\end{figure}

With a flattened representation we need non-trivial code to read a struct
through a pointer, particularly when some of its fields are themselves flattened
structs. We implemented this code by representing a struct declaration with not
only the fields, but their types as well; the representation of these Goose
types is given in \autoref{fig:goose:types}. The exact types are not important,
but we do need the entire tree of how big each field is and the shape of each
field in order to determine the location and extent of any given field. Using
types to represent these shapes makes the translation much simpler, since we
have access to the type of every sub-expression from the Go type checker. Any
load of a value from memory is translated to a call to $\LoadTyped$. This is a
Gallina definition of type $\textdom{GooseType} \to \textdom{Expr}$, where that
expression is itself a GooseLang function that takes a pointer. Hence you can
think of it is a macro for GooseLang that is represented in Gallina as a meta
language. $\LoadTyped(t)$ uses $t$ passed in Gallina to determine how to load
and assemble the fields of a struct of type $t$.

For the purpose of proofs we represent a pointer to an arbitrary type
$t$ with a typed points-to fact of the form $l \mapsto_t v$. This
definition expands to a number of primitive points-to facts, one for
each base element. The specification for loading says
$\hoare{\ell \mapsto_t v}{\LoadTyped(t) \app \ell}{\Ret{v} \ell \mapsto_t v}$, which
(much like the non-atomic primitive $\goosekw{Load}(\ell)$) hides the fact that something
non-atomic is happening and looks like an ordinary dereference.
Similarly, StoreTyped also takes a type, although the specification
requires the caller to prove that the value has the right shape (in
reality it always will because the Go code we translate from is
well-typed).

The payoff of structs being many independent locations is that it is
possible to model references to individual struct fields. From a pointer
to the root of the struct, a field pointer is simply an offset from that
pointer (skipping the flattened representations of the previous fields).
This offset calculation is much like the code to read a struct from
memory, except that it merely computes a single offset rather than
iterating over all the fields and offsets.

The Go language reference specifies that each field acts like an
independent ``variable'' (which is stored in the GooseLang heap when it
is mutable in Go), so this model should accurately reflect the
specification. Moreover modeling structs as independent locations is
also justified as being similar to how the implementation works. Structs
in memory are in reality represented by contiguous memory, and field
access is implemented by computing a pointer from the base of the
struct. The main difference between the physical implementation and the
model is that we use a single, abstract memory location for each field,
whereas the implementation encodes all data into bytes.

Recall that $l \mapsto_t v$ is internally composed of untyped
points-to facts for all the base elements of $v$. In order to reason
about $v$'s fields, we introduce a new struct field points-to fact,
written $l \mapsto_{t.f} v$, which asserts ownership of just field
$f$ of a struct of type $t$ rooted at $l$, and gives that field's
value as $v$. A recursive function gives an ``exploded'' set of struct
fields by iterating over $t$'s fields and $v$ simultaneously. Then,
we give a proof that $l \mapsto_t v$ is equivalent to the separating
conjunction of this exploded list. The result is a convenient lemma for
reasoning about a struct using its fields: in the forward direction, the
equivalence breaks a large typed points-to into individual fields (with
the values computed from $v$), while in the other direction it allows
to prove a $l \mapsto_t v$ by gathering up all the fields.

The struct field points-to is indispensable in proofs, because the
pattern of \cc{x.f} in Go when $x$ is a pointer is in fact a field
load (in C, this would be written \cc{x->f}). The model
for loading a struct field is a function $\loadField(t, f) \app x$
which is implemented in two steps, first computing the offset to field
$f$ and then dereferencing the computed pointer (in both cases the struct type $t$
describes how to interpret field $f$). Having a field points-to gives
a natural specification for this type of load:
$\hoare{\ell \mapsto_{t.f} v}{\loadField(t, f) \app \ell}{\Ret{v} \ell \mapsto_{t.f} v}$.

The lemmas about breaking apart and recombining structs are all proven
against a simpler model of structs that only requires flattening and
offset calculations. In a sense the model is the trusted code, but the
fact that the struct maps-to exploding lemma is true that all of the
expected Hoare triples hold provides strong evidence that the model is
also doing the right thing. For example, the exploding lemma shows that
field offsets are disjoint, since the struct maps-to can be broken into
field points-to facts for each field.

Something to emphasize above: all of the struct code is generic for
struct type $t$, which in the code is concretely the ``schema''
described above, a list of fields and types (the code calls this a
``descriptor'' and uses $d$ as the metavariable, to avoid confusion
with a generic type $t$). \tej{using $t$ for struct types rather than
  descriptors $d$ is very confusing, maybe we should actually use $d$ or $s$ for
struct descriptors}

\subsection{Slices}

One of the most commonly used data structures in Go is the slice
\cc{[]T}, which is a dynamically-sized array of values of type
\cc{T}. Goose supports a wide range of operations on slices,
including appending and sub-slicing; it turns out that the semantics of
mutable slices is non-trivial in Go, resulting in an interesting
semantics and reasoning principles.

A slice is a combination of a pointer, length, and capacity. Slices are
views into a contiguous memory allocation; this view can be narrowed
with sub-slicing operations of the form \cc{s[i:j]}, resulting
in aliased slices. The elements between the length and capacity are not
directly accessible but are used to support efficient amortized appends.
Go's built-in slice operations include bounds checks on all slice
operations and panic if a memory access or sub-slice operation goes out
of bounds.

GooseLang has a primitive for contiguous memory, which we use to model
the allocation underlying a slice (though these are not directly
accessible to Go code, since they do not carry enough information for
bounds checking). On top of these we model slices as a tuple of a base
pointer, length, and capacity.

The GooseLang slice model is directly inspired by the implementation of
slices, including modeling slice capacity. Initially we had a more
abstract model that ignored capacity (which would appear to be just an
optimization), but were surprised to find that this was insufficient to
even accurately model subslicing and appending. Directly modeling slice
capacity was the simplest solution to obtain a model that is faithful to
the Go implementation. The Go language reference isn't specific about
what the slice capacity after various operations should be, so our
GooseLang model picks a non-deterministic capacity in several places
(within appropriate bounds).

\newidentmacro{ptr}
\newidentmacro{len}
\newidentmacro{cap}

The most basic operations on slices are indexing and storing. The
GooseLang model of $s$ is a three-tuple, but for clarity we will refer
to its elements as $\ptr(s)$, $\len(s)$, and $\cap(s)$. The
translation of \cc{s[i]} is essentially a load from
$\ptr(s) + i$ (or undefined behavior if this offset is out-of-bounds).
Similarly \cc{s[i] = x} stores to the same location. We
translate Go's \cc{len(s)} directly to $\len(s)$. Go also supports
accessing a slice's capacity, but this is rarely used and Goose does not
support it.

The Go \cc{append} operation is the most sophisticated to model. The
behavior of \cc{append(s, x)} where \cc{s: []T} and
\cc{x: T} depends on whether there is extra capacity to store the
new element \cc{x}. If there is capacity, then \cc{x} is stored
there and the append returns a new slice with the same pointer but a
larger length. If there is no capacity, then \cc{append} must
allocate a new slice, copy the existing elements to it, and then store
\cc{x}. In the latter case \cc{append} returns a slice with a
fresh pointer.

The difficulty with Go slices arise when supporting subslicing. Consider
\cc{s[:i]}, where \cc{i} is less than \cc{len(s)}.
Clearly this slice should have the same pointer and length \cc{i},
but what should its capacity be? Surprisingly, the capacity of this
prefix is the full capacity of \cc{s}, which means that the unused
elements of \cc{s[:i]} \emph{include the elements of \cc{s}}
beyond the index \cc{i}. As a result, \cc{append(s[:i], x)}
in fact modifies \cc{s[i]}. GooseLang takes care to model this
behavior by implementing \cc{append} exactly as above, taking into
account that \cc{append(s, x)} might be an in-place operation.

The GooseLang model is specifically designed to be sound by sticking to
the Go implementation as closely as possible, but we want reasoning
about slices to be convenient and high-level, without worrying about
slice capacity directly. The design of GooseLang nicely separates the
model from the reasoning principles --- we verify specifications against
the concrete model, so that only the model is trusted and not the
separation logic specifications.

\newcommand{\sliceRep}{\mathtt{sliceRep}}
\newcommand{\sliceCap}{\mathtt{sliceCap}}

\newcommand{\lappend}{\mdoubleplus}

The GooseLang model of slices is based on two abstract predicates:
$\sliceRep(s, l)$ and $\sliceCap(s)$. To model the slice values
themselves we use $s : Slice$ where $Slice$ is a Gallina record; a
function $SliceVal(s) : Val$ converts the Gallina representation to
the GooseLang tuple that the slice model uses. We will only present the
\emph{untyped} version of this specification where $l : list val$, but
GooseLang also has a typed version where $l : list T$ where there is
some (Gallina) function $\mathtt{to\_val} : T -> Val$. The typed version is
practically convenient in proofs but is only a small extension to the
untyped version.

The first predicate $\sliceRep(s, l)$ gives the abstract value of
$s$, the list of values it contains, excluding additional capacity. It
also represents ownership over all these elements, in terms of the
underlying struct points-to facts. We use this predicate to specify
loads and stores:

\[
  \hoareV{\sliceRep(s, l) * i < |l|}%
{\mathtt{s[i]}}%
{\Ret{v} v = l !! i * \sliceRep(s, l)}
\]
\[
  \hoareV{\sliceRep(s, l) * i < |l|}%
 {\mathtt{s[i] = v}}%
{\Ret{v} v = l !! i * \sliceRep(s, l[i := v])}
\]

Next, $\sliceCap(s)$ is an abstract predicate that represents
\emph{ownership over the capacity} of $s$. It is necessary to append,
since appending might need to write to the capacity, but unneeded to
read and write to a slice.
\[
\hoareV{\sliceRep(s, l) * \sliceCap(s)}%
{\mathtt{append(s, x)}}%
{\Ret{s'} \sliceRep(s', l \lappend [x]) * \sliceCap(s')}
\]

\newidentmacro{sliceFull}

This specification is fairly simple. In fact, we often use a shorthand
$\sliceFull(s, l) = \sliceRep(s, l) * \sliceCap(s)$ when the proof will
always retain ownership of slice capacity, in which case the spec looks
even simpler. However, the proof is non-trivial, since in one case it
moves ownership from $\sliceCap(s)$ to $\sliceRep(s', l \lappend [x])$
(where $ptr(s') = ptr(s)$), while in the other it constructs a
completely new allocation for $s'$.

\newidentmacro{sliceTake}
\newidentmacro{sliceDrop}

The most interesting rules are for subslicing and how they interact with
capacity. Consider \cc{s[:i]} again. While Go has no formal
notion of ownership, our specifications do. We can model the
\emph{value} for \cc{s[:i]} easily enough; call it
$\sliceTake(s, i)$ (it simply reduces the length and keeps the capacity
of $s$, as specified by Go). Now we need to decide how ownership of
$\sliceRep(s, l) * \sliceCap(s)$ should relate to ownership of
$\sliceRep(sliceTake(s, i), take(l, i))$. It turns out there are two
possibilities: we can either give up ownership of the remainder of $s$
in exchange for $\cap(\sliceTake(s, i))$, or we can ignore the
capacity of the subslice and keep
$\sliceRep(\sliceDrop(s, i), drop(l, i))$. These are incomparable and
unexpressed in the code: the decision is based on whether we intend to
append to the subslice but stop using the old slice, or whether we want
to continue using the remainder of \cc{s}.

\tej{why not just use \cc{s[:i]} for $\sliceTake(s, i)$ and $l[:i]$ for
$take(l, i)$? overloading will make everything much easier to read}

Concretely, GooseLang verifies the following entailment for reasoning
about subslicing in terms of the slice model:

$\sliceFull(s, l) \vdash \sliceFull(\sliceTake(s, i), take(l, i))$

This entailment precisely captures how retaining ownership of the
capacity of $\sliceTake(s, i)$ requires giving up the remainder of
$s$.

\begin{align*}
  &\sliceRep(s, l) \dashv\vdash \\
  &\quad \sliceRep(\sliceTake(s, i), take(l, i)) \sep {} \\
  &\quad \sliceRep(\sliceDrop(s, i), drop(l, i))
\end{align*}

This alternative bidirectional entailment splits $s$ into two parts,
but gives up ownership over $\sliceTake(s, i)$'s capacity in exchange
for using those elements in \cc{s[:i]}. From this point it will
not be possible to prove the safety of appending to \cc{s[:i]},
since this would conflict with the separate ownership over
\cc{s[i:]}.

\subsection{Maps}

\newidentmacro{mapVal}
\newidentmacro{mapRep}
\newidentmacro{mapDelete}
\newidentmacro{mapInsert}
\newidentmacro{mapIter}

After slices, maps are the next most commonly used collection type in
Go. We implement maps as lists of key-value pairs, stored in a single
memory location in reverse insertion order. Go's builtin maps are
\emph{not} thread-safe, so the model enforces single-threaded access by
marking the map as being read while reading from it; this re-uses the
race detection for other pointers to ensure that racy access to a map is
undefined behavior, while allowing concurrent read-read access. Maps
support all the Go operations: insertions, reads (including returning
whether the key is present), \cc{len} to get the number of elements
in the map, deletion, and iteration. Go map iteration is
non-deterministic and in practice random, but we did not model this
since it would be challenging to do so; however, the reasoning
principles for map iteration do not expose an iteration order.

The implementation of maps is the most involved out of any of the Go
primitives. It required directly implementing maps (albeit
inefficiently, using an association list) using recursive GooseLang
code. GooseLang is an untyped language, so our first attempts had basic
errors like missing arguments. We improved our confidence in this
implementation both by testing it and by verifying it. Both of these
essentially rule out type errors (regardless of what specification we give),
and the specification is simple enough
to be a reliable test of behavior. Both simple tests and verification
cover easy mistakes like reading the oldest write to a key rather than
the latest, or duplicate keys during iteration (the implementation must
skip over a key after observing it once).

The proof and specification for maps is relatively easy since they are
not safe to use concurrently, so the proof assumes ownership over the entire map. We
treat a map as a pointer to an abstract map value, a GooseLang value
that encodes the entire map data as a list of key-value pairs. The
specification is based on a pure relation $\mapVal(v, m)$ that relates
this encoded value to a Gallina map $m$, which uses \cc{gmap} from
stdpp; for simplicity we use \cc{gmap u64 val} and limit map
keys to integers. Values are not a visible notion to the Go code, since
it always interacts with maps via their pointer, so the specifications
all use $\mapRep(\ell, m) = \exists v.\, \ell \mapsto v * \mapVal(v, m)$. The
indirection is important, since the Go map value
\cc{m : map[uint64]V} is in fact a reference to a map that is
mutated in-place (unlike a slice, which has both pure data --- pointer,
length, and capacity --- and heap data).

For example, this is the specification for map deletion:
\[
\hoare{\mapRep(l, m)}{\mapDelete(l, k)}{\mapRep(l, delete(m, k))}
\]

Map iteration has a more sophisticated specification. Suppose we have a generic loop
over a map in Go like the following:

\begin{verbatim}
for k, v := range m {
  body(k, v)
}
\end{verbatim}

The model for this entire construct is given by $\cc{MapIter}(m, body)$, where
$m$ is a reference to the map and $body$ is an expression for the body of the
loop. Goose translates generic loop bodies, so the Go code does not literally
need to consist of a call to a separate function. The possibility of
\emph{iterator invalidation} adds one subtlety to Go's map iteration --- it
would be incorrect for the body of the loop to modify the map (it might be sound
to write to the map without modifying the domain, but we do not attempt to model
this). The model of maps puts the entire contents of a map in one heap location,
so naively we would not enforce this property. The solution is to mark the map's
reference as being read for the entire duration of iteration, using the
$\goosekw{StartRead}$ and $\goosekw{FinishRead}$ GooseLang primitives at the
beginning and end of $\cc{MapIter}$. If the map value in the heap is
$\goosekw{Reading} \app n \app v$, these operations increment and decrement
(respectively) the reader count $n$, so that any writes within the body have
undefined behavior.

Iteration gets a \emph{higher-order} specification that assumes a specification
for the body, showing it preserves a loop invariant $P$ over the part of the map
consumed so far:

\[
  \infer{
    \forall m_{0}, k, v.\,
    k \notin m_{0} \land m[k] = v \to \\\\
    \hoare{P(m_0)}{body \app k \app v)}{P( m_{0}\mapupd{k}{v} )}
}
{
  \hoare{\mapRep(\ell, m) \sep P(\emptyset)}%
{\cc{MapIter}(\ell, body)}%
{\mapRep(\ell, m) \sep P(m)}
}
\]

On top of this generic specification we prove some alternate specifications that
express the invariant in slightly different ways --- for example, it is often
useful to express the invariant in terms of both the map iterated over so far
and the remaining subset of the map.

Map iteration in Go happens in a non-deterministic order\footnote{In fact the
runtime randomizes the starting position of the iteration, to avoid callers
accidentally relying on any particular behavior. See
\href{https://github.com/golang/go/blob/c379c3d58d5482f4c8fe97466a99ce70e630ad44/src/runtime/map.go\#L844-L850}%
{\cc{mapiterinit} from src/runtime/map.go}.}
Thus strictly speaking $\cc{MapIter}$ should shuffle the elements of the map
before iterating over it, in order to model the non-determinism of the
implementation. We do not (currently) do this, simply because the shuffle would
be hard to implement in GooseLang. However, the specification for map iteration
does not expose an iteration order and would apply unchanged to this more
precise model. All proofs go through this specification, so our proofs should
remain unchanged if $\cc{MapIter}$ started modeling non-determinitic ordering.

\section{Testing Goose}
\label{sec:goose:testing}

Goose is a trusted component in the entire verification process. For the
overall system's proof to be sound, we rely on the model to produce all
of the behaviors of the Go code; that is, the behaviors of the Go code
(in practice, using the Go compiler) should be a subset of the behaviors
of its translated GooseLang (according to the Coq semantics). As long as
this is case, the proof is sound in that if the modeled system always
satisfies some property the code will, too.

One subtlety in the trust we place in Goose is that it only applies when
Goose translates code successfully, that code compiles in Coq, and the
model has no undefined behavior. If any of these fail, then the proof of
the system would either not be possible or not go through. Therefore the
most important bugs are those where the translation's behavior differs
from that of Go; these can compromise soundness of the system and lead
to a proof that is not borne out in practice.

To increase out confidence in Goose, we implemented a large suite of
unit tests. While these tests check that Goose continues to translate
existing code (and check that the translation has not unexpectedly
change), for soundness the relevant test is to compare Go to the Goose
output. Unfortunately GooseLang is not natively an executable language.
Its semantics is expressed as a Coq relation that specifies how an
expression is evaluated (or gets stuck, indicating undefined behavior).
To test GooseLang code, we implemented an interpreter in Coq, which can
run GooseLang code and produce either an error due to undefined behavior
or a result. While the interpreter is not very efficient, it has good
enough performance to run the Goose unit tests.

The interpreter is an important part of the testing strategy, but
ultimately the comparison is intended to be between Go and the GooseLang
semantics. Thus we verified that the interpreter produces executions in
accordance with the semantics. The correctness theorem is slightly
subtle in that the interpreter produces only one possible execution, but
the non-determinism is only due to the choice of what locations to use
for pointers, which should not affect any visible behavior.

The technical challenge with implementing and verifying the interpreter
is that the semantics uses a convenient but non-executable way of
expressing the order of evaluation. GooseLang is a lambda calculus, so
its semantics is expressed as a transition system between expressions.
It is easy to give the semantics of a primitive at the \emph{head} of an
expression; for example, we can say what $\goosekw{Store}(l, v)$ does in a given
heap if $l$ and $v$ are already values (it stores $v$ in the heap
and evaluates to \texttt{\#()}, the unit value). It is also easy to
interpret \emph{pure} reductions like \cc{x + y} where \cc{x}
and \cc{y} are values since the semantics of these pure expressions
is already given as a Gallina function.

The challenge in the interpreter comes from \emph{context} reductions,
which specify how to find a sub-expression within \cc{e} to reduce
if the head is not immediately a value. The semantics follows a standard
presentation of context reduction using \emph{evaluation contexts}. The
idea is to define a type of evaluation contexts $E \in \mathcal{E}$ that
represent an expression with a hole; $E[e]$ represents filling that hole with
the expression $e$. The possible evaluation
contexts give all the context reductions in one compact rule, \ruleref{context-reduce}: if $e$
can step to $e'$, then $E[e]$ can step to $E[e']$. This rule applies whenever
such an $E$ exists, while the
interpreter recurses through an expression (in the right order) and
evaluates a sub-expression, then fills it into the context. We prove
this correct, showing that the interpreter and semantics agree on an
evaluation order. Specifically, the interpreter proof shows the interpreter
produces a valid evaluation order, and a separate proof shows that evaluation
contexts are unique.\footnote{See the lemma
\href{https://github.com/mit-pdos/perennial/blob/6f5ed5e7c2d3e8d657a0022c51e1d1e32a81e671/src/goose_lang/lang.v\#L1443-L1447}%
{\cc{head_redex_unique} in src/goose\_lang/lang.v}.} There is other non-determinism in the semantics that the
interpreter does not fully explore, though.

The test suite is structured as a number of test functions, each producing a
boolean that should be true. To check that the test itself is written correctly,
we test that it produces \cc{true} in Go first. Then to check the semantics of
the translation, in GooseLang we check that the interpreter succeeds and returns
true for each test function. While we could compare more sophisticated results
like integers or structs between the two, this strategy is especially easy to
implement, since there is no need to correlate Go and GooseLang outputs and
compare structured data.

The interpreter and test framework was designed and implemented by
Sydney Gibson, and is described in greater detail in her master's
thesis. The thesis includes more details on evaluating the interpreter
itself, for example documenting bugs caught by the test suite and other
bugs that are now part of our regression tests.


\section{Limitations}%
\label{sec:goose:limitations}

Notably missing in Goose but prominent in Go is support for interfaces
and channels. We believe both are easy enough to support, but interfaces
were not necessary for our implementation, and rather than channels the code
verified in this thesis
use mutexes and condition variables for more low-level control over
synchronization.

Control flow is slightly tricky since a Go function is translated
to a single GooseLang expression that should evaluate to the function's
return value. Goose supports many specific patterns, especially common
cases like early returns inside \cc{if} statements and loops with
\cc{break} and \cc{continue}, but more complex control flow ---
particularly returning from within a loop --- is not supported. It would be
easiest to express general control flow in continuation-passing style (in which
every GooseLang takes a continuation, and calling this continuation corresponds
to returning from the function in Go), but this
would complicate every specification and the translation of function calls.

Goose do not support Go's \cc{defer} statement. It would be nice to support some
common and simple patterns, particularly for unlocking, by translating
\cc{defer} statically. The behavior of Go's \cc{defer} statement in general is
to push the deferred function to a stack of calls associated with the current function that are executed in reverse order at return
time. GooseLang does not have a first-class notion of a Go function to associate
the stack of deferred functions with, nor the concept of returning
from a function. However, it would be useful to simple static uses of \cc{defer}
at the top-level of a function.

Named return values are recommended to document return parameters, and sometimes
simplify and clarify the body of a function.\footnote{See the description in
\href{https://go.dev/doc/effective_go\#named-results}{Effective Go}.} However, in
general they are quite subtle, due to interaction with \cc{defer} statements and
concurrency~\cite{chabbi:golang-races}. One source of difficulty is that the
return values are treated like local variables declared at the top of the
function, and it is easy to accidentally have races on these variables if they
are accessed concurrently.

Goose does not support mutual recursion between Go functions, and
additionally requires the translation to be in the right order so
definitions appear before they are used. Files in a package are especially prone
to this issue, since they are processed in alphabetical order; we sometimes name
files something like ``0constants.go'' to make the order works out correctly.
The subtlety here is that
definition management in Go, as in most imperative languages,
conceptually treats all top-level definitions as simultaneous, whereas
Coq processes definitions sequentially. Using Coq definition management
to model Go definition management imposes a limitation compared to Go,
but is much simpler to work with compared to modeling a Go package as a
set of mutually recursive definitions. Reasoning about code written in such a
model would require setting up specification for all the definitions, then
proving them in a recursive way, all while ensuring that no specification is
used before it is proven.

GooseLang does have one extant bug related to evaluation contexts. The contexts
$e \app E$ and $E \app v$ define a right-to-left evaluation order for functions,
which is the opposite of Go. We haven't yet fixed this, either by adjusting the
GooseLang semantics or changing the translation to emit code that explicitly
evaluates all the arguments in the correct order before calling the function.

Map iteration in Go happens in a non-deterministic order.\footnote{In fact the
runtime randomizes the starting position of the iteration, to avoid callers
accidentally relying on any particular behavior. See
\href{https://github.com/golang/go/blob/c379c3d58d5482f4c8fe97466a99ce70e630ad44/src/runtime/map.go\#L844-L850}%
{\cc{mapiterinit} from src/runtime/map.go}.} Thus, strictly speaking, the model
of map iteration described in \cref{sec:goose:maps} given by \cc{MapIter} should
shuffle the elements of the map before iterating over it, in order to model the
non-determinism of the implementation. Goose does not (currently) do this,
simply because the shuffle would be hard to implement in GooseLang. However, the
specification for map iteration does not expose an iteration order and would
apply unchanged to this more non-deterministic model. All proofs go through a
common iteration specification, so our proofs should remain unchanged if
\cc{MapIter} started modeling non-deterministic ordering.

\section{Implementation}%
\label{sec:goose:impl}

Goose is implemented in Go. It takes advantage of a family of packages in under
\cc{go} in the Go standard library, such as \cc{go/ast} and \cc{go/types}, to
parse and type-check Go code; the \cc{golang.org/x/tools/go/packages} package
even makes it possible to work with Go packages and modules. These greatly
simplify implementing Goose and also make it more trustworthy, since at least
the source code is parsed in exactly the same way as the compiler (though we can
still introduce bugs in interpreting what the AST means).
The translator has an intermediate representation of the Coq source for
GooseLang, and splits translation into generating the Go structs for this
representation and subsequently printing this representation as a Coq file.

The translator makes as much of an attempt as possible to identify and report
errors, identifying where a feature is unsupported. Each function is translated
almost independently, allowing translation to move on to the next function and
report a whole batch of errors. Some issues are not checked by the translation,
such as the topological order on definitions mentioned in limitations above (in
section \cref{sec:goose:limitations}); in these cases the resulting Coq code
does not compile, which preserves soundness but is a slightly worse user
experience.

GooseLang has a notion of ``external'' operations and state, which is generally
instantiated with a disk. The translator checks that the source package imports
only one of the allowed external imports, including through its transitive
dependencies. It also supports code that uses no external operations,
translating it to a form that is itself parameterized, allowing the result to be
used as part of any other code.

\section{Conclusion}

Goose is an approach for verifying Go code. We define GooseLang as a model of Go
and automatically translate a subset of Go to this language. GooseLang comes
with a number of reasoning principles for handling features of Go. The benefit
of this approach is the ability to write high-performance code in a
productive language, with convenient reasoning while verifying
that code. Several aspects of the design contribute to making the approach
sound, ranging from the subset of Go supported, to the design of GooseLang, and
the use of standard Go tools for analyzing the source code.

Our main use case for Goose for this thesis was to verify GoTxn, but
the tool and approach are more generally applicable, even without concurrency or
crash-safety reasoning. These ideas could also be productively applied to
languages other than Go --- I am personally excited about the prospect of having
a version for Rust.
