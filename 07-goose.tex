This chapter is about Goose, which solves a practical
problem of reasoning about efficient code. To use Goose, a developer writes
code in Go, uses an automatic translator to convert the code to a model in the Coq proof
assistant, then carries out the proof on top of the model in the Perennial logic. Goose encompasses
the entire process: it includes the translation tool itself, the way it models Go
code, and finally the reasoning principles for proving properties of translated
Go. (We will also use ``Goose'' in some places to refer to the subset of Go
supported by the translation tool.)

This chapter is intentionally fairly independent of the rest of the thesis for a
reader interested in verifying Go code but not crash safety or the specifics of
the GoTxn and DaisyNFS proofs. Crashes are modeled simply by stopping
execution and wiping out all of the state except for the disk, which does not
relate to the specifics of Go.

\section{Goals and motivation}
\label{sec:goose:goals}

There are three main goals for Goose: supporting \textbf{efficient} code,
\textbf{soundness} for the translation process, and \textbf{convenient
reasoning}. The subset of Go that Goose supports is
intended to enable writing and verifying high-performance code. Goose
is sound if the model captures the behaviors of the code, which is required for
the proofs to say something meaningful about the executable Go code. If the model
misses some buggy behavior in the code, then a correctness proof wouldn't mean
anything about the code. Finally, Goose aims to make reasoning convenient by developing reasoning
principles for the model of Go primitives like structs, slices, and maps.

There are alternate setups for verification where the connection between the
proofs and the code is not given by a model and translation. Goose supports
efficient code both by connecting to ordinary Go, which benefits from the Go
compiler and runtime, and by supporting a variety of features. Sometimes
efficient code is more complex to prove, but this is a tradeoff the user can
make.

The
design of Goose tries to achieve soundness through simplicity and careful choice
of what to model. There is sometimes a tradeoff between simplicity and
supporting efficient code. If a language feature is needed for good performance (for
example, taking pointers to individual struct fields), then Goose models it,
even if the feature is complex. If
a feature would only result in more idiomatic code and modeling it seems
subtle, then it might not be implemented (for example, simple uses of
\texttt{defer} could be modeled but aren't because the feature is complicated in
general). The result is that Goose is generally pleasant and productive enough
to write in, but requires some practice for a Go programmer.

Convenient reasoning remains a goal for Goose, but not one that was always
achieved. All of the verified libraries are usable but pain points remain; some
of these are simply a matter of engineering effort or fixing bugs, but we have
also found code patterns that weren't well captured in the reasoning.

\subsection{Why Go?}

Go is a convenient language for building verified systems. It is productive
enough to build systems that get good performance. The language is simple,
facilitating a sound translation.

For our first goal, efficiency, Go has enough features to build good systems
in. It has efficient and useful built-in slice and map data structures. The
runtime handles concurrency efficiently and has good support for synchronization
using locks and condition variables, allowing a low-level implementation.

There is also an advantage to Go as a programming environment rather than
programming language. The tooling for testing, debugging, and profiling is
extremely good, making it easy to fix bugs (before verification or in unverified
code) and find performance problems while optimizing. We were able to use
low-level interfaces to the operating system to access the disk --- these are
easier to understand in isolation, compared to understanding the combination of
a file-system library and the operating system's behavior. Garbage collection
simplifies writing code, particularly with concurrency, and carries a relatively
low performance impact due to the efficient runtime.

For the second goal, soundness, it helps that Go is a simple language. The Goose
translator effectively gives a semantics to the source code; in a complex
language this can be a daunting task (such as attempts to formalize JavaScript and
Python~\cite{guha:lambda-js,politz:python-semantics}). It isn't too
difficult to give Go a semantics, especially the Goose subset. Go's tooling
helped, including libraries for parsing and type-checking Go source code. Not
only do these libraries save time in implementing Goose, they greatly improve
reliability since they are written by experts (the Go compiler team, extracting
code from the compiler itself).

\subsection{Why not C or Rust?}

C is not too different from Go as a basis for Goose. The main differences are
the need to implement and verify manual memory management, and it would be more
challenging to parse and type-check C code.

Using Rust as a source language seems attractive but would likely not be much
better than Go. One subtlety is that while the source code is type checked, the
model is an untyped program. It would be difficult to take advantage of the fact
that the code is type-checked, and thus the verification engineer would anyway
re-prove memory safety in the same way as Goose requires with Go. The main
benefit would be reduced bugs before reaching verification.

Another difficulty with Rust would be the size and complexity of the language
--- the subset supported might be restrictive enough that the experience no
longer feels like Rust. For example, the \cc{Vec<T>} type has 152 methods
without even including trait implementations. Using any of these methods would
require assuming a semantics for it, which is trusted to be sound (that is,
getting the semantics wrong could compromise the whole verification). Thus in
practice the expansive standard library would mostly not be available; for
soundness only a core subset would be modeled and the rest manually implemented
and verified.

\section{Related work}%
\label{s:goose:rel-work}

There are several areas of related work. First and foremost Goose is an approach
to verifying executable programs, so we start by discussing alternate
approaches. Second, Goose implicitly gives a semantics to Go, and there are
related projects giving semantics to other programming languages.

\subsection{Verification approaches}

At a high level, all verification systems need to solve the problem of
connecting the world of the proof assistant to the ``real world'' that runs the
code. The proofs are always over a model of the code, so somewhere in that
process we trust that the running code has been modeled correctly. The approach
imposes some requirements on the code that can be written and how proofs are
written. Thus each approach makes a tradeoff between flexibility, soundness, and
convenient reasoning, as outlined above in \autoref{sec:goose:goals}.

There are basically three ways to verify programs (with several interesting
caveats). We can translate code to a model (as in Goose), pretty-print a
model to code, or compile from a language intended for both verification and
implementation. In addition, verified compilation and translation validation can
implement parts of the translation process from model to code in a verified way;
this can usefully make reasoning more convenient without sacrificing soundness.

It is easiest to understand this characterization in the context of concrete
examples. Goose translates from Go code to a model in Coq. Both
CFML~\cite{chargueraud:cfml} and hs-to-coq~\cite{spector-zabusky:hstocoq} work
in a similar way, for OCaml and Haskell respectively.
Fiat-Crypto~\cite{erbsen:fiat-crypto} prints a model of C code to a string that
is compiled with a C compiler. Programs in Dafny~\cite{leino:dafny} are
typically run using Dafny's built-in compiler (which has backends for C\#, Java,
JavaScript, and Go at the time of this writing). Similarly, languages like Coq
and \fstar have support for \emph{extraction} that translate functional programs
in those languages to something executable like OCaml.

There are approaches that are somewhere in between using built-in compilation
and writing code in a model. For example, FSCQ~\cite{chen:fscq} uses Coq
extraction, but it works from a model of I/O interaction which is implemented by
combining the extracted code with a Haskell library. This combines Coq's
extraction (compilation) with translating from a model to code. For \fstar,
there is a specialized toolchain called KaReMeL~\cite{protzenko:lowstar} that
targets a subset of \fstar called \lowstar and extracts imperative C code.

Verified compilation can reduce trust when using either approach.
VST~\cite{cao:vst-floyd} in its most basic form looks like a code-to-model
translation, using a tool called \cc{clightgen} to translate C code to an
abstract syntax tree (AST) in Coq, which the user can then specify and verify in
VST. However, it is then possible to run this AST through the CompCert verified
compiler and produce assembly code (or more specifically, a model of assembly
represented in Coq). The proof than covers this assembly model, which is
pretty-printed and run through an assembler to run it. While we started with C
code, this system only requires trusting the model of assembly. On the other
hand, in terms of flexibility this system is limited to verifying what
\cc{clightgen} supports and its performance is in part determined by the quality
of the CompCert compiler.

Verified compilation is used in a way similar to VST in
Vale~\cite{bond:vale,fromherz:vale-fstar}. Code is written in a specialized
assembly-like language called Vale, with proof annotations. This is then
compiled to Dafny or \fstar (Vale has two backends) where the proofs are carried
out before finally being printed back down to assembly. Again, the original Vale
code is no longer relevant from a soundness perspective, but it is this DSL that
determines how flexible the system is.

Cogent~\cite{amani:cogent,oconnor:cogent-lang} also uses a form of verified
compilation to make proofs easier without making reasoning more difficult. Code
written in the Cogent language (an imperative language with linear types) is
translated to a model in Isabelle/HOL along with a functional specification for
the code and a proof of correspondence. Thus the user can write proofs on top of
the functional specification, but the proofs go down to an imperative model.

\subsection{Modeling programming languages}

% We wanted control over the translation process to simplify the resulting
% model that we needed to write proofs for. Using an existing translator
% that essentially translates syntax would still leave the task of giving
% a semantics to the output code and proving the right specifications in
% Perennial to reason about various parts of the semantics.

While Goose is designed around verification, it is also a semantics for a subset
of Go. Each function in Go is translated to a term in GooseLang, and we then
give that term a semantics in Coq in the form of a transition system. Indeed
this must be the case, because the proof is about the behavior of the code, so
we first need to say what the behavior of the code is before verifying it.

There are several projects that have tackled the problem of giving a semantics
to real-world languages. Many of these are focused on the goal of completeness,
which makes them much more challenging than Goose. By only giving a treatment to
a subset of the language, we were able to make the work more manageable, while
still meeting our goal of reasoning about efficient code.

CH20~\cite{krebbers:c-coq} is a fairly complete formalization of the C standard.
CH20 defines C in operational and axiomatic styles, along with an executable
semantics to test the semantics on (small) concrete programs. All of these
semantics are formally related to each other. Having multiple, related semantics
helps give confidence in each of them, since they independently express the same
thing in different ways.

VST~\cite{cao:vst-floyd} uses CompCert C to model C code. One similarity to
Goose is support for structs in terms of their individual fields. Both systems
need to model the semantics of features like taking a pointer to an individual
struct field (as CH20 also handles). What sets VST and Goose apart from CH20 is
to also have reasoning principles for verifying code that uses structs in
interesting ways, such as structs where a subset of fields are protected by a
lock.

RustBelt~\cite{jung:rustbelt} is intended as a model of Rust code, at the Rust
MIR (mid-level IR) level of abstraction. The modeling language in RustBelt,
\lambdarust, has many similarities to GooseLang, and both are used together with
Iris~\cite{jung:iris-1} and the Iris Proof Mode~\cite{krebbers:ipm}. RustBelt
has different goals, being intended for reasoning about Rust as a whole rather
than program verification. This is a somewhat subtle difference: the goal in
RustBelt is to prove properties about \emph{all} \lambdarust programs, whereas
we only prove correctness of \emph{particular} GooseLang programs. As a result
it is important for \lambdarust, together with its type system, to rule out
programs that Rust does not allow, whereas GooseLang is more expressive than Go
with unsafe features like pointer arithmetic. Due to different goals, RustBelt
also does not have an automatic translation from Rust to \lambdarust, though
some libraries have been translated by hand and verified using Iris.

Finally, it is worth mentioning WebAssembly~\cite{haas:wasm} as a rare example
of a production language with a formal semantics, and even more rarely that was
designed with the formal semantics in mind. As a result, the WebAssembly
semantics is relatively clean. Compare this situation to that of C, which
CH20~\cite{krebbers:c-coq} describes as ``oriented towards being efficiently
implementable rather than being abstract in a mathematical sense.''

\section{High-level overview}

The goal of the translation is to model a Go program using GooseLang,
which is a programming language defined in Coq for this purpose. That is,
GooseLang is defined using expressions defined in Coq, equipped with a
semantics also given in Coq. Since GooseLang programs support references
to model the Go heap, the semantics is written in terms of transitions
of (expression, heap) pairs where the heap maps pointers to values. The
intention of the translation is that the semantics of the translated
function should cover all the behaviors of the Go code, in terms of
return values and effect on the heap. As long as this is true, a proof
that the translated code always satisfies some specification means that
the real running code will, too.

GooseLang is a low-level language, so many constructs in Go translate to
(small) implementations in GooseLang. This implementation choice proved
to be much more convenient than adding primitives to the language for
every Go construct. For example, a slice is represented as a tuple of
pointer, length, and capacity, and appending to a slice requires
checking for available capacity and copying if none is available.
Appending to a slice is a complicated operation, and it was easier to
write it correctly as a program rather than directly as a transition in
the semantics. The one cost to this design strategy is that an arbitrary
GooseLang program is much more general than translated Go programs --- for
example, GooseLang has support for pointer arithmetic. This
has no impact on verifying any \emph{specific} Go program.

The extra generality of GooseLang is a downside when a theorem talks about an
\emph{arbitrary} GooseLang program, as shows up in the transaction system's
program refinement and simulation-transfer theorems. In order to make these
theorems true, they must rule out
some ill-defined GooseLang programs, which are
not possible to produce by writing Go and translating it to GooseLang. Both
specifications make these assumptions using a
standard technique of using a type system for GooseLang developed in Coq, and encoding
syntactic restrictions via that type system.

\tej{add a diagram, perhaps like the ones in the Goose talk, showing the
translation process and how Go/Goose/GooseLang all fit together}

An important aspect of GooseLang is supporting interactive proofs on top
of the translated code. The interactive proofs use separation logic, a
variant of Hoare logic, so specifications describe the behavior of each
individual function. In order to support verification of any translated
code, GooseLang comes with a specification for any primitive or function
that the translated code might refer to, including libraries like slices
used to model more sophisticated Go features. GooseLang has many
``pure'' operations that have no effect on the heap, due to many
primitive data types and operations (for example, there are both 8-,
32-, and 64-bit integers, and arithmetic and logical operations for
each). The specifications for these operations are handled with a single
lemma, which is applied automatically with a tactic \cc{wp_pures}.

Since our goal is to support interactive rather than automated proofs,
it is helpful to make the model simple to work with. The translation process maintains
a strong correspondence between the model and source code: each Go
package translates to a single Coq file, and each top-level declaration
in the Go code maps to a Gallina definition (a GooseLang constant or
function). Goose has a special case for translating immutable variables
to let bindings in GooseLang (rather than allocating a pointer that will
only be read). As a result, factoring out a sub-expression to a variable
has little impact on proofs.

The translation process sometimes translates a Go operation to a sophisticated
model in order to capture some corner-case behavior, for example in the model of
slices described in \cref{sec:goose:slices}. This complex models don't have an
undue affect on proofs as long as Goose's reasoning principles can abstract away
the complexity for common cases. The subsequent sections in this chapter
walk through several features of Go. Each section first describes a model that
implements the feature in GooseLang, which primarily aims to
be faithful to Go. Next, the section describes reasoning principles we developed for that
feature, in the form of separation logic assertions (for example, to
represent a slice) and Hoare triples (for example, to specify the
behavior of the \cc{append()} operation). The goal for the model is to capture
Go's behavior, whereas the reasoning
principles aim to make proofs using the model practical.

\section{GooseLang}

\newcommand{\goosekw}[1]{\mathsf{\textcolor{blue}{#1}}}
% spacing for application
\newcommand{\app}{\:}
% any pure binary op
\newcommand{\binop}{\circledcirc}
% any pure unary op
\newcommand{\unop}{\circleddash}

\newcommand{\gooseif}[3]{\goosekw{if} \app #1 \app%
  \goosekw{then} \app #2 \app \goosekw{else} \app #3}

\newcommand{\recfx}{\goosekw{rec} \, f(x) = e}

\begin{figure}
  \begin{mathpar}
  \begin{array}{llcl}
    &x,f &\in &\textdom{Var} \\
    &\ell &\in &\textdom{Loc} \\
    &n &\in & \textdom{U64} \cup \textdom{U32} \cup \textdom{U8} \\
    &s &\in &\textdom{String} \\
    &\binop &::= & + \ALT - \ALT * \ALT = \ALT < \ALT \dots \\
    &\unop &::= & - \ALT \goosekw{ToString} \ALT \goosekw{ToU64} \ALT
                         \dots \\
    \textdom{Var}& v &::= & () \ALT \goosekw{true} \ALT \goosekw{false} \ALT n
                            \ALT \ell \ALT s \\
    &&\ALT & \goosekw{inj}_1 \app v \ALT \goosekw{inj}_2 \app v \ALT
             \recfx \\
    \textdom{Exp}& e &::= & x \ALT v \ALT e \app e %
                            \ALT \goosekw{Fork} \app e
                            \ALT \goosekw{Panic} \app s
                            \ALT \recfx \\
    &&\ALT & \gooseif{e}{e}{e} \\
    &&\ALT & (e, e) \ALT \pi_1 \app e \ALT \pi_2 \app e \\
    &&\ALT & \goosekw{inj}_1 \app e \ALT \goosekw{inj}_2 \app e
    \ALT %
      \goosekw{case} \app e \app \goosekw{of} \app \goosekw{inj}_1 \app x
      \Rightarrow e \app \goosekw{or} \app \goosekw{inj}_2 \app x \Rightarrow e
    \\
    &&\ALT & e \binop e \ALT \unop e \ALT \goosekw{ArbitraryU64} \\
    &&\ALT & \goosekw{AllocN}(e, e)
             \ALT \goosekw{CmpXchg}(e, e, e) %
             \ALT \goosekw{Load}(e) \\
    &&\ALT & \goosekw{PrepareWrite}(e) %
             \ALT \goosekw{FinishStore}(e, e) %
             \ALT \goosekw{StartRead}(e) %
             \ALT \goosekw{FinishRead}(e) %
  \end{array}
  \end{mathpar}
  \caption{GooseLang syntax}
  \label{goose:syntax}
\end{figure}

\newcommand{\reduces}{\rightsquigarrow}
\newcommand{\purereduction}{\overset{\mathrm{pure}}{\reduces}}

\begin{figure}
  \textbf{Semantics preliminary definitions}
  \begin{mathpar}
  \begin{array}{llcl}
    \textdom{ECtx}& E &::= & \square \ALT E \app v \ALT e \app E  %
                             \ALT E \binop e \ALT v \binop E \ALT \unop E \ALT \\
                  &&\ALT & \gooseif{E}{e}{e} \\
                  &&\ALT & (E, e) \ALT (v, E) \ALT \pi_i \app E \ALT
                           \goosekw{inj}_i \app E \ALT \dots \\
    \textdom{NonAtom}& z &::= & \goosekw{Reading} \app n \app v \ALT
                                \goosekw{Writing} \app v \\
    \textdom{Heap}& h &\in& \textdom{Loc} \overset{\mathrm{fin}}{\to}
                        \textdom{NonAtom} \\
    \textdom{World}& w \\
    %\textdom{State}& \sigma &::= (h, w) \\
    \textdom{TPool}& \mathcal{E} &\in \textdom{List}\app\textdom{Exp} \\
    %\textdom{Config}& \rho &::= (\sigma, \mathcal{E}) \\
  \end{array}
  \end{mathpar}

  \textbf{Pure reduction}%
  \hfill %
  \boxedassert[line width=0.4pt]{e \purereduction e'} %
  \hspace{20pt}

  \begin{mathpar}
  \begin{array}{rcll}
    v \binop v' &\purereduction & v'' & \mathrm{if} \: v'' = v \binop v' \\
    \unop v &\purereduction & v' & \mathrm{if} \: v' = \unop v \\
    \gooseif{\goosekw{true}}{e_1}{e_2} &\purereduction & e_1 \\
    \gooseif{\goosekw{false}}{e_1}{e_2} &\purereduction & e_2 \\
    \pi_i(v_1, v_2) &\purereduction &v_i \\
    \goosekw{case} \app \goosekw{inj}_i v \app\goosekw{of}\app %
    \goosekw{inj}_1 \app x_1 \Rightarrow e_1 \app\goosekw{or}\app%
    \goosekw{inj}_2 \app x_2 \Rightarrow e_2%
                &\purereduction & \subst{e_i}{x_i}{v} \\
    (\recfx) \app v &\purereduction & \subst{\subst{e}{f}{(\recfx)}}{x}{v} \\
    \goosekw{ArbitraryInt} &\purereduction &n & \forall n \in \textdom{U64} \\
  \end{array}
  \end{mathpar}

  \newcommand{\mapupd}[2]{[#1 \mapsto #2]}

  \textbf{Per-thread one-step reduction}%
  \hfill %
  \boxedassert[line width=0.4pt]{((h, w), e) \reduces ((h', w'), e') } %
  \hspace{20pt}

  \begin{mathpar}
  \begin{array}{rcll}
    ((h, w), e) &\reduces & ((h, w), e') &\mathrm{if} \: e \purereduction e' \\
    ((h, w), \goosekw{CmpXchg}(\ell, v_1, v_2)) &\reduces %
                          &((h\mapupd{\ell}{v}, w), \goosekw{true})%
                                         &\mathrm{if} \: h(\ell) = v_1 \\
    ((h, w), \goosekw{CmpXchg}(\ell, v_1, v_2)) &\reduces %
                          &((h, w), \goosekw{false})%
                                         &\mathrm{if} \: h(\ell) = v_2 \\
    ((h, w), \goosekw{AllocN}(n, v)) &\reduces %
                          &((h[\ell + i \mapsto v \mid 0 \leq i < n], w), \ell) %
                                         &\mathrm{if} \: \forall 0 \leq i < n,
                                           \, \ell + i \notin \mathrm{dom}(h) \\
    ((h, w), \goosekw{PrepareWrite}(\ell)) &\reduces %
                          &((h\mapupd{l}{\goosekw{Writing} \app v}, w), ()) %
                                         &\mathrm{if} \: h[\ell] = \goosekw{Reading} \app 0
                                           \app v \\
    ((h, w), \goosekw{FinishStore}(\ell, v)) &\reduces %
                          &((h\mapupd{l}{\goosekw{Reading} \app 0 \app v}, w), ()) %
                                         &\mathrm{if} \: h[\ell] =
                                           \goosekw{Writing} \app v' \\
    ((h, w), \goosekw{StartRead}(\ell)) &\reduces %
                          &((h\mapupd{l}{\goosekw{Reading} \app (n+1) \app v}, w), ()) %
                                         &\mathrm{if} \: h[\ell] =
                                           \goosekw{Reading} \app n \app v \\
    ((h, w), \goosekw{FinishRead}(\ell)) &\reduces %
                          &((h\mapupd{l}{\goosekw{Reading} \app n \app v}, w), v) %
                                         &\mathrm{if} \: h[\ell] =
                                           \goosekw{Reading} \app (n+1) \app v \\
    ((h, w), \goosekw{Load}(\ell)) &\reduces %
                          &((h, w), v) &\mathrm{if} \: %
                                         h[\ell] = \goosekw{Reading} \app n \app v \\
  \end{array}
  \end{mathpar}

  \textbf{Context reduction}%
  \hfill%
  \boxedassert[line width=0.4pt]{((h, w), \mathcal{E}) \reduces ((h', w'), \mathcal{E}')}%
  \hspace{20pt}

  \begin{mathpar}
    \inferH[context-reduce]%
    {((h, w), e) \reduces ((h,', w'), e') \\ \text{$E$ is an evaluation context}}%
    {((h, w), \mathcal{E}\mapupd{i}{E[e]}) \reduces%
      ((h', w'), \mathcal{E}\mapupd{i}{E[e']})}

    \inferH[fork-reduce]%
    {j \notin \mathrm{dom}(\mathcal{E}) \cup \{j\} \\ \text{$E$ is an evaluation context}}%
    {((h, w), \mathcal{E}\mapupd{i}{E[\goosekw{Fork}(e)]}) \reduces%
      ((h, w), \mathcal{E}\mapupd{i}{E[()]}\mapupd{j}{e})
    }
  \end{mathpar}

  \caption{GooseLang semantics}%
  \label{goose:semantics}
\end{figure}

\tej{describe disk semantics}

\tej{also need to describe Hoare triples and separation logic, might as well do
that here since it's part of the foundations of reasoning about GooseLang}

\resume

\section{Modeling and reasoning about Go}%
\label{sec:goose:reasoning}

A key principle in the design of Goose is to model features of Go as code,
written as libraries on top of the base GooseLang language described in
\autoref{sec:goose:lang}. In this section we describe some of the features of Go
we model, as well as reasoning principles we develop on top to verify code that
uses these features.

Specifications in this section are written in Perennial, which is a concurrent
separation logic. The basic specification of this logic is the triple
$\hoare{P}{\cc{f()}}{Q}$, which says that if \cc{f()} is run in a state
satisfying the precondition $P$ and terminates, the final state will satisfy the
postcondition $Q$. Separation logic additionally means that $P$ and $Q$ can
describe only the state actually involved in \cc{f()}, implicitly stating other
state is unmodified (we can give so-called ``small footprint'' specifications).
Assertions make use of the \emph{points-to assertion} $\ell \mapsto v$, which
gives the value of one address in memory. For example, some basic specifications
in separation logic are those for the load and store operations:
%
\begin{mathpar}
  \hoare{\ell \mapsto v}{!\ell}{\Ret{v} \ell \mapsto v}

  \hoare{\ell \mapsto v_0}{\ell \gets v}{\ell \mapsto v}
\end{mathpar}

The specification for load uses $\Ret{v}$ to specify what the return value is.

The specifications in this section can be appreciated with only basic
familiarity with sequential separation logic; concurrency concerns do not arise
much at this level, and there is nothing specific to crash safety. All of these
specifications are actually proven in the Perennial logic and can be used to
reason about concurrency and crash safety, though, as we did for GoTxn.

\subsection{Modeling pointers}%
\label{sec:goose:pointers}

Pointers turn out to be slightly subtle because of concurrency. In
short, GooseLang disallows concurrent reads and writes to the same
location by making such racy access undefined behavior (any
specification for a program implies that if the precondition holds, the
program never exhibits undefined behavior). The hardware provides some
guarantees, but they are relatively weak: for example, different cores
can observe writes at different times. Go's own memory model specifies
even weaker guarantees. Rather than attempt to formalize Go's rules
(which are complex and involve defining a partial order over all program
instructions), we side step the issue and make any races undefined,
which works for our intended use cases since we always synchronize
concurrent access with locks.

To disallow concurrent reads and writes we first detect them. The key is to make
$x \gets v$, the primitive store operation, \emph{non-atomic} by splitting it
into two operations. The GooseLang semantics tracks the behavior of these
operations by augmenting the heap with extra information; each address in the
semantics has a $\textdom{NonAtom}$ which can be
$\goosekw{Reading} \app n \app v$ if there are $n$ readers and the value is $v$,
or $\goosekw{Writing} \app v$ if a thread is currently writing. We actually only
use non-atomic stores for normal pointers, but we have support for non-atomic
reads as well for iterating over maps.

Ordinarily values in the heap are of the form $\goosekw{Reading} \app 0 \app v$,
to indicate no readers or writers. Writes are split into
$\goosekw{PrepareWrite}(\ell)$, which sets the value of $\ell$ to
$\goosekw{Writing} \app v_{0}$, and $\goosekw{FinishStore}(\ell, v)$ which sets
it to $\goosekw{Reading} \app 0 \app v$. A concurrent write will be undefined
since $\goosekw{PrepareWrite}(\ell)$ requires no concurrent writers, and
similarly for a concurrent read which is undefined if the address is being
written. Non-atomic reads are similar with $\goosekw{StartRead}$ and
$\goosekw{FinishRead}$; these increment and decrement the number of readers,
respectively, so that multiple readers can run concurrently but any concurrent
writer has undefined behavior.

Next, we need reasoning principles to abstract away this complexity from
program verification. Separation logic turns out to provide the right
language to reason about racy access. When a thread owns
$\ell \mapsto v$, we know no other thread can have access to location
$\ell$, so the specifications for reads and writes are unaffected by the
operations being non-atomic (although their proofs are a bit more
complicated to deal with the new semantics). The only change is that the
Read operation is no longer an atomic primitive but a function that
takes two execution steps. In Iris this means that two threads cannot
share memory with an invariant and must mediate access with a lock,
which transfers ownership of the $\ell \mapsto v$ for multiple execution
steps.

\subsection{Locks}

\newcommand{\Acquire}{\goosedef{Acquire}}
\newcommand{\CAS}{\goosedef{CAS}}

As is typical in Goose, locks are not built-in to GooseLang but modeled
using an implementation based on simpler primitives. Since locks are the
only synchronization primitive, implementing them requires shared
concurrent access, which ordinary pointers do not have in GooseLang.
Instead, the language also includes a primitive atomic compare-and-swap
operation that is only used to implement a model of locks. We could also
use the same operation to model Go's low-level atomic operations, like
\cc{atomic.CompareAndSwapUint64} and \cc{atomic.LoadUint64}, but
have not implemented this yet since we don't have code that uses these
low-level synchronization primitives.

The model of locks is simple enough to give the code in its entirety. The lock
is represented as a pointer to a boolean that is true if the lock is held. As a
helper we define $\CAS$ (compare-and-swap), a variant of compare-and-exchange
that only returns a boolean on success and not also the previous value.

\begin{align*}
  \CAS &\defeq \gooselambda{x, v1, v2} \pi_{2}\app \goosekw{CmpXchg}(x, v1, v2) \\
  \goosedef{NewLock} & \defeq \gooselambda{\_} \goosekw{ref} \app \goosekw{false} \\
  \Acquire &\defeq \gooselambda{l} \\
       &\goosekw{let}\app f = (\goosekw{rec}\: \textlog{tryAcquire}(\_) = \\
       &\qquad \gooseif{\CAS \app l \app \goosekw{false} \app \goosekw{true}}%
         {()}{\textlog{tryAcquire} \app ()}) \app\goosekw{in} \\
       &f \app () \\
  \goosedef{Release} &\defeq \gooselambda{l} l \gets \goosekw{false} \\
\end{align*}

Acquiring a lock is modeled as repeatedly using
$\CAS \app l \app \goosekw{false} \app \goosekw{true}$ to
atomically test that the lock is false and set it to true if so, while release
stores false to the lock. This implementation as a spin lock is merely an
operational model that captures what the lock does: acquire blocks until the
lock is free and sets it to locked, while release frees the lock. This code is
used to model Go's builtin \cc{*sync.Mutex}, which is implemented more
efficiently than spin locks with cooperation from the runtime and operating
system.

The specification for locks is a typical one for concurrent separation logic,
based on associating a \emph{lock invariant} with each lock, which is a predicate that holds when the lock
is free. Because this is a separation logic, we can also interpret the lock
invariant as ownership over some data (for example, some region of memory); the
lock mediates access to this ownership, handing it out when the lock is acquired
and requiring it back when the lock is released. We prove this specification
sound against the GooseLang spin-lock implementation.

\subsection{Structs}

\newidentmacro{LoadTyped}
\newidentmacro{StoreTyped}
\newidentmacro{loadField}
\newidentmacro{storeField}
\newdefmacro{structType}

One of the most important features of Go to support is structs. Goose support
for structs uses a form of \emph{shallow embedding} using a combination of
Gallina (Coq's functional language) and GooseLang. We encode
higher-level primitives like field access on top of GooseLang primitives like
tuples and continguous memory.

Struct types are represented using a combination of two Gallina types,
$t \in \textdom{GooseType}$ gives the type of a struct, while
$s \in \textdom{StructDecl}$ is an (anonymous) struct declaration that gives its
field names and their types. The definitions of these two types are given in
\autoref{fig:goose:types}. The exact types are not important in the semantics,
but we do use the shape of the struct to determine how they are laid out in
memory to support pointers to individual fields. Using types to represent these
shapes makes the translation much simpler, since we have access to types from
the Go type checker via the \cc{go/types} package. As an example of how
approximate these types are, it is sufficient to have a \goosekw{ptrT} type for
all pointers; we will use the Go types to determine its type and the types of
its fields if it is a pointer to a struct.

\begin{figure}[ht!]
\begin{mathpar}
  \begin{array}{lrcl}
    \textdom{GooseType} & t &::=& \goosekw{uint64T} \ALT \goosekw{boolT} \ALT
                                  \goosekw{unitT} \ALT
                                  \dots \\
                        &&\ALT & t \times t \ALT \goosekw{ptrT} \ALT \dots \\
    \textdom{StructDecl} & s &\in& \textlog{list}(\textlog{string} \times
                                   \textlog{GooseType}) \\
    \multicolumn{2}{r}{\structType([])} &\defeq& \goosekw{unitT} \\
    \multicolumn{2}{r}{\structType((f, t) :: s)} &\defeq& t \times \structType(s)
  \end{array}
\end{mathpar}
\caption{GooseLang types and struct declarations. These are used in the
  semantics only to give a ``shape'' to data for accessing struct fields, and
  not to represent the Go type system.}%
\label{fig:goose:types}
\end{figure}

First we need to handle struct values. We treat a struct value as just a tuple
of its fields. The definition of the struct gives an ordering of the fields.
This is enough to construct a struct from its fields and to access a field by
name. Here the shallow embedding comes in: we define struct declarations in
Gallina, and struct construction and field access are Gallina definitions that
produce GooseLang expressions, rather than all of this being directly
implemented in GooseLang.

Structs in memory are more interesting than struct values. Structs could
be stored in a single location; due to our non-atomic semantics for
memory, this would be sound even for structs larger than a machine word.
However, this model would be too restrictive: it is safe for threads to
concurrently access \emph{different fields}, just not the same field,
and we actually take advantage of this property (largely to write more
natural Go code; working around this restriction requires splitting
structs up if they are stored in memory).

To support this concurrency, we model a struct in memory with a
flattened representation, with each base element in a separate memory
cell. The flattening applies recursively to fields that are themselves
structs, until a base literal is reached (like an integer or boolean);
base elements are at most a machine word, but can be smaller. When
allocating a new pointer, the semantics flattens composite values and
stores the elements in a sequence of contiguous addresses.

With a flattened representation we need non-trivial code to read a struct
through a pointer, particularly when some of its fields are themselves flattened
structs. Any load of a value from memory is translated to a call to
$\LoadTyped$. This is a Gallina definition of type
$\textdom{GooseType} \to \textdom{Expr}$, where that expression is itself a
GooseLang function that takes a pointer. Hence you can think of it is a macro
for GooseLang that is represented in Gallina as a meta language. $\LoadTyped(t)$
is directed by a type $t$ passed in Gallina to determine how to load and
assemble the fields of a struct of type $t$.

For the purpose of proofs we represent a pointer to an arbitrary type
$t$ with a typed points-to fact of the form $\ell \mapsto_t v$. This
definition expands to a number of primitive points-to facts, one for
each base element. The specification for loading says
$\hoare{\ell \mapsto_t v}{\LoadTyped(t) \app \ell}{\Ret{v} \ell \mapsto_t v}$, which
(much like the primitive load $!\ell$) hides the fact that something
non-atomic is happening and looks like an ordinary dereference.
Similarly, $\StoreTyped$ also takes a type, although the specification
requires the caller to prove that the value has the right shape (in
reality it always will because the Go code we translate from is
well-typed).

The payoff of structs being many independent locations is that it is
possible to model references to individual struct fields. From a pointer
to the root of the struct, a field pointer is simply an offset from that
pointer (skipping the flattened representations of the previous fields).
This offset calculation is much like the code to read a struct from
memory, except that it merely computes a single offset rather than
iterating over all the fields and offsets.

The Go language reference specifies that each field acts like an
independent ``variable'' (which is stored in the GooseLang heap when it
is mutable in Go), so this model should accurately reflect the
specification. Moreover modeling structs as independent locations is
also justified as being similar to how the implementation works. Structs
in memory are in reality represented by contiguous memory, and field
access is implemented by computing a pointer from the base of the
struct. The main difference between the physical implementation and the
model is that we use a single, abstract memory location for each field,
whereas the implementation encodes all data into bytes.

Recall that $\ell \mapsto_t v$ is internally composed of untyped
points-to facts for all the base elements of $v$. In order to reason
about $v$'s fields, we introduce a new struct field points-to fact,
written $\ell \mapsto_{s.f} v$, which asserts ownership of just field
$f$ of a struct with descriptor $s$ rooted at $\ell$, and gives that field's
value as $v$. A recursive function gives an ``exploded'' set of struct
fields by iterating over $t$'s fields and $v$ simultaneously. Then,
we give a proof that $\ell \mapsto_{\structType(s)} v$ is equivalent to the separating
conjunction of this exploded list. The result is a convenient lemma for
reasoning about a struct using its fields: in the forward direction, the
equivalence breaks a large typed points-to into individual fields (with
the values computed from $v$), while in the other direction it allows
to prove a $\ell \mapsto_{\structType(s)} v$ by gathering up all the fields.

The struct field points-to is indispensable in proofs, because the
pattern of \cc{x.f} in Go when $x$ is a pointer is in fact a field
load (in C, this would be written \cc{x->f}). The model
for loading a struct field is a function $\loadField(s, f) \app x$
which is implemented in two steps, first computing the offset to field
$f$ and then dereferencing the computed pointer (in both cases the struct descriptor $s$
describes how to interpret field $f$). Having a field points-to gives
a natural specification for this type of load:
$\hoare{\ell \mapsto_{s.f} v}{\loadField(t, f) \app \ell}{\Ret{v} \ell \mapsto_{s.f} v}$.

The lemmas about breaking apart and recombining structs are all proven
against a simpler model of structs that only requires flattening and
offset calculations. In a sense the model is the trusted code, but the
fact that the struct maps-to exploding lemma is true that all of the
expected Hoare triples hold provides strong evidence that the model is
also doing the right thing. For example, the exploding lemma shows that
field offsets are disjoint, since the struct maps-to can be broken into
field points-to facts for each field.

\subsection{Slices}

One of the most commonly used data structures in Go is the slice
\cc{[]T}, which is a dynamically-sized array of values of type
\cc{T}. Goose supports a wide range of operations on slices,
including appending and sub-slicing; it turns out that the semantics of
mutable slices is non-trivial in Go, resulting in an interesting
semantics and reasoning principles.

A slice is a combination of a pointer, length, and capacity. Slices are
views into a contiguous memory allocation; this view can be narrowed
with sub-slicing operations of the form \cc{s[i:j]}, resulting
in aliased slices. The elements between the length and capacity are not
directly accessible but are used to support efficient amortized appends.
Go's built-in slice operations include bounds checks on all slice
operations and panic if a memory access or sub-slice operation goes out
of bounds.

GooseLang has a primitive for contiguous memory, which we use to model
the allocation underlying a slice (though these are not directly
accessible to Go code, since they do not carry enough information for
bounds checking). On top of these we model slices as a tuple of a base
pointer, length, and capacity.

The GooseLang slice model is directly inspired by the implementation of
slices, including modeling slice capacity. Initially we had a more
abstract model that ignored capacity (which would appear to be just an
optimization), but were surprised to find that this was insufficient to
even accurately model subslicing and appending. Directly modeling slice
capacity was the simplest solution to obtain a model that is faithful to
the Go implementation. The Go language reference isn't specific about
what the slice capacity after various operations should be, so our
GooseLang model picks a non-deterministic capacity in several places
(within appropriate bounds).

\newidentmacro{ptr}
\newidentmacro{len}
\newidentmacro{cap}

The most basic operations on slices are indexing and storing. The
GooseLang model of $s$ is a three-tuple, but for clarity we will refer
to its elements as $\ptr(s)$, $\len(s)$, and $\cap(s)$. The
translation of \cc{s[i]} is essentially a load from
$\ptr(s) + i$ (or undefined behavior if this offset is out-of-bounds).
Similarly \cc{s[i] = x} stores to the same location. We
translate Go's \cc{len(s)} directly to $\len(s)$. Go also supports
accessing a slice's capacity, but this is rarely used and Goose does not
support it.

The Go \cc{append} operation is the most sophisticated to model. The
behavior of \cc{append(s, x)} where \cc{s: []T} and
\cc{x: T} depends on whether there is extra capacity to store the
new element \cc{x}. If there is capacity, then \cc{x} is stored
there and the append returns a new slice with the same pointer but a
larger length. If there is no capacity, then \cc{append} must
allocate a new slice, copy the existing elements to it, and then store
\cc{x}. In the latter case \cc{append} returns a slice with a
fresh pointer.

The difficulty with Go slices arise when supporting subslicing. Consider
\cc{s[:i]}, where \cc{i} is less than \cc{len(s)}.
Clearly this slice should have the same pointer and length \cc{i},
but what should its capacity be? Surprisingly, the capacity of this
prefix is the full capacity of \cc{s}, which means that the unused
elements of \cc{s[:i]} \emph{include the elements of \cc{s}}
beyond the index \cc{i}. As a result, \cc{append(s[:i], x)}
in fact modifies \cc{s[i]}. GooseLang takes care to model this
behavior by implementing \cc{append} exactly as above, taking into
account that \cc{append(s, x)} might be an in-place operation.

The GooseLang model is specifically designed to be sound by sticking to
the Go implementation as closely as possible, but we want reasoning
about slices to be convenient and high-level, without worrying about
slice capacity directly. The design of GooseLang nicely separates the
model from the reasoning principles --- we verify specifications against
the concrete model, so that only the model is trusted and not the
separation logic specifications.

\newcommand{\sliceRep}{\mathtt{sliceRep}}
\newcommand{\sliceCap}{\mathtt{sliceCap}}

\newcommand{\lappend}{\mdoubleplus}

The GooseLang model of slices is based on two abstract predicates:
$\sliceRep(s, l)$ and $\sliceCap(s)$. To model the slice values
themselves we use $s : \textlog{Slice}$ where $\textlog{Slice}$ is a Gallina record; a
function $\textlog{SliceVal}(s) : Val$ converts the Gallina representation to
the GooseLang tuple that the slice model uses. We will only present the
\emph{untyped} version of this specification where $l : \textlog{list}(\textdom{Val})$, but
GooseLang also has a typed version where $l : \textlog{list}(T)$ where there is
some (Gallina) function $\mathtt{to\_val} : T \to \textdom{Val}$. The typed version is
practically convenient in proofs but is only a small extension to the
untyped version.

The first predicate $\sliceRep(s, l)$ gives the abstract value of
$s$, the list of values it contains, excluding additional capacity. It
also represents ownership over all these elements, in terms of the
underlying struct points-to facts. We use this predicate to specify
loads and stores:

\[
  \hoareV{\sliceRep(s, l) * i < |l|}%
{\mathtt{s[i]}}%
{\Ret{v} v = l !! i * \sliceRep(s, l)}
\]
\[
  \hoareV{\sliceRep(s, l) * i < |l|}%
 {\mathtt{s[i] = v}}%
{\Ret{v} v = l !! i * \sliceRep(s, l[i := v])}
\]

Next, $\sliceCap(s)$ is an abstract predicate that represents
\emph{ownership over the capacity} of $s$. It is necessary to append,
since appending might need to write to the capacity, but unneeded to
read and write to a slice.
\[
\hoareV{\sliceRep(s, l) * \sliceCap(s)}%
{\mathtt{append(s, x)}}%
{\Ret{s'} \sliceRep(s', l \lappend [x]) * \sliceCap(s')}
\]

\newidentmacro{sliceFull}

This specification is fairly simple. In fact, we often use a shorthand
$\sliceFull(s, l) = \sliceRep(s, l) * \sliceCap(s)$ when the proof will
always retain ownership of slice capacity, in which case the spec looks
even simpler. However, the proof is non-trivial, since in one case it
moves ownership from $\sliceCap(s)$ to $\sliceRep(s', l \lappend [x])$
(where $ptr(s') = ptr(s)$), while in the other it constructs a
completely new allocation for $s'$.

\newidentmacro{sliceTake}
\newidentmacro{sliceDrop}

The most interesting rules are for subslicing and how they interact with
capacity. Consider \cc{s[:i]} again. While Go has no formal
notion of ownership, our specifications do. We can model the
\emph{value} for \cc{s[:i]} easily enough; call it
$\sliceTake(s, i)$ (it simply reduces the length and keeps the capacity
of $s$, as specified by Go). Now we need to decide how ownership of
$\sliceRep(s, l) * \sliceCap(s)$ should relate to ownership of
$\sliceRep(sliceTake(s, i), take(l, i))$. It turns out there are two
possibilities: we can either give up ownership of the remainder of $s$
in exchange for $\cap(\sliceTake(s, i))$, or we can ignore the
capacity of the subslice and keep
$\sliceRep(\sliceDrop(s, i), drop(l, i))$. These are incomparable and
unexpressed in the code: the decision is based on whether we intend to
append to the subslice but stop using the old slice, or whether we want
to continue using the remainder of \cc{s}.

\tej{why not just use \cc{s[:i]} for $\sliceTake(s, i)$ and $l[:i]$ for
$take(l, i)$? overloading will make everything much easier to read}

Concretely, GooseLang verifies the following entailment for reasoning
about subslicing in terms of the slice model:

$\sliceFull(s, l) \vdash \sliceFull(\sliceTake(s, i), take(l, i))$

This entailment precisely captures how retaining ownership of the
capacity of $\sliceTake(s, i)$ requires giving up the remainder of
$s$.

\begin{align*}
  &\sliceRep(s, l) \dashv\vdash \\
  &\quad \sliceRep(\sliceTake(s, i), take(l, i)) \sep {} \\
  &\quad \sliceRep(\sliceDrop(s, i), drop(l, i))
\end{align*}

This alternative bidirectional entailment splits $s$ into two parts,
but gives up ownership over $\sliceTake(s, i)$'s capacity in exchange
for using those elements in \cc{s[:i]}. From this point it will
not be possible to prove the safety of appending to \cc{s[:i]},
since this would conflict with the separate ownership over
\cc{s[i:]}.

\subsection{Maps}

\newidentmacro{mapVal}
\newidentmacro{mapRep}
\newidentmacro{mapDelete}
\newidentmacro{mapInsert}
\newidentmacro{mapIter}

After slices, maps are the next most commonly used collection type in
Go. We implement maps as lists of key-value pairs, stored in a single
memory location in reverse insertion order. Go's builtin maps are
\emph{not} thread-safe, so the model enforces single-threaded access by
marking the map as being read while reading from it; this re-uses the
race detection for other pointers to ensure that racy access to a map is
undefined behavior, while allowing concurrent read-read access. Maps
support all the Go operations: insertions, reads (including returning
whether the key is present), \cc{len} to get the number of elements
in the map, deletion, and iteration. Go map iteration is
non-deterministic and in practice random, but we did not model this
since it would be challenging to do so; however, the reasoning
principles for map iteration do not expose an iteration order.

The implementation of maps is the most involved out of any of the Go
primitives. It required directly implementing maps (albeit
inefficiently, using an association list) using recursive GooseLang
code. GooseLang is an untyped language, so our first attempts had basic
errors like missing arguments. We improved our confidence in this
implementation both by testing it and by verifying it. Both of these
essentially rule out type errors (regardless of what specification we give),
and the specification is simple enough
to be a reliable test of behavior. Both simple tests and verification
cover easy mistakes like reading the oldest write to a key rather than
the latest, or duplicate keys during iteration (the implementation must
skip over a key after observing it once).

The proof and specification for maps is relatively easy since they are
not safe to use concurrently, so the proof assumes ownership over the entire map. We
treat a map as a pointer to an abstract map value, a GooseLang value
that encodes the entire map data as a list of key-value pairs. The
specification is based on a pure relation $\mapVal(v, m)$ that relates
this encoded value to a Gallina map $m$, which uses \cc{gmap} from
stdpp; for simplicity we use \cc{gmap u64 val} and limit map
keys to integers. Values are not a visible notion to the Go code, since
it always interacts with maps via their pointer, so the specifications
all use $\mapRep(\ell, m) = \exists v.\, \ell \mapsto v * \mapVal(v, m)$. The
indirection is important, since the Go map value
\cc{m : map[uint64]V} is in fact a reference to a map that is
mutated in-place (unlike a slice, which has both pure data --- pointer,
length, and capacity --- and heap data).

For example, this is the specification for map deletion:
\[
\hoare{\mapRep(l, m)}{\mapDelete(l, k)}{\mapRep(l, delete(m, k))}
\]

Map iteration has a more sophisticated specification. Suppose we have a generic loop
over a map in Go like the following:

\begin{verbatim}
for k, v := range m {
  body(k, v)
}
\end{verbatim}

The model for this entire construct is given by $\cc{MapIter}(m, body)$, where
$m$ is a reference to the map and $body$ is an expression for the body of the
loop. Goose translates generic loop bodies, so the Go code does not literally
need to consist of a call to a separate function. The possibility of
\emph{iterator invalidation} adds one subtlety to Go's map iteration --- it
would be incorrect for the body of the loop to modify the map (it might be sound
to write to the map without modifying the domain, but we do not attempt to model
this). The model of maps puts the entire contents of a map in one heap location,
so naively we would not enforce this property. The solution is to mark the map's
reference as being read for the entire duration of iteration, using the
$\goosekw{StartRead}$ and $\goosekw{FinishRead}$ GooseLang primitives at the
beginning and end of $\cc{MapIter}$. If the map value in the heap is
$\goosekw{Reading} \app n \app v$, these operations increment and decrement
(respectively) the reader count $n$, so that any writes within the body have
undefined behavior.

Iteration gets a \emph{higher-order} specification that assumes a specification
for the body, showing it preserves a loop invariant $P$ over the part of the map
consumed so far:

\[
  \infer{
    \forall m_{0}, k, v.\,
    k \notin m_{0} \land m[k] = v \to \\\\
    \hoare{P(m_0)}{body \app k \app v)}{P( m_{0}\mapupd{k}{v} )}
}
{
  \hoare{\mapRep(\ell, m) \sep P(\emptyset)}%
{\cc{MapIter}(\ell, body)}%
{\mapRep(\ell, m) \sep P(m)}
}
\]

On top of this generic specification we prove some alternate specifications that
express the invariant in slightly different ways --- for example, it is often
useful to express the invariant in terms of both the map iterated over so far
and the remaining subset of the map.

Map iteration in Go happens in a non-deterministic order\footnote{In fact the
runtime randomizes the starting position of the iteration, to avoid callers
accidentally relying on any particular behavior. See
\href{https://github.com/golang/go/blob/c379c3d58d5482f4c8fe97466a99ce70e630ad44/src/runtime/map.go\#L844-L850}%
{\cc{mapiterinit} from src/runtime/map.go}.}
Thus strictly speaking $\cc{MapIter}$ should shuffle the elements of the map
before iterating over it, in order to model the non-determinism of the
implementation. We do not (currently) do this, simply because the shuffle would
be hard to implement in GooseLang. However, the specification for map iteration
does not expose an iteration order and would apply unchanged to this more
precise model. All proofs go through this specification, so our proofs should
remain unchanged if $\cc{MapIter}$ started modeling non-determinitic ordering.

\section{Testing Goose}%
\label{sec:goose:testing}

Goose is a trusted component in the entire verification process. For the
overall system's proof to be sound, we rely on the model to produce all
of the behaviors of the Go code; that is, the behaviors of the Go code
(in practice, using the Go compiler) should be a subset of the behaviors
of its translated GooseLang (according to the Coq semantics). As long as
this is case, the proof is sound in that if the modeled system always
satisfies some property the code will, too.

One subtlety to Goose soundness is that the system is automatically sound if
Goose fails to translate some code, or the code does not compile in Coq, or the
model has undefined behavior; in each of these cases it is impossible to verify
incorrect code. These might still reflect a bug in Goose, or at least an
unsupported feature, but they do not compromise soundness. Therefore the
most important bugs are those where the translation is well-defined but its behavior differs
from that of Go; these can compromise soundness of the system and lead
to a proof that is not borne out in practice.

To increase out confidence in Goose, we implemented a large suite of
unit tests. On their own these tests check that Goose continues to translate
existing code (and check that the translation has not unexpectedly
change). To test the soundness of the translation, the relevant comparison is
between Go and GooseLang. Unfortunately GooseLang is not natively an executable language.
Its semantics is expressed as a Coq relation that describes the valid executions
of an expression, but not how to run a particular expression.

To test GooseLang code, we implemented an interpreter in Coq, which can
run GooseLang code and produce either an error due to undefined behavior
or a result. The interpreter is itself
verified to match the semantics. The specification for the interpreter is slightly
subtle in that the interpreter produces only one possible execution rather than
all the executions allowed by the semantics, but
the non-determinism is only due to the choice of what locations to use
for pointers, which should not affect any visible behavior.

GooseLang is a lambda calculus, so
its semantics is expressed as a transition system between expressions.
It is easy to
interpret \emph{pure} reductions like \cc{x + y} where \cc{x}
and \cc{y} are values, since the semantics of these pure expressions on their own
is already given as a Gallina function rather than a relation. The semantics of each core primitive is
given by a transition relation, which the interpreter implements as a function
and its correctness theorem shows this function produces an allowed transition.
For example, the semantics of allocation stores a value in an unused address,
whereas the interpreter concretely identifies an unused location (the maximum
used address, plus one).

The challenge in the interpreter's correctness theorem comes from \emph{context} reductions,
which specify how to find a sub-expression within \cc{e} to reduce
if the head is not immediately a value. The semantics follows a standard
presentation of context reduction using \emph{evaluation contexts}. The
idea is to define a type of evaluation contexts $E \in \mathcal{E}$ that
represent an expression with a hole; $E[e]$ represents filling that hole with
the expression $e$. The possible evaluation
contexts give all the context reductions in one compact rule, \ruleref{context-reduce}: if $e$
can step to $e'$, then $E[e]$ can step to $E[e']$. This rule applies whenever
such an $E$ exists, while the
interpreter recurses through an expression (in the right order) and
evaluates a sub-expression, then fills it into the context. The interpreter's
proof shows that
this traversal is correct, proving that the interpreter and semantics agree on an
evaluation order. Specifically, the interpreter proof shows the interpreter
produces a valid evaluation order, and a separate proof shows that evaluation
contexts are unique.\footnote{See the lemma
\href{https://github.com/mit-pdos/perennial/blob/6f5ed5e7c2d3e8d657a0022c51e1d1e32a81e671/src/goose_lang/lang.v\#L1443-L1447}%
{\cc{head_redex_unique} in src/goose\_lang/lang.v}.} There is other non-determinism in the semantics that the
interpreter does not fully explore, though, such as for allocation.

The test suite is structured as a number of test functions, each producing a
boolean that should be true. To check that the test itself is written correctly,
a Go test checks that it produces \cc{true}. Then to check the semantics of
the translation, the GooseLang test infrastructure translates the test and runs
it through the interpreter, checking that this produces $\goosetrue$ in
GooseLang. While the interpreter is not extremely efficient, it is fast enough
to run the tests in the test suite.

The interpreter and test framework was designed and implemented by
Sydney Gibson, and is described in greater detail in her master's
thesis~\cite{gibsons-meng}. That thesis includes more details on evaluating the
interpreter itself, for example documenting bugs caught by the test suite and
other bugs that are now part of our regression tests.


\section{Limitations}%
\label{sec:goose:limitations}

Notably missing in Goose but prominent in Go is support for interfaces
and channels. We believe both are easy enough to support, but interfaces
were not necessary for our implementation, and rather than channels the code
verified in this thesis
use mutexes and condition variables for more low-level control over
synchronization.

Control flow is slightly tricky since a Go function is translated
to a single GooseLang expression that should evaluate to the function's
return value. Goose supports many specific patterns, especially common
cases like early returns inside \cc{if} statements and loops with
\cc{break} and \cc{continue}, but more complex control flow ---
particularly returning from within a loop --- is not supported. It would be
easiest to express general control flow in continuation-passing style (in which
every GooseLang takes a continuation, and calling this continuation corresponds
to returning from the function in Go), but this
would complicate every specification and the translation of function calls.

Goose do not support Go's \cc{defer} statement. It would be nice to support some
common and simple patterns, particularly for unlocking, by translating
\cc{defer} statically. The behavior of Go's \cc{defer} statement in general is
to push the deferred function to a stack of calls associated with the current function that are executed in reverse order at return
time. GooseLang does not have a first-class notion of a Go function to associate
the stack of deferred functions with, nor the concept of returning
from a function. However, it would be useful to simple static uses of \cc{defer}
at the top-level of a function.

Named return values are recommended to document return parameters, and sometimes
simplify and clarify the body of a function.\footnote{See the description in
\href{https://go.dev/doc/effective_go\#named-results}{Effective Go}.} However, in
general they are quite subtle, due to interaction with \cc{defer} statements and
concurrency~\cite{chabbi:golang-races}. One source of difficulty is that the
return values are treated like local variables declared at the top of the
function, and it is easy to accidentally have races on these variables if they
are accessed concurrently.

Goose does not support mutual recursion between Go functions, and
additionally requires the translation to be in the right order so
definitions appear before they are used. The subtlety here is that
definition management in Go, as in most imperative languages,
conceptually treats all top-level definitions as simultaneous, whereas
Coq processes definitions sequentially. Using Coq definition management
to model Go definition management imposes a limitation compared to Go,
but is much simpler to work with compared to modeling a Go package as a
set of mutually recursive definitions. Reasoning about code written in such a
model would require setting up specification for all the definitions, then
proving them in a recursive way, all while ensuring that no specification is
used before it is proven.

GooseLang does have one extant bug related to evaluation contexts. The contexts
$e \app E$ and $E \app v$ define a right-to-left evaluation order for functions,
which is the opposite of Go. We haven't yet fixed this, either by adjusting the
GooseLang semantics or changing the translation to emit code that explicitly
evaluates all the arguments in the correct order before calling the function.

Map iteration in Go happens in a non-deterministic order.\footnote{In fact the
runtime randomizes the starting position of the iteration, to avoid callers
accidentally relying on any particular behavior. See
\href{https://github.com/golang/go/blob/c379c3d58d5482f4c8fe97466a99ce70e630ad44/src/runtime/map.go\#L844-L850}%
{\cc{mapiterinit} from src/runtime/map.go}.} Thus, strictly speaking, the model
of map iteration described in \cref{sec:goose:maps} given by \cc{MapIter} should
shuffle the elements of the map before iterating over it, in order to model the
non-determinism of the implementation. Goose does not (currently) do this,
simply because the shuffle would be hard to implement in GooseLang. However, the
specification for map iteration does not expose an iteration order and would
apply unchanged to this more non-deterministic model. All proofs go through a
common iteration specification, so our proofs should remain unchanged if
\cc{MapIter} started modeling non-deterministic ordering.

\section{Conclusion}

Goose is an approach for verifying Go code. We define GooseLang as a model of Go
and automatically translate a subset of Go to this language. GooseLang comes
with a number of reasoning principles for handling features of Go. The benefit
of this approach is the ability to write high-performance code in a
productive language, with convenient reasoning while verifying
that code. Several aspects of the design contribute to making the approach
sound, ranging from the subset of Go supported, to the design of GooseLang, and
the use of standard Go tools for analyzing the source code.

Our main use case for Goose for this thesis was to verify GoTxn, but we believe
the tool and approach are more generally applicable, even without concurrency or
crash-safety reasoning. These ideas could also be productively applied to
languages other than Go --- I am personally excited about the prospect of having
a version for Rust.\footnote{I think the translator for Rust to Coq should be
called Roost.}
