\section{Conclusion}
\label{sec:gotxn:concl}

\txn is the first concurrent crash-safe journaling
system with a machine-checked proof, built on top of the Perennial 2.0 framework.
\txn uses Perennial's techniques, including lifting and crash framing, to
carry over the atomic benefits of journaling to its formal specification.
This enables storage applications to use mostly crash-free reasoning in
their proofs.  For example, in the verified \simplenfs server, only \simplenfsCrashLOC{}
lines of code, out of \simplenfsLOC{}, required crash reasoning.  \txn is sophisticated
enough to implement a functional (but unverified) NFSv3 server, \gnfs, that achieves
90\% of the performance of a Linux ext4 NFSv3 server on a development workload, far higher than
any previous verified file systems, and \txn's concurrency enables \gnfs
to scale with concurrent client requests.  To simplify \txn's proofs,
Perennial provides logically atomic crash specifications, which capture the
crash properties of internal interfaces as single logical transitions,
enabling modular proofs for \txn's internal layers.
