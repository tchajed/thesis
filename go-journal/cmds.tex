% Other name ideas:
% CSI (Crash-Safe Iris)
% CS-Iris
% Cyris
% Ciris

% Rhizome
% Scion
% One of the so-called ``resurrection plants'', which can survive dehydration for long periods (astericus, lichen, ...)?
% RootStock (a kind of rhizome, but a little less alien sounding. unfortunately, also the name of a cloud provider.)

% Perennial
% long form: Perennial Iris (though note that irises are perennial)

% Succulent (David suggested Sucqulent, maybe Suqulent?)

% Geophyte: "Plants that have an underground storage organ are called geophytes
% in the Raunkiær plant life-form classification system. [2][3] Storage organs
% often, but not always, act as perennating organs which enable plants to survive
% adverse conditions (such as cold, excessive heat, lack of light or drought)."
% (Wikipedia)

%% abbreviations
\newcommand{\txn}{GoJournal\xspace}
\newcommand{\simplenfs}{SimpleNFS\xspace}
\newcommand{\gnfs}{GoNFS\xspace}

\newcommand{\fstar}{F${}^\star$\xspace}

%% editing markup
\newcommand{\insertnote}[3]{\noindent\textcolor{#1}{\textbf{#2:} #3}}
\newcommand{\note}[1]{\insertnote{blue}{NOTE}{#1}}
\newcommand{\todo}[1]{\insertnote{red}{TODO}{#1}}
\newcommand{\tej}[1]{\insertnote{red}{TC}{#1}}
\newcommand{\joe}[1]{\insertnote{red}{JDT}{#1}}
\newcommand{\mfk}[1]{\insertnote{red}{MFK}{#1}}
\newcommand{\ralf}[1]{\insertnote{red}{RJ}{#1}}
%% for checking length without todos/notes
%\renewcommand{\insertnote}[3]{}
\newcommand{\todocite}[1]{\textcolor{red}{[cite #1]}}

%% semantic formatting
\newcommand{\code}[1]{\texttt{\detokenize{#1}}}
\newcommand{\parahdr}[1]{\noindent\textbf{#1}}
\newcommand{\cc}[1]{\mbox{\smaller[0.5]\texttt{\detokenize{#1}}}}
\newcommand{\scc}[1]{\mbox{\textsc{\detokenize{#1}}}}

\newcommand{\loc}[1]{\num[group-separator={,},group-minimum-digits=4]{#1}}

%% \theoremstyle{definition}
%% \newtheorem{theorem}{Theorem}

\newcommand{\sep}{*}
\NewDocumentCommand \hoareC {m m m m}{
  \curlybracket{#1}\spac #2 \spac \curlybracket{#3}\curlybracket{#4}%
}

\NewDocumentCommand \hoareCV {O{c} m m m m}{
  {\begin{aligned}[#1]
  &\curlybracket{#2} \\
  &\quad{#3} \\
  &\curlybracket{#4} \\
  &\curlybracket{#5}
  \end{aligned}}%
}

\newcommand{\SKIP}{\mathit{noop}}
\renewcommand{\Ret}[1]{\text{\textbf{ret}}\,\, #1,\,}


\renewcommand{\topfraction}{0.9}
\renewcommand{\dbltopfraction}{0.9}
\renewcommand{\bottomfraction}{0.8}
\renewcommand{\textfraction}{0.05}
\renewcommand{\floatpagefraction}{0.9}
\renewcommand{\dblfloatpagefraction}{0.9}
\setcounter{topnumber}{10}
\setcounter{bottomnumber}{10}
\setcounter{totalnumber}{10}
\setcounter{dbltopnumber}{10}

\bibpunct[: ]{[}{]}{,}{n}{XXX}{XXX}

% \newcommand{\circularLOC}{\loc{108}}
% \newcommand{\circularLOP}{\loc{1911}}
% \newcommand{\circularRatio}{18}
%
% \newcommand{\walLOC}{\loc{571}}
% \newcommand{\walLOP}{\loc{12940}}
% \newcommand{\walRatio}{23}
%
% % Use a different naming convention to avoid confusing txn layer with gotxn
% \newcommand{\txnlayerLOC}{\loc{134}}
% \newcommand{\txnlayerLOP}{\loc{2440}}
% \newcommand{\txnlayerRatio}{18}
%
% \newcommand{\buftxnLOC}{\loc{117}}
% \newcommand{\buftxnLOP}{\loc{1180}}
% \newcommand{\sepbuftxnLOP}{\loc{1326}}
% \newcommand{\allbuftxnRatio}{21}
%
% \newcommand{\lockmapLOC}{\loc{118}}
% \newcommand{\lockmapLOP}{\loc{867}}
% \newcommand{\lockmapRatio}{7}
%
% \newcommand{\miscLOC}{\loc{293}}
% \newcommand{\miscLOP}{\loc{3785}}
% \newcommand{\miscRatio}{13}
%
% GoJournal total
\newcommand{\gotxnLOC}{\loc{1345}}
\newcommand{\gotxnLOP}{\loc{25797}}
% \newcommand{\gotxnRatio}{18}
%
% \newcommand{\gnfsLOC}{\loc{2941}}
%
\newcommand{\simplenfsLOC}{\loc{462}}
\newcommand{\simplenfsLOP}{\loc{3749}}
\newcommand{\simplenfsRatio}{8}
%
\newcommand{\simplenfsCrashLOC}{\loc{44}}
