\section{Programming with GoTxn}
\label{s:gotxn-api}

\begin{figure}
\begin{minted}{go}
type Addr struct {
  Blkno  uint64
  Offset uint64
}

// starting and stopping a transaction
func Begin() *Txn
func Abort(tx *Txn)
func Commit(tx *Txn)

// operations within a transaction
func Read(tx *Txn, a Addr, sz uint64) []byte
func ReadBit(tx *Txn, a Addr) bool
func Write(tx *Txn, a Addr, d []byte)
func WriteBit(tx *Txn, a Addr, d bool)

// allocator API
func NewAllocator(max uint64) *Allocator
func Alloc(a *Allocator) uint64
func Free(a *Allocator, n uint64)
\end{minted}
  \caption{The API for the transaction system and allocator. Reads and writes
    between \cc{Begin} and \cc{Commit} appears to execute atomically on disk and
    for other threads, while \cc{Abort} guarantees the transaction has no
    effect. The allocator's \cc{Alloc} and \cc{Free} operations are safe to call
    concurrently.}
\label{fig:txn-api}
\end{figure}

The transaction system gives the caller a way to read and write on-disk
objects (which can be smaller than the 4KB block size of the underlying disk),
in such a way that a whole sequence of reads and writes appears atomic. The
programming interface is listed in \cref{fig:txn-api}. In practice this makes
it easier to implement a file system by handling the challenges of crash safety
and concurrency, leaving the file system to implement its data structures and
business logic using disk objects.

To use GoTxn, the file system follows a similar pattern for each operation of
wrapping all of the code for that operation in a single transaction. Operations
start by creating an empty transaction with \cc{Begin()}.
The entire
transaction appears to execute atomically when the operation finishes it
with \cc{Commit}, or the transaction is discarded with no effect on
\cc{Abort}. Reads and writes operate on addresses which specify a
position by giving a block number and an offset in bits (always less
than $4096 \cdot 8$, the number of bits in a block). The \cc{Read}
method requires an explicit size argument while the size of a
\cc{Write} is implicit in the size of the \cc{data} slice. We separate
out the bit-sized operations to \cc{ReadBit} and \cc{WriteBit} (rather
than using a single-element byte slice) to simplify the specification.

\Cref{fig:txn-api} also includes an API for allocation alongside the
transaction API.\@ \Cref{sec:daisy-nfs} will explain in more detail why this
API must be part of the transaction system, since it is relevant to the
file-system design. The allocator should be considered part of the transaction
system API insofar as transactions are allowed to use this concurrent allocator.
In general using shared memory within a transaction is not permitted since it
would compromise the atomicity guarantees --- the transaction system acquires
locks to make reads and writes seem atomic, but it doesn't have any locking
discipline for other state.

%% The state of the transaction system (the transactional disk transactions
%% manipulate) looks much like a flat array of bytes.
%% However, the caller cannot
%% read and write arbitrary regions of this array due to restrictions in the
%% gojournal code and proof. all reads and writes must be within a single 4kb block
%% on disk, and of a power-of-two number of bytes or a single bit.

% In practice the file system uses three kinds of objects: full blocks are used
% for data (both for directories and data files), bit objects comprise the inode
% and block allocators, and 128-byte objects are used to represent inodes. The
% file-system statically allocates regions for the inodes, allocator bitmaps,
% and data blocks, so that object sizes never change.

The transaction system uses two-phase locking on top of a verified journal,
GoJournal. The journaling system makes writes atomic but relies on the two-phase
locking implementation to avoid conflicts between concurrent transactions.
While a transaction is running, it acquires locks for
any addresses it reads or writes, and on abort or commit, it releases
all locks held. Transactions that don't conflict can prepare in
parallel, and GoJournal will batch concurrently committed
transactions for efficiency.

Acquiring multiple locks during a transaction creates the possibility
for deadlocks, if two threads acquire a pair of locks in the opposite
order. The two-phase locking implementation does not implement a
specific lock acquisition order, leaving it to the file system to
avoid deadlock --- for example, the implementation of \cc{RENAME}
makes sure to lock the smaller inode number first (by reading from it) if the
rename is between different directories.
