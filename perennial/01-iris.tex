\section{Primer on Iris and separation logic}

A program logic is a formal system for specifying and reasoning about programs.
One of the simplest program logics is Hoare logic, still the basis for much
sequential reasoning today. The judgments of Hoare logic consist of
specifications of the form $\hoare{P}{e}{Q}$, which is interpreted as meaning
``if $e$ is run in a state where $P$ holds and it terminates, then the final
state will satisfy $Q$''. The logic has various rules for proving and combining
these specifications.

\subsection{Separation logic}

Separation logic is an extension of Hoare logic that has proven profitable for
reasoning about heap-manipulating programs with pointers and concurrency
(surprisingly, the same techniques help solve both problems). A good
introduction to the basic ideas of separation logic is found in O'Hearn's
``Separation Logic'' article~\cite{ohearn:seplogic}. Here we will give a more
terse overview to introduce the relevant notation.

Separation logic introduces some notation for the logical assertions that
describe the heap. The core assertion to talk about pointers is $p \mapsto v$,
which says that the pointer $p$ when dereferenced has value $v$, or more simply
``$p$ points to $v$''. Separation logic introduces a new logical connective
$P \sep Q$, pronounced ``$P$ and separately $Q$'', which says that the heap can
be divided into two disjoint pieces, one satisfying $P$ and the other satisfying
$Q$. We can relate two propositions with $P \proves Q$, where $\proves$ can be
read as either ``entails'' or ``proves'', which says that in any heap where $P$
holds, $Q$ must also hold.

When working in separation logic, specifications like $\hoare{P}{e}{Q}$ are
generally stated in a ``small footprint'' style where $P$ mentions only the
state $e$ relies on for its execution. This intuition is backed by the
celebrated frame rule, which says that if $\hoare{P}{e}{Q}$ holds, any disjoint
state is unaffected, namely $\hoare{P \sep F}{e}{Q \sep F}$.

Instead of working with Hoare triples, it is convenient to instead define
specifications in a different style of \emph{weakest preconditions} (WPs). We will use
$\wpre{e}{Q}$ to denote the weakest precondition of $e$ with postcondition $Q$;
if $e$ is run in a state satisfying $\wpre{e}{Q}$ and terminates, the final
state will satisfy $Q$. For some inuition behind weakest preconditions note that
$P \proves \wpre{e}{Q}$ is equivalent to $\hoare{P}{e}{Q}$. An excellent
explanation of weakest preconditions vs Hoare triples can be found in
``Separation logic for sequential programs''~\cite{chargueraud:seq-seplogic}.

As an example of weakest precondition reasoning, the frame rule becomes
\ruleref{wp-frame} in terms of WPs. Reading this rule forwards, if we have $F$
and separately $\wpre{e}{Q}$, then we can move $F$ to the postcondition because
separation logic guarantees the proof of the WP does not affect or invalidate
the resources in the frame $F$.

\begin{figure}
\begin{mathpar}
\inferH{wp-frame}%
{}%
{F * \wpre{e}{Q} \proves \wpre{e}{F * Q}}

\inferH{wp-mono}%
{P \proves P' \and \forall v. ([v/x] Q' \proves [v/x] Q) \and \hoare{P'}{e}{Q'}}%
{\hoare{P}{e}{Q}}

\inferH{wp-seq}%
{\hoare{P}{e_1}{Q} \and \hoare{Q}{e_2}{R}}%
{\hoare{P}{e_1;\, e_2}{R}}

\inferH{wp-load}%
{}{\hoare{p \mapsto v}{\load{p}}{\Ret{v} p \mapsto v}}

\inferH{wp-store}%
{}{\hoare{p \mapsto v}{\store{p}{v'}}{p \mapsto v'}}

\end{mathpar}
\caption{Selection of proof rules for sequential separation logic.}
\label{fig:wp-rules}
\end{figure}

\subsection{Concurrency with Iris}

Iris generalizes separation logic to also reason about concurrency. A full
explanation of the Iris logic is out-of-scope for the thesis; ``Iris from the
ground up''~\cite{jung:iris-jfp} is a comprehensive introduction while the
original ``Iris 1.0'' paper~\cite{jung:iris-1} is a shorter introduction for a
reader already familiar with separation logic. Two features of Iris are most
relevant since they are used in the GoTxn proof: ghost state and invariants.

Ghost state is a technique for reasoning about a program by augmenting its
physical state (local variables and the heap) with some additional \emph{ghost
state} which is maintained only for the sake of the proof and has no affect on
the program's execution (hence the term ``ghost''). It is easier to understand
ghost state as a programming language feature, as in Dafny, than in Iris, so let
us first see how they help there and then return to Iris.

In Dafny, a variable can be marked \cc{ghost}. Ghost variables can be written
and read in the proof, but Dafny enforces that the ghost variables' values never
influences execution; they can only be used to inform uses of lemmas,
assertions, and other proof annotations. Then at runtime ghost variables and all
uses of them are \emph{erased} before running the program. Why would we add a
ghost variable to a program? The simplest examples are code where a ghost
variable holds the old value of some variable, say prior to a loop; this lets
the proof refer to the old value while clarifying that the regular execution
does not need it.

%% for a concrete example see bubble sort in
%% https://www.doc.ic.ac.uk/~scd/Dafny_Material/Lectures.pdf

We can think about the technique of ghost variables as carrying out the proof in
two steps. First, for every normal execution (without ghost variables), the
proof constructs a ``parallel'' execution that tracks the ghost state alongside
physical state. The rest of the proof is carried out on the parallel execution,
but the proofs apply to the normal execution because by design they have the
same behavior. We have reason to believe verifying the program with ghost
variables is easier to verify because the ghost variables can track important
information about the history of the program, such as in the example above of
the pre-loop values of the local variables.

In Iris, the proof is a separate entity from the code. The program logic still
has a way to use ghost variables, with proof rules that construct and update a
ghost variable, applied at the appropriate points in the proof rather than added
to the code. The high-level idea for why this works --- the parallel execution ---
remains the same. In fact the argument is a bit simpler because it is obvious
that the ghost variables do not affect program execution, since their creation
and updates only appear in the proof.

So far, we've explained ghost state as being ghost variables, with the familiar
API where they can be read and written. Iris ghost state is a bit more
sophisticated in order to support concurrency reasoning. First, Iris has
separation logic assertions for \emph{ownership} of ghost state, which can be
split and divided among threads. Second, ghost state can have restrictions on
how it may be updated.

\newcommand{\dashedbox}[1]{\boxedassert[densely dashed]{#1}[]}
\newcommand{\ghostvar}[2][]{\dashedbox{\gamma \mapsto_{#1} #2}}

A simple example to illustrate both principles is Iris's fractional ghost
variables. We will write $\ghostvar[q]{v}$ to indicate that $\gamma$ has value
$v$ (of any fixed type) and that we own a (positive) fraction $q$ of it.
Complete ownership $\ghostvar[1]{v}$ is common enough that we will also write it
$\ghostvar{v}$, with no fraction. There are several rules for manipulating and
using this fractional ghost state:

\begin{mathpar}
  \inferH{frac-alloc}{}%
  {\proves \upd \exists \gamma.\, \ghostvar{v}}

  \inferH{frac-update}{}%
  {\ghostvar{v} \proves \upd \ghostvar{v'}}

  \inferH{frac-split}{}%
  {\ghostvar[q_1 + q_2]{v} \provesIff \ghostvar[q_1]{v} \sep \ghostvar[q_2]{v}}

  \inferH{frac-agree}{}%
  {\ghostvar[q_1]{v_1} \sep \ghostvar[q_2]{v_2} \proves v_1 = v_2}

  \inferH{upd-fire}%
  {P \proves \wpre{e}{Q}}%
  {\upd P \proves \wpre{e}{Q}}
\end{mathpar}

The new notation $\upd P$ is an Iris \emph{update modality}. The assertion
$\upd P$ expresses ownership of resources which could be used to become $P$ with
some update to ghost state. The dashed box is intended to emphasize that this
assertion is about ghost state and not about the heap, as in the points-to
assertion $p \mapsto v$. As an example of proving an update modality,
\ruleref{frac-alloc} shows that starting with no assertions it is possible to
allocate a new ghost variable $\gamma$ with value $v$ and complete ownership
over it; this is analogous to how the Hoare triple for allocation has no
precondition. The formal rule that allows the user to get access to $P$ is
\ruleref{upd-fire}. It corresponds to advancing the proof of $\wpre{e}{Q}$ by
changing whatever ghost state is needed to turn $\upd P$ into $P$. As long as
the user is proving a weakest precondition, we can always apply this rule to
``eliminate'' an update modality.

Fractional ghost state can be updated after creation with \autoref{frac-update}.
The update requires full ownership. Fractional ghost variables can instead be
split into smaller pieces with \autoref{frac-split}; these pieces can no longer
be updated (since this would invalidate the rest of the assertions), but two
assertions for the same ghost variable must be for equal values since the
underlying variable has only one value.

In this thesis we describe constructing ghost state to carry out parts of the
proof. In reality all ghost state in Iris is constructed by defining an
algebraic structured $M$ called a ``resource algebra'', and ownership is always
ownership of an element $a \in M$, but we will not explain the details --- see
``Iris from the ground up''~\cite{jung:iris-jfp}. When we explain some ghost
state, it will be in terms of some resources and rules that allow updating those
resources. The Iris logic ensures that the updates are ``sound'', enforcing a
global property that updates to a resource in one part of the proof never
invalidate resources owned by concurrent threads at the same time.

A fundamental reasoning principle for concurrency is the notion of an
\emph{invariant}. Threads will eventually need to share state, and invariants
are the main way to reason about how threads coordinate on that shared state.
Invariants in Iris are another resource: we write $\knowInv{}{I}$ for the
assertion that says $I$ is an invariant. Once this invariant is established, the
proof rules in Iris guarantee that $I$ holds at all steps of the program. A
thread that has $\knowInv{}{I}$ in its precondition can make use of the
invariant by ``opening'' it, but only for a single program step; it must be
returned afterward to guarantee the invariant holds for other threads. Finally,
invariants are freely \emph{duplicable} --- that is,
$\knowInv{}{I} \proves \knowInv{}{I} \sep \knowInv{}{I}$ --- reflecting that
knowledge of an invariant, once it is established, is an assertion that all
threads agree on which cannot be invalidated.

\tej{Might not need persistently modality, shouldn't need later modality}

\tej{Look at Perennial 1.0 paper and Later Credits paper for some inspiration on
introducing Iris}
