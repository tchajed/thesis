\section{Reasoning techniques using Perennial}
\label{sec:perennial:techniques}

\tej{May not want this section; instead write some text in GoTxn chapter about
how circ proof works in detail (because it happens to illustrate most of the
techniques), and how refinement works. Then see if there's anything general to
say about Perennial's mechanisms here for that to make sense. It might also make
sense to explain circ here, since it's pretty easy to explain so it doesn't need
to be motivated by the rest of GoJournal or even the WAL.}

Using invariants in clever ways, for both standard concurrency reasoning and
crashes

Logically-atomic crash specs expose atomic updates to the state of a module

Notion of a ``durable'' resource via post-crash modality + combining with an
invariant

Refinement reasoning uses lower-level tools from Perennial for a different
soundness theorem
