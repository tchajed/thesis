DaisyNFS is a verified, concurrent, and crash-safe file system, built on top of
GoTxn. This chapter describes its specification, implementation, and proof. A
key aspect of DaisyNFS is that the proofs of the file-system code use \emph{sequential
reasoning}, even though DaisyNFS is a concurrent file system --- this is possible
due to a \emph{simulation-transfer theorem}. The theorem uses the strong
guarantee of GoTxn to enable verifying a system written with transactions using
entirely different techniques, so that the file-system proofs are carried out in
Dafny, a sequential verification-oriented programming language.

\section{Motivation and goals}%
\label{sec:daisy-nfs:motivation}

DaisyNFS is an implementation of the Network File System (NFS) API.\@ This is a
standard file-system interface, specified in RFC 1813~\cite{RFC:1813}
(specifically this is the standard for NFS version 3, which is what DaisyNFS
implements). NFS is widely used, generally to export a file system across a
network to multiple clients, and widely supported by operating systems ---
Windows Server, macOS, and Linux each include an implementation of both an NFS
server and client.\footnote{Windows 10 and 11 can act as NFS clients but do not
include an NFS server.}

In order to verify DaisyNFS, we first need a specification. A natural starting
point is RFC 1813, which is supposed to prescribe what a valid NFSv3 server
does. However, the RFC is a prose document with about 130 pages of English text,
which is unsuitable for a mathematical proof. Thus the first step is
to turn the RFC into something more precise. Our NFS specification uses a
transition system defined in Dafny for this purpose, with an abstract state that
can capture the state of the server at all times, and transitions that define
for each NFS operation how the state evolves and what the return value is. The
transition system allows non-determinism in the specification to give the
implementation some flexibility.

The transition system describes the abstraction of an NFS server, but what does
it mean for the \cc{daisy-nfsd} binary to implement this protocol? To formalize
DaisyNFS's correctness we use \emph{refinement}, as used in the GoTxn
specification as well. The server binary is a refinement of the NFS transition
system (the specification transition system) if every execution of the code has
user-visible behavior that the specification could also produce (also phrased as
a behavior that the specification allows). In our specification the visible
behavior of both systems is defined to be network requests and responses.

Proving refinement for DaisyNFS's implementation directly would be quite challenging.

\tej{still say ``this paper'' in several places}

\section{Motivation and approach}%
\label{sec:daisy:motivation}

DaisyNFS is an implementation of the Network File System (NFS) API.\@ This is a
standard file-system interface, specified in RFC 1813~\cite{RFC:1813}
(specifically this is the standard for NFS version 3, which is what DaisyNFS
implements). NFS is widely used, generally to export a file system across a
network to multiple clients, and widely supported by operating systems ---
Windows Server, macOS, and Linux each include an implementation of both an NFS
server and client.\footnote{Windows 10 and 11 can act as NFS clients but do not
include an NFS server.}

In order to verify DaisyNFS, we first need a specification. A natural starting
point is RFC 1813, which prescribes what a valid NFSv3 server
does. However, the RFC is a prose document with about 130 pages of English text,
which is unsuitable for a mathematical proof. Thus the first step is
to turn the RFC into something more precise. Our NFS specification (described in
\cref{sec:daisy:nfs}) uses a
transition system defined in Dafny for this purpose, with an abstract state that
can capture the state of the server at all times, and transitions that define
for each NFS operation how the state evolves and what the return value is. The
transition system allows non-determinism in the specification to give the
implementation some flexibility.

The transition system describes the abstraction of an NFS server, but what does
it mean for the \cc{daisy-nfsd} binary to implement this protocol? To formalize
DaisyNFS's correctness we use \emph{refinement}, as used in the GoTxn
specification as well. The server binary is a refinement of the NFS transition
system (the specification transition system) if every execution of the code has
user-visible behavior that the specification could also produce (also phrased as
a behavior that the specification allows). In our specification the visible
behavior of both systems is defined to be network requests and responses. This
notion of refinement also considers crashing executions (where the server
suddenly stops, the machine reboots, and the server resumes), and concurrent NFS
requests.

DaisyNFS's concurrent refinement is a much more sophisticated property to verify
than sequential refinement. In both cases, the basic technique is to construct a
\emph{forward simulation} from the code execution to the specification
transition system, which requires an invariant connecting their states and a
proof that shows the invariant is preserved. In a sequential, non-crash
simulation it is sufficient to show that each operation restores the invariant
when it returns, since its intermediate states are invisible.
\Cref{fig:concurrent-refinement} illustrates the obligation for a concurrent
refinement, while \cref{fig:refinement} shows the much simpler obligation needed
for a sequential simulation. The complication in a concurrent simulation is that the code can have many concurrent
threads, each running a different operation at the specification level. The
proof of any given operation must cover its intermediate states since at any
time other threads might run, and similarly the proof must consider interference
from other threads at any time.

\begin{figure}
  \includegraphics{fig/concurrent-refinement.png}
  \caption[Proving a concurrent simulation]{This figure depicts the
    concurrent-simulation obligation, from the perspective of the green thread.
    Proving concurrent refinement requires a proof that shows every operation
    (1) preserves the invariant at all intermediate points, and (2) simulates
    the abstract specification for the operation at some \emph{linearization
    point}. Unlike sequential refinement (see \cref{fig:refinement}), the
    proof must show the invariant holds at intermediate points in order to
    reason about potential interference with other threads.}
  \label{fig:concurrent-refinement}
\end{figure}


The design of DaisyNFS uses transactions, and in particular GoTxn, to simplify
the proof of concurrent refinement. Transactions appear to run
sequentially, and thus should permit reasoning about the body of each
transaction sequentially even though the actual execution interleaves multiple
transactions. This chapter describes a formalization of this intuition in the
form of a \emph{simulation-transfer theorem} (described in
\cref{sec:daisy:simulation-transfer}) which proves that a system implemented
with transactions that is verified with a sequential forward simulation against
some specification refines the same specification in the sense of a concurrent,
crash-safe refinement when run through GoTxn. The proof of this theorem is an
extension of the program refinement specification from \cref{sec:txn:spec}.

Due to simulation transfer, we are able to use a simpler verification
methodology of sequential simulation for the DaisyNFS file-system code, compared
to the Perennial program logic used to verify the transaction system underneath.
To fully take advantage of this difference, DaisyNFS is verified using
Dafny~\cite{leino:dafny}, an entirely different tool. Dafny is a
verification-oriented programming language that is restricted to sequential
proofs. The use of Dafny greatly reduces the proof burden for verifying
DaisyNFS, because sequential proofs are well-suited to automation and Dafny's
automation is well-developed (in contrast automation for concurrent proofs is
still nascent, and would need to be integrated into Perennial to be used for
these proofs).

The value of sequential proofs can be seen in the proof-to-code ratio for the
transaction system, which is $18\times$, versus the Dafny proofs which required
about $2\times$ as many lines of proof as code. Further evidence can be seen in
the incremental development of DaisyNFS, which \cref{sec:eval:incremental}
further elaborates on.

Simulation transfer greatly reduces the proof burden for verifying DaisyNFS.
There still remains a challenge of actually implementing the file system using
transactions. \Cref{sec:daisy:design} discusses some key challenges to address,
in particular how to ensure transactions are bounded in size (since GoTxn
transactions cannot exceed the size of the on-disk circular buffer) and avoiding
deadlocks in the transaction system.

The design of DaisyNFS does impose limitations. The proof approach relies on
transactions appearing to run sequentially, which prevents modifying state
outside the transaction system. There are cases where that would get better
performance in exchange for a more difficult proof. The transaction system does
not have a proof of liveness, and the file-system proof does not show that
transactions avoid deadlock. Our NFS implementation does not cover some
features, such as symbolic links, hard links, and paginated \cc{READDIR}; we
believe all of these could be implemented and specified with the same approach
but have not done so in our prototype.

\section{System design}%
\label{sec:system}

% \begin{figure}
%   \center
%   \includegraphics{drawn-diagrams/system-overview.png}
%   \caption{The structure of the code.}
%   \label{fig:system}
% \end{figure}

\begin{figure}
  \center
  \begin{tikzpicture}[>=latex, node distance=1.25cm]

 \tikzset{
    genericnode/.style={rectangle,draw,minimum width=2cm, minimum height=.85cm, align=center,},
    layer/.style={
      genericnode,
      alias=genericnode,% <- alias added
      label={[anchor=south west,shift={(genericnode.north west)},inner sep=2pt]{\tiny #1}}% position the label using the alias
    }}

%\tikzstyle{layer}=[rectangle, draw, minimum width=2cm, minimum height=.85cm, align=center];
\tikzstyle{genlayer}=[dashed, layer={}];
\tikzstyle{edge}=[->,thick];

\draw node (dispatch) [layer=Go] {Dispatch Loop};
\draw node (goout) [genlayer,below of=dispatch] {Go output};
\draw node (txn) [layer=Go,below of=goout] {GoTxn};

\draw node (coq) [left=1.5cm of txn, align=center] {Verification \\ in Perennial};
\draw node (dafny) [left=1.25cm of goout, layer=Dafny] {File-system \\ Operations};

\draw node (out) [below=1cm of txn] {\texttt{daisy-nfsd} binary};

\draw [thick] (dispatch.south) -- (goout.north);
\draw [thick] (goout.south) -- (txn.north);
\draw [edge] (txn.south) -- node[right] {\texttt{go build}} (out.north);
\draw [edge] (txn.west) -- (coq.east);

\draw [edge] (dafny.east) -- node[above] {\texttt{dafny}} (goout.west);


\end{tikzpicture}

  \caption{The structure of the code.}
  \label{fig:system}
\end{figure}

As shown in \autoref{fig:system}, \sys is implemented in three layers:
1) a dispatch loop that speaks the NFS wire protocol and calls the
appropriate method for each operation; 2) a Dafny class that
implements each method; and 3) a transaction system that applies the
updates of each method to the disk atomically.  The dispatch loop is
unverified; we assume that the server correctly decodes messages,
calls the right method for an operation, and encodes the response. The
middle layer implementing the file-system operations is implemented
and verified in Dafny, which has a backend for Go.  The
third layer is directly written in Go and verified using Coq and
Perennial.  By implementing the file system on top of the transaction
system, we can implement each NFS method in Dafny as sequential code
calling into a concurrent transaction system library. The NFS
operations supported by \sys are listed in \autoref{fig:nfs}.

\subsection{Dafny file system}

%% We focus on the
%% design of the transaction system here, but the file system also has several
%% internal abstractions. These abstractions are primarily interesting in a
%% verification context so we discuss them later in \autoref{sec:design}.

%% The file-system implementation calls the transaction system to store all
%% file-system data, ensuring that it is written atomically and durably.

% \begin{figure}
%   \begin{center}
%   \includegraphics[width=\columnwidth]{drawn-diagrams/disk-layout.png}
%   \end{center}
%   \caption{The layout of the file system on top of the transaction system's
%     disk. The number of inode blocks and data bitmap blocks is a compile-time
%     constant, but easy to change without affecting the proofs.}
%   \label{fig:layout}
% \end{figure}

\begin{figure}
  %\includegraphics[width=\columnwidth]{drawn-diagrams/disk-layout.png}
  \begin{tikzpicture}[>=latex,scale=1.1]

  \tikzstyle{circlog}=[thick,rectangle, draw,minimum height=1cm, align=center];

  \node[circlog,minimum width=1cm,
       label={[label position=below, align=center]Super\\block}] (logger) {};
  \node[circlog,minimum width=1.7cm, right=0cm of logger,
    inner sep=0pt,
    rectangle split,
    rectangle split every empty part={},
    rectangle split parts=6,
    rectangle split empty part width={.2cm},
    rectangle split horizontal,
       label={[label position=below, align=center]20 inode\\ blocks}] (installer) {};

  \node[circlog, right=0cm of installer,
    inner sep=0pt,
    rectangle split,
    rectangle split every empty part={},
    rectangle split parts=20,
    rectangle split empty part width={.1cm},
    rectangle split horizontal,
       label={[label position=below, align=center]30 allocator\\ bitmap blocks}] (log) {};


  \node[circlog,minimum width=3.2cm, right=0cm of log,
    inner sep=0pt,
    rectangle split,
    rectangle split every empty part={},
    rectangle split parts=3,
    rectangle split empty part width={1.15cm},
    rectangle split horizontal,
       label={[label position=below, align=center]data blocks\\ (remainder of disk)}] (log2) {};

  \draw [line width=.05cm,black,transform canvas={yshift=.25cm}] (installer.south west)+(0,-.5cm) -- (installer.north west);
  \draw [line width=.05cm,black,transform canvas={yshift=.25cm}] (log.south west)+(0,-.5cm) -- (log.north west);
  \draw [line width=.05cm,black,transform canvas={yshift=.25cm}] (log2.south west)+(0,-.5cm) -- (log2.north west);

%  \tikzstyle{circlog}=[thick,rectangle, draw,minimum height=1.5cm, align=center];
%
%  \node[circlog,minimum width=1.2cm,
%       label={[label position=below, align=center]Super\\block}] (logger) {};
%  \node[circlog,minimum width=1.7cm, right=0cm of logger,
%       label={[label position=below, align=center]20 inode\\ blocks}] (installer) {};
%
%  \node[circlog,minimum width=2.0cm, right=0cm of installer,
%       label={[label position=below, align=center]30 allocator\\ bitmap blocks}] (log) {};
%
%  \node[circlog,minimum width=3.4cm, right=0cm of log,
%       label={[label position=below, align=center]data blocks\\ (remainder of disk)}] (log) {};

%  \node[] at (-0.45cm,-1cm) {$\uparrow$ 0};
%  \node[] at (4.2cm,-1cm) {$\uparrow$ 513};

%  \draw [decorate,decoration={brace,mirror,amplitude=10pt},xshift=-4pt,yshift=0pt]
%    (-0.45cm,-1.2cm) -- (4cm,-1.2cm) node [black,midway,yshift=-0.6cm]
%    {\scc{circular}};

\end{tikzpicture}

  \caption{The layout of the file system on top of the transaction system's
    disk. The number of inode blocks and data bitmap blocks is a compile-time
    constant, but easy to change without affecting the proofs.}
  \label{fig:layout}
\end{figure}

The file system is responsible for implementing files and directories
onto an array of disk blocks that is exported by the transaction
system.  The disk layout used by the file system is shown in
\autoref{fig:layout}, with regions for inode blocks, bitmap blocks,
and data blocks for files and directories. This figure is in terms of
the disk exported by the transaction system; the transaction system
itself has a 513-block write-ahead log to support multi-block atomic
writes to the disk.

The high-level organization of the file system separates three concerns, each
building upon the previous: (1) implementing large (about 512GB), fixed-size
files with zeros in place of unallocated data; (2) implementing files with
byte-level granularity by tracking a size field; and (3) implementing
directories by encoding them as files with a special type together with
operations to manipulate those files. \autoref{sec:design} explains the
internals of the file-system design in more detail, alongside the structure of
the Dafny proof.

%% Each
%% operation takes place in a single transaction at run time, but this transaction
%% is built up by calling methods through several abstraction layers before
%% eventually producing a sequence of transactional reads and writes.

% The proof requires that each 4KB block be accessed with a consistent size. We
% represent that in the specification with a fixed ``schema'' specifying the
% size of every block address, which is fixed by the caller while calling the
% initialization function. The schema is never passed to the code and only used
% to enforce the consistent-size restriction in the precondition of \cc{Read}
% and \cc{Write}. These restrictions are reflected by representing the
% transaction system not as an array of bytes but as a mapping from
% ``addresses'' specifying a block number and offset to ``objects'' which can be
% either a boolean or a list of bytes. The schema is a static mapping from block
% number to size such that all the objects are of that size.

\subsection{Transaction system}


\begin{figure}
\begin{verbatim}
type Addr struct {
  Blkno  uint64
  Offset uint64
}

// starting and stopping a transaction
func Begin() *Txn
func Abort(tx *Txn)
func Commit(tx *Txn)

// operations within a transaction
func Read(tx *Txn, a Addr, sz uint64) []byte
func ReadBit(tx *Txn, a Addr) bool
func Write(tx *Txn, a Addr, d []byte)
func WriteBit(tx *Txn, a Addr, d bool)

// allocator API
func NewAllocator(max uint64) *Allocator
func Alloc(a *Allocator) uint64
func Free(a *Allocator, n uint64)
\end{verbatim}
  \caption{The API for the transaction system and allocator, both of which are
    available within the Dafny file-system implementation. Reads and writes
    between \cc{Begin} and \cc{Commit} appears to execute atomically on disk and
    for other threads, while \cc{Abort} guarantees the transaction has no
    effect. The allocator's \cc{Alloc} and \cc{Free} operations are safe to call
    concurrently.}
\label{fig:txn-api}
\end{figure}

The transaction system handles concurrency and crash safety, and its
API is listed in full in \autoref{fig:txn-api}.  The file system
creates an empty transaction by calling \cc{Begin()}. The entire
transaction appears to execute atomically when the caller finishes
with \cc{Commit}, or the transaction is discarded with no effect on
\cc{Abort}. Reads and writes operate on addresses which specify a
position by giving a block number and an offset in bits (always less
than $4096 \cdot 8$, the number of bits in a block). The \cc{Read}
method requires an explicit size argument while the size of a
\cc{Write} is implicit in the size of the \cc{data} slice. We separate
out the bit-sized operations to \cc{ReadBit} and \cc{WriteBit} (rather
than using a single-element byte slice) to simplify the specification.

\autoref{fig:txn-api} also shows the allocator API alongside the
transaction API because its implementation is also part of the
concurrent code that the Dafny file system has access to.

%% The state of the transaction system (the transactional disk transactions
%% manipulate) looks much like a flat array of bytes.
%% However, the caller cannot
%% read and write arbitrary regions of this array due to restrictions in the
%% gojournal code and proof. all reads and writes must be within a single 4kb block
%% on disk, and of a power-of-two number of bytes or a single bit.

% In practice the file system uses three kinds of objects: full blocks are used
% for data (both for directories and data files), bit objects comprise the inode
% and block allocators, and 128-byte objects are used to represent inodes. The
% file-system statically allocates regions for the inodes, allocator bitmaps,
% and data blocks, so that object sizes never change.

The transaction system uses two-phase locking and GoJournal, which was
verified in prior work~\cite{chajed:gojournal}, to implement
transactions.  While a transaction is running, it acquires locks for
any addresses it reads or writes, and on abort or commit, it releases
all locks held. Transactions that don't conflict can prepare in
parallel, and GoJournal will batch concurrently committed
transactions for efficiency.

Acquiring multiple locks during a transaction creates the possibility
for deadlocks, if two threads acquire a pair of locks in the opposite
order. The two-phase locking implementation does not implement a
specific lock acquisition order, leaving it to the file system to
avoid deadlock --- for example, the implementation of \cc{RENAME}
makes sure to lock the smaller inode number first if the rename is
between different directories.

%% The journal provides a way to write multiple addresses atomically,
%% but it is illegal to access the same address concurrently from two different
%% transactions.

\section{Specifying DaisyNFS}%
\label{sec:daisy:spec}

Specifying DaisyNFS takes two steps: first formalizing the NFS protocol, and
then stating how the implementation relates to this formalization.
The specification for NFS is a state machine describing an ideal NFS server in
the form of an abstract state and a transition for each operation. The
implementation of DaisyNFS is a binary \cc{daisy-nfsd} that implements the NFS
protocol, running on top of a
disk. Then the DaisyNFS correctness
theorem is a \emph{refinement} property, which intuitively says that
for any interaction with the
implementation, the ideal, atomic NFS state machine could produce the responses;
this section shortly gives a more formal definition.
As a result a client interacting with the server can pretend
that it is the NFS state machine and ignore the complexities of its
implementation.

\subsection{Formalizing NFS}

RFC 1813 specifies the NFS protocol, which we make mathematically precise with a
state-machine representation defined in Dafny.
The formalization requires first
defining what state operations modify, and then a transition for each
NFS operation that specifies how it changes the state and what return
values are allowed. While most of the specification is deterministic,
some operations have to be specified with non-determinism; for
example, we allow returning an out-of-space error in many operations,
and the specification allows any timestamp to be picked for the
current time. The RFC is precise about arguments and allowed return
values, and the text is good about explaining the intended behavior,
but it does not describe the state an NFS server maintains.  We define
the NFS server state as shown in \cref{fig:dafny-state}.

\begin{figure}[h!]
\begin{small}
\begin{verbatim}
type Ino = uint64
type Path = seq<byte>
datatype Attrs = Attrs(mode: uint32, ...)
datatype File =
  | ByteFile(data: seq<byte>, attrs: Attrs)
  | Dir(dir: map<Path, Ino>, attrs: Attrs)

// the abstract state of the file system
type FilesysData = map<Ino, File>
\end{verbatim}
\end{small}
\caption{Dafny definition of the NFS server state (simplified).}
\label{fig:dafny-state}
\end{figure}

This definition says that an NFS server conceptually maintains a mapping from
inode numbers to files, where a file can either be a regular file with
bytes, or a directory. Both types of files have a number of attributes, storing
metadata like the file's mode (permission bits) and modification time. A
directory is a partial map from file names
(which are just bytes) to inode numbers. Note that DaisyNFS doesn't
represent the file system as a tree but as a collection of
links, which is sufficient to model all NFS operations, because
NFS clients resolve path names.

% \mfk{the rest of this
%   paragraph is lacking a clear narrative.}
% In any case a
% tree wouldn't be a sufficient state for NFS since modifying one file
% affects any other hard links to the same file (note though that DaisyNFS
% does not currently support hard links).

The NFS state machine models each operation as a non-deterministic transition
that answers when it is allowed for an operation to change the state from
\cc{fs} to \cc{fs'} and return \cc{r}. The return value is always wrapped in a
\cc{Result} type, which can be either \cc{Ok(v)} for a normal return or an error
code for one of the errors defined in the standard. The file system systematically guarantees
that the state is unchanged when an operation returns an error (though this is
stronger than what the RFC mandates); the transaction system makes this easy to
achieve by aborting the whole transaction. For example,
\cref{fig:getsz} shows the
specification for a (hypothetical) \cc{GETSZ} operation that returns the size of
the inode \cc{ino}.

\begin{figure}
\small
\begin{verbatim}
predicate GETSZ_spec(ino: Ino, fs: FilesysData,
  fs': FilesysData, r: Result<uint64>)
{
  fs' == fs &&
  (r.ErrBadHandle? ==> ino !in fs) &&
  (r.ErrIsDir? ==> ino in fs && fs[ino].Dir?) &&
  (r.Ok? ==> ino in fs && fs[ino].ByteFile? &&
             r.v == |fs[ino].data|)
}
\end{verbatim}
\caption{Specification of a hypothetical \cc{GETSZ} operation, a simplification
  of the real \cc{GETATTR} operation.}
\label{fig:getsz}
\end{figure}

There are four clauses in the specification. The first just says that this
operation is read-only. The second is one possible error: if the server returns
\cc{ErrBadHandle}, then \cc{ino} is not allocated. The third is a different
error, which says this operation might return \cc{ErrIsDir} for directories.
Finally the fourth clause says that if the operation is successful, it returns the
length of the data in \cc{fs[ino]}. Dafny checks several consistency properties
of this specification itself; for example, a use of \cc{fs[ino]} only compiles
if the specification earlier implies \cc{ino in fs}.

We developed a state-machine model of the regular file and directory operations
in NFS in this style, including specifying what certain errors
signify. \Cref{fig:nfs} lists the entire NFS API and what parts are verified in
DaisyNFS. \tej{this part is a bit out-of-place in this subsection, consider rearranging}

\renewcommand{\check}{\textcolor{ForestGreen}{\checkmark}}
\newcommand{\nope}{\textcolor{Maroon}{\ding{55}}}

\begin{figure}
\small \centering
\begin{tabular}{@{~}ll@{}c@{~}}
  \toprule
  \bf Category & \bf Operations & \bf Verified \\
  \midrule
  \textit{File and directory ops}
  & \cc{GETATTR}, \cc{SETATTR}, \cc{READ}, \cc{WRITE} & \check \\
  & \cc{CREATE}, \cc{REMOVE}, \cc{MKDIR}, \cc{RENAME} & \check \\
  & \cc{LOOKUP}, \cc{READDIR} & \check \\

  \textit{Unsupported features}
  & \cc{READLINK}, \cc{SYMLINK}, \cc{LINK}, \cc{MKNOD} & \nope \\
  & \cc{READDIRPLUS}, \cc{ACCESS} & \nope \\

  \textit{Configuration}
  & \cc{FSINFO}, \cc{PATHCONF}, \cc{FSSTAT} & \nope \\

  \textit{Trivial operations}
  & \cc{NULL}, \cc{COMMIT} & \check \\

  \bottomrule
\end{tabular}
\caption{NFS API and which operations DaisyNFS supports and verifies.}
\label{fig:nfs}
\end{figure}

DaisyNFS implements \cc{FSINFO} and \cc{PATHCONF}, which give the client static
configuration information about the file system (for example, the maximum size
of a write or the maximum path length). These return constants and thus have no
specification. DaisyNFS also implements \cc{FSSTAT} to report total and free space,
but it does not have a meaningful specification.

DaisyNFS could support some of the remaining operations with some more effort.
Symbolic links (symlinks) are essentially a file that holds a path, which can be read with
\cc{READLINK}. \cc{MKNOD} similarly creates a new type of special file.
Specifying these operations would require mostly mechanical changes to the
specification to accommodate the new file types.
\cc{LINK} is more complicated because in addition to tracking
the link count of every file in the state, the specification for \cc{REMOVE}
needs to say that the link count is decremented and that the file is deleted if
its link count drops to zero.

\subsection{Specifying correctness for DaisyNFS}

\begin{figure}[ht]
\small
\centering
\begin{tabular}{@{~}llr@{~}}
\toprule
\bf Layer & \bf Operations & \\
\midrule
  NFS
      & \cc{CREATE(d_ino, name)}, \cc{READDIR(d_ino)}, \dots & \cref{sec:daisy:spec} \\
  Txn
      & \cc{Read(tx, a, sz)}, \cc{Commit(tx)}, \cc{Alloc(a)},
        \dots & \cref{sec:txn:api} \\
  Disk
      & \cc{Read(a)}, \cc{Write(a, b)} & \\
\bottomrule
\end{tabular}
\caption{API layers of DaisyNFS.}
\label{fig:layers}
\end{figure}

The proof is about the server loop at three layers of abstraction, as outlined
in \cref{fig:layers}. The Disk layer at the bottom is where the whole server
runs, while the NFS layer serves as the specification. In addition, there
is an intermediate layer Txn for the transaction system which is the abstraction
on which the Dafny code is written and verified.

Now we have enough to state the final DaisyNFS correctness theorem:
\begin{theorem}[DaisyNFS correctness]
  $\mathrm{link}(\sdfy, \txncode) \refines \snfs$.%
  \label{thm:correctness}
\end{theorem}

%
In this correctness theorem, initialization requires running a Dafny method on
an empty disk. Subsequently the system boots by first recovering the transaction
system, then restoring the file system. \Cref{thm:correctness} will follow
from the correctness of the transaction system combined with the results from
Dafny.

\section{Verifying DaisyNFS}
\label{sec:proof}

How do we prove that the implementation of the NFS server running on a disk
implements the NFS transition system formalized in Dafny? The overall structure
of the proof resembles the division in the implementation: we prove that the
transaction system really makes the calling code's transactions atomic, and
separately prove that the file system's transactions are implemented correctly.
By isolating the concurrency and crash-safety reasoning to the transaction system, we can use
Dafny, with highly automated proofs, to reason about the file-system
implementation (since it behaves sequentially) while using Perennial to reason
about that concurrency and crash safety in the transaction system. Since the
proofs happen in separate formal systems, we also prove a linking theorem about
the entire file system whose proofs connects the transaction system's
correctness to the correctness of the file-system code.

Using a careful choice of interface between the transaction system and Dafny, we
minimize the work that goes into linking, which has to be proven on paper.
In particular the transaction system's proof
shows that any calling program appears to have atomic transaction, provided the
transactions follow some rules. For example, transactions can only modify state
protected by the two-phase locking system; if the caller accesses a global
variable, the transaction system can't see this access and acquire a lock so the
transaction would not be atomic. At a high level the on-paper reasoning
addresses why the Dafny transactions follow the rules required by the proof
mechanized in Coq.
% Dafny always assume methods run
% sequentially, so it is necessary that the implementation meet any restrictions
% of the transaction system for the Dafny proof to be meaningful.

% to understand how it guarantees transactions behave
% atomically and what the requirements on the caller are.

% While there is no fully
% machine-checked theorem covering the end-to-end system, all of the proofs within
% the transaction system and within the Dafny code are machine-checked and this
% proof merely connects those theorems. The manual step carried out here is
% an audit over the Dafny code to establish that it fits the transaction system's
% rules.

% It would be an interesting direction for future work to automate this step.

\begin{figure}[ht]
  \centering
\begin{tabular}{lp{4.4cm}r}
\toprule
Layer & Operations & \\
\midrule
  NFS
      & \cc{CREATE(d_ino, name)}, \cc{READDIR(d_ino)}, \dots & \autoref{fig:nfs} \\
  Txn
      & \cc{Read(tx, a, sz)}, \cc{Commit(tx)}, \cc{Alloc(a)},
        \dots & \autoref{fig:txn-api} \\
  Disk
      & \cc{Read(a)}, \cc{Write(a, b)} & \\
\bottomrule
\end{tabular}
\caption{API layers of \sys.}
\label{fig:layers}
\end{figure}

The proof relates the \sys server loop at three different layers of abstraction,
corresponding to the three API layers in \autoref{fig:layers}. Refinement from a code program to
a specification program says that the behaviors of the code are a subset of the
behaviors of the specification; we will shortly give a more precise definition.
At the top layer, the specification end, we'd like to think of \sys as atomically responding to
NFS operations according to the state machine developed in Dafny.
The NFS operations are implemented using methods from transaction API, the middle layer.
Finally, the running code links the Dafny code with the transaction system's
implementation and runs on top of a raw disk-block API.

There are three programs involved in defining and proving the overall
correctness of \sys, corresponding to the server loop at each abstraction layer.
At the top, the specification is a loop $\snfs : \gooselayer{NFS}$ which
atomically processes each NFS operation according to the NFS state machine. The
next level is $\sdfy : \gooselayer{Txn}$, where each handler is the atomic body of the
corresponding Dafny method, operating on top of the transaction system. Finally,
the executable code is written
$\mathrm{link}(\sdfy, \txncode) : \gooselayer{Disk}$, indicating ``linking'' the
Txn-layer server $\sdfy$ with the transaction system by taking each call to a
Txn API and plugging in its implementation on top of a disk.

\begin{figure}[ht]
  \center
  \begin{tikzpicture}[>=latex, node distance=1.5cm]

 \tikzset{
    genericnode/.style={rectangle,draw,minimum width=2cm, minimum height=.85cm, align=center,},
    layer/.style={
      genericnode,
      alias=genericnode,% <- alias added
      label={[anchor=south west,shift={(genericnode.north west)},inner sep=2pt]{\tiny #1}}% position the label using the alias
    }}

%\tikzstyle{layer}=[rectangle, draw, minimum width=2cm, minimum height=.85cm, align=center];
\tikzstyle{genlayer}=[dashed, layer={}];
\tikzstyle{edge}=[->,thick];

\draw node (dispatch) [layer={$\gooselayer{NFS}$}] {$\snfs$};
\draw node (goout) [layer={$\gooselayer{Txn}$},below of=dispatch] {$\sdfy$};
\draw node (txn) [layer={$\gooselayer{Disk}$},below of=goout] {$\mathrm{link}(\sdfy, \txncode)$};

\draw node (dafny) [left=1.25cm of goout, layer=Dafny] {File-system \\ Operations};

%\draw [thick] (goout.south) -- (txn.north);
\path (txn) edge[draw=none]
                node (incl1) [sloped, auto=false,
                 allow upside down] {$\refines$} (goout);
\path (goout) edge[draw=none]
                node (incl2) [sloped, auto=false,
                 allow upside down] {$\refines$} (dispatch);


\path (goout) edge[draw=none]
                node (thm2) [xshift=1.3cm, auto=true] {Thm \ref{thm:dafny} (Dafny)} (dispatch);
\path (txn) edge[draw=none]
                node (thm1) [xshift=1.5cm, auto=true] {Thm \ref{thm:txn} (Perennial)} (goout);

\draw [edge] (dafny.east) -- node[above] {\texttt{dafny}} (goout.west);


\end{tikzpicture}

  \caption{The overall structure of the proof combines theorems proven in Perennial
    and Dafny, each tool used for the reasoning it is best suited to.}
  \label{fig:proof-overview}
\end{figure}

\autoref{fig:proof-overview} gives the overall structure of the proof, which
relates these three programs via refinement. The high-level strategy is to break the proof down
into three steps: 1) state and prove \autoref{thm:txn} in Perennial, the correctness
of the transaction system, 2) state and prove \autoref{thm:dafny} in Dafny, the
correctness of the Dafny methods (as transactions), and 3) prove a specification
for the whole compiled system, \autoref{thm:correctness}, by applying the other
two theorems.

\subsection{Correctness theorem for sequential file-system transactions}
\label{sec:proof:dafny}

The top-level file-system is a program denoted $\sdfy : \gooselayer{Txn}$
written against the transaction-system API, where this program models the
top-level dispatch loop that repeatedly accepts an NFS request and responds to
it in a separate thread. What we prove using Dafny is that this implementation
refines a more abstract dispatch loop $\snfs$ where the transitions atomically
follow the NFS transition system:

\begin{theorem}
  $\sdfy \refines \snfs$. To establish the init relation in this definition,
  the caller runs a dedicated method in Dafny that assumes an all-zero
  transaction system and establishes the invariant all other operations rely on
  (including recovery).
  \label{thm:dafny}
\end{theorem}

We prove this theorem in Dafny using a standard \emph{forward simulation}
technique.  Instead of directly reasoning about the $\sdfy$ concurrent program
(which is not possible in Dafny) we instead consider each of its handler methods
separately. Because transactions run atomically, it is sound to use Dafny's
sequential specifications in terms of pre- and post-conditions to reason about
an individual method. If every method satisfies a refinement obligation using a
common abstraction relation, we know that the entire program satisfies
refinement as defined above. This proof method of forward simulation is
so common that many systems verified in Dafny do not mention it, or treat
simulation and refinement synonymously.

Crash safety does add one interesting case to forward simulation. In general for
a sequential storage sytem, recovery should satisfy a specification that assumes
a \emph{crash invariant} that the whole system maintains at each intermediate
point and establishes the abstraction relation, with a new abstract state
consistent with the specification's crash behavior~\cite{chajed:argosy}. Due to
our transaction system we can make one simplification: on crash, the system will
satisfy the abstraction relation and not just a weaker crash invariant, since
operations are atomic; furthermore, the file system automatically maintains the
abstraction relation for crashes during recovery since recovery itself uses a
single transaction. Recovery must still do some work since the abstraction
relation held prior to the crash using in-memory state that has been lost. We
give the precise specification proven in Dafny in \autoref{appendix:proof}, and
also argue how it fits into the forward simulation proof.

\subsection{Linking the transaction system and Dafny correctness theorems}
\label{sec:proof:linking}

Using the mechanized theorems, we can show that the overall system is correct,
namely that the running code implements the abstract NFS dispatch loop:

\begin{theorem}[\sys correctness]
  $\mathrm{link}(\sdfy, \txncode) \refines \snfs$. Initialization requires
  starting from an empty disk, then running the initialization implemented in
  Dafny. After that, the system boots by first recovering the transaction
  system's state, then running file-system recovery.
  \label{thm:correctness}
\end{theorem}

This theorem's proof is a straightforward consequence of the mechanized theorems
described as long as $\sdfy$ is safe; that is, the Dafny code follows the rules of
the transaction system. As long as this
holds we simply apply the theorems in order to show
$\mathrm{link}(\sdfy, \txncode) \refines \sdfy \refines \snfs$, since
the definition of refinement is transitive.

The proof requires Dafny operations to be encapsulated in transactions
and follow the transaction system's safety restrictions.
The code in Dafny does not generally manage starting and
committing a transaction; this is handled by a single wrapper function written
in Go. The wrapper creates a transaction, calls a Dafny method on it,
and then aborts if the method returns an error code and commits otherwise.
It is easy to confirm that this function follows the calling sequence required
by the transaction system.

Many transaction safety restrictions are preconditions on the
transaction-system APIs, which are enforced as Dafny preconditions.  The most
restrictive part of safety is that transactions are not allowed to read or write
shared memory, other than through the Txn layer. The Dafny code does allocate
heap objects and read them, but these are only used locally, and thus are
unaffected by concurrency; we work through this argument more formally in
\autoref{appendix:proof}. To confirm that there is no shared mutable state, we
checked that the Dafny class implementing the file system has no mutable
variables, other than the transaction system and its allocators.  Dafny does not
support mutable global variables outside of any class, so checking the class is sufficient.

Notice that this linking proof makes minimal assumptions about the code in
Dafny, other than the fact that it is verified. In particular, the same argument
would equally apply to a different NFS implementation, or even a system with an
entirely different specification, as long as the equivalent of
\autoref{thm:dafny} was still proven in Dafny, and the Dafny code did not use
shared mutable state.

% Formally, even if $\sdfy$ is not safe, it is sufficient if there
% exists any other program $\widetilde{\server}_{\mathrm{dfy}}$ with equivalent behavior which is
% safe; we can apply \autoref{thm:txn} to this alternate program and get the same
% refinement result. Such a program would be a systematic transformation of the
% Dafny code that replaced any allocation with a local variable forwarded to the
% remainder of the code, and inlined any global constants. This transformation
% would could fail if Dafny wrote outside of its local allocations, but there are
% no mutable variables to write to. Finally, we manually check that any buffers
% returned from Dafny are used in a read-only manner.

The Txn layer's allocator methods are important for this approach to work.
The allocator cannot be implemented in Dafny because it is shared mutable state.
Instead, we expose the allocator API as part of a transaction, albeit with a loose
specification that says \cc{Alloc} may return
any number and \cc{Free} may be called on any unused block.
As we explain in \autoref{sec:txn-proof}, under this specification, \cc{Alloc} and \cc{Free} both behave
atomically in a transaction. The
true allocation state is stored on disk, in the bitmap blocks for
example, and the Dafny code must validate that the address returned by \cc{Alloc} is actually
free.

% It is possible that one transaction calls
% \cc{Free} and another allocates the same number before the disk is updated,
% because freeing does not happen at commit time, but the allocator's policy is
% designed to delay allocating recently freed blocks so this is extremely
% unlikely.

% 06-txn has been moved to txn chapter
\section{Verifying the Dafny implementation}%
\label{sec:daisy:design}

We follow the standard approach for verifying software in Dafny: each
file-system operation is implemented as a method on a class and its
specification is given using pre- and post-conditions. In \cref{sec:proof},
we explained how the Dafny proof shows the code is a correct implementation of NFS in terms of sequential refinement. This
section provides details about the file-system design and proof.

% The proof is given by annotating the code with proof steps, which include
% updates to ghost state, assertions to assist the automated verification, and
% calls to lemmas.

DaisyNFS is implemented and verified in several layers of abstraction, depicted in
\cref{fig:dafny-layers}. Each layer is implemented as a class that wraps the
lower layer as a field. The transaction system is an assumed interface in Dafny,
while the complete server implements the NFS wire protocol and calls into the
top-level Dafny class for each operation.

% The approach we follow is inspired by xv6. We borrow some tricks from FSCQ and
% Yggdrasil to accomodate verification (especially using more layers than the xv6
% C implementation). We also needed to adapt the proof a bit so that rather than
% using separation logic we encode disjointness in an SMT-friendly way. The
% approach we take is standard and the same as used in Yggdrasil (see how layer 3
% works).

\begin{figure}
\small \centering
\begin{tabular}{ll}
  \toprule
  \textbf{Layer} & \textbf{Functionality} \\
  \midrule
  dir & Directories and top-level NFS API. \\
  typed & Inode allocation. \\
  byte & Implement byte-level operations using blocks. \\
  block & Gather blocks for each file into a single sequence. \\
  indirect & Triple-indirect blocks organized in a tree. \\
  inode & In-memory, high-level inodes; block allocation. \\
  txn & Assumed interface to external transaction system. \\
  \bottomrule
\end{tabular}
\caption{Layers in the Dafny implementation and proof of the file-system
operations.}
\label{fig:dafny-layers}
\end{figure}

Between the layers of the file system
there are three difficult pieces of
functionality: organizing data blocks into metadata and data (the
indirect and block layers), translating byte-level operations into
block operations (the byte and typed layers), and implementing
directories as special files that the file system itself reads and
writes (the dir layer). The modularity was essential to complete the proof in
manageable chunks (to avoid overwhelming the developer and prover), and it would
have been natural even without verification.

\subsection{Implementing the file system using transactions}

The design of DaisyNFS is broadly similar to the file system in xv6~\cite{xv6},
as well as Yggdrasil~\cite{sigurbjarnarson:yggdrasil}, a verified sequential
file system. We also adopt the recursive strategy for implementing and
verifying indirect blocks from DFSCQ~\cite{akonradi-meng}; recursion simplifies
the implementation of triply-indirect blocks, which are needed to reach a
reasonable maximum file size of 512GB.\@ Unlike most file systems, DaisyNFS is designed
to fit every operation into a transaction in order to support our goal of
sequential reasoning. This is a non-standard design and we encountered some
unique challenges in doing so. In this section we highlight difficulties in
fitting two features into transactions: rename and freeing space from deleted
files.

\subsubsection{Rename}
\label{sec:dafny:rename}

The NFS RENAME operation is similar to the \cc{rename} system call: it moves a
source file or directory to a destination location. What makes it tricky is that
it involves more than one inode and hence introduces the possibility for
deadlock.
% , which we would like to avoid even if the theorems do not forbid it.
We
use the standard strategy of enforcing a global ordering where inodes are always
locked in numerical order (smaller inode numbers first); this avoids a deadlock
where a cycle of threads is waiting on each other.

At this point it is worth discussing the performance considerations that lead to
handling lock ordering in the file
system, rather than generically in GoTxn. The transaction system could
avoid deadlocks by either enforcing a global order over addresses or by
timing-out operations. Enforcing a global order is inefficient for the file
system; data blocks will never cause deadlock because the file system only
accesses a block after locking the (unique) inode that owns it. Timing-out
operations would lead to slow and spurious transaction failures that could more
rapidly be avoided in the higher-level code, hence we do not attempt to detect
deadlock dynamically.

In a rename operation, the source and destination are each specified by a
combination of the parent directory inode and name within that directory. Rename
has an additional functionality of overwriting the destination if the source and
destination are files, or if both are directories and the destination is empty.
It is this latter check that makes deadlock avoidance difficult: it is necessary
to lock the source and destination directories first to lookup the source and
destination names, but those might be files that are earlier in the inode lock
order. We address this in the code by returning an error from the Dafny
transaction before the lock order would be violated. The error comes with the
set of inodes that should have been acquired.  The rename is then re-run with
this set of inodes as a lock hint; these are first acquired in the correct
order, then compared against the current source and destination in case they
have been renamed concurrently.

%% \subsubsection{Large files}
%% \label{sec:dafny:indirect}
%%
%% The indirect and block layers together implement an abstraction of
%% files as a sequence of blocks, using some blocks in the transaction
%% system as indirect blocks that contain pointers to other blocks.  Most
%% of the code implementing indirect blocks is implemented recursively
%% using an approach borrowed from DFSCQ~\cite{akonradi-meng}.
%% In practice we configure the system with several direct and (single)
%% indirect blocks in an inode to make small files more efficient, plus a
%% triple-indirect block that allows files to grow to 512GB.

%% Using only a single level of
%% indirection would be convenient for verification, but it would limit the size of
%% files to 28MB. Instead, we use multiple levels of indirection. Indirect blocks
%% have a natural recursive structure, where a $k$-indirect block holds pointers to
%% $(k-1)$-indirect blocks, down to $k=0$ for the data blocks.

%% Files at the block layer are always of the maximum size, to avoid
%% reasoning about sizes until the higher layers. To efficiently
%% represent such large files, the zero block number is treated specially
%% as encoding a zero block, including for indirect blocks; an idea we
%% borrowed from DFSCQ~\cite{akonradi-meng}. Because this works
%% recursively, the single zero for the root of the triple-indirect block
%% address efficiently stores the many GBs of zeros in a typical small
%% file.

%% The representation of file data using zero addresses and
%% recursive indirect block implementation were directly inspired by the
%% DFSCQ indirect-block~\cite{akonradi-meng}.

%% Indirect blocks pose a challenge for verification due to the classic problem of
%% \emph{aliasing}. The proof must show that modifying a data block or indirect
%% block has no effect on other files. In the DFSCQ proof, the invariant
%% captures the non-aliasing between files using separation logic, which makes
%% disjointness easy to express. In Dafny we have no such logical
%% technique, so we instead use a standard SMT-friendly trick for the invariant: in
%% addition to the physical mapping that tracks how to dereference a block address,
%% the indirect layer proof tracks a ghost \emph{reverse} mapping that tracks where
%% each in-use block number is stored. The invariant states that the forward and reverse
%% mappings are inverses of each other, which implies that modifying an address
%% only affects its owner and nothing else.

%% To encode the reverse mapping, we need a data type to represent the location of
%% a block within an inode. With indirect blocks, the metadata blocks themselves
%% also need to be considered locations, since the invariant must also rule out
%% metadata aliasing with data or other metadata. We encode locations with a
%% position datatype that encodes an inode, an indirection level, and an offset for
%% the blocks at that indirection level. If we imagine that an inode's block
%% pointers are organized in a tree, the roots are stored directly in the inode
%% while the leaves are direct blocks. An indirection level which is higher than
%% the leaf level describes a metadata block.

%% The indirect block proof is split into the indirect and blocks layers. In the
%% indirect layer, the abstract state maps positions (including the inode number)
%% to data, and off to the side tracks the size and attributes of each inode. The
%% block layer changes this representation to a flat sequence of blocks by mapping
%% each leaf position to its linear index within the inode. Separating these two
%% made it easier to work on the indirect layer while giving the upper layers the
%% much more natural abstraction of a file as a sequence of blocks.

\subsubsection{Freeing space}
\label{sec:dafny:freeing}

Freeing space becomes surprisingly tricky with large files. The problem is that
a large-enough file may reference too many blocks to be
freed in a single transaction.
DaisyNFS handles freeing by removing a file from its directory and marking it free in
one transaction, and in separate transactions reclaiming the space it took by deallocating
its blocks.

Removal is implemented as a combination of two transactions,
one which performs the logical operation but leaks space, and an operation
\cc{ZeroFreeSpace(ino)} which frees and zeros the unused space in an inode that
we prove has no effect on the file-system state. Because this operation is a
logical no-op, it is safe to call it at any time. In practice the implementation
is careful to call it after any operation that leaves unused blocks, in
particular \cc{SETATTR}, which can shrink a file by reducing its size, and
\cc{REMOVE}, which deletes a file. Furthermore since \cc{ZeroFreeSpace} doesn't
affect the user-visible data, it may return early to avoid overflowing a
transaction, which GoJournal limits to 511 blocks.

There is one case where freeing blocks is important for correctness and not just to reclaim space. Growing a file is supposed to logically fill the new space with
zeros. If the file had old data in that space, it would not be zero but some
previously written and deleted data, which both violates the specification and
is a potential security risk. The way we handle this with background freeing is
with validation: when the \cc{SETATTR} operation grows a file checks, it checks if the
free space is already zero first, and if not fails with a special error code. The
unverified code interprets this as a signal to immediately call
\cc{ZeroFreeSpace} and try the operation again. The same support also handles
holes created by writing past the end of a file, which are similarly supposed to
be zero.

The freeing implementation is an interesting example of using validation in
verification. The specification for much of the freeing code is loose, allowing
any data to be written to the free space. We only needed a strong specification
for the code that checks if the zeroing is done; the rest of the code needs to
be correct for this check to ever succeed, but we aren't required to prove it.

\subsection{Achieving good performance}

An important aspect of the Dafny proof was to write code in a way that produces
high-performance Go code.
% problem because Dafny's built-in immutable collections (sequences and maps) are
% extremely inefficient in Go due to an impedance mismatch between Dafny and Go
% semantics.
Compared to Dafny's C\# backend, the generated Go code for Dafny's built-in
immutable collections has much
additional pointer indirection and defensive copying. Using these data
structures for byte sequences would simplify proofs, but has unacceptably poor
performance in Go.

To avoid this performance problem we use an axiomatized interface to
Go byte slices (\cc{[]byte} in Go) whenever raw data is required, including file
data and paths, and then modify these slices in-place. It was possible to
axiomatize this API without any changes to Dafny; we use a standard Dafny
feature of \cc{:extern} classes to specify a Dafny class \cc{Bytes} in terms of
ghost state of type \cc{seq<byte>} but then implement it as in Go as a thin
wrapper around the native \cc{[]byte} type. This API is trusted, so we
test it. To catch off-by-one errors in the specification, we wrote
tests like \verb![]byte{1,2,3}[2]! and ran them in Go and
(equivalent) Dafny.
%(this test should return 3 because both languages are
% zero-indexed).

% We implement the file system fairly efficiently, taking advantage of Dafny's
% support for imperative code.
The on-disk data structures---inodes, indirect
blocks, and directories---are represented in memory in their serialized form and
modified by updating this representation directly, avoiding copies to move
between representations. These were first written with slower purely
functional code, which was then migrated to imperative code that
used the functional code as a specification.


Dafny's default integer type \cc{int} is unbounded and compiled to big-integer
operations. We used Dafny's
\cc{nativeType} support to instead define a type of 64-bit integers (that
is, natural numbers less than $2^{64}$) and compile this to Go's \cc{uint64}.
This requires overflow reasoning, but
automation makes this palatable in the proof and the performance gain is
significant.



% \subsection{Random notes on development process}
%
% \begin{itemize}
%   \item Used inefficient functional Dafny code at first, then slowly migrated to
%         in-memory data structures and improved performance.
%   \item Hard to debug and fix timeouts. Profiling verification performance is
%         hard.
%   \item Profiling Go code is great as usual. The generated code looks strange,
%         but I think after code generation it's pretty ordinary (the weird things
%         are mostly bad variable names, lot of unused assignments, and anonymous
%         functions that are immediately called).
%   \item Used some unit tests, but very few and only at the top level. Mainly
%         debugged Go compilation issues and cases where errors were
%         unintentionally being returned.
%   \item Trusted code isn't easy, had bugs in it before testing it thoroughly.
%         Also violated preconditions in top-level specs, triggering memory-safety
%         bugs.
% \end{itemize}

% 08-impl has been moved to dedicated impl chapter
\section{Conclusion}

This paper presented \sys, a verified crash-safe,
concurrent file system. \sys was built with verification in mind in
two parts: a transaction system and a file system whose operations run
as a transaction. This design allowed us to use the sharpest tool for
each part: Perennial for concurrency and crash-safety reasoning and
Dafny for sequential reasoning with much proof automation inside a
transaction.  The specification of the transaction system was
carefully designed to meet the assumptions that Dafny makes, but
requires some by-hand auditing to confirm.  Overall this approach
results in proof overhead of about $2\times$ for the file system part
(vs. $18\times$ for the transaction system), allowing us to verify and
build a functional and performing file system.

% That's all, folks!

\section{Proof of refinement for \sys}
\label{appendix:proof}

% make theorem re-statements get the same numbers
\setcounter{theorem}{0}

Previously in \cref{sec:daisy:proof} we gave an informal overview of \sys's
correctness proof. This section develops the proof in more detail. There are two
gaps we will fill in compared to the proof above: first, we will be more precise
about what the requirements of the transaction system proof are and why DaisyNFS
satisfies them, and second, we will specify what the theorem says about recovery.

To make this section self-contained, we will re-state theorems 1--3 (with slight
tweaks to describe initialization better). First, we re-state program refinement
for the transaction system:

\begin{theorem}
  The transaction system's implementation $\txncode$ is a program refinement,
  meaning for all $p : \gooselayer{Txn}$, if $\mathrm{safe}(p)$, then
  $\mathrm{link}(p, \txncode) \refines p$. Initialization relates an all-zero
  disk to an all-zero transactional disk.
  \label{thm:txn-appendix}
\end{theorem}

So far, we have been vague about what exactly $p : \gooselayer{L}$ means. In
Coq, $\gooselayer{L}$ is a datatype in a language we called GooseLang, which is
intended to represent a Go program with access to operations from layer $L$. Any
GooseLang program has access to standard Go constructs like heap allocation,
concurrency, and operations on basic types like integers and booleans. All of
the concrete GooseLang code we reason about for the transaction system is in
$\gooselayer{Disk}$. $\gooselayer{Txn}$ is used to give a specification for code
using the transaction system, while $\gooselayer{NFS}$ is only used to model the
NFS server in the top-level specification.

The overall correctness statement for DaisyNFS refers to
$\stxn : \gooselayer{Txn}$, which models the Dafny implementation at the level
of transactions. We express the result from Dafny as the following refinement
theorem:

\begin{theorem}
  $\sdfy \refines \snfs$. Initialization requires an invariant stated in Dafny,
  which is established by running a separate \cc{Init} procedure verified in
  Dafny starting from an all-zero transactional disk.
  \label{thm:dafny-appendix}
\end{theorem}

The overall correctness theorem says
$\linkedcode \refines \snfs$ (ignoring initialization). Both the code and spec
are expressed as a single program, a top-level loop whose
observable behavior is to receive some input from the outside world, which is an
NFS operation and its arguments, then to dispatch a handler thread to process
the request and respond over the network with the handler's result. The
difference between the two is that the specification $\snfs$ processes requests
atomically and according to a high-level NFS transition system, whereas in the
code $\linkedcode$ handlers represent the executable code running on top of a
disk, derived by compiled from Dafny and linking with the transaction-system
implementation.

Notice that even the code dispatch loop in this specification abstracts over the
details of interacting with the outside world; in reality, the executable binary connects
to clients over TCP and parses a stream of bytes to produce NFS operations. The
NFS wire protocol is outside the scope of our verification. We want to
interpret the specification as saying something about the binary that we run at
the end of the day; in doing so, we assume that the code correctly implements
the protocol, translating the byte stream into the right method calls for each
handler, and then translating the responses back.

Combining the above results, the top-level specification is:

\begin{theorem}
  $\linkedcode \refines \snfs$. Initialization requires
  starting from an empty disk, then running the initialization implemented in
  Dafny. After that, the system boots by first recovering the transaction
  system's state, then running file-system recovery.
  \label{thm:correctness-appendix}
\end{theorem}

Recovery is part of this specification because the semantics of running any
program includes the possibility to crash and restart, with a crash considered
visible behavior. With crashing as just another possible behavior, the normal
definition of refinement enforces that $\linkedcode$ has crash atomicity, since
its crash and restart behaviors must correspond to one from $\snfs$ which does
not allow crashes during a transaction.

The overall strategy to prove this theorem is
$\linkedcode \refines \sdfy \refines \snfs$; refinement is clearly transitive
because it is a subset relation on program behaviors. This argument
centers on $\sdfy : \gooselayer{Txn}$. Note that we are assuming that
such a program actually exists, one that models the Dafny code using GooseLang
--- it isn't something we'll actually construct because we don't translate Dafny
to GooseLang. We believe this assumption is reasonable because GooseLang is
fairly complete and we use only basic features of Go to implement the file
system. At this level of abstraction, transactions are not represented with the
usual \cc{Begin}, \cc{Commit}, and \cc{Abort} calls but with a
specification-only Atomically operation.

In the following two sections we expand on how to apply \cref{thm:txn-appendix} to
$\sdfy$ and how the initialization and recovery code is incorporated to prove
\cref{thm:correctness-appendix}.

\subsection{Safety}

The formal definition of safety in Coq is split into two properties. The first
is a static restriction that describes the
relationship between the specification's $\mathrm{Atomically}(f)$ construct and
the code that implements a transaction. Static restrictions rule out constructs
that would break atomicity, notably use of the heap could produce effects
outside the transaction system. They also enforce a correct discipline over
transaction objects; for example, a single transaction should use only one
\cc{tx} object and when finished should call one of \cc{Commit} or \cc{Abort}
and then dispose of \cc{tx}. The translation of a transaction body is
non-trivial because the specification has operations that implicitly refer to
the ``current'' transaction, while the code must manipulate a transaction object
\cc{tx}. The Coq implementation enforces the static restrictions using a type
system, which allows us to prove program refinement
using a standard proof technique called \emph{logical relations}.

The model of the Dafny code $\sdfy$ uses Atomically blocks, but its
implementation uses the usual transaction-system API\@. We can be fairly confident
that the real Dafny code after linking with Go is equivalent to $\linkedcode$.
The main requirement is that the Dafny code manage transaction objects the same
way as the type system does. Each handler is written as a method that takes a
transaction object (and it never starts another transaction) and returns a
result or signals an error. Then a common utility function \cc{runTxn(f)} takes
care of beginning a transaction, passing it to \cc{f}, and determining whether
to commit or abort based on the return value. This utility function captures
exactly the translation of an Atomically block in the transaction system specification.

The more serious restriction is that transactions cannot access the heap at all.
The Dafny code does use the heap, but only ever in a local way to access objects
allocated earlier in the transaction. We can justify this in the proof by
distinguishing $\sdfy$, which is as faithful to the syntactic Dafny code as
possible, from a program $\sdfyAlt$ that is equivalent to $\sdfy$ but where the
use of the heap has been eliminated. $\sdfyAlt$ can be constructed from $\sdfy$
by using a state monad to turn every heap access into purely functional code.
We expect such a construction to succeed because
a simple audit shows that the Dafny code is in a class that has no mutable
variables other than the transaction system and ghost variables,
so it is impossible for a transaction to have a non-local effect.

Applying \cref{thm:txn-appendix} to this revised Dafny program, we obtain that
$\linked{\sdfyAlt} \refines \sdfyAlt$. Given that $\sdfyAlt \progeq \sdfy$, it is
reasonable that the linked versions are also equivalent, completing the first
part of the refinement proof:
\[
  \linked{\sdfy} \progeq \linked{\sdfyAlt} \refines \sdfyAlt \progeq \sdfy
\]

The other side of the refinement comes from reasoning about $\sdfy$ in Dafny.
The theorems proven in Dafny imply that $\sdfy \refines \snfs$. Reasoning about
refinement between these two programs is simple because the granularity of their
concurrency is the same: both have the same structure, but $\sdfy$ executes a
whole implementation for each transaction while $\snfs$ has only one transition.
The Dafny proofs show that $\sdfy$ maintains ghost state corresponding to the
NFS transition system and an invariant that is strong enough to conclude all the
handlers simulate the NFS specification.

There is one subtlety in connecting this proof to the transaction system proof,
even though it appears to be simple transitivity to connect them and show
$\linkedcode \refines \snfs$. The subtlety is that the Dafny proof is about a
model of the transaction system expressed in Dafny, whereas the transaction
system's program refinement proof uses a specification written in Coq. It is
important that the specifications line up, or at least that the behaviors in Coq
are a subset of those in Dafny. These being different formal systems we cannot
directly compare them, but we aimed to formalize the two as closely as possible.
The interfaces are also sufficiently narrow that we can inspect and compare both
by hand.

We have discussed the static restrictions for a safe Txn program, but there are
also dynamic restrictions, namely that the caller should not trigger undefined
behavior in the transaction system. We handle this systematically by enforcing
all of the required preconditions in the Dafny interface to the transaction
system, and these preconditions are respected by virtue of the Dafny proof
succeeding. Similar to trusting that the semantics of the transaction system are
the same, we also trust that Dafny enforces the preconditions correctly.

\subsection{Initialization and recovery}

So far, when we write a refinement of the form $\sdfy \refines \snfs$, we have
been vague about how initialization and recovery factor in. Now we'll
be more precise about how these are handled, first in the program refinement
definition for the transaction system and then for the overall DaisyNFS
correctness theorem.

The transaction system does not require any initialization code, as long as the
disk is initially all-zero. The theorem relates an all-zero disk to a
transaction-system state full of zero objects. Our proof requires the caller to
use each disk block with a consistent object size (since changes in object size
are difficult to coordinate between threads). We call a mapping of block numbers
to object sizes a \emph{schema}, and enforce that the schema picked during
initialization is static for the rest of the execution. The Dafny interface also
uses a constant schema, which DaisyNFS sets to include a super block, some
number of inode blocks, then some allocator blocks, and finally data blocks
for the remainder of the disk.

Unlike the transaction system, the file system does require initialization code
separate from recovery, because an all-zero set of objects is not a valid
file-system state (at minimum, the root inode should be an allocated directory).
More precisely then, DaisyNFS's specification as stated above requires that the
caller first run \cc{Init} on an all-zero disk, initially establishing the
file-system invariant. After that, the system should run the recovery method on
each reboot to behave like $\sdfy$ according to \cref{thm:txn-appendix}.

While we can use \cref{thm:txn-appendix} to show the code behaves like $\sdfy$, why
does $\sdfy$ behave like $\snfs$ across a crash? The specification we prove in
Dafny about recovery is this:

\begin{verbatim}
constructor Recover(txs: TxnSystem,
    ghost fs: Filesys)
  requires fs.Valid()
  requires same_txn_disk(txs, fs.txs)
  ensures this.data == fs.data
  ensures Valid()
\end{verbatim}

The key to this proof is the argument \texttt{fs: Filesys}, which encodes the
assumptions about the file system just before the crash. Because this file
system is now lost, it is a \emph{ghost} argument; we cannot implement recovery
using the
old in-memory state, but do get to assume that the transaction system \cc{txs}
is the same as before the crash. Each operation is handled
atomically, so recovery also assumes that the file system satisfies its
\cc{fs.Valid()} invariant. What we prove in this specification Dafny
is that \cc{Recover()} produces a new, valid \cc{Filesys} in memory with the
same abstract state as the old one. With the file-system invariant restored, we
can say that operations after a crash and recovery continue to follow the NFS
state machine.

\subsection{A hypothetical mechanized proof}

One question one might have is why this proof needs to happen on paper. The
agenda for mechanizing it would look something like this: compile the Dafny code
to Go, then abstract away the details of the network interaction to produce
$\server_{\mathrm{code}}$. The same compiled code also gives an explicit
definition of $\sdfy$ where we use Atomically to wrap each transaction. Then, we
might give explicit definitions for each intermediate step, in particular
$\sdfyAlt$ and its compiled version $\linkedcode$. We would prove that
$\sdfyAlt$ is safe; perhaps we could assume dynamic safety of $\sdfy$, trusting
that part to the Dafny proof. We could then show in Coq that
$\server_{\mathrm{code}} \approx \linkedcode \refines \sdfyAlt \equiv \sdfy$,
applying Theorem 1 for the middle step.

There are many barriers to this process. The first is that Goose cannot
translate all of the patterns produced by the Dafny compiler in order to
initially import the Dafny code into Coq. The second is that $\sdfyAlt$ is a
non-trivial transformation of $\sdfy$ to really carry out. Finally, even after
constructing the intermediate programs we would need infrastructure for
reasoning about program equivalences in GooseLang. All of these are surmountable
problems, but it is unlikely that they increase confidence in the proof beyond
careful manual inspection.

